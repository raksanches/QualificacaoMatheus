\documentclass[12pt,
	openright,	% capítulos começam em pág ímpar (insere página vazia caso preciso)
	twoside,    % para impressão em anverso (frente) e verso. Oposto a oneside - Nota: utilizar \imprimirfolhaderosto*
	%oneside,   % para impressão em páginas separadas (somente anverso) -  Nota: utilizar \imprimirfolhaderosto
	a4paper,			% tamanho do papel. 
	sumario=tradicional,
	% -- opções da classe abntex2 --
	% -- opções do pacote babel --
	english,			% idioma adicional para hifenização
	french, 			% idioma adicional para hifenização
	%spanish,			% idioma adicional para hifenização
	brazil				% o último idioma é o principal do documento
	% {USPSC} configura o cabeçalho contendo apenas o número da página
]{USPSC}
%]{USPSC1} para utilizar o cabeçalho diferenciado para as páginas pares e ímpares como indicado abaixo:
%- páginas ímpares: cabeçalho com seções ou subseções e o número da página
%- páginas pares: cabeçalho com o número da página e o título do capítulo 

%% pacotes para tabelas
\usepackage{booktabs}  % table
\usepackage{float} % Fixa tabelas e figuras no local exato
\usepackage{multirow} % múltiplas linhas
\usepackage[table]{xcolor}

% Pacotes básicos - Fundamentais 
\usepackage[T1]{fontenc}		% Seleção de códigos de fonte.
\usepackage[utf8]{inputenc}		% Codificação do documento (conversão automática dos acentos)
\usepackage{lmodern}			% Usa a fonte Latin Modern

% Para utilizar a fonte Times New Roman, inclua uma % no início do comando acima  "\usepackage{lmodern}"
\usepackage{times}			% Usa a fonte Times New Roman	
% Lembre-se de alterar a fonte no comando que imprime o preâmbulo no arquivo da Classe USPSC.cls	

%Chamada de pacotes para tratamento de termos matemáticos
%\usepackage{mtpro2}
\usepackage{amsmath}
\usepackage{amssymb}
\usepackage{amsfonts}
\usepackage{lastpage}			% Usado pela Ficha catalográfica
%\usepackage{indentfirst}		% Indenta o primeiro parágrafo de cada seção.
%\usepackage{color}				% Controle das cores
\usepackage{graphicx}			% Inclusão de gráficos
\usepackage{microtype}			% para melhorias de justificação
\usepackage{pdfpages}
\usepackage{makeidx}		% para gerar índice remissimo
\usepackage{subfiles}		% para controlar subarquivos
\usepackage[ampersand]{easylist} % para defininar listas
\usepackage{titlesec, blindtext}
%\usepackage{mathptmx}
%\renewcommand{\rmdefault}{ptm}
\usepackage{nccmath}
\usepackage{extarrows} % imply with text below
%\usepackage{indentfirst}
\usepackage{caption}
\usepackage{subcaption}
\usepackage{multicol,lipsum}
\usepackage{fancybox}
\usepackage{longtable}
\usepackage{mathtools}
\usepackage{enumitem}  % Enumerate, Itemize
\usepackage{tabu}  % tabela adequada a fórmulas
\usepackage{pbox} % caixa dentro de uma célula da tabela
\usepackage[b]{esvect} % seta de vetor sobrescrita
\usepackage{framed} % quadro ao redor de um texto
\usepackage{verbatim}
\usepackage[percent]{overpic}
\usepackage{makecell} %\multirowcell
\usepackage{vwcol}
\usepackage{arydshln} % dashed line on matrix
\usepackage[portuguese,onelanguage,ruled,linesnumbered]{algorithm2e}
\usepackage{hhline}
\usepackage{dblfloatfix}    % To enable figures at the bottom of page
\usepackage[section]{placeins} % make floating elements don't switch from sections

% Pacotes de citações - Citações padrão ABNT - Sistema autor-data
\usepackage[alf,abnt-emphasize=bf, abnt-thesis-year=both,
	abnt-repeated-author-omit=yes, abnt-last-names=abnt,
	abnt-etal-cite,abnt-etal-list=3, abnt-etal-text=default,
	abnt-and-type=e,
	abnt-doi=doi, abnt-url-package=none,
	abnt-verbatim-entry=no]{abntex2cite}

% Pacotes de Nota de rodape
% O presente modelo adota o formato numérico para as notas de rodapés quando utiliza o sistema de chamada autor-data para citações e referências.
%\renewcommand{\footnotesize}{\small} %Comando para diminuir a fonte das notas de rodapé	

% pacotes de tabelas
\usepackage{multicol}	% Suporte a mesclagens em colunas
\usepackage{multirow}	% Suporte a mesclagens em linhas
\usepackage{longtable}	% Tabelas com várias páginas
\usepackage{threeparttablex}	% notas no longtable
\usepackage{array}

\definecolor{LightGray}{gray}{0.9}

% DADOS INICIAIS 
\include{Elementos_pre-textuais}

% Configurações de aparência do PDF final
% alterando o aspecto da cor azul
\definecolor{blue}{RGB}{41,5,195}

% informações do PDF
\makeatletter
\hypersetup{
	%pagebackref=true,
	pdftitle={\@title},
	pdfauthor={\@author},
	pdfsubject={\imprimirpreambulo},
	pdfcreator={LaTeX with abnTeX2},
	pdfkeywords={abnt}{latex}{abntex}{USPSC}{trabalho acadêmico},
	colorlinks=true,		% false: boxed links; true: colored links
	linkcolor=blue,        % BLUE  	  % color of internal links 
	citecolor=blue,        % BLUE	  % color of links to bibliography
	filecolor=magenta,     % MAGENTA  % color of file links
	urlcolor=blue,	       % BLUE
	bookmarksdepth=4
}
\makeatother

% Espaçamentos entre linhas e parágrafos 
\setlength{\parindent}{1.3cm} % Tamanho do parágrafo
\setlength{\parskip}{0.2cm}   % % Controle do espaçamento entre um parágrafo e outro (tente também \onelineskip)

% compila o sumário e índice
\makeindex

\renewcommand\chaptertitlename{Capítulo~}
\titleformat{\chapter}[display]
{\normalfont\huge\bfseries}{\chaptertitlename\ \thechapter}{20pt}{\Huge}
\titleformat*{\section}{\Large\bfseries}
\titleformat*{\subsection}{\large\bfseries}
\titleformat*{\subsubsection}{\bfseries}
\titleformat*{\paragraph}{\bfseries}
\titleformat*{\subparagraph}{\bfseries}

\renewcommand{\bar}[1]{\overline{#1}}

\newcommand{\script}[1]{\mathcal{#1}}
\newcommand{\abs}[1]{\left|{#1}\right|}
\newcommand{\tr}{\mathop{\mathrm{tr}}\nolimits}
\newcommand{\dev}[1]{\mathop{\mathrm{dev}}\nolimits{#1}}
\newcommand{\Rey}{\mathrm{Re}}
\newcommand{\NN}{\mathbf{\nabla_y}}
\newcommand{\NNx}{\mathbf{\nabla_x}}
\newcommand{\NNT}{\mathbf{\nabla}^T_\mathbf{y}}
\newcommand{\Lapl}{\mathbf{\nabla}^2_\mathbf{y}}
\newcommand{\apder}[2]{\left.\frac{\partial{#1}}{\partial{#2}}\right|_{\BB{y}}}
\newcommand{\apderx}[2]{\left.\frac{\partial{#1}}{\partial{#2}}\right|_{\BB{x}}}
\newcommand{\apderxh}[2]{\left.\frac{\partial{#1}}{\partial{#2}}\right|_{\bhat{x}}}
\newcommand{\der}[2]{\frac{\partial{#1}}{\partial{#2}}}
\newcommand{\dder}[2]{\frac{\partial^2{#1}}{\partial{#2}^2}}
\newcommand{\BB}[1]{\mathbf{#1}}
\newcommand{\BBB}[1]{\overline{\mathbf{#1}}}
\newcommand{\dBB}[1]{\dot{\mathbf{#1}}}
\newcommand{\ddBB}[1]{\ddot{\mathbf{#1}}}
\newcommand{\uhat}{\hat{\mathbf{u}}}
\newcommand{\bhat}[1]{\hat{\mathbf{#1}}}
\newcommand{\that}[1]{\tilde{\mathbf{#1}}}
\newcommand{\Def}{\BB{\varepsilon}}
\newcommand{\deftax}{\dot{\mathbf{\varepsilon}}}
\newcommand{\tens}{\mathbf{\sigma}}
\newcommand{\norm}[1]{\left\lVert{#1}\right\lVert}
\newcommand{\bigpar}[1]{\left({#1}\right)}
%LARGE EDDY SIMULATION:
\newcommand{\Dfil}{D_\Delta(\BB{y})}
\newcommand{\yfil}{\BB{y}_\Delta}
\newcommand{\deffil}{\dot{\bar{\mathbf{\varepsilon}}}}
\newcommand{\deffilij}[1]{\dot{\bar{\varepsilon}}_{#1}}
%VARIATIONAL MULTISCALE METHOD
\newcommand{\supg}{\mathrm{SUPG}}
\newcommand{\pspg}{\mathrm{PSPG}}
\newcommand{\lsic}{\mathrm{LSIC}}
\newcommand{\sups}{\mathrm{SUPS}}
\newcommand{\sugn}{\mathrm{SUGN}}
\newcommand{\rgn}{\mathrm{RGN}}
\newcommand{\rM}{\BB{r}_\mathrm{M}(\BBB{u},\bar{p})}
\newcommand{\rC}{r_\mathrm{C}(\BBB{u})}
\newcommand{\rMi}[1]{(r_\mathrm{M})_{#1}}
\newcommand{\rCi}{r_\mathrm{C}}
\newcommand{\NM}{\BB{N}_\mathrm{M}}
\newcommand{\NC}{\BB{N}_\mathrm{C}}
\newcommand{\intDom}[1]{\int_{\Omega}{#1 d\Omega}}
\newcommand{\intFront}[1]{\int_{\Gamma}{#1 d\Gamma}}
\newcommand{\intNeumann}[1]{\int_{\Gamma_N}{#1 d\Gamma_N}}
\newcommand{\intDirichlet}[1]{\int_{\Gamma_D}{#1 d\Gamma_D}}
\newcommand{\SintDom}[1]{\sum_{e=1}^{n_{el}}{\int_{\Omega^e}{#1 d\Omega^e}}}

%\newcommand{\agdt}{\alpha_f\gamma\Delta t}
\newcommand{\agdt}{\beta_f}
\newcommand{\am}{\alpha_m}
\newcommand{\dij}{\delta_{ij}}
\newcommand{\tsups}{\tau_\sups}
\newcommand{\nlsic}{\nu_\lsic}
\newcommand{\uub}[1]{\bigpar{\bar{u}_{#1}-\hat{u}_{#1}}}
\newcommand{\intDomna}[1]{\int_{\Omega^{n+\alpha_f}}{#1 d\Omega}}
%REYNOLDS-AVERAGED NAVIER-STOKES
\newcommand{\umed}{\bar{\BB{u}}}
\newcommand{\ufl}{\BB{u}'}
\newcommand{\epmean}{\dot{\bar{\varepsilon}}}

% Início do documento
\begin{document}

% ----------------------------------------------------------
% CONFIGURAÇÕES INICIAS
% ----------------------------------------------------------
% Espaços antes e depois das equações
\setlength{\abovedisplayskip}{0pt}
\setlength{\belowdisplayskip}{0pt}
% ----------------------------------------------------------	
\selectlanguage{brazil} % Seleciona o idioma do documento (conforme pacotes do babel)
\frenchspacing % Retira espaço extra obsoleto entre as frases. 
% Formatação dos Títulos

% ----------------------------------------------------------
% ELEMENTOS PRÉ-TEXTUAIS
% ----------------------------------------------------------
% Capa
\imprimircapa
% Folha de rosto
\imprimirfolhaderosto % (o * indica impressão em anverso (frente) e verso )

% Resumo
\subfile{Resumo}

% Abstract
\subfile{Abstract}

% inserir lista de figurass
\pdfbookmark[0]{\listfigurename}{lof}
\listoffigures*
\cleardoublepage
% inserir lista de tabelas
\pdfbookmark[0]{\listtablename}{lot}
\listoftables*
\cleardoublepage
% inserir lista de quadros
%\pdfbookmark[0]{\listofquadroname}{loq}
%\listofquadro*
%\cleardoublepage
% inserir lista de abreviaturas e siglas
\begin{siglas}
	\item[ALE] Lagrangiana-Euleriana Arbitrária - \textit{Arbitrary Lagrangian-Eulerian}
    \item[LBB] Ladyzhenskaya-Babuška-Brezzi
	\item[CFD] Dinâmica dos Fluidos Computacional - \textit{Computational Fluid Dynamics}
	\item[CSD] Dinâmica dos Sólidos Computacional - \textit{Computational Solid Dynamics}
    \item[GLS] Galerkin Least-Squares
	\item[IFE] Interação Fluido-Estrutura
	\item[LES] Simulação de Grandes Vórtices - \textit{Large Eddy Simulation}
    \item[LSIC] \textit{Least-Squares on Incompressibility Constraint}
	\item[MEF] Método dos Elementos Finitos
    \item[PSPG] \textit{Pressure-Stabilizating/Petrov-Galerkin}
	\item[RANS] \textit{Reynolds-Averaged Navier-Stokes}
    \item[SGS] Subgrid-Scales
    \item[SUPG] \textit{Streamline-Upwind/Petrov-Galerkin}
	\item[VMS] Métodos Variacionais Multiescala - \textit{Variational Multi-Scale}
\end{siglas}
% inserir lista de símbolos
\begin{simbolos}
    \item[Operadores]
    \item[$\dev(\cdot)$] Parte desviadora do tensor
    \item[$H^1$] Espaço de Sobolev de ordem 1
    \item[$L^2$] Espaço das funções de quadrado integrável
    \item[$\tr(\cdot)$] Traço de um tensor
    \item[$\mathbf{\nabla}(\cdot)$] Gradiente
    \item[$\mathbf{\nabla}\cdot(\cdot)$] Divergente
    \item[$\mathbf{\nabla}^2(\cdot)$] Laplaciano
    \item[$\cdot$] Produto interno
    \item[$:$] Contração dupla
    \item[$\times$] Produto vetorial
    \item[$\otimes$] Produto tensorial
    \item[$\sum$] Somatório
    \item[$\prod$] Produtório
    
    \item[Parâmetros Gerais]
    \item[$\hat{\mathbf{e}}_i$] Vetor versor na direção $i$
    \item[$\mathbb{I}$] Tensor identidade de quarta ordem
    \item[$\mathbf{I}$] Tensor identidade de segunda ordem
    \item[$m$] Massa
    \item[$n_{sd}$] Número de dimensões espaciais do problema
    \item[$\Rey$] Número de Reynolds
    \item[$t$] Tempo
    \item[$V$] Volume
    \item[$\delta_{ij}$] Delta de Kronecker
    \item[$\Delta t$] Intervalo discreto de tempo
    \item[$\rho$] Massa específica
    \item[$\mathbf{\sigma}$] Tensor de tensões de Cauchy
    \item[$\phi$] Propriedade qualquer
    
    \item[Configurações do Contínuo]
    \item[$\mathbf{x}$] Posição inicial (ou posição material)
    \item[$\mathbf{\hat{x}}$] Posição de referência
    \item[$\mathbf{y}$] Posição atual (ou posição espacial)
    \item[$\Gamma_D$] Fronteira de Dirichlet
    \item[$\Gamma_N$] Fronteira de Neumann
    \item[$\Omega$] Domínio de análise na configuração atual
    \item[$\Omega_0$] Domínio de análise na configuração inicial
    \item[$\hat{\Omega}$] Domínio de análise na configuração de referência
    
    \item[Dinâmica dos Sólidos Computacional]
    \item[$\mathbb{A}$] Medida de deformação de Almansi
    \item[$\mathbf{A}$] Gradiente da função de mudança de configuração
    \item[$\mathbb{E}$] Medida de deformação de Green-Lagrange
    \item[$\mathbf{f}$] Função de mudança de configuração
    \item[$\mathbb{H}$] Medida de deformação de Hencky
    \item[$J$] Jacobiano da mudança de configuração
    \item[$\mathbb{K}$] Energia cinética
    \item[$n$] Vetor normal à superfície na configuração atual
    \item[$N$] Vetor normal à superfície na configuração inicial
    \item[$\mathbb{P}$] Energia das Forças Externas
    \item[$\mathbf{P}$] Tensor de tensões de Piola-Kirchhoff de primeira espécie
    \item[$u_e$] Energia específica de deformação
    \item[$\mathbb{U}$] Energia de deformação
    \item[$\mathbf{\varepsilon}$] Tensor de deformação linear
    \item[$\varepsilon_V$] Deformação volumétrica
    \item[$\Pi$] Energia total
    
    \item[Dinâmica dos Fluidos Computacional]
    \item[$\mathbf{A}$] Gradiente da função de mudança de configuração
    \item[$\mathbf{c}$] Força de corpo
    \item[$\script{D}$] Tensor constitutivo de quarta ordem
    \item[$\mathbf{f}$] Força por unidade de massa
    \item[$\mathbf{f}$] Função mudança de configuração
    \item[$\mathbf{F}$] Resultante das forças externas
    \item[$\mathbf{g}$] Velocidades prescritas na fronteira de Dirichlet
    \item[$\mathbf{h}$] Forças de superfícies prescritas na fronteira de Neumann
    \item[$n$] Vetor normal à superfície
    \item[$p$] Campo de pressões
    \item[$\mathbf{q}$] Resultante das forças externas por unidade de volume
    \item[$\script{S}_u$] Espaço de funções tentativas para o campo de velocidades
    \item[$\script{S}_p$] Espaço de funções tentativas para o campo de pressões
    \item[$\mathbf{u}$] Campo de velocidades
    \item[$\script{V}_u$] Espaço de funções teste para o campo de velocidades
    \item[$\script{V}_p$] Espaço de funções teste para o campo de pressões
    \item[$\mathbf{\hat{u}}$] Campo de velocidades da malha
    \item[$\mathbf{\dot{\varepsilon}}$] Tensor de taxa de deformação
    \item[$\mu$] Viscosidade cinemática
    \item[$\nu$] Viscosidade dinâmica
    \item[$\tau$] Tensor desviador

    \item[\textit{Large Eddy Simulation}]
    \item[$\mathbf{C}$] Tensor de termos cruzados
    \item[$C_S$] Constante de Smagorinsky
    \item[$\Dfil$] Domínio de abrangência do filtro
    \item[$g$] Filtro
    \item[$\mathbf{L}$] Tensor de Leonard
    \item[$\mathbf{R}$] Tensor SGS de Reynolds
    \item[$\BB{T}$] Tensor SGS
    \item[$\BB{T}_S$] Tensor SGS de Smagorinsky
    \item[$\yfil$] Ponto na vizinhança de $\BB{y}$ interno à $\Dfil$
    \item[$\Delta$] Tamanho da malha
    \item[$\deffil$] Taxa de deformação em grandes escalas
    \item[$\nu_T$] Viscosidade de vórtice
    \item[$\bar{\phi}$] Propriedade filtrada
    \item[$\phi'$] Propriedade não filtrada
    
    \item[\textit{Variational Multi-Scale}]
    
    \item[\textit{Reynolds-Averaged Navier-Stokes}]
\end{simbolos}

% inserir o sumario
\pdfbookmark[0]{\contentsname}{toc}
\tableofcontents*
\cleardoublepage

% ----------------------------------------------------------
% ELEMENTOS TEXTUAIS
% ----------------------------------------------------------
\textual

\renewcommand\chaptername{}

\subfile{Capitulos/C1-Introducao}
\subfile{Capitulos/C2-EstadoArte}
\subfile{Capitulos/C3-FundTeorica/C3-FundTeorica}
\subfile{Capitulos/C4-MetodologiaCronograma}

% ----------------------------------------------------------
% ELEMENTOS PÓS-TEXTUAIS
% ----------------------------------------------------------
\postextual

% REFERÊNCIAS BIBLIOGRÁFICAS
\bibliography{Referencias}

\end{document}