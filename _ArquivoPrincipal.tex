\documentclass[12pt,
	openright,	% capítulos começam em pág ímpar (insere página vazia caso preciso)
	twoside,    % para impressão em anverso (frente) e verso. Oposto a oneside - Nota: utilizar \imprimirfolhaderosto*
	%oneside,   % para impressão em páginas separadas (somente anverso) -  Nota: utilizar \imprimirfolhaderosto
	a4paper,			% tamanho do papel. 
	sumario=tradicional,
	% -- opções da classe abntex2 --
	% -- opções do pacote babel --
	english,			% idioma adicional para hifenização
	french, 			% idioma adicional para hifenização
	%spanish,			% idioma adicional para hifenização
	brazil				% o último idioma é o principal do documento
	% {USPSC} configura o cabeçalho contendo apenas o número da página
]{USPSC}
%]{USPSC1} para utilizar o cabeçalho diferenciado para as páginas pares e ímpares como indicado abaixo:
%- páginas ímpares: cabeçalho com seções ou subseções e o número da página
%- páginas pares: cabeçalho com o número da página e o título do capítulo 

%% pacotes para tabelas
\usepackage{booktabs}  % table
\usepackage{float} % Fixa tabelas e figuras no local exato
\usepackage{multirow} % múltiplas linhas
\usepackage[table]{xcolor}

% Pacotes básicos - Fundamentais 
\usepackage[T1]{fontenc}		% Seleção de códigos de fonte.
\usepackage[utf8]{inputenc}		% Codificação do documento (conversão automática dos acentos)
\usepackage{lmodern}			% Usa a fonte Latin Modern

% Para utilizar a fonte Times New Roman, inclua uma % no início do comando acima  "\usepackage{lmodern}"
\usepackage{times}			% Usa a fonte Times New Roman	
% Lembre-se de alterar a fonte no comando que imprime o preâmbulo no arquivo da Classe USPSC.cls	

%Chamada de pacotes para tratamento de termos matemáticos
%\usepackage{mtpro2}
\usepackage{amsmath}
\usepackage{amssymb}
\usepackage{amsfonts}
\usepackage{lastpage}			% Usado pela Ficha catalográfica
%\usepackage{indentfirst}		% Indenta o primeiro parágrafo de cada seção.
%\usepackage{color}				% Controle das cores
\usepackage{graphicx}			% Inclusão de gráficos
\usepackage{microtype}			% para melhorias de justificação
\usepackage{pdfpages}
\usepackage{makeidx}		% para gerar índice remissimo
\usepackage{subfiles}		% para controlar subarquivos
\usepackage[ampersand]{easylist} % para defininar listas
\usepackage{titlesec, blindtext}
%\usepackage{mathptmx}
%\renewcommand{\rmdefault}{ptm}
\usepackage{nccmath}
\usepackage{extarrows} % imply with text below
%\usepackage{indentfirst}
\usepackage{caption}
\usepackage{subcaption}
\usepackage{multicol,lipsum}
\usepackage{fancybox}
\usepackage{longtable}
\usepackage{mathtools}
\usepackage{enumitem}  % Enumerate, Itemize
\usepackage{tabu}  % tabela adequada a fórmulas
\usepackage{pbox} % caixa dentro de uma célula da tabela
\usepackage[b]{esvect} % seta de vetor sobrescrita
\usepackage{framed} % quadro ao redor de um texto
\usepackage{verbatim}
\usepackage[percent]{overpic}
\usepackage{makecell} %\multirowcell
\usepackage{vwcol}
\usepackage{arydshln} % dashed line on matrix
\usepackage[portuguese,onelanguage,ruled,linesnumbered]{algorithm2e}
\usepackage{hhline}
\usepackage{dblfloatfix}    % To enable figures at the bottom of page
\usepackage[section]{placeins} % make floating elements don't switch from sections

% Pacotes de citações - Citações padrão ABNT - Sistema autor-data
\usepackage[alf,abnt-emphasize=bf, abnt-thesis-year=both,
	abnt-repeated-author-omit=yes, abnt-last-names=abnt,
	abnt-etal-cite,abnt-etal-list=3, abnt-etal-text=default,
	abnt-and-type=e,
	abnt-doi=doi, abnt-url-package=none,
	abnt-verbatim-entry=no]{abntex2cite}

% Pacotes de Nota de rodape
% O presente modelo adota o formato numérico para as notas de rodapés quando utiliza o sistema de chamada autor-data para citações e referências.
%\renewcommand{\footnotesize}{\small} %Comando para diminuir a fonte das notas de rodapé	

% pacotes de tabelas
\usepackage{multicol}	% Suporte a mesclagens em colunas
\usepackage{multirow}	% Suporte a mesclagens em linhas
\usepackage{longtable}	% Tabelas com várias páginas
\usepackage{threeparttablex}	% notas no longtable
\usepackage{array}

\definecolor{LightGray}{gray}{0.9}

% DADOS INICIAIS 
\include{Elementos_pre-textuais}

% Configurações de aparência do PDF final
% alterando o aspecto da cor azul
\definecolor{blue}{RGB}{41,5,195}

% informações do PDF
\makeatletter
\hypersetup{
	%pagebackref=true,
	pdftitle={\@title},
	pdfauthor={\@author},
	pdfsubject={\imprimirpreambulo},
	pdfcreator={LaTeX with abnTeX2},
	pdfkeywords={abnt}{latex}{abntex}{USPSC}{trabalho acadêmico},
	colorlinks=true,		% false: boxed links; true: colored links
	linkcolor=blue,        % BLUE  	  % color of internal links 
	citecolor=blue,        % BLUE	  % color of links to bibliography
	filecolor=magenta,     % MAGENTA  % color of file links
	urlcolor=blue,	       % BLUE
	bookmarksdepth=4
}
\makeatother

% Espaçamentos entre linhas e parágrafos 
\setlength{\parindent}{1.3cm} % Tamanho do parágrafo
\setlength{\parskip}{0.2cm}   % % Controle do espaçamento entre um parágrafo e outro (tente também \onelineskip)

% compila o sumário e índice
\makeindex

\renewcommand\chaptertitlename{Capítulo~}
\titleformat{\chapter}[display]
{\normalfont\huge\bfseries}{\chaptertitlename\ \thechapter}{20pt}{\Huge}
\titleformat*{\section}{\Large\bfseries}
\titleformat*{\subsection}{\large\bfseries}
\titleformat*{\subsubsection}{\bfseries}
\titleformat*{\paragraph}{\bfseries}
\titleformat*{\subparagraph}{\bfseries}

\renewcommand{\bar}[1]{\overline{#1}}

\newcommand{\DNS}{\textit{Direct Numerical Simulation}}
\newcommand{\RANS}{\textit{Reynolds-Averaged Navier-Stokes}}
\newcommand{\LES}{\textit{Large Eddy Simulation}}
\newcommand{\VMS}{\textit{Variational Multi-Scale}}

\newcommand{\script}[1]{\mathcal{#1}}
\newcommand{\abs}[1]{\left|{#1}\right|}
\newcommand{\tr}{\mathop{\mathrm{tr}}\nolimits}
\newcommand{\dev}[1]{\mathop{\mathrm{dev}}\nolimits{#1}}
\newcommand{\Rey}{\mathrm{Re}}
\newcommand{\NN}{\mathbf{\nabla_y}}
\newcommand{\NNx}{\mathbf{\nabla_x}}
\newcommand{\NNT}{\mathbf{\nabla}^T_\mathbf{y}}
\newcommand{\Lapl}{\mathbf{\nabla}^2_\mathbf{y}}
\newcommand{\apder}[2]{\left.\frac{\partial{#1}}{\partial{#2}}\right|_{\BB{y}}}
\newcommand{\apderx}[2]{\left.\frac{\partial{#1}}{\partial{#2}}\right|_{\BB{x}}}
\newcommand{\apderxh}[2]{\left.\frac{\partial{#1}}{\partial{#2}}\right|_{\bhat{x}}}
\newcommand{\der}[2]{\frac{\partial{#1}}{\partial{#2}}}
\newcommand{\Dder}[3]{\frac{\partial^2{#1}}{\partial{#2}\partial{#3}}}
\newcommand{\dder}[2]{\frac{\partial^2{#1}}{\partial{#2}^2}}
\newcommand{\BB}[1]{\mathbf{#1}}
\newcommand{\BBB}[1]{\overline{\mathbf{#1}}}
\newcommand{\dBB}[1]{\dot{\mathbf{#1}}}
\newcommand{\ddBB}[1]{\ddot{\mathbf{#1}}}
\newcommand{\uhat}{\hat{\mathbf{u}}}
\newcommand{\bhat}[1]{\hat{\mathbf{#1}}}
\newcommand{\that}[1]{\tilde{\mathbf{#1}}}
\newcommand{\Def}{\BB{\varepsilon}}
\newcommand{\deftax}{\dot{\mathbf{\varepsilon}}}
\newcommand{\tens}{\mathbf{\sigma}}
\newcommand{\norm}[1]{\left\lVert{#1}\right\lVert}
\newcommand{\bigpar}[1]{\left({#1}\right)}
%LARGE EDDY SIMULATION:
\newcommand{\Dfil}{D_\Delta(\BB{y})}
\newcommand{\yfil}{\BB{y}_\Delta}
\newcommand{\deffil}{\bar{\mathbf{S}}}
\newcommand{\deffilij}[1]{\bar{S}_{#1}}
%\newcommand{\deffil}{\dot{\bar{\mathbf{\varepsilon}}}}
%\newcommand{\deffilij}[1]{\dot{\bar{\varepsilon}}_{#1}}

%VARIATIONAL MULTISCALE METHOD
\newcommand{\supg}{\mathrm{SUPG}}
\newcommand{\pspg}{\mathrm{PSPG}}
\newcommand{\lsic}{\mathrm{LSIC}}
\newcommand{\sups}{\mathrm{SUPS}}
\newcommand{\sugn}{\mathrm{SUGN}}
\newcommand{\rgn}{\mathrm{RGN}}
\newcommand{\rM}{\BB{r}_\mathrm{M}(\BBB{u},\bar{p})}
\newcommand{\rC}{r_\mathrm{C}(\BBB{u})}
\newcommand{\rMi}[1]{(r_\mathrm{M})_{#1}}
\newcommand{\rCi}{r_\mathrm{C}}
\newcommand{\NM}{\BB{N}_\mathrm{M}}
\newcommand{\NC}{\BB{N}_\mathrm{C}}
\newcommand{\intDom}[1]{\int_{\Omega}{#1 d\Omega}}
\newcommand{\intFront}[1]{\int_{\Gamma}{#1 d\Gamma}}
\newcommand{\intDomi}[1]{\int_{\Omega_0}{#1 d\Omega_0}}
\newcommand{\intFronti}[1]{\int_{\Gamma_0}{#1 d\Gamma_0}}
\newcommand{\intNeumann}[1]{\int_{\Gamma_N}{#1 d\Gamma_N}}
\newcommand{\intDirichlet}[1]{\int_{\Gamma_D}{#1 d\Gamma_D}}
\newcommand{\SintDom}[1]{\sum_{e=1}^{n_{el}}{\int_{\Omega^e}{#1 d\Omega^e}}}

%\newcommand{\agdt}{\alpha_f\gamma\Delta t}
\newcommand{\agdt}{\beta_f}
\newcommand{\am}{\alpha_m}
\newcommand{\dij}{\delta_{ij}}
\newcommand{\tsups}{\tau_\sups}
\newcommand{\nlsic}{\nu_\lsic}
\newcommand{\uub}[1]{\bigpar{\bar{u}_{#1}-\hat{u}_{#1}}}
\newcommand{\intDomna}[1]{\int_{\Omega^{n+\alpha_f}}{#1 d\Omega}}
%REYNOLDS-AVERAGED NAVIER-STOKES
\newcommand{\umed}{\bar{\BB{u}}}
\newcommand{\ufl}{\BB{u}'}
\newcommand{\epmean}{\bar{S}}
\newcommand{\ke}{k-\varepsilon}
\newcommand{\kw}{k-\omega}
\newcommand{\te}{$\tau_{ij}-\varepsilon$}
\newcommand{\ep}{\varepsilon}
\newcommand{\pr}{\script{P}}

\newcommand{\Alpha}{\mathrm{A}}

% Início do documento
\begin{document}

% ----------------------------------------------------------
% CONFIGURAÇÕES INICIAS
% ----------------------------------------------------------
% Espaços antes e depois das equações
\setlength{\abovedisplayskip}{0pt}
\setlength{\belowdisplayskip}{0pt}
% ----------------------------------------------------------	
\selectlanguage{brazil} % Seleciona o idioma do documento (conforme pacotes do babel)
\frenchspacing % Retira espaço extra obsoleto entre as frases. 
% Formatação dos Títulos

% ----------------------------------------------------------
% ELEMENTOS PRÉ-TEXTUAIS
% ----------------------------------------------------------
% Capa
\imprimircapa
% Folha de rosto
\imprimirfolhaderosto % (o * indica impressão em anverso (frente) e verso )

% Resumo
\subfile{Resumo}

% Abstract
\subfile{Abstract}

% inserir lista de figurass
\pdfbookmark[0]{\listfigurename}{lof}
\listoffigures*
\cleardoublepage
% inserir lista de tabelas
\pdfbookmark[0]{\listtablename}{lot}
\listoftables*
\cleardoublepage
% inserir lista de quadros
%\pdfbookmark[0]{\listofquadroname}{loq}
%\listofquadro*
%\cleardoublepage
% inserir lista de abreviaturas e siglas
\begin{siglas}
    \item[ALE] Lagrangiana-Euleriana Arbitrária - \textit{Arbitrary Lagrangian-Eulerian}
    \item[LBB] Ladyzhenskaya-Babuška-Brezzi
    \item[CFD] Dinâmica dos Fluidos Computacional - \textit{Computational Fluid Dynamics}
    \item[CSD] Dinâmica dos Sólidos Computacional - \textit{Computational Solid Dynamics}
    \item[DNS] Simulação Numérica Direta - \textit{Direct Numerical Simulation}
    \item[GLS] Galerkin Least-Squares
    \item[IFE] Interação Fluido-Estrutura
    \item[IP] Problema de Interação - \textit{Interaction Problem}
    \item[LBB] Condições de \textit{Ladyzhenskaya-Babuška-Brezzi}
    \item[LES] Simulação de Grandes Vórtices - \textit{Large Eddy Simulation}
    \item[LSIC] \textit{Least-Squares on Incompressibility Constraint}
    \item[MEF] Método dos Elementos Finitos
    \item[PSPG] \textit{Pressure-Stabilizating/Petrov-Galerkin}
    \item[RANS] \textit{Reynolds-Averaged Navier-Stokes}
    \item[RBVMS] \textit{Residual-Based Variational Multi-Scale}
    \item[SGS] \textit{Subgrid-Scales}
    \item[SUPG] \textit{Streamline-Upwind/Petrov-Galerkin}
    \item[SVK] Saint-Venant-Kirchhoff
    \item[VMS] Métodos Variacionais Multiescala - \textit{Variational Multi-Scale}
\end{siglas}
% inserir lista de símbolos
\begin{simbolos}
    \item[\textbf{Operadores}]
    \item[$\dev(\cdot)$] Parte desviadora do tensor
    \item[$\det{(\cdot)}$] Determinante
    \item[$D(\cdot)/Dt$] Derivada material
    \item[$H^1$] Espaço de Sobolev de ordem 1
    \item[$L^2$] Espaço das funções de quadrado integrável
    \item[$\tr(\cdot)$] Traço de um tensor
    \item[$\BB{\nabla}(\cdot)$] Gradiente
    \item[$\BB{\nabla}\cdot(\cdot)$] Divergente
    \item[$\BB{\nabla}^2(\cdot)$] Laplaciano
    \item[$\cdot$] Produto interno
    \item[$:$] Contração dupla
    \item[$\times$] Produto vetorial
    \item[$\otimes$] Produto tensorial
    \item[$\sum$] Somatório
    \item[$\prod$] Produtório
    \item[$\norm{(\cdot)}$] Norma

    \item[\textbf{Parâmetros Gerais}]
    \item[$\hat{\BB{e}}_i$] Vetor versor na direção $i$
    \item[$\mathbb{I}$] Tensor identidade de quarta ordem
    \item[$\BB{I}$] Tensor identidade de segunda ordem
    \item[$m$] Massa
    \item[$N_a$] Função de forma
    \item[$n_{sd}$] Número de dimensões espaciais do problema
    \item[$t$] Tempo
    \item[$V$] Volume
    \item[$V_0$] Volume inicial
    \item[$\delta_{ij}$] Delta de Kronecker
    \item[$\Delta t$] Intervalo discreto de tempo
    \item[$\BB{\xi}$] Coordenada paramétrica
    \item[$\rho$] Massa específica
    \item[$\BB{\sigma}$] Tensor de tensões de Cauchy
    \item[$\phi$] Propriedade qualquer

    \item[\textbf{Configurações do Contínuo}]
    \item[$\BB{x}$] Posição inicial (ou posição material)
    \item[$\BB{\hat{x}}$] Posição de referência
    \item[$\BB{y}$] Posição atual (ou posição espacial)
    \item[$\Gamma$] Fronteira do domínio de análise
    \item[$\Gamma_D$] Fronteira de Dirichlet
    \item[$\Gamma_N$] Fronteira de Neumann
    \item[$\Omega$] Domínio de análise na configuração atual
    \item[$\Omega_0$] Domínio de análise na configuração inicial
    \item[$\Omega_\xi$] Domínio de análise no espaço paramétrico
    \item[$\hat{\Omega}$] Domínio de análise na configuração de referência

    \item[\textbf{Dinâmica dos Fluidos Computacional}]
    \item[$\BB{A}$] Gradiente da função de mudança de configuração
    \item[$\BB{c}$] Força de corpo
    \item[$\script{D}$] Tensor constitutivo de quarta ordem
    \item[$\BB{f}$] Força por unidade de massa
    \item[$\BB{f}$] Função mudança de configuração
    \item[$\BB{F}$] Resultante das forças externas
    \item[$\BB{g}$] Velocidades prescritas na fronteira de Dirichlet
    \item[$\BB{h}$] Forças de superfícies prescritas na fronteira de Neumann
    \item[$n$] Vetor normal à superfície
    \item[$p$] Campo de pressões
    \item[$q$] Função teste relacionado à equação da continuidade
    \item[$\BB{q}$] Resultante das forças externas por unidade de volume
    \item[$\Rey$] Número de Reynolds
    \item[$\script{S}_u$] Espaço de funções tentativas para o campo de velocidades
    \item[$\script{S}_p$] Espaço de funções tentativas para o campo de pressões
    \item[$\BB{u}$] Campo de velocidades
    \item[$\dot{\BB{u}}$] Campo de acelerações
    \item[$\script{V}_u$] Espaço de funções teste para o campo de velocidades
    \item[$\script{V}_p$] Espaço de funções teste para o campo de pressões
    \item[$\BB{w}$] Função teste relacionada à equação da conservação do momento
    \item[$\BB{w}$] Velocidade da malha em relação à configuração inicial
    \item[$\BB{\hat{u}}$] Campo de velocidades da malha
    \item[$\BB{\dot{\varepsilon}}$] Tensor de taxa de deformação
    \item[$\mu$] Viscosidade cinemática
    \item[$\nu$] Viscosidade dinâmica
    \item[$\tau$] Tensor de tensões desviadoras

    \item[\textbf{Dinâmica dos Sólidos Computacional}]
    \item[$\mathbb{A}$] Medida de deformação de Almansi
    \item[$\BB{A}$] Gradiente da função de mudança de configuração
    \item[$\BB{c}$] Forças de corpo
    \item[$\BB{c}^0$] Forças de corpo na configuração inicial
    \item[$\BB{C}$] Tensor de alongamento à direita de Cauchy-Green
    \item[$\script{C}$] Tensor constitutivo de quarta ordem
    \item[$E$] Módulo de elasticidade longitudinal (ou Módulo de Young)
    \item[$\mathbb{E}$] Medida de deformação de Green-Lagrange
    \item[$\BB{F}$] Resultante das forças externas
    \item[$\BB{f}$] Função de mudança de configuração
    \item[$\mathbb{H}$] Medida de deformação de Hencky
    \item[$J$] Jacobiano da mudança de configuração
    \item[$\mathbb{K}$] Energia cinética
    \item[$n$] Vetor normal à superfície na configuração atual
    \item[$N$] Vetor normal à superfície na configuração inicial
    \item[$\mathbb{P}$] Energia das Forças Externas
    \item[$\BB{P}$] Tensor de tensões de Piola-Kirchhoff de primeira espécie
    \item[$u_e$] Energia específica de deformação
    \item[$\mathbb{U}$] Energia de deformação
    \item[$\dot{\BB{y}}$] Campo de velocidades
    \item[$\ddot{\BB{y}}$] Campo de acelerações
    \item[$\BB{\varepsilon}$] Tensor de deformação linear
    \item[$\varepsilon_V$] Deformação volumétrica
    \item[$\Pi$] Energia total

    \item[\textbf{Método dos Elementos Finitos para Elemento de Casca}]
    \item[$\BB{F}$] Força concentrada
    \item[$\BB{g}$] Vetor de desbalanceamento mecânico
    \item[$G$] Módulo de elasticidade transversal
    \item[$h_0$] Espessura inicial da casca
    \item[$\BB{H}$] Matriz Hessiana
    \item[$\BB{q}$] Força não-conservativa
    \item[$\mathbb{Q}$] Dissipação de energia
    \item[$\BB{S}$] Tensor de tensões de Piola-Kirchhoff de segunda espécie
    \item[$\BB{t}$] força distribuída sobre a superfície média
    \item[$\BB{v}^0$] Vetor unitário normal è superfície média
    \item[$\BB{v}^1$] Vetor generalizado na configuração atual
    \item[$\BB{x}^m$] Coordenada inicial de um ponto na superfície média
    \item[$\BB{y}^m$] Coordenada atual de um ponto na superfície média
    \item[$\alpha$] Taxa de variação da espessura
    \item[$\nu$] Coeficiente de Poisson
    \item[$\lambda_m$] Constante de amortecimento

    \item[\textbf{Acoplamento Fluido-Estrutura}]
    \item[$\script{G}$] Soma de todas as equações diferenciais do problem
    \item[$\BB{H}$] Matriz tangente
    \item[$\BB{h}$] Vetor resíduo
    \item[$\BB{\alpha}$] Vetor com parâmetros nodais
    \item[$\BB{\beta}$] Vetor com parâmetros nodais das funções teste

    \item[\textbf{\textit{Variational Multi-Scale}}]
    \item[$\NM$] Vetor de resíduo
    \item[$\NC$] Vetor de resíduo
    \item[$\BB{P}$] Vetor de pressões nodais
    \item[$\rC$] Resíduo da equação da continuidade
    \item[$\rM$] Resíduo da equação da conservação da quantidade de movimento
    \item[$\BB{U}$] Vetor de velocidades nodais
    \item[$\dot{\BB{U}}$] Vetor de acelerações nodais
    \item[$\nu_\lsic$] Estabilizador LSIC
    \item[$\rho_\infty$] Raio espectral
    \item[$\tau_\sups$] Estabilizador SUPS
    \item[$\bar{\phi}$] Valor de uma propriedade no subespaço de escalas grosseiras
    \item[$\phi'$] Valor de uma propriedade no subespaço de escalas finas

    \item[\textbf{\textit{Large Eddy Simulation}}]
    \item[$\BB{C}$] Tensor de termos cruzados
    \item[$C_S$] Constante de Smagorinsky
    \item[$\Dfil$] Domínio de abrangência do filtro
    \item[$E(k)$] Amplitude espectral da energia cinética
    \item[$g$] Filtro
    \item[$k$] Raio da esfera parametrizada
    \item[$\BB{L}$] Tensor de Leonard
    \item[$\BB{R}$] Tensor SGS de Reynolds
    \item[$\deffil$] Taxa de deformação em grandes escalas
    \item[$\BB{T}$] Tensor SGS
    \item[$\BB{T}_S$] Tensor SGS de Smagorinsky
    \item[$\yfil$] Ponto na vizinhança de $\BB{y}$ interno à $\Dfil$
    \item[$\alpha$] Constante de Kolmogoroff
    \item[$\Delta$] Tamanho da malha
    \item[$\ep$] Dissipação turbulenta
    \item[$\nu_T$] Viscosidade de vórtice
    \item[$\bar{\phi}$] Propriedade filtrada
    \item[$\phi'$] Propriedade não filtrada

    \item[\textbf{\textit{Reynolds-Averaged Navier-Stokes}}]
    \item[$\BB{b}$] Tensor de tensões anisotrópico de Reynolds
    \item[$\BB{C}$] Gradiente de difusão turbulenta
    \item[$D_\ep$] Difusão turbulenta
    \item[$k$] Energia cinética média gerada pelo campo de flutuações
    \item[$\script{P}$] Pressão dividida pela massa específica
    \item[$\BB{P}$] Produção do tensor de Reynolds
    \item[$P_\ep$] Produção da dissipação
    \item[$\ep$] Dissipação turbulenta da energia cinética
    \item[$\BB{\ep}$] Tensor de taxa de dissipação turbulenta da energia cinética
    \item[$\BB{\Pi}$] Correlação entre pressão e o tensor de taxa de deformação do campo de flutuações
    \item[$\bar{\phi}$] Média temporal de uma propriedade
    \item[$\phi'$] Flutuação de uma propriedade no espaço-tempo
    \item[$\Phi_\ep$] Destruição turbulenta da dissipação
    \item[$\BB{\tau}$] Tensor de tensões de Reynolds
    \item[$\omega$] Escala de tempo turbulento
\end{simbolos}

% inserir o sumario
\pdfbookmark[0]{\contentsname}{toc}
\tableofcontents*
\cleardoublepage

% ----------------------------------------------------------
% ELEMENTOS TEXTUAIS
% ----------------------------------------------------------
\textual

\renewcommand\chaptername{}

\subfile{Capitulos/C1-Introducao}
\subfile{Capitulos/C2-EstadoArte}
\subfile{Capitulos/C3-FundTeorica/C3-FundTeorica}
\subfile{Capitulos/C5-ResultadosEsperados}
\subfile{Capitulos/C4-MetodologiaCronograma}

% ----------------------------------------------------------
% ELEMENTOS PÓS-TEXTUAIS
% ----------------------------------------------------------
\postextual

% REFERÊNCIAS BIBLIOGRÁFICAS
\bibliography{Referencias}

\end{document}