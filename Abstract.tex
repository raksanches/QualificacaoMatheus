% Abstract

\documentclass[_ArquivoPrincipal.tex]{subfiles}

\begin{document}
	\autor{Silva, M. J.}
	\begin{resumo}[Abstract]
		\begin{otherlanguage*}{english}
			\begin{flushleft} 
				\setlength{\absparsep}{0pt} % ajusta o espaçamento dos parágrafos do resumo		
				\SingleSpacing 
				\imprimirautorabr~ ~\textbf{\imprimirtitleabstract}.	\imprimirdata.  \pageref{LastPage}p. 
				%Substitua p. por f. quando utilizar oneside em \documentclass
				%\pageref{LastPage}f.
				\imprimirtipotrabalho~-~\imprimirinstituicao, \imprimirlocal, 	\imprimirdata. 
			\end{flushleft}
			\OnehalfSpacing 
			The study of Fluid-Structure Interaction is having an increasing degree of importance, since the structures are becoming increasingly lighter and slender, due to the constant advances in the different areas of engineering. Among these interactions, emphasizes those in which the fluid is in turbulent flow, which occurs, for example, in buildings subject to the winds actions. Therefore, it is necessary to develop increasingly efficient techniques to determine the behavior of both the structure and the flow of fluids that interact with it. In this sense, there is a great variety of techniques for determining the behavior of flexible structures, the most notable being those based on the Finite Element Method. Among these methods, one that is gaining prominence is the one that considers nodal positions as parameters of analysis, called Positional Finite Element Method. Likewise, it is observed that flow analysis can also be performed in different ways, such as: construction of scaled samples for practical tests; and problem modeling via mathematical methods. In this context, it appears that the construction of real samples in scale is very expensive in view of the need for a large infrastructure to obtain valid data, as well as, in some cases, the data obtained are dependent on the scale of the sample, not pointing to real results, therefore. Thus, mathematical models become the best solution to analyze these problems. However, depending on the degree of complexity of the problem and/or the model used, this analysis leads to a very high computational cost, making its resolution unfeasible. Thus, the present work seeks to carry out a comparative study between some of the most common techniques of Fluid-Structure Interaction analysis, such as Reynolds-Averaged Navier-Stokes, Large Eddy Simulation and Variational Multi-Scale, considering turbulent flows in flexible structures, using the Positional Finite Element Method for this evaluation.
			\vspace{\onelineskip}
			
			\noindent 
			\textbf{Keywords}: Fluid-Structure Interaction. Turbulent Flow. Positional Finite Element Method. Large Eddy Simulation. Variational Multi-Scale Methods. Reynolds-Averaged Navier-Stokes.
		\end{otherlanguage*}
	\end{resumo}
\end{document}

