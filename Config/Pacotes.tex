%% pacotes para tabelas
\usepackage{booktabs}  % table
\usepackage{float} % Fixa tabelas e figuras no local exato
\usepackage{multirow} % múltiplas linhas
\usepackage[table]{xcolor}

% Pacotes básicos - Fundamentais 
\usepackage[T1]{fontenc}		% Seleção de códigos de fonte.
\usepackage[utf8]{inputenc}		% Codificação do documento (conversão automática dos acentos)
\usepackage{lmodern}			% Usa a fonte Latin Modern

% Para utilizar a fonte Times New Roman, inclua uma % no início do comando acima  "\usepackage{lmodern}"
\usepackage{times}			% Usa a fonte Times New Roman	
% Lembre-se de alterar a fonte no comando que imprime o preâmbulo no arquivo da Classe USPSC.cls	

%Chamada de pacotes para tratamento de termos matemáticos
%\usepackage{mtpro2}
\usepackage{amsmath}
\usepackage{amssymb}
\usepackage{amsfonts}
\usepackage{lastpage}			% Usado pela Ficha catalográfica
%\usepackage{indentfirst}		% Indenta o primeiro parágrafo de cada seção.
\usepackage{graphicx}			% Inclusão de gráficos
\usepackage{microtype}			% para melhorias de justificação
\usepackage{pdfpages}
\usepackage{makeidx}		% para gerar índice remissimo
\usepackage{subfiles}		% para controlar subarquivos
\usepackage[ampersand]{easylist} % para defininar listas
\usepackage{titlesec, blindtext}
%\usepackage{mathptmx}
%\renewcommand{\rmdefault}{ptm}
\usepackage{nccmath}
\usepackage{extarrows} % imply with text below
%\usepackage{indentfirst}
\usepackage{caption}
\usepackage{subcaption}
\usepackage{multicol,lipsum}
\usepackage{fancybox}
\usepackage{longtable}
\usepackage{mathtools}
\usepackage{enumitem}  % Enumerate, Itemize
\usepackage{tabu}  % tabela adequada a fórmulas
\usepackage{pbox} % caixa dentro de uma célula da tabela
\usepackage[b]{esvect} % seta de vetor sobrescrita
\usepackage{framed} % quadro ao redor de um texto
\usepackage{verbatim}
\usepackage[percent]{overpic}
\usepackage{makecell} %\multirowcell
\usepackage{vwcol}
\usepackage{arydshln} % dashed line on matrix
\usepackage[portuguese,onelanguage,ruled,linesnumbered]{algorithm2e}
\usepackage{hhline}
\usepackage{dblfloatfix}    % To enable figures at the bottom of page
\usepackage[section]{placeins} % make floating elements don't switch from sections
\usepackage{etoolbox}

% Pacotes de citações - Citações padrão ABNT - Sistema autor-data
\usepackage[alf,abnt-emphasize=bf, abnt-thesis-year=both,
	abnt-repeated-author-omit=yes, abnt-last-names=abnt,
	abnt-etal-cite,abnt-etal-list=3, abnt-etal-text=default,
	abnt-and-type=e,
	abnt-doi=doi, abnt-url-package=none,
	abnt-verbatim-entry=no]{abntex2cite}

% Pacotes de Nota de rodape
% O presente modelo adota o formato numérico para as notas de rodapés quando utiliza o sistema de chamada autor-data para citações e referências.
%\renewcommand{\footnotesize}{\small} %Comando para diminuir a fonte das notas de rodapé	

% pacotes de tabelas
\usepackage{multicol}	% Suporte a mesclagens em colunas
\usepackage{multirow}	% Suporte a mesclagens em linhas
\usepackage{longtable}	% Tabelas com várias páginas
\usepackage{threeparttablex}	% notas no longtable
\usepackage{array}


\usepackage[T1]{fontenc}
\usepackage[utf8]{inputenc}
\usepackage[lmargin=3cm,tmargin=3cm,rmargin=2cm,bmargin=2cm]{geometry}
\usepackage[brazil]{babel}
\usepackage{times} %fonte Adobe Times Roman 
\usepackage{lastpage} % usado pela ficha catalográfica
\usepackage{color} % controle das cores		
\usepackage{graphicx} % inclusão de gráficos
\usepackage{float} % força as figuras a ficarem depois do texto que você colocou
\usepackage{chemfig}
%\usepackage{chemmacros}
\usepackage{upgreek}
\usepackage{microtype} % para melhorias de justificação	
\usepackage{pdfpages}
\usepackage{lastpage}
\usepackage{makeidx}
\usepackage{quoting}
\usepackage{afterpage}
\usepackage{indentfirst} % indenta todos os primeiros parágrafos
\usepackage{setspace}
\usepackage{titlesec}
\renewcommand{\ABNTEXchapterfontsize}{\bfseries\normalsize}
\renewcommand{\ABNTEXsectionfontsize}{\bfseries\normalsize}
\renewcommand{\ABNTEXsubsectionfontsize}{\normalsize}
\usepackage{hyperref} % para permitir a criação de links
\usepackage[scaled]{helvet} % usa a fonte arial
\usepackage{fancyhdr}
\usepackage{verbatim}
\usepackage{amsmath,amssymb,amsfonts,textcomp} % vários pacotes juntos, utilizados para usar funções e outros símbolos matemáticos
\fancyhf{} % numeração no canto superior direito
\fancyheadoffset{0cm}
\renewcommand{\headrulewidth}{0pt}
\renewcommand{\footrulewidth}{0pt}
\fancyhead[R]{\thepage}
\fancypagestyle{plain}{%
	\fancyhf{}%
	\fancyhead[R]{\thepage}%
}
\usepackage[alf]{abntex2cite} % Citações padrão ABNT
\usepackage{appendix}
\usepackage{colortbl}
\usepackage{icomma}