\usepackage[T1]{fontenc}		% Seleção de códigos de fonte.
\usepackage[utf8]{inputenc}		% Codificação do documento (conversão automática dos acentos)
%\usepackage{lmodern}			% Usa a fonte Latin Modern
% Para utilizar a fonte Times New Roman, inclua uma % no início do comando acima  "\usepackage{lmodern}"
% Abaixo, tire a % antes do comando  \usepackage{times}
\usepackage{times}		    	% Usa a fonte Times New Roman	
% Para usar a fonte , lembre-se de tirar a % do comando %\renewcommand{\ABNTEXchapterfont}{\rmfamily}, localizado mais abaixo, logo após "Outras opções para nota de rodapé no Sistema Numérico" 						
\usepackage{lastpage}			% Usado pela Ficha catalográfica
\usepackage{indentfirst}		% Indenta o primeiro parágrafo de cada seção.
\usepackage{color}				% Controle das cores
\usepackage{graphicx}			% Inclusão de gráficos
\usepackage{float} 				% Fixa tabelas e figuras no local exato
\usepackage{microtype} 			% para melhorias de justificação
\usepackage{pdfpages}
\usepackage{makeidx}            % para gerar índice remissivo
\usepackage{hyphenat}           % Pacote para retirar a hifenizacao do texto
\usepackage[absolute]{textpos}  % Pacote permite o posicionamento do texto
\usepackage{eso-pic}            % Pacote para incluir imagem de fundo
\usepackage{makebox}            % Pacote para criar caixa de texto

\usepackage[alf, abnt-emphasize=bf, abnt-thesis-year=both, abnt-repeated-author-omit=no, abnt-last-names=abnt, abnt-etal-cite=3, abnt-etal-list=3, abnt-etal-text=it, abnt-and-type=e, abnt-doi=doi, abnt-url-package=none, abnt-verbatim-entry=no]{abntex2cite}
\bibliographystyle{USPSC-classe/abntex2-alf-USPSC}

\renewcommand{\footnotesize}{\small}
\renewcommand{\ABNTEXchapterfont}{\rmfamily}
\usepackage{lipsum}				% para geração de dummy text

\usepackage{multicol}	% Suporte a mesclagens em colunas
\usepackage{multirow}	% Suporte a mesclagens em linhas
\usepackage{longtable}	% Tabelas com várias páginas
\usepackage{threeparttablex}    % notas no longtable
\usepackage{array}

\usepackage{USPSC-classe/ABNT6023-10520}

\usepackage{amsmath,amssymb,amsfonts,textcomp}

\usepackage[portuguese,onelanguage,ruled,linesnumbered]{algorithm2e}

\usepackage{caption,subcaption}

\usepackage{enumitem} % Enumerate, Itemize
\usepackage{icomma}

%==============================================================================================================
% 
% \usepackage{booktabs}  % table
% % \usepackage[table]{xcolor}
% 
% \usepackage{subfiles}	         % para controlar subarquivos
% \usepackage[ampersand]{easylist} % para definir listas
% \usepackage{titlesec, blindtext}
% %\usepackage{mathptmx}
% %\renewcommand{\rmdefault}{ptm}
% \usepackage{nccmath}
% \usepackage{extarrows} % imply with text below
% \usepackage{fancybox}
% \usepackage{mathtools}
% \usepackage{tabu}      % tabela adequada a fórmulas
% \usepackage{pbox}      % caixa dentro de uma célula da tabela
% \usepackage[b]{esvect} % seta de vetor sobrescrita
% \usepackage{framed}    % quadro ao redor de um texto
% \usepackage{verbatim}
% \usepackage[percent]{overpic}
% \usepackage{makecell} %\multirowcell
% \usepackage{vwcol}
% \usepackage{arydshln} % dashed line on matrix
% \usepackage{hhline}
% \usepackage{dblfloatfix}       % To enable figures at the bottom of page
% \usepackage[section]{placeins} % make floating elements don't switch from sections
% \usepackage{etoolbox}
% 
% \usepackage[brazil]{babel}
% \usepackage{upgreek}
% %\usepackage{quoting}
% \usepackage{afterpage}
% \usepackage{setspace}
% \usepackage{hyperref}       % para permitir a criação de links
% \usepackage[scaled]{helvet} % usa a fonte arial
% \usepackage{fancyhdr}
% \usepackage{appendix}
% \usepackage{colortbl}
% \usepackage{tablefootnote}