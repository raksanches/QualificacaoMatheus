%==================================================================================================
\chapter{Estado Da Arte}
%==================================================================================================

No presente capítulo serão abordados os principais temas relacionados à problemas de Interação Fluido-Estrutura (IFE), tendo como ênfase os modos de turbulência. Serão destacados os principais métodos de cálculos envolvendo a dinâmica das estruturas computacional (\ref{CSD}), a dinâmica dos fluidos computacional (\ref{CFD}) e os modelos de turbulências baseados na decomposição de Reynolds (RANS) e na decomposição de grandes vórtices (LES), assim como o modelo de estabilização multiescala (VMS) (\ref{MT}).

%==================================================================================================
\section{Dinâmica Das Estruturas Computacional} \label{CSD}
%==================================================================================================

A mecânica dos sólidos busca descrever o comportamento de elementos estruturais quando sujeitos a solicitações externas, identificando os campos de tensão, deformação e deslocamento. A utilização de métodos computacionais para a solução desses problemas iniciou-se principalmente com o método das diferenças finitas, mas no decorrer do tempo, o Método dos Elementos Finitos (MEF), cujo nome foi cunhado por \citeonline{clough1960finite}, se destacou como uma ferramenta mais versátil, sendo atualmente a ferramenta mais difundida na mecânica dos sólidos.

De início a análise das estruturas era restrita a diversas assunções, como o caso de se considerar apenas situações de pequenos deslocamentos e deformações em regime linear elástico, limitando as análises que poderiam ser desenvolvidas. Posteriormente se expandiram as análises à situações de grandes deslocamentos, como observado no trabalho de \citeonline{turner1960large}, surgindo assim os conceitos de não linearidade geométrica e sendo bem aplicada em trabalhos como de \citeonline{bathe1975finite,brendel1980linear,hughes1981nonlinear,hughes1983nonlinear,simo1986finite,de2012nonlinear}.

Na sequência desenvolve-se o MEF corrotacional, o qual decompõe o movimento de um sólido em uma parcela referente ao movimento de corpo rígido e outra referente à deformação do mesmo, sendo consideradas as posições e rotações nodais como parâmetros de análise, como pode ser observado no trabalho de \citeonline{WEMPNER1969117}, onde foi verificado o comportamento de cascas em regime de grandes deslocamentos e pequenas deformações. Outros trabalhos relevantes a serem ressaltados nesse tema são os de \citeonline{hughes1981nonlinear, argyris1982excursion, ibrahimbegovic2002role, pimenta2004fully, battini2006choice} e \citeonline{pimenta2008exact}.

No entanto a utilização de uma análise que considera tais parâmetros nodais só se mostra eficiente ao se estudar estruturas que desenvolvem pequenos deslocamentos e deformações. Isso deve-se ao fato de essa abordagem utilizar rotações finitas como parâmetros nodais, o que pode gerar controversas em problemas envolvendo grandes rotações, uma vez que não há comutatividade entre essas grandezas. Ainda observa-se que a avaliação dinâmica de estruturas reticuladas se torna problemática, do ponto de vista da conservação de energia, além da matriz de massa ser variável, tornando o processo de integração temporal muito complexo \cite{sanches2013unconstrained}.

Tendo isso em vista, \citeonline{coda2003analise}, instigado pelo trabalho de \citeonline{bonet2000finite}, apresentou uma formulação alternativa do MEF, a qual baseia-se em posições como parâmetros nodais. Tal abordagem possui vantagens ao se trabalhar com não linearidade geométrica, por ser independente das rotações nodais, assim como possui uma matriz de massa constante, o que facilita a análise dinâmicas das estruturas. Outro ponto a ser ressaltado é a simplicidade da formulação, assim como sua facilidade de implementação. Com isso, vale ressaltar os trabalhos de \citeonline{coda2004simple, coda2007alternative, coda2010improved, coda2009two, coda2009unconstrained} e \citeonline{carrazedo2010alternative} que utilizam de forma bem-sucedida o MEF Posicional para análise de estruturas reticuladas e de cascas aplicadas à grandes deslocamentos. Este método se diferencia dos demais ao considerar as posições nodais, obtidas a partir de vetores indeformados, como parâmetros de análise, facilitando o cálculo de efeitos de não-linearidade geométrica.

Para a aproximação temporal das variáveis, pode-se utilizar, por exemplo, os integradores temporais de Newmark e $\alpha$-generalizado. Sendo assim, verifica-se que a aproximação de Newmark insere dois parâmetros livres que podem ser calibrados para diferentes problemas e faz a aproximação de velocidades e acelerações em função de valores passados e posições futuras. Trabalhos como de \citeonline{sanches2013unconstrained,rosa2021tecnica} apresentaram aplicações da aproximação de Newmark em utilizações do método dos elementos finitos posicional e constataram a presença de uma matriz de massa constante, possibilitando a utilização estável do integrador temporal. \citeonline{sanches2013unconstrained} ainda expõe que nesse cenário o integrador de Newmark conserva o momento linear e angular em análises dinâmicas. Além disso, esse trabalho também analisou problemas envolvendo IFE, obtendo resultados favoráveis, indicando a boa aplicação desse método em problemas dessa natureza.

Já o método $\alpha$-generalizado, introduzido por \citeonline{chung1993time}, é um esquema de integração temporal implícito que se utiliza de valores de velocidades e acelerações em instantes de tempo intermediários para a determinação de valores futuros. Outra característica interessante dessa técnica está na possibilidade de se controlar facilmente a difusão numérica no processo de marcha no tempo por meio do ajuste do raio espectral. Alguns trabalhos que aplicaram o MEF posicional em conjunto com o método $\alpha$-generalizado são \citeonline{siqueira2019ligaccoes,moreira2021analise,Acanvini2023formulacao}.

Dentre as aplicações dinâmicas da formulação posicional do MEF, pode-se citar inicialmente o trabalho de \citeonline{marques2006estudo}, o qual analisa o comportamento de sólidos bidimensionais em situação de não linearidade geométrica. Na sequência aplicou-se a formulação posicional à elementos de pórticos planos e sólidos tridimensionais por \citeonline{maciel2008analise}. Também foi avaliado o comportamento de elementos de cascas em problemas estáticos \cite{coda2007alternative} e dinâmicos \cite{coda2009unconstrained}.

No âmbito da IFE, destaca-se a necessidade de se utilizar metodologias que capturem eficientemente grandes deslocamentos, uma vez que são observadas aplicações como, por exemplo, \textit{flutter}, aplicações biomecânicas, estruturas infláveis \cite{karagiozis2011computational}, simulações de turbinas \cite{bazilevs20113d}, dentre outras.

%==================================================================================================
\section{Dinâmica Dos Fluidos Computacional} \label{CFD}
%==================================================================================================

Ao contrário de problemas envolvendo elementos sólidos, que possuem um estado inicial bem definido, os fluidos, em especial os Newtonianos, não o possuem, uma vez que não incapazes de resistir à tensões desviadoras, deformando-se, assim, indefinidamente quando sujeitos à essas tensões. Dessa forma, torna-se apropriada a utilização de uma descrição Euleriana para descrever os fluidos, em que os parâmetros nodais são principalmente as velocidades do mesmo \cite{fernandes2020tecnica}.

Em geral, problemas envolvendo Dinâmica dos Sólidos (\textit{Computational Solid Dynamics} - CSD) partem do princípio da estacionariedade de energias, buscando a determinação do ponto em que a energia do sistema seja mínima. Esses métodos apresentam a particularidade de surgir um sistema de equações com uma matriz simétrica e, em alguns casos, de vetores cujas componentes possuem significado físico. Porém, em problemas que envolvem a Dinâmica dos Fluidos (\textit{Computational Fluid Dynamics} - CFD), o sistema de equações possui uma matriz, na maioria dos casos, assimétrica, devido à presença de termos convectivos nas equações governantes \cite{bazilevs2013computational,brooks1982streamline}.

Nesse sentido, são desenvolvidos novos métodos, em busca de uma solução mais representativa com uma malha menos refinada. Dentre os principais desenvolvidos, vale mencionar o \textit{Streamline-Upwind/Petrov-Galerkin} (SUPG) \cite{brooks1982streamline}, \textit{Galerkin Least-Squares} (GLS) \cite{hughes1989new,tezduyar1991stabilized}, \textit{Subgrid Scales} (SGS) e \cite{hughes1995multiscale}.

Um campo que vale destacar na CFD é o que estuda os fenômenos de turbulência, uma vez que esses fenômenos podem se apresentar nas mais variadas escalas, sendo necessário a geração de uma malha muito refinada para detectar a formação dos vórtices, o que ocasiona um aumento radical no custo computacional da análise. Assim, são desenvolvidas novas técnicas afim de se obter melhores resultados, sem que haja um grande aumento no volume de cálculos. Dentre esses métodos destacam-se os métodos de \RANS\ (RANS) \cite{speziale1991analytical,alfonsi2009reynolds,ling2015evaluation}, as Simulações de Grandes Vórtices (\LES\ - LES) \cite{germano1991dynamic,piomelli1999large,hughes2000large,vsekutkovski2021partitioned} e o método Variacional Multiescala (\VMS\ - VMS) \cite{hughes1995multiscale,hughes1998variational,hughes2002variational,bazilevs2010large,bazilevs2013computational}. O capítulo \ref{FT} descreve de forma mais detalhada cada um desses métodos.

%e os métodos de atualização de malha \cite{de1993petrov}

%==================================================================================================
\subsection{Modelos De Turbulência} \label{MT}
%==================================================================================================

Um fluido pode apresentar um escoamento de duas formas distintas, a depender do número de Reynolds que este apresenta. Em casos cujo número de Reynolds é baixo, o escoamento é considerado laminar, ou seja, o escoamento se dá de forma semelhante ao movimento de lâminas independentes, não havendo mistura macroscópica entre as mesmas. Esse tipo de escoamento possui soluções muito mais simples de se obter, no entanto representam uma ocorrência baixa na maioria dos problemas observados na natureza. Já em casos cujo número de Reynolds é muito elevado, o escoamento é considerado turbulento. Nesse cenário há a formação de vórtices sobre o escoamento, que podem ocorrer de forma instável, desordenada e em várias escalas diferentes \cite{popiolek2005analise,shaughnessy2005introduction}. Esse fenômeno pode ser descrito de acordo com as equações de Navier-Stokes, no entanto sua resolução se apresenta com um alto grau de complexidade, uma vez que possui termos não lineares em sua composição. Assim são necessárias técnicas de solução, para que se possa obter uma solução de forma aproximada à essas equações.

Uma possível forma de simulação de escoamentos turbulentos é o denominado \textit{Direct Numerical Simulation} (DNS), o qual resolve as equações de Navier-Stokes diretamente em todas as escalas de presentes, sendo, portanto, o método mais preciso de se simular escoamentos turbulentos. No entanto possui um alto custo computacional, restringindo sua aplicação à problemas pequenos e de geometria simples para verificação de modelos de turbulência que sejam mais viáveis \cite{piomelli1999large,yokokawa200216}. Alguns trabalhos realizados que utilizam o DNS são: \citeonline{yokokawa200216,picano2015turbulent,olad2022towards,frey2021machine}.

Outra forma de se simular escoamentos turbulentos é o \textit{Reynolds-Averaged Navier-Stokes} (RANS). A principal premissa desse modelo diz que é possível separar as propriedades do escoamento em duas partes: uma parte relacionada à média temporal da propriedade, sendo essa a parcela predominante e a principal a ser determinada, e outra relacionada à variações no espaço-tempo da mesma. Por meio dessa consideração, a equação de Navier-Stokes se transforma de tal forma a surgir um termo adicional relacionado à interações entre as parcelas flutuantes, a qual representa a interferência dos efeitos turbulentos na propriedade média. Existem diversos modelos para descrever o comportamento, em que o mais comum de se encontrar é baseado no tensor de tensões de Reynolds, levando em consideração efeitos de viscosidade turbulenta \cite{piomelli1999large,alfonsi2009reynolds,bazilevs2010large,ling2015evaluation}.

Por sua vez, o modelo de \textit{Large-Eddy Simulation} (LES), que considera também a separação dos parâmetros em duas parcelas: uma parcela filtrada; e outra não-filtrada. Nesse cenário se faz necessária a modelagem dos termos não-filtrados, enquanto a parcela filtrada é determinada diretamente por meio da resolução das equações de Navier-Stokes. Uma possível forma de se modelar os termos não-filtrado baseia-se no modelo de \textit{Sub-Grid Scale} (SGS), que faz a interação entre os campos de grandes e de pequenas escalas, sendo aprimorada de forma a capturar os efeitos turbulentos em função do tamanho da malha \cite{ghosal1995basic,hughes2000large,moeng2015large}.
\citeonline{olad2022towards} aponta que a formulação a base de LES se mostrou mais preciso na determinação dos campos de velocidades e de tensões desviadoras, porém com um custo computacional consideravelmente superior ao RANS.

No entanto, as simulações baseadas em RANS e LES possuem a dificuldade de trabalharem com sistemas de equações nas quais surgem sub-blocos nulos na matriz do problema. Assim, surgem modelos de estabilização como o \textit{Variational Multi-Scale} (VMS), introduzido por \citeonline{hughes1995multiscale,hughes1998variational,hughes2000large} que contornam essa dificuldade. Esse modelo faz uso dos princípios variacionais, em que tanto os espaços tentativas quanto os espaços testes são divididos em: parcela de grandes escalas; e parcela de pequenas escalas. Com isso se faz a modelagem do espaço de pequenas escalas em termos de resíduos das equações de conservação de massa e de conservação da quantidade de movimento. Tal consideração preenche os termos da diagonal principal, fazendo com que o problema se torne positivo-definido, facilitando a escolha dos espaços de aproximação de velocidades e pressões, além da resolução do problema se mostrar de forma mais estável \cite{bazilevs2013computational,sondak2015new}.

%==================================================================================================
\section{Interação Fluido-Estrutura} \label{IFE}
%==================================================================================================

Segundo \citeonline{sanches2014fluid} problemas numéricos envolvendo IFE são divididos em três áreas, sendo elas a CFD, CDS e Problemas de Interação (\textit{Interaction Problem} - IP). Fazer essa interação entre CFD e CSD pode se tornar uma tarefa complexa, pois se caracteriza pela sua multidisciplinaridade \cite{hou2012numerical} além de acoplar esses dois problemas se tornar complicado, uma vez que pode-se utilizar descrições diferentes para cada problema, por exemplo a utilização de uma descrição Lagrangiana para modelar o sólido e uma descrição Euleriana para o fluido, além de que os parâmetros nodais determinados em cada um dos dois problemas são distintos, como deslocamentos ou posições em sólidos e velocidades e pressões no fluido.

Outra questão a ser observada em problemas de IFE é que, devido à movimentação da estrutura, deve-se realizar algum procedimento para que o domínio do fluido perceba essa movimentação. \citeonline{bazilevs2013computational} apontam duas técnicas possíveis de serem utilizadas para se considerar esse efeito: a técnica de malha conforme (ou \textit{interface-tracking}); e a técnica de malha não-conforme (ou \textit{interface-capturing}). Na técnica de malhas conformes a interface entre o sólido e o fluido é caracterizada pela presença de condições de contorno na mesma, necessitando, assim, que a malha do fluido se deforma para se acomodar à nova configuração, deformando-se no processo ou, caso necessário, passando por remalhamento \cite{terahara2020heart}. Esse tipo de técnica é interessante, já que permite a utilização de uma malha razoavelmente complexa próxima à interface, para se obter resultados mais precisos nessa região. No entanto percebe-se que alguns casos de deformações excessivas do domínio do fluido, pode ocorrer de não ser viável ou meramente possível essa movimentação. Por sua vez, a técnica de malhas não-conformes considera as condições de contorno diretamente nas equações governantes, permitindo que os problemas sejam resolvidos separadamente sem a necessidade de movimentação da malha, evitando, assim, os problemas relacionados à movimentação excessiva da malha. Porém, em certos casos de problemas de geometria complexa, os custos de remalhamento podem ser compensados com o aumento de precisão obtido próximo à interface \cite{bazilevs2013computational,hou2012numerical,bazilevs2015ale}.

Dentro do âmbito de malhas não-conformes, nota-se a utilização de técnicas baseadas em contornos imersos em diversos trabalhos, como: o de \citeonline{zhao2016numerical}, o qual analisou os resultados da técnica por comparação com respostas analíticas, numéricas e experimentais, tendo obtido bons resultados; o de \citeonline{zheng2020numerical}, que utilizou uma formulação modificada de um método de contornos imersos, comparando os resultados com os obtidos experimentalmente em situações simétricas e assimétricas, obtendo resultados satisfatórios; o de \citeonline{xiao2022immersed}, sendo estudado escoamentos com transferência de massa, calor e momento, e também são apontadas pelos autores as dificuldades provenientes, dentre outras causas, da não conformidade da malha, especialmente em problemas com alto número de Reynolds; dentre outros, como de \citeonline{wang2011algorithms,ruess2013weakly,yan2021three}.

Tendo em vista a técnica de malha conforme, destaca-se a utilização formulação Lagrangiana-Euleriana Arbitrária (\textit{Arbitrary Lagrangian-Eulerian} - ALE) \cite{donea1982arbitrary,kanchi20073d,fernandes2019ale} e a formulação Espaço Tempo (\textit{Space-Time} - ST) \cite{takizawa2011multiscale,terahara2020heart,takizawa2011stabilized}. Assim, o problema pode ser subdividido na determinação dos parâmetros referentes ao: sólido; fluido; e malha.

Como mencionado por \citeonline{hou2012numerical}, problemas de IFE podem ser ainda classificados em termos de sua abordagem: o modelo monolítico; e o modelo particionado, o qual também pode ser subdividido em particionado forte e particionado fraco. No modelo monolítico todos os parâmetros do problema são calculados diretamente no mesmo bloco de equações. Uma grande vantagem dessa abordagem é a obtenção direta dos valores de interesse, além de alcançar uma precisão maior nos resultados, às custas de um maior custo computacional, uma vez que o sistema a ser resolvido é consideravelmente maior. Já o modelo particionado resolve de forma independente os diferentes problemas envolvidos, o que se mostra como uma grande vantagem desse método, já que é possível reutilizar códigos funcionais para diferentes modelos e apenas integrá-los em um processo que fará a interação, flexibilizando o código à novos modelos com uma probabilidade menor de erros no código \cite{roux2008domain,hou2012numerical}. O modelo particionado fraco é suficientemente satisfatório em simulações com intervalos pequenos de passo de tempo, dispensando, assim, a interação em \textit{loop}, sendo classificado, portanto, como um método explícito, ao contrário do modelo particionado forte, onde ocorrem interações em \textit{loop} para corrigir eventuais erros na IFE, caracterizando-se como um método implícito \cite{fernandes2020tecnica}.

Dentre os trabalhos que se utilizam do modelo monolítico cita-se os trabalhos de \citeonline{michler2004monolithic,hron2007fluid,wick2021optimization}. Já os trabalhos que utilizam o modelo particionado cita-se \citeonline{sanches2013unconstrained,sanches2014fluid,fernandes2019ale}