%==================================================================================================
\chapter{Estado Da Arte}
%==================================================================================================

No presente capítulo serão abordados os principais temas relacionados à problemas de Interação Fluido-Estrutura (IFE), tendo como ênfase os modos de turbulência. Serão destacados os principais métodos de cálculos envolvendo a dinâmica das estruturas computacional (\ref{CSD}), a dinâmica dos fluidos computacional (\ref{CFD}) e os modelos de turbulências baseados na decomposição de Reynolds (RANS) e na decomposição de grandes vórtices (LES), assim como o modelo de estabilização multiescala (VMS) (\ref{MT}).

%==================================================================================================
\section{Dinâmica Das Estruturas Computacional} \label{CSD}
%==================================================================================================

A mecânica dos sólidos busca descrever o comportamento de elementos estruturais quando sujeitos a solicitações externas, identificando os campos de tensão, deformação e deslocamento. A utilização de métodos computacionais para a solução desses problemas iniciou-se principalmente com o método das diferenças finitas, mas no decorrer do tempo, o Método dos Elementos Finitos (MEF), cujo nome foi cunhado por \citeonline{clough1960finite}, se destacou como uma ferramenta mais versátil, sendo atualmente a ferramenta mais difundida na mecânica dos sólidos.

Em geral, problemas envolvendo Dinâmica dos Sólidos (\textit{Computational Solid Dynamics} - CSD) são formulados numericamente empregando métodos energéticos, que buscam a imposição da conservação da energia. Isso se deve principalmente ao fato das leis constitutivas elásticas possuírem um potencial energético adequado. No entanto, os métodos de resíduos ponderados, tais como Bubnov-Galerkin ou Petrov-Galerkin podem ser igualmente empregados. Cabe ressaltar que, para os problemas elásticos clássicos, as equações governantes são elípticas (caso estático) ou parabólicas (caso dinâmico), permitindo o uso do método de Bubnov-Galerkin de forma estável, e conduzindo a sistemas com matrizes simétricas \cite{de2012nonlinear}.

De início a análise das estruturas era restrita a diversas considerações, como o caso de se considerar apenas situações de pequenos deslocamentos e deformações em regime linear elástico, limitando as análises que poderiam ser desenvolvidas. Posteriormente se expandiram as análises à situações de grandes deslocamentos, como observado no trabalho de \citeonline{turner1960large}, surgindo assim os conceitos de não linearidade geométrica e sendo bem aplicada em trabalhos como de \citeonline{bathe1975finite,brendel1980linear,hughes1981nonlinear,hughes1983nonlinear,simo1986finite,de2012nonlinear}.

Na sequência desenvolve-se o MEF corrotacional, o qual decompõe o movimento de um sólido em uma parcela referente ao movimento de corpo rígido e outra referente à deformação do mesmo, sendo consideradas as posições e rotações nodais como parâmetros de análise, como pode ser observado no trabalho de \citeonline{WEMPNER1969117}, onde foi verificado o comportamento de cascas em regime de grandes deslocamentos e pequenas deformações. Outros trabalhos relevantes a serem ressaltados nesse tema são os de \citeonline{hughes1981nonlinear, argyris1982excursion, ibrahimbegovic2002role, pimenta2004fully, battini2006choice} e \citeonline{pimenta2008exact}.

No entanto a utilização de uma análise que considera tais parâmetros nodais só se mostra eficiente ao se estudar estruturas que desenvolvem pequenos deslocamentos e deformações. Isso deve-se ao fato de essa abordagem utilizar rotações finitas como parâmetros nodais, o que pode gerar controversas em problemas envolvendo grandes rotações, uma vez que não há comutatividade entre essas grandezas. Ainda observa-se que a avaliação dinâmica de estruturas reticuladas se torna problemática, do ponto de vista da conservação de energia, além da matriz de massa ser variável, tornando o processo de integração temporal muito complexo \cite{sanches2013unconstrained}.

Tendo isso em vista, \citeonline{coda2003analise}, instigado pelo trabalho de \citeonline{bonet2000finite}, apresentou uma formulação alternativa do MEF, a qual baseia-se em posições como parâmetros nodais. Tal abordagem possui vantagens ao se trabalhar com não linearidade geométrica, por ser independente das rotações nodais, assim como possui uma matriz de massa constante, o que facilita a análise dinâmicas das estruturas. Outro ponto a ser ressaltado é a simplicidade da formulação, assim como sua facilidade de implementação. Com isso, vale ressaltar os trabalhos de \citeonline{coda2004simple, coda2007alternative, coda2010improved, coda2009two, coda2009unconstrained} e \citeonline{carrazedo2010alternative} que utilizam de forma bem-sucedida o MEF Posicional para análise de estruturas reticuladas e de cascas aplicadas à grandes deslocamentos. Este método se diferencia dos demais ao considerar as posições nodais, obtidas a partir de vetores indeformados, como parâmetros de análise, facilitando o cálculo de efeitos de não-linearidade geométrica.

Para a aproximação temporal das variáveis, pode-se utilizar, por exemplo, os integradores temporais de Newmark e $\alpha$-generalizado. Sendo assim, verifica-se que a aproximação de Newmark insere dois parâmetros livres que podem ser calibrados para diferentes problemas e faz a aproximação de velocidades e acelerações em função de valores passados e posições futuras. Trabalhos como de \citeonline{sanches2013unconstrained,rosa2021tecnica} apresentaram aplicações da aproximação de Newmark em utilizações do método dos elementos finitos posicional e constataram a presença de uma matriz de massa constante, possibilitando a utilização estável do integrador temporal. \citeonline{sanches2013unconstrained} ainda expõe que nesse cenário o integrador de Newmark conserva o momento linear e angular em análises dinâmicas. Além disso, esse trabalho também analisou problemas envolvendo IFE, obtendo resultados favoráveis, indicando a boa aplicação desse método em problemas dessa natureza.

Já o método $\alpha$-generalizado, introduzido por \citeonline{chung1993time}, é um esquema de integração temporal implícito que se utiliza de valores de velocidades e acelerações em instantes de tempo intermediários para a determinação de valores futuros. Outra característica interessante dessa técnica está na possibilidade de se controlar facilmente a difusão numérica no processo de marcha no tempo por meio do ajuste do raio espectral. Alguns trabalhos que aplicaram o MEF posicional em conjunto com o método $\alpha$-generalizado são \citeonline{siqueira2019ligaccoes,moreira2021analise,Acanvini2023formulacao}.

Dentre as aplicações dinâmicas da formulação posicional do MEF, pode-se citar inicialmente o trabalho de \citeonline{marques2006estudo}, o qual analisa o comportamento de sólidos bidimensionais em situação de não linearidade geométrica. Na sequência aplicou-se a formulação posicional à elementos de pórticos planos e sólidos tridimensionais por \citeonline{maciel2008analise}. Também foi avaliado o comportamento de elementos de cascas em problemas estáticos \cite{coda2007alternative} e dinâmicos \cite{coda2009unconstrained}.

No âmbito da IFE, destaca-se a necessidade de se utilizar metodologias que capturem eficientemente grandes deslocamentos, uma vez que são observadas aplicações como, por exemplo, \textit{flutter}, aplicações biomecânicas, estruturas infláveis \cite{karagiozis2011computational}, simulações de turbinas \cite{bazilevs20113d}, dentre outras.

%==================================================================================================
\section{Dinâmica Dos Fluidos Computacional} \label{CFD}
%==================================================================================================

Ao contrário de problemas da mecânica dos sólidos, que possuem um estado inicial bem definido, os fluidos, em especial os Newtonianos, não o possuem, uma vez que são incapazes de resistir a tensões desviadoras, deformando-se, assim, indefinidamente quando sujeitos à essas tensões. Dessa forma, torna-se apropriada a utilização de uma descrição Euleriana para descrever os fluidos, em que os parâmetros nodais são principalmente as velocidades do mesmo \cite{fernandes2020tecnica}.

Quando aplicado o método de Bubnov-Galerkin às equações da mecânica dos fluidos em descrição Lagrangiana, os termos convectivos (hiperbólicos), geram matrizes assimétricas e induzem oscilações espúrias na solução à medida em que se tornam dominantes sobre os termos dissipativos \cite{bazilevs2013computational,brooks1982streamline}. Isso contribuiu para que o método dos elementos finitos fosse preterido no início dos estudos da Dinâmica dos fluidos Computacional (CFD), sendo largamente empregados outros métodos como diferenças finitas e volumes finitos como pode ser visto, por exemplo, em \citeonline{anderson1995computational,chung2002computational}.

No entanto, com o passar do tempo, surgiram trabalhos que buscaram estabilizar a solução da convecção por meio do MEF, sendo o mais empregado o método \textit{Streamline-Upwind/Petrov-Galerkin} (SUPG) \cite{brooks1982streamline}, que baseia-se na adição do resíduo da equação da quantidade de movimento ponderado por uma função desenvolvida para introduzir estabilização na direção das linhas de corrente. Outras alternativas importantes, são baseadas em mínimo quadrados (\textit{Galerkin Least-Squares} (GLS) \cite{hughes1989new,tezduyar1991stabilized}) ou na ideia de análise multiescala (por exemplo o \textit{Subgrid Scales} (SGS) \cite{piomelli1999large,hughes2000large} e \textit{Variational Multiscale Method} (VMS) \cite{hughes1995multiscale,hughes1998variational,hughes2000large}). Assim, dadas às suas vantagens como ferramenta de discretização espacial e facilidades para imposições de condições de contorno, o MEF passou a ser bastante explorado também no contexto da CFD.

Também podem ocorrer instabilidades, no caso de escoamentos compressíveis, relacionados à variações abruptas das propriedades do escoamento, denominadas de ondas de choque. Sendo assim, desenvolvem-se tećnicas capazes de estabilizar tais problemas, como a captura de choque, utilizada por \citeonline{Nithiarasuetal1998,ZienkTaylor2000v3,Nithiarasuetal2006}, a qual considera a adição de um termo de viscosidade artificial, tendo seu valor máximo no ponto de descontinuidade e que vai sendo anulado conforme se afasta desse ponto.

Além disso, o cálculo das pressões pode apresentar resultados espúrios caso a escolha dos espaços aproximadores não seja feita cuidadosamente. Sendo assim, essa escolha deve atender as condições de \LBB\ (LBB), sendo que uma condição necessária, porém não suficiente, aponta que o espaço de aproximação para o campo de pressões não pode ser o mesmo que o de velocidades \cite{BrezziF1991,donea2003finite,fernandes2020tecnica}. Alternativamente, foram desenvolvidos termos estabilizadores capazes de flexibilizar a escolha dos espaços aproximadores, como os termos \textit{Pressure-Stabilizing/Petrov-Galerkin} (PSPG) e \textit{Streamline-Upwind/Petrov-Galerkin} (SUPG) \cite{TezduyarS2003}.

Outra possibilidade de se resolver esses problemas de instabilidade reside na utilização de modelos de estabilização multiescala, como o \textit{Variational Multi-Scale} (VMS), introduzido por \citeonline{hughes1995multiscale,hughes1998variational,hughes2000large}. Esse modelo faz uso dos princípios variacionais, em que tanto os espaços tentativas quanto os espaços testes são divididos em: parcela de grandes escalas; e parcela de pequenas escalas. Com isso se faz a modelagem do espaço de pequenas escalas em termos de resíduos das equações de conservação de massa e de conservação da quantidade de movimento. Dessa forma a escolha dos espaços aproximadores se torna mais flexível ao mesmo tempo que problemas de instabilidades em situação de convecção dominante também é controlado \cite{bazilevs2013computational,sondak2015new}.

Outro problema surge à medida que o número de Reynolds do escoamento aumenta, passando de um regime laminar de escoamento para um regime turbulento. Esses fenômenos podem se apresentar nas mais variadas escalas, podendo demandar a geração de uma malha muito refinada para capturar os vórtices e solucionar o problema de forma estável. Isso ocasiona um aumento radical no custo computacional da análise. Como espera-se que um método numérico seja capaz de apresentar soluções estáveis para os problemas, independentemente do nível de discretização, diversos trabalhos foram conduzidos a fim de desenvolverem formulações estáveis para escoamentos com elevados números de Reynolds, buscando minimizar os custos computacionais. Dentre esses métodos destacam-se os métodos de \RANS\ (RANS) \cite{speziale1991analytical,alfonsi2009reynolds,ling2015evaluation}, as Simulações de Grandes Vórtices (\LES\ - LES) \cite{germano1991dynamic,piomelli1999large,hughes2000large,vsekutkovski2021partitioned} e o método Variacional Multiescala (\VMS\ - VMS) \cite{hughes1995multiscale,hughes1998variational,hughes2002variational,bazilevs2010large,bazilevs2013computational}.

%==================================================================================================
\subsection{Modelos De Turbulência} \label{MT}
%==================================================================================================

Um fluido pode apresentar um escoamento considerado laminar, ou seja, que se dá de forma semelhante ao movimento de lâminas independentes, não havendo mistura macroscópica entre as mesmas, ou pode apresentar formação de vórtices, que podem ocorrer inclusive de forma instável, desordenada e em várias escalas diferentes, quando o escoamento é classificado como turbulento \cite{popiolek2005analise,shaughnessy2005introduction}. Os escoamentos turbulentos estão relacionados a casos com elevados números de Reynolds de duas formas distintas, a depender do valor desse parâmetro. Em casos cujo número de Reynolds (quociente de forças de inércia por forças viscosas) é baixo, o escoamento é laminar, passando por um momento de transição e atingindo o regime turbulento quando o número de Reynolds torna-se elevado. As equações de Navier-Stokes governam também os escoamentos turbulentos, no entanto sua resolução numérica passa a apresentar alto grau de complexidade, podendo demandar discretizações muito refinadas para a obtenção de uma resposta estável e consistente \cite{neto2002fundamentos}.

A aplicação direta do MEF sobre as equações de Navier-Stokes para a simulação de problemas turbulentos é conhecida como \textit{Direct Numerical Simulation} (DNS). Isso considera diretamente em todas as escalas de presentes, sendo, portanto, o método mais preciso de se simular escoamentos turbulentos, sendo utilizado em trabalhos como os de \citeonline{yokokawa200216,picano2015turbulent,olad2022towards,frey2021machine}. No entanto, o custo computacional acaba por restringir sua aplicação a problemas pequenos e de geometria simples, sendo muito empregado para verificação de modelos numéricos que sejam mais viáveis \cite{piomelli1999large,yokokawa200216}.

Uma maneira alternativa de se tratar os escoamentos turbulentos é o \textit{Reynolds-Averaged Navier-Stokes} (RANS), cujos conceitos iniciais foram introduzidos por \citeonline{reynolds1895iv}. A principal premissa desse modelo diz que é possível separar as propriedades do escoamento em duas partes: uma parte relacionada à média da propriedade, sendo essa a parcela predominante e a principal a ser determinada, e outra relacionada à variações no espaço-tempo da mesma. Isso introduz ao equacionamento um termo adicional relacionado às interações entre as parcelas flutuantes, que representa a interferência dos efeitos turbulentos na propriedade média. Por conta dessa relação existente entre as parcelas médias e de flutuações, surge a necessidade de adicionar mais equações ao problema, de forma a fechá-lo. Assim, diferentes abordagens se mostram possíveis e que podem ser classificadas em função da quantidade de equações de transporte adicionadas, sendo elas: modelos de zero equações, uma equação, duas equações e modelos de tensões \cite{piomelli1999large,alfonsi2009reynolds,bazilevs2010large,ling2015evaluation}.

Outra forma de abordar tais problemas é com o emprego do modelo de \textit{Large-Eddy Simulation} (LES), introduzido por \citeonline{smagorinsky1963general}. Esse modelo considera a separação das grandezas em duas parcelas: uma parcela filtrada, a qual captura os grandes vórtices, possíveis de ser calculada diretamente por meio das equações de Navier-Stokes filtradas; e outra não-filtrada, a qual representa os pequenos vórtices, que apesar de não poder ser calculada diretamente, possui um comportamento isotrópico, viabilizando sua modelagem. Nesse cenário se faz necessária a modelagem dos termos não-filtrados, enquanto a parcela filtrada é determinada diretamente por meio da resolução das equações de Navier-Stokes. Uma possível forma de se modelar os termos não-filtrado baseia-se no modelo de \textit{Sub-Grid Scale} (SGS), que faz a interação entre os campos de grandes e de pequenas escalas, sendo aprimorada de forma a capturar os efeitos turbulentos em função do tamanho dos elementos \cite{ghosal1995basic,hughes2000large,moeng2015large}. \citeonline{olad2022towards} aponta que, em relação ao RANS, a formulação base do LES se mostra mais precisa na determinação dos campos de velocidades e de tensões desviadoras, porém com um custo computacional consideravelmente superior.

%==================================================================================================
\section{Interação Fluido-Estrutura} \label{IFE}
%==================================================================================================

Segundo \citeonline{sanches2014fluid} a solução numérica de problemas de IFE compreende a solução de 3 sub problemas, sendo esses a CFD, a CDS e o Problema de Interação (\textit{Interaction Problem} - IP). Fazer o acoplamento adequado entre CFD e CSD pode se tornar uma tarefa complexa, pois se caracteriza pela multidisciplinaridade \cite{hou2012numerical}, onde pode ser necessário lidar com descrições e métodos diferentes para cada problema, por exemplo a utilização de uma descrição Lagrangiana para modelar o sólido e uma descrição Euleriana para o fluido. Além disso, os parâmetros nodais determinados em cada um dos dois problemas são, em geral, distintos, como deslocamentos ou posições em sólidos e velocidades e pressões no fluido.

Outra questão a ser observada em problemas de IFE é que, devido à movimentação da estrutura, deve-se realizar algum procedimento para que o domínio do fluido perceba essa movimentação. Nesse sentido, os métodos disponíveis são divididos basicamente em dois grupos: os métodos de malha móvel, onde a malha do fluido é conforme ao contorno da estrutura, sendo dinamicamente deformada, e os métodos de malhas fixas, onde a malha do fluido não é conforme ao contorno do sólido e permanece fixa (indeformável) durante a análise.

Nos métodos de malhas móveis, demanda-se a solução de um problema adicional para movimentar/deformar a malha do fluido de forma a acomodar a nova deformação da estrutura. É importante notar que em casos onde haja grande distorção do domínio do fluido, ou mudanças topológicas do mesmo (ex. contato entre sólidos) é necessário o remalhamento do fluido \cite{terahara2020heart}. Esse tipo de técnica é bastante precisa e robusta, já que permite a utilização de uma malha razoavelmente complexa próxima à interface fluido-estrutura para se obter resultados mais precisos nessa região. Nos trabalhos que empregam métodos de malhas móveis, destacam-se a utilização formulação Lagrangiana-Euleriana Arbitrária (\textit{Arbitrary Lagrangian-Eulerian} - ALE) \cite{donea1982arbitrary,kanchi20073d,fernandes2019ale} e da formulação Espaço Tempo (\textit{Space-Time} - ST) \cite{takizawa2011multiscale,terahara2020heart,takizawa2011stabilized}.

Por sua vez, os métodos de malhas não-conformes consideram as condições de interface impostas diretamente nas equações governantes, em geral através de técnicas de contorno imerso, permitindo que a malha do sólidos mova-se imersa na malha do fluido que permanece fixa. Essa técnica evitando os problemas relacionados à movimentação excessiva da malha, sendo adequada para problemas com mudanças topológicas do domínio. Porém, em certos casos de problemas de geometria complexa, a economia com os custos de remalhamento pode não compensar a perda de precisão próxima à interface \cite{bazilevs2013computational,hou2012numerical,bazilevs2015ale}. Dentro do âmbito de malhas fixas, nota-se a utilização de técnicas baseadas em contornos imersos em diversos trabalhos, como: o de \citeonline{zhao2016numerical}, o qual analisou os resultados da técnica por comparação com respostas analíticas, numéricas e experimentais, tendo obtido bons resultados; o de \citeonline{zheng2020numerical}, que utilizou uma formulação modificada de um método de contornos imersos, comparando os resultados com os obtidos experimentalmente em situações simétricas e assimétricas, obtendo resultados satisfatórios; o de \citeonline{xiao2022immersed}, sendo estudado escoamentos com transferência de massa, calor e momento, e também são apontadas pelos autores as dificuldades provenientes, dentre outras causas, da não conformidade da malha, especialmente em problemas com alto número de Reynolds; dentre outros, como de \citeonline{wang2011algorithms,ruess2013weakly,yan2021three}.

Como pode ser visto em \citeonline{hou2012numerical}, existem diferentes possibilidades de se impor as condições de acoplamento na solução numérica de problemas de IFE, podendo resultar em acoplamento monolítico, onde os meios sólido e fluido são tratados como uma única entidade, levando a um único sistema de equações ou particionado, onde sólido e fluido são solucionado separadamente com as condições de interface sendo transferidas de um meio para o outro como condições de contorno. Como exemplos de trabalhos que adotam o modelo monolítico pode-se citar:
Dentre os trabalhos que se utilizam do modelo monolítico cita-se os trabalhos de \citeonline{michler2004monolithic,hron2007fluid,wick2021optimization} e \citeonline{Acanvini2023formulacao}.

Os acoplamentos particionados ainda podem ser subdivididos em particionado forte, ou implícito, e particionado fraco ou explícito \cite{Felippaetal2001}. No acoplamento fraco, as condições de acoplamento são transmitidas de um meio para o outro a cada passo de tempo, sendo vantajoso principalmente em casos de escoamentos compressíveis quando o passo de tempo precisa ser muito pequeno para capturar propagações de ondas de choque \cite{sanches2011acoplamento,sanches2014fluid,sanches2013unconstrained}. No entanto, especialmente em problemas incompressíveis, o acoplamento fraco pode conduzir a efeitos numéricos indesejáveis, conduzindo a respostas erradas ou instáveis \cite{Felippaetal2001, fernandes2019ale}. No acoplamento forte por sua vez, as condições de interface são transmitida dentro de um processo iterativo processo de solução dos sistemas não lineares para sólido e fluido, levando a uma solução precisa. No entanto, essa técnica pode apresentar dificuldade de convergência em algumas situações, especialmente à medida em que as massas específicas do sólido e do fluido, onde os problemas são considerados fortemente acoplados e a pequena perturbação em um dos meios causa grandes mudanças no equilíbrio do outro meio. Para contornar esses problemas existem técnicas como o emprego de relaxações de processo iterativo baseado em Gauss-Seidel com relaxações de Aitken \cite{fernandes2019ale} ou a técnica \textit{augmented} A22 \cite{tezduyar2005finite}. Como exemplos de trabalhos que empregam o modelo particionado pode-se citar: \citeonline{sanches2013unconstrained,sanches2014fluid,fernandes2019ale}.

A principal vantagem do modelo particionado está na garantida da modularidade do código e na redução do custo computacional, ao se resolver sistemas menores e melhor condicionados \cite{sanches2011acoplamento,fernandes2020tecnica}. Por outro lado, o modelo monolítico apresenta como vantagens melhor precisão e evitar problemas de instabilidade e convergência \cite{Acanvini2023formulacao}.