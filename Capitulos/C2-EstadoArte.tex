%==================================================================================================
\section{Estado Da Arte}
%==================================================================================================

Na presente seção é apresentada uma breve revisão do estado da arte dos principais temas relacionados à análise numérica de IFE, tendo como ênfase os problemas com escoamentos turbulentos. São destacados os desenvolvimentos básicos para a dinâmica das estruturas computacional com grandes deslocamentos (\ref{CSD}), para a dinâmica dos fluidos computacional (\ref{CFD}) com contornos móveis, onde aborda-se também as técnicas de estabilização, assim como o modelo de estabilização variacional multiescala (VMS) e os modelos de turbulência (\ref{MT}), e para a solução de problemas de interação fluido-estrutura (\ref{IFE}).

%==================================================================================================
\subsection{Dinâmica Das Estruturas Computacional} \label{CSD}
%==================================================================================================

A mecânica dos sólidos busca descrever o comportamento de elementos estruturais quando sujeitos a solicitações externas, identificando os campos de tensão, deformação e deslocamento. A utilização de métodos computacionais para a solução desses problemas iniciou-se principalmente com o método das diferenças finitas, mas no decorrer do tempo, o Método dos Elementos Finitos (MEF), cujo nome foi cunhado por \citeonline{clough1960finite}, se destacou como uma ferramenta mais versátil, sendo atualmente a ferramenta mais difundida na mecânica dos sólidos.

Em geral, problemas envolvendo a mecânica dos sólidos, são formulados numericamente empregando métodos energéticos, que buscam a imposição da conservação da energia. Isso se deve principalmente ao fato de as leis constitutivas elásticas possuírem um potencial energético adequado. No entanto, os métodos de resíduos ponderados, tais como Bubnov-Galerkin ou Petrov-Galerkin podem ser igualmente empregados. Cabe ressaltar que, para os problemas elásticos clássicos, as equações governantes são elípticas (caso estático) ou parabólicas (caso dinâmico), permitindo o uso do método de Bubnov-Galerkin de forma estável, e conduzindo a sistemas com matrizes simétricas \cite{de2012nonlinear}.

Os primeiros trabalhos no âmbito da mecânica dos sólidos computacional são restritos aos casos de pequenos deslocamentos e pequenas deformações. em regime elástico linear. No entanto, mesmo em casos em que as estruturas se comportem dentro do regime elástico, surgem situações onde torna-se necessário considerar a configuração correta (deformada) do equilíbrio, como é o caso de diversos problemas de IFE. Assim, logo as análises começaram a considerar situações de grandes deslocamentos, como observado no trabalho de \citeonline{turner1960large}, surgindo o conceito de não linearidade geométrica, que foi explorada e tratada em diversos trabalhos importantes, tais como os de \citeonline{bathe1975finite,brendel1980linear,hughes1981nonlinear,hughes1983nonlinear,simo1986finite,de2012nonlinear}.

Tradicionalmente, as estruturas reticuladas têm sido representadas por elementos finitos que utilizam os deslocamentos e as rotações dos nós como graus de liberdade para a descrição do movimento \cite{reddy2005introduction,assan2020metodo}. No entanto, isso só se mostra eficiente para estruturas que desenvolvem pequenos deslocamentos e rotações. Isso deve-se ao fato de que não há comutatividade entre essas últimas grandezas.

Uma das abordagens que possibilitou a solução de diversos problemas de grandes deslocamentos é o MEF corrotacional, o qual decompõe o movimento de um sólido em uma parcela referente ao movimento de corpo rígido e outra referente à deformação do mesmo, sendo considerados os deslocamentos e as rotações nodais como graus de liberdade (\textit{Degree of freedom} - DOF), como pode ser observado no trabalho de \citeonline{WEMPNER1969117}, onde é estudado o comportamento de cascas em regime de grandes deslocamentos e pequenas deformações. Outros trabalhos relevantes a serem ressaltados nesse tema são os de \citeonline{hughes1981nonlinear, argyris1982excursion, ibrahimbegovic2002role, pimenta2004fully, battini2006choice} e \citeonline{pimenta2008exact}.

Tendo isso em vista a necessidades de aproximações de rotações finitas na formulação corrotacional, bem como efeitos que podem tornar a matriz de massa ser variável, tornando o processo de integração temporal complexo \cite{sanches2013unconstrained}, \citeonline{coda2003analise}, motivado pelo trabalho de \citeonline{bonet2000finite}, apresentou uma formulação alternativa do MEF, a qual baseia-se em posições. Tal abordagem possui vantagens ao se trabalhar com não linearidade geométrica, por ser independente das rotações nodais, assim como possui uma matriz de massa constante, o que facilita a análise dinâmicas das estruturas. Outro ponto a ser ressaltado é a simplicidade da formulação, assim como sua facilidade de implementação.

A formulação posicional do MEF tem sido amplamente empregada para problemas da mecânica dos sólidos e das estruturas, demostrando bastante robustez e precisão. Isso é atestado pelos trabalhos de \citeonline{coda2004simple}, o qual estudou a formulação posicional em elementos reticulados em análise estática, \citeonline{coda2007alternative}, que estendeu a formulação para elementos de casca, onde ainda foi adicionado um termo referente ao empenamento no trabalho de \citeonline{coda2010improved}, \citeonline{coda2009two}, que estudaram em um caso bidimensional o comportamento dinâmico de uma estrutura inflável, \citeonline{coda2009unconstrained}, que utilizou a formulação em elementos de cascas em análises dinâmicas e \citeonline{carrazedo2010alternative}, que estudaram os efeitos de impactos termomecânicos em elementos de treliça.

A maior parte dos trabalhos referentes à formulação posicional do MEF para problemas dinâmicos, utiliza o integrador temporal de Newmark, que insere dois parâmetros livres que são escolhidos pelo usuário, modificando a forma da aproximação de velocidades de acelerações e conferindo diferentes características de estabilidade e dissipação. A formulação posicional resulta em uma matriz de massa constante, de forma que \citeonline{sanches2013unconstrained} mostram que nesse cenário o integrador de Newmark conserva o momento linear e angular em análises dinâmicas e apresenta conservação da energia suficiente para os problemas de IFE considerados.

Uma alternativa ao método de Newmark é o método $\alpha$-generalizado, introduzido por \citeonline{chung1993time}, sendo também um esquema de integração temporal implícito que se utiliza de valores de velocidades e acelerações em instantes de tempo intermediários para a determinação de valores futuros. A característica interessante dessa técnica está na possibilidade de se controlar facilmente a difusão numérica das altas frequências no processo de marcha no tempo por meio do ajuste de um único parâmetro. Alguns trabalhos mais recentes que aplicaram com sucesso o MEF posicional em conjunto com o método $\alpha$-generalizado são \citeonline{siqueira2019ligaccoes,moreira2021analise,Avancini2023formulacao}.

Dentre as aplicações dinâmicas da formulação posicional do MEF, pode-se citar inicialmente o trabalho de \citeonline{marques2006estudo}, o qual analisa o comportamento de sólidos bidimensionais em situação de não linearidade geométrica. Na sequência aplica-se a formulação posicional à elementos de pórticos planos e sólidos tridimensionais por \citeonline{maciel2008analise}. Também foi avaliado o comportamento de elementos de cascas em problemas estáticos \cite{coda2007alternative} e dinâmicos \cite{coda2009unconstrained}.

No âmbito da IFE, destaca-se a necessidade de se utilizar metodologias que capturem eficientemente grandes deslocamentos, uma vez que são observados efeitos de não linearidade geométrica em problemas como: \textit{flutter}, aplicações biomecânicas, estruturas infláveis \cite{karagiozis2011computational}, simulações de turbinas \cite{bazilevs20113d}, dentre outros.

Já a utilização do MEF baseado em posições em problemas de IFE pode ser vista nos trabalhos de \citeonline{sanches2013unconstrained}, onde considera-se o acoplamento de estruturas de cascas com escoamentos compressíveis, \citeonline{fernandes2019ale} onde considera-se acoplamento de cascas com escoamentos incompressíveis, \citeonline{fernandes2020tecnica}, o qual aplicou a formulação em elementos reticulados, \citeonline{Avancini2023formulacao}, que utilizou a formulação em problemas de escoamento com superfície livre, dentre outros. Assim, percebe-se o bom desempenho dessa formulação nesse contexto. No entanto, a utilização dessa formulação em problemas de IFE com elevados números de Reynolds ainda não foi explorada.


%==================================================================================================
\subsection{Dinâmica Dos Fluidos Computacional} \label{CFD}
%==================================================================================================

Ao contrário dos problemas da mecânica dos sólidos, que possuem um estado inicial bem definido, os fluidos, em especial os Newtonianos, não o possuem, uma vez que são incapazes de resistir a tensões desviadoras, deformando-se, assim, indefinidamente quando sujeitos à essas tensões. Dessa forma, torna-se apropriada a utilização de uma descrição Euleriana para seu equacionamento, os graus de liberdade são, em geral, velocidade e pressão \cite{fernandes2020tecnica}.

Quando aplicado o método de Bubnov-Galerkin às equações da mecânica dos fluidos em descrição Euleriana, os termos convectivos (hiperbólicos) à medida em que se tornam dominantes sobre os termos dissipativos, induzem oscilações espúrias na solução \cite{bazilevs2013computational,brooks1982streamline}. Isso contribuiu para que o método dos elementos finitos fosse preterido no início dos estudos de CFD, tendo sido largamente empregados outros métodos como os das diferenças finitas e dos volumes finitos, como pode ser visto, por exemplo, em \citeonline{anderson1995computational,chung2002computational}.

No entanto, com o passar do tempo, surgiram trabalhos que buscaram estabilizar os efeitos da convecção no MEF de forma eficiente, sendo o mais empregado o método \textit{Streamline-Upwind/Petrov-Galerkin} (SUPG) \cite{brooks1982streamline}, que se baseia na adição do resíduo da equação da quantidade de movimento ponderado por uma função desenvolvida para introduzir estabilização na direção das linhas de corrente. Outras alternativas importantes são baseadas em mínimo quadrados (\textit{Galerkin Least-Squares} - GLS \cite{hughes1989new,tezduyar1991stabilized}) ou na ideia de análise multiescala (por exemplo o \textit{Subgrid Scales} - SGS \cite{piomelli1999large,hughes2000large} e \textit{Variational Multiscale Method} - VMS \cite{hughes1995multiscale,hughes1998variational,hughes2000large}). Assim, dadas às suas vantagens como ferramenta de discretização espacial e facilidades para imposições de condições de contorno, o MEF passou a ser bastante explorado também no contexto da CFD.

% Também podem ocorrer instabilidades, no caso de escoamentos compressíveis, relacionados a variações abruptas das propriedades do escoamento, denominadas de ondas de choque. Sendo assim, desenvolveram-se técnicas capazes de estabilizar tais problemas, como a captura de choque, utilizada por \citeonline{Nithiarasuetal1998,ZienkTaylor2000v3,Nithiarasuetal2006}, a qual considera a adição de um termo de viscosidade artificial, tendo seu valor máximo no ponto de descontinuidade e que vai sendo anulado conforme se afasta desse ponto.

No caso da aplicação do MEF para escoamentos incompressíveis, a escolha dos espaços aproximadores para pressão e velocidade pode conduzir a resultados espúrios para o campo de pressão. Para que se obtenha resposta estável, essa escolha deve atender às condições de \LBB\ (LBB), sendo que uma condição necessária, porém não suficiente, aponta que a ordem do espaço de aproximação para o campo de pressões deve ser inferior à do espaço de aproximação de velocidades \cite{BrezziF1991,donea2003finite,fernandes2020tecnica}.

Visando conferir maior flexibilidade aos métodos, permitindo mesma ordem de aproximação para velocidade e pressão, foram desenvolvidas formulações estabilizadas, tais como os métodos \textit{Pressure-Stabilizing/Petrov-Galerkin} (PSPG), que parte de uma ideia similar ao SUPG, adicionando um termo estabilizador à forma fraca do problema e melhorando a estabilidade do método no campo de pressões, e \textit{Galerkin/Least-Squares} (GLS), o qual adiciona resíduos na forma de mínimos quadrados ao método de Galerkin, obtendo, dessa forma, resultados estáveis ao mesmo tempo que a precisão do método não é afetada \cite{hughes1989new,TezduyarS2003}.

Outro problema surge à medida em que o número de Reynolds do escoamento aumenta, passando de um regime laminar de escoamento para um regime turbulento, é a formação e desprendimento de vórtices. Esses fenômenos podem se apresentar nas mais variadas escalas, podendo demandar a geração de uma malha muito refinada para capturar os vórtices e solucionar o problema de forma estável. Isso ocasiona um aumento radical no custo computacional da análise. Como espera-se que um método numérico seja capaz de apresentar soluções estáveis para os problemas, independentemente do nível de discretização, diversos trabalhos foram conduzidos a fim de desenvolverem formulações estáveis para escoamentos com elevados números de Reynolds, buscando minimizar os custos computacionais, destacando-se  os métodos \RANS\ (RANS) \cite{speziale1991analytical,alfonsi2009reynolds,ling2015evaluation}, que parte da ideia de decompor as variáveis envolvidas em parcela média e parcela de flutuações, em que a média das variáveis pode ser tomada de diferentes maneiras, a depender da natureza do problema, e as Simulações de Grandes Vórtices (\LES\ - LES) \cite{germano1991dynamic,piomelli1999large,hughes2000large,vsekutkovski2021partitioned}, as quais decompõem as parcelas em uma parte filtrada e outra não filtrada a partir do produto convolucional dos campos analisados com o filtro adotado.

O método Variacional Multiescala (\VMS\ - VMS) \cite{hughes1995multiscale,hughes1998variational,hughes2002variational,bazilevs2010large,bazilevs2013computational}, por sua vez, visa garantir ao mesmo tempo estabilização para os efeitos de convecção, para o campo de pressão, e para o problemas de vorticidade. Esse modelo faz uso dos princípios variacionais, em que tanto os espaços tentativas quanto os espaços testes são divididos em parcela de grandes escalas e e parcela de pequenas escalas. Com isso se faz a modelagem do espaço de pequenas escalas em termos de resíduos das equações de conservação de massa e de conservação da quantidade de movimento. Dessa forma obtém-se uma formulação consistente, onde os termos estabilizantes das formulações PSPG e SUPG se fazem presentes, ao mesmo tempo em que outros termos consistentes surgem \cite{bazilevs2013computational,sondak2015new}.

\begin{comment}
\textcolor{red}{Faltou falar sobre o problema de contornos móveis (introduzir ALE, Spaço-tempo e contorno imerso como possibilidades de solução )}
\textcolor{red}{ALE: \cite{donea1982arbitrary,hughes1981lagrangian,bazilevs2015ale}\\}
\textcolor{red}{Espaço-tempo: \cite{aliabadi1993space,hughes1996space,hughes2000large}\\}
\textcolor{red}{Contorno imerso: \cite{peskin1972flow,peskin1977numerical,akkerman2012free}\\}
\end{comment}

%==================================================================================================
\subsubsection{Efeitos de turbulência} \label{MT}
%==================================================================================================

Para se compreender as estruturas de um regime de escoamento, é comum empregar uma parâmetro adimensional dado pela razão entre as forças inerciais e viscosas, denominado de número de Reynolds. Os escoamentos podem se apresentar de três formas distintas a medida em que o seu número de Reynolds evolui. Primeiramente (a baixos números de Reynolds) observa-se o regime de escoamento laminar, caracterizado por um escoamento ordenado, tendo sua movimentação semelhante a lâminas deslizando independentemente entre si, sem mistura macroscópica entre as camadas de fluido \cite{popiolek2005analise,shaughnessy2005introduction}. Conforme o número de Reynolds aumenta, o escoamento passa por um momento de transição, com a formação esporádica de vórtices. Já quando o número de Reynolds se torna elevado, o escoamento passa a ser classificado como turbulento, sendo caracterizado por uma mistura intensa entre as camadas de fluido, com vórtices de diferentes escalas e formados de maneira desordenada.

As equações de Navier-Stokes são capazes de descrever esses escoamentos, no entanto, sua solução numérica passa a apresentar alto grau de complexidade, podendo demandar discretizações muito refinadas para a obtenção de uma resposta estável e consistente \cite{neto2002fundamentos}.

A aplicação direta do MEF sobre as equações de Navier-Stokes para a simulação de problemas turbulentos é conhecida como \textit{Direct Numerical Simulation} (DNS). Essa abordagem resolve diretamente em todas as escalas presentes, sendo, portanto uma abordagem precisa para se simular escoamentos turbulentos, mas que necessita de discretização muito refinada, sendo utilizado em trabalhos como os de \citeonline{yokokawa200216,picano2015turbulent,olad2022towards,frey2021machine}. No entanto, o custo computacional acaba por restringir sua aplicação a problemas pequenos e de geometria simples, sendo muito empregado para verificação de modelos numéricos que sejam mais viáveis \cite{piomelli1999large,yokokawa200216}.

Sendo assim surge a necessidade de se utilizar modelos de turbulência, como aqueles baseados na decomposição de Reynolds, ou na simulação de grandes vórtices. Com isso os modelos de turbulência podem ser classificados, dentre outras formas, como: modelos simples onde se obtém expressões algébricas para o cálculo de viscosidade de vórtice, onde se assume que a geração e dissipação de estruturas turbulentas ocorrem no mesmo local; modelos que adicionam equações diferenciais para o cálculo da viscosidade de vórtice; e modelos que adicionam equações de transporte para o cálculo das tensões de Reynolds \cite{souza2011revisao,alfonsi2009reynolds,teixeira2001simulaccao}.

Em geral o que se busca com a aplicação dos modelos de turbulência é a decomposição das variáveis envolvidas, que pode ser feita de diferentes maneiras a depender do modelo adotado, de forma a se modelar as parcelas turbulentas e resolver diretamente a parcela resultante. Isso faz com que seja possível a utilização de malhas mais grosseiras, reduzindo o custo computacional.

%==================================================================================================
\subsubsubsection{\textit{Reynolds-Averaged Navier-Stokes}} \label{RANS}
%==================================================================================================

Uma maneira alternativa de se tratar os escoamentos turbulentos é o \textit{Reynolds-Averaged Navier-Stokes} (RANS), cujos conceitos iniciais foram introduzidos por \citeonline{reynolds1895iv}. A principal premissa desse modelo diz que é possível separar as variáveis envolvidas no escoamento em duas parcelas: uma relacionada à média, sendo essa a parcela predominante e a principal a ser determinada, e outra relacionada a variações no espaço-tempo.

A maneira como a média é tomada nesse caso pode variar de acordo com a natureza do problema, podendo ser uma média temporal, apropriada para escoamentos estacionários, uma média espacial, apropriada para escoamentos com distribuição homogênea de estruturas turbulentas, e uma média de conjunto de experimentos, adequado para casos mais gerais \cite{speziale1991analytical,alfonsi2009reynolds}.

Isso introduz ao equacionamento um termo adicional relacionado às interações entre as parcelas flutuantes, que representa a interferência dos efeitos turbulentos na propriedade média. Por conta dessa relação existente entre as parcelas médias e de flutuações, surge a necessidade de adicionar mais equações ao problema, de forma a fechá-lo. Assim, diferentes abordagens se mostram possíveis e que podem ser classificadas em função da quantidade de equações de transporte adicionadas, sendo elas: modelos de zero equações, uma equação, duas equações e modelos de tensões \cite{piomelli1999large,alfonsi2009reynolds,bazilevs2010large,ling2015evaluation}.

%==================================================================================================
\subsubsubsection{\textit{Large-Eddy Simulation}} \label{LES}
%==================================================================================================

Outra forma de abordar tais problemas é com o emprego do modelo de \textit{Large-Eddy Simulation} (LES), introduzido pelo trabalho de \citeonline{smagorinsky1963general}, que o propôs no intuito de se estudar simulações de camadas limite atmosféricas. Esse modelo considera a separação das grandezas em duas parcelas: uma parcela filtrada, a qual captura os grandes vórtices, possíveis de ser calculada diretamente por meio das equações de Navier-Stokes filtradas; e outra não-filtrada, a qual representa os pequenos vórtices, que apesar de não poder ser calculada diretamente, possui um comportamento isotrópico, viabilizando sua modelagem. Uma possível forma de se modelar os termos não-filtrado baseia-se no modelo de \textit{Sub-Grid Scale} (SGS), que faz a interação entre os campos de grandes e de pequenas escalas, sendo aprimorada de forma a capturar os efeitos turbulentos em função do tamanho dos elementos \cite{ghosal1995basic,hughes2000large,moeng2015large}.

Outra possibilidade se baseia no modelo de Smagorinsky, o qual faz a consideração das estruturas formadas na subescala por meio da adição de um termo de viscosidade de vórtice, inserindo no problema uma constante, denominada de constante de Smagorinsky. Posteriormente, \citeonline{deardorff1971magnitude} estimou o coeficiente de Smagorinsky ao estudar escoamentos em canais. A ideia central do modelo parte da consideração de que o escoamento pode ser bem caracterizado por meio de uma separação de escalas, na qual as grandes escalas são responsáveis pela transferência da energia cinética, sendo influenciadas diretamente pela natureza do escoamento, assim como pelas condições de contorno, enquanto as pequenas escalas são responsáveis pela dissipação de energia e possuem propriedades isotrópicas e homogêneas no escoamento. Assim, o modelo de Smagorinsky consiste em uma decomposição do campo de velocidades em duas parcelas, uma de grandes escalas e outra de pequenas escalas, sendo a parcela de pequenas escalas modelada por um termo viscoso, o qual é determinado a partir de um modelo de viscosidade de vórtice.

%==================================================================================================
\subsection{Interação Fluido-Estrutura} \label{IFE}
%==================================================================================================

A solução numérica de problemas de IFE compreende a solução de 3 subproblemas, sendo esses a CFD, a CSD e o Problema de Interação (\textit{Interaction Problem} - IP). Fazer o acoplamento adequado entre CFD e CSD pode se tornar uma tarefa complexa, pois o problema se caracteriza pela multidisciplinaridade \cite{hou2012numerical}, onde pode ser necessário lidar com descrições e métodos diferentes para cada problema, por exemplo a utilização de uma descrição Lagrangiana para modelar o sólido e uma descrição Euleriana para o fluido. Além disso, as variáveis incógnitas em cada um dos dois problemas são, em geral, distintas, como deslocamentos ou posições em sólidos e velocidades e pressões no fluido.

Outra questão a ser observada em problemas de IFE é que, devido à movimentação da estrutura, deve-se realizar algum procedimento para que o domínio do fluido perceba essa movimentação. Nesse sentido, alguns dos métodos disponíveis são: os métodos de malha móvel, onde a malha do fluido é conforme ao contorno da estrutura, sendo dinamicamente deformada, os métodos de malhas fixas, onde a malha do fluido não é conforme ao contorno do sólido e permanece fixa (indeformável) durante a análise, e os métodos híbridos, que se utiliza de superposição de malhas fixa e móvel \cite{fernandes2020tecnica}.

Nos métodos de malhas móveis, demanda-se a solução de um problema adicional para movimentar/deformar a malha do fluido de forma a acomodar a nova configuração da estrutura. Esse tipo de técnica é bastante precisa e robusta, já que permite a utilização de uma malha razoavelmente complexa próxima à interface fluido-estrutura para se obter resultados mais precisos nessa região. No entanto, é importante notar que em casos em que haja grande distorção do domínio do fluido, ou mudanças topológicas do mesmo (ex. contato entre sólidos) é necessário o remalhamento do fluido \cite{terahara2020heart}.  Nos trabalhos que empregam métodos de malhas móveis, destacam-se a utilização formulação Lagrangiana-Euleriana Arbitrária (\textit{Arbitrary Lagrangian-Eulerian} - ALE) \cite{donea1982arbitrary,kanchi20073d,fernandes2019ale} e da formulação Espaço Tempo (\textit{Space-Time} - ST) \cite{takizawa2011multiscale,terahara2020heart,takizawa2011stabilized} para permitir a movimentação da malha do fluido de forma independente do movimento das partículas.

Por sua vez, os métodos de malhas não-conformes consideram as condições de interface impostas diretamente nas equações governantes, em geral através de técnicas de contorno imerso, permitindo que a malha do sólido se mova imersa na malha do fluido, a qual permanece fixa. Essa técnica evita os problemas relacionados à movimentação excessiva da malha, sendo adequada para problemas com mudanças topológicas do domínio. Porém, em certos casos de problemas de geometria complexa, a economia com os custos de remalhamento pode não compensar a perda de precisão próxima à interface \cite{bazilevs2013computational,hou2012numerical,bazilevs2015ale}.

No âmbito dos acoplamentos com malhas fixas, nota-se a utilização de técnicas baseadas em contornos imersos em diversos trabalhos, como: o de \citeonline{zhao2016numerical}, o qual analisou os resultados da técnica por comparação com respostas analíticas, numéricas e experimentais, tendo obtido bons resultados; o de \citeonline{zheng2020numerical}, que utilizou uma formulação modificada de um método de contornos imersos, comparando os resultados com os obtidos experimentalmente em situações simétricas e assimétricas, obtendo resultados satisfatórios; o de \citeonline{xiao2022immersed}, sendo estudado escoamentos com transferência de massa, calor e momento, e também são apontadas pelos autores as dificuldades provenientes, dentre outras causas, da não conformidade da malha, especialmente em problemas com alto número de Reynolds; dentre outros, como de \citeonline{wang2011algorithms,ruess2013weakly,yan2021three}.

Como pode ser visto em \citeonline{hou2012numerical}, existem diferentes possibilidades de se impor as condições de acoplamento na solução numérica de problemas de IFE, podendo resultar em acoplamento monolítico, onde os meios sólido e fluido são tratados como uma única entidade, levando a um único sistema de equações; ou particionado, onde sólido e fluido são solucionados separadamente com as condições de interface sendo transferidas de um meio para o outro como condições de contorno.

O acoplamento monolítico é muito robusto e elimina problemas de estabilidade que podem surgir com a solução particionada. No entanto, é pouco flexível do ponto de vista de permitir diferentes métodos para os domínios fluido e sólido, além de resultar num sistema de equações maior e que pode ser mal condicionado. Como exemplos de trabalhos que adotam o modelo monolítico pode-se citar os trabalhos de \citeonline{michler2004monolithic,hron2007fluid,wick2021optimization} e \citeonline{Avancini2023formulacao}.

Já os acoplamentos particionados podem ser subdivididos em modelos fortes, ou implícitos, e modelos fracos, ou explícitos \cite{Felippaetal2001}. Nos acoplamento fracos, as condições de acoplamento são transmitidas de um meio para o outro a cada passo de tempo, sendo vantajoso principalmente em casos de escoamentos compressíveis quando o passo de tempo precisa ser muito pequeno para capturar propagações de ondas de choque \cite{sanches2011acoplamento,sanches2014fluid,sanches2013unconstrained}. No entanto, especialmente em problemas incompressíveis, o acoplamento fraco pode conduzir a efeitos numéricos indesejáveis, conduzindo a respostas erradas ou instáveis \cite{Felippaetal2001, fernandes2019ale}. No acoplamento forte por sua vez, as condições de interface são transmitidas dentro de um processo iterativo de solução dos sistemas não lineares para sólido e fluido, levando a uma solução precisa. No entanto, essa técnica pode apresentar dificuldade de convergência em algumas situações, especialmente à medida em que as massas específicas do sólido e do fluido se aproximam, onde os problemas são considerados fortemente acoplados e a pequena perturbação em um dos meios causa grandes mudanças no equilíbrio do outro. Para contornar esses problemas existem técnicas como o emprego de relaxações de processo iterativo baseado em Gauss-Seidel com relaxações de Aitken \cite{fernandes2019ale} ou a técnica \textit{augmented} A22 \cite{tezduyar2005finite}. Como exemplos de trabalhos que empregam o modelo particionado pode-se citar: \citeonline{sanches2013unconstrained,sanches2014fluid,fernandes2019ale}.

A principal vantagem do modelo particionado está na modularidade do código e na redução do custo computacional, ao se resolver sistemas menores e mais bem condicionados \cite{sanches2011acoplamento,fernandes2020tecnica}. Por outro lado, o modelo monolítico apresenta como vantagens melhor precisão e evita problemas de instabilidade e convergência \cite{Avancini2023formulacao}.