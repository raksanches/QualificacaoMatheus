%==================================================================================================
\section{Metodologia} \label{MetodologiaCronograma}
%==================================================================================================

Tendo em vista a complexidade dos problemas em estudo, delimita-se este trabalho aos estudos dos problemas de interação fluido-estrutura considerando escoamento incompressível viscoso, interagindo com estruturas de casca com grandes deslocamentos. Adota-se acoplamento particionado forte do tipo bloco-iterativo, com método de malha móvel para o fluido baseado na descrição Lagrangiana-Euleriana Arbitrária (ALE). Essa abordagem é adotada por ser reconhecidamente robusta, modular e amplamente utilizada para análises de IFE.

Assim como os problemas a serem estudados, as implementações necessárias também são bastante complexas, sendo importante a adoção de uma metodologia de programação que aproveite os códigos disponíveis e ao mesmo tempo facilite alterações de tipos de elementos, modelos constitutivos, métodos de solução de sistemas e operações algébricas, e, mais importante no contexto desta proposta, diferentes métodos de estabilização e modelos de turbulência.

Dessa forma adota-se a linguagem de programação C++ orientada a objeto, uma vez que se é possível aproveitar diversos códigos já desenvolvidos pelo grupo de pesquisa, \ie\ programa de elementos finitos para análise de escoamentos incompressíveis com contornos móveis e programa para análise dinâmica não linear geométrica de estruturas de casca finas ou espessas.

Para a solução da mecânica dos sólidos, emprega-se uma formulação o MEF baseada em posições, aproveitando-se um programa para análise de cascas com cinemática de Reissner-Mindlin, já desenvolvido pelo grupo de pesquisa. A formulação baseada em posições emprega uma descrição Lagrangiana Total, sendo adotado o modelo constitutivo de Saint-Venant Kirchhoff, com a medida de deformação de Green-Lagrange. O elemento de casca utilizado possui 7 graus de liberdade por nó, \ie\ 3 componentes de posição nodal, 3 componentes de um vetor generalizado, inicialmente perpendicular à superfície média da casca, e 1 parâmetro de enriquecimento para permitir variação linear da deformação na direção da espessura. O programa emprega elementos finitos isoparamétricos triangulares de aproximação linear, quadrática ou cúbica, sendo a integração temporal realizada por meio do método de marcha no tempo $\alpha$-generalizado, e o sistema não linear resultante é resolvido por meio do método de Newton-Raphson.% A aproximação das derivadas temporais do problema é feita com integrador temporal $\alpha$-generalizado. O processo de solução do sistema não linear é obtida por meio do método iterativo de Newton-Raphson.

Já para a solução da mecânica dos fluidos, parte-se de um código computacional para escoamentos bidimensionais e tridimensionais empregando a formulação estabilizada SUPG/PSPG do MEF e descrição ALE, o que possibilita a representação de contornos móveis. São empregados elementos triangulares (2D) e tetraédricos (3D) com aproximações linear ou quadrática. Assim como no sólido, a aproximação temporal é dada pelo integrador $\alpha$-generalizado e o método de Newton-Raphson é empregado para a solução do sistema não linear. Neste código é parte desta proposta a implementação de elementos Taylor-Hood, com aproximação quadrática para velocidades e linear para pressão, a implementação da técnica de estabilização \LSIC\ (LSIC) para captura de vórtices, assim como o modelo de turbulência LES e o modelo de estabilização VMS.

Após o estudo e a verificação isolada das ferramentas para solução do fluido e do sólido, parte-se para o acoplamento. Para isso adota-se um modelo de acoplamento particionado forte do tipo bloco-iterativo, sendo a movimentação da malha do fluido realizada com base na equação de Laplace, em que os elementos menores possuem rigidez mais elevada, para evitar distorções excessivas, enquanto os elementos maiores absorvem a maior parte das deformações.

Frente ao custo computacional dos problemas considerados, emprega-se o protocolo de processamento paralelo \textit{Message Passing Interface} (MPI), com utilização da biblioteca PETSc (\textit{Portable, Extensible Toolkit for Scientific Computation}) \cite{petsc-web-page}, a qual é reconhecidamente eficiente para solução de sistemas esparsos que necessitem de alto desempenho computacional. As malhas são geradas a partir do programa Gmsh \cite{geuzaine2009gmsh}, e os resultados são interpretados graficamente por meio do visualizador Paraview \cite{ahrens2005paraview} e por meio do programa gnuPlot, no caso da geração de gráficos de linha 2D ou 3D. Nota-se que se prioriza o uso de ferramentas computacionais livres e de código aberto para facilitar a distribuição e as atualizações da ferramenta computacional resultante.