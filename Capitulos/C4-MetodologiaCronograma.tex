%==================================================================================================
\chapter{Metodologia e cronograma} \label{MetodologiaCronograma}
%==================================================================================================

Tendo em vista a complexidade dos problemas em estudo, delimita-se os estudos dos problemas de interação fluido-estrutura considerando escoamento incompressível viscoso interagindo com estruturas reticuladas com grandes deslocamentos. Adota-se acoplamento particionado forte do tipo bloco-iterativo com método de malha móvel para o fluido empregando-se a descrição Lagrangiana-Euleriana Arbitrária (ALE). Essa abordagem é adotada por ser reconhecidamente robusta, modular e amplamente utilizada para análises de IFE.

Assim como os problemas a serem estudados, as implementações necessárias também são bastante complexas, sendo importante a adoção de uma metodologia de programação que aproveite os códigos disponíveis e ao mesmo tempo facilite alterações de tipos de elementos, modelos constitutivos, métodos de solução de sistemas e operações algébricas, e mais importante no contexto desta proposta, diferentes métodos de estabilização e diferentes modelos de turbulência.

Dessa forma, adota-se a linguagem de programação C++ orientada a objeto, em um sistema operacional Linux, uma vez que se é possível aproveitar diversos códigos já desenvolvidos pelo grupo de pesquisa, a saber, programa de elementos finitos para análise de escoamentos incompressíveis empregando-se elementos triangulares (2D) e tetraédricos (3D) com aproximações linear ou quadrática e programa para análise dinâmica não linear geométrica de estruturas de casca finas ou espessas empregando elementos triangulares cúbicos.

Para a solução da mecânica dos sólidos será empregado o MEF baseado em posições, aproveitando-se um programa para análise de cascas com cinemática de Reissner-Mindlin, já desenvolvido pelo grupo de pesquisa. A formulação baseada em posições emprega uma descrição Lagrangiana Total, sendo adotado o modelo constitutivo de Saint-Venant Kirchhoff  com a medida de deformação de Green-Lagrange. O elemento de casca utilizado possui 7 graus de liberdade por nó, a saber, 3 coordenadas de posição, 3 coordenadas de um vetor generalizado, inicialmente perpendicular à superfície média da casca, e 1 parâmetro de enriquecimento para permitir variação linear da deformação na direção da espessura. A aproximação das derivadas temporais do problema é feita com integrador temporal $alpha$-generalizado. O processo de solução dos sistema não linear é obtida por meio do método iterativo de Newton-Raphson. 

Já para a solução da mecânica dos fluidos fluido, parte-se de um código computacional para escoamentos bidimensionais e tridimensionais empregado a formulação estabilizada PSPG/SUPG do MEF e a descrição ALE, o que possibilita a representação de contornos móveis. Assim como no sólido, a aproximação temporal é dada pelo integrador $\alpha$-generalizado e o método de Newton-Raphson é empregado para a solução do sistema não linear. Neste código é parte desta proposta a implementação de elementos Taylor-Hood, com aproximação quadrática para velocidades e linear para pressão, a implementação da técnica de estabilização \LSIC (LSIC) para captura de vórtices, assim comoos modelos de turbulência RANS e LES e o modelo de estabilização VMS (implementações essas que já foram finalizadas).

Após o estudo e verificação isolada das ferramentas para solução do fluido e do sólido, parte-se para o acoplamento.
Para isso será adotado um modelo de acoplamento particionado forte do tipo bloco-iterativo, sendo a movimentação da malha do fluido realizada por meio de um problema fictício de elasticidade, considerando-se a malha do fluido como um sólido elástico com deslocamentos prescritos no contorno.

% \textcolor{red}{-Falar como você vai resolver o sólido (citar o programa de cascas disponível, a formulação empregada e a integração temporal adotada). Especificar se você vai implementar alguma coisa nele ou só utilizar.}

% \textcolor{red}{-Falar como você vai resolver o fluido considerando contornos móveis (citar o programa de fluido disponível, a formulação/elementos empregados e a integração temporal adotada). Especificar o que você vai implementar}

% \textcolor{red}{-Falar sobre a técnica de acoplamento a ser adotada para unir os códigos}

Será empregado o protocolo de processamento paralelo \textit{Message Passing Interface} (MPI), com utilização da biblioteca PETSc (\textit{Portable, Extensible Toolkit for Scientific Computation}) \cite{petsc-web-page}, a qual é reconhecidamente eficiente para solução de sistemas esparços que necessitem de alto desempenho computacional, tendo suporte para C++ e MPI.  As malhas serão geradas a partir do programa Gmsh \cite{geuzaine2009gmsh}, e os resultados serão interpretados graficamente por meio do visualizador Paraview \cite{ahrens2005paraview} e por meio do programa gnuPlot, no caso da geração de gráficos de linha 2D ou 3D. Nota-se que prioriza-se o uso de ferramentas computacionais livres e de código aberto para facilitar a distribuição e as atualizações da ferramenta computacional resultante.

%==================================================================================================
\section{Cronograma}
%==================================================================================================

% \textcolor{red}{As atividades estavam insuficientes para descrever o seu trabalho durante o curso (pode incluir as disciplinas, já que o cronograma é do trabalho de mestrado como todo). Adicionei outras atividades, verifique se não falta nada e atualize o cronograma (Pode incluir o exame de qualificação...).}

As atividades necessárias para se alcançar os objetivos desta pesquisa são divididas da seguinte maneira:

\begin{enumerate}[label=\alph*.]
	\item\label{M:1} Cursar disciplinas do programa de mestrado;
	\item\label{M:2} Revisão bibliográfica a ser realizada de forma contínua ao longo de todo o trabalho, para que o mesmo se mantenha atualizado frente ao estado da arte durante toda sua execução;
	\item\label{M:3} Estudo das formulações estabilizadas do método dos elementos finitos para a solução de escoamentos incompressíveis com contornos móveis;
	\item\label{M:4} Estudo dos modelos de turbulência RANS e LES;
	\item\label{M:5} Estudo do código computacional para escoamentos incompressíveis disponível e implementações do modelo de estabilização VMS;
	\item\label{M:6} Implementação computacional dos modelos de turbulência no código de escoamentos incompressíveis;
	\item\label{M:7} Verificação dos modelos de turbulência implementados no contexto de escoamentos com contornos fixos a partir de exemplos presentes na literatura;
	\item\label{M:8} Estudo da formulação posicional do MEF para elementos de casca com cinemática de Reissner-Mindlin;
	\item\label{M:9} Exame de qualificação de mestrado;
	\item\label{M:10} Estudo e implementação do modelo de acoplamento particionado do tipo bloco-iterativo;
	\item\label{M:11} Verificação e Estudo dos modelos implementados no contexto dos problemas de iteração fluido-estrutura com elevados números de Reynolds;
	\item\label{M:12} Redação da dissertação e elaboração de artigos para revistas;
	\item \label{M:13} Defesa do mestrado.
\end{enumerate}

Essas atividades são organizadas cronologicamente, ao longo de todo o período planejado para a conclusão do mestrado, de acordo com o cronograma da Tabela \ref{Cronograma}.

\definecolor{lightgray}{RGB}{153,153,153}
\definecolor{lightblue}{RGB}{126,194,216}

\begin{table}[h!]
	\caption{Cronograma de atividades}
	\fontsize{8}{14}\selectfont
	\centering
	\newcommand{\Rx}{\cellcolor{lightblue}}
	\newcommand{\Px}{\cellcolor{lightgray}}
	\begin{tabular}{|c|ccccccccc|cccccccccccc|cccc|}
		\hline
		\MR{2}{*}{Item} & \MC{9}{c|}{2022} & \MC{12}{c|}{2023} & \MC{4}{c|}{2024}                                                                                                                                     \\ \cline{2-26}
		                & A                & M                 & J                & J   & A   & S   & O   & N   & D   & J   & F   & M   & A   & M   & J   & J   & A   & S   & O   & N   & D   & J   & F   & M   & A   \\ \hline
		\ref{M:1}       & \Rx              & \Rx               & \Rx              & \Rx & \Rx & \Rx & \Rx & \Rx & \Rx &     &     &     &     &     &     &     &     &     &     &     &     &     &     &     &     \\ \hline
		\ref{M:2}       & \Rx              & \Rx               & \Rx              & \Rx & \Rx & \Rx & \Rx & \Rx & \Rx & \Rx & \Rx & \Rx & \Rx & \Rx & \Rx & \Rx & \Px & \Px & \Px & \Px & \Px & \Px & \Px & \Px &     \\ \hline
		\ref{M:3}       &                  &                   &                  &     & \Rx & \Rx & \Rx & \Rx & \Rx & \Rx & \Rx &     &     &     &     &     &     &     &     &     &     &     &     &     &     \\ \hline
		\ref{M:4}       &                  &                   &                  &     &     &     &     & \Rx & \Rx & \Rx & \Rx & \Rx & \Rx & \Rx &     &     &     &     &     &     &     &     &     &     &     \\ \hline
		\ref{M:5}       &                  &                   &                  &     &     &     &     &     &     &     & \Rx & \Rx & \Rx & \Rx &     &     &     &     &     &     &     &     &     &     &     \\ \hline
		\ref{M:6}       &                  &                   &                  &     &     &     &     &     &     &     &     &     &     & \Rx & \Rx & \Rx & \Px & \Px & \Px &     &     &     &     &     &     \\ \hline
		\ref{M:7}       &                  &                   &                  &     &     &     &     &     &     &     &     &     &     &     & \Rx & \Rx &     &     &     &     &     &     &     &     &     \\ \hline
		\ref{M:8}       &                  &                   &                  &     &     &     &     &     &     &     &     &     &     & \Rx & \Rx & \Rx &     &     &     &     &     &     &     &     &     \\ \hline
		\ref{M:9}       &                  &                   &                  &     &     &     &     &     &     &     &     &     &     &     &     &     & \Px &     &     &     &     &     &     &     &     \\ \hline
		\ref{M:10}      &                  &                   &                  &     &     &     &     &     &     &     &     &     &     &     &     &     & \Px & \Px & \Px & \Px &     &     &     &     &     \\ \hline
		\ref{M:11}      &                  &                   &                  &     &     &     &     &     &     &     &     &     &     &     &     &     &     &     &     & \Px & \Px & \Px & \Px &     &     \\ \hline
		\ref{M:12}      & \Rx              & \Rx               & \Rx              & \Rx & \Rx & \Rx & \Rx & \Rx & \Rx & \Rx & \Rx & \Rx & \Rx & \Rx & \Rx & \Rx & \Px & \Px & \Px & \Px & \Px & \Px & \Px & \Px &     \\ \hline
		\ref{M:13}      &                  &                   &                  &     &     &     &     &     &     &     &     &     &     &     &     &     &     &     &     &     &     &     &     &     & \Px \\ \hline
	\end{tabular}
	\label{Cronograma}
\end{table}

\begin{minipage}{\textwidth}
	\fontsize{8}{14}\selectfont
	\centering
	\colorbox{lightblue}{\rule{0pt}{10pt}\rule{10pt}{0pt}} Já realizado \quad
	\colorbox{lightgray}{\rule{0pt}{10pt}\rule{10pt}{0pt}} Pendente \quad
\end{minipage}