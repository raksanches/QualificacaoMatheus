% Introdução

\documentclass[_ArquivoPrincipal.tex]{subfiles}

\begin{document}

%==================================================================================================
\chapter{Metodologia e cronograma} \label{MetodologiaCronograma}
%==================================================================================================

O trabalho será conduzido inicialmente pelo estudo e formulação matemática dos diferentes métodos de modelagem de escoamentos turbulentos e de dinâmica das estruturas utilizando o Método dos Elementos Finitos Posicional, seguida pela discretização desses modelos e implementação computacional para obtenção de soluções numéricas. Esses resultados serão validados através da comparação com resultados consagrados na literatura, assim como aplicação em problemas relevantes na engenharia.

A linguagem de programação que será utilizada é a C++ orientada a objeto, em um sistema operacional Linux, uma vez que se é possível aproveitar diversas funções já desenvolvidas pelo grupo de pesquisa. Serão utilizadas as bibliotecas \textit{Boost} e PETSc, que permite a utilização de processamento em paralelo MPI. As malhas serão geradas a partir do \textit{software} Gmsh \cite{geuzaine2009gmsh}, e os resultados serão apresentados graficamente pela exportação para o \textit{software} Paraview \cite{ahrens2005paraview}.

O cronograma apresentado na Tabela \ref{Cronograma} é planejado tomando-se um período de Novembro de 2022 à Fevereiro de 2024, o qual considera as seguintes etapas para devida execução da pesquisa:

\begin{enumerate}[label=\alph*.]
	\item\label{M:2} Revisão bibliográfica, a qual será realizada ao longo de todo o trabalho, para que o mesmo se mantenha atualizado durante toda sua execução;
	\item\label{M:3} Formulação matemática das técnicas de análise de escoamentos turbulentos e de Interação Fluido-Estrutura;
	\item\label{M:5} Implementação computacional das técnicas formuladas;
	\item\label{M:6} Validação dos resultados obtidos a partir de exemplos presentes na literatura e em problemas de engenharia;
	\item\label{M:7} Otimização das rotinas computacionais;
	\item\label{M:8} Redação da dissertação e elaboração de artigos para revistas.
\end{enumerate}

\begin{table}[H]
	\caption{Cronograma de atividades}
	\fontsize{10}{14}\selectfont
	\centering
	\definecolor{X}{rgb}{0.4,0.5,0.7}
	\newcommand{\CM}{\cellcolor{X}}
	\newcommand{\MR}{\multirow}
	\newcommand{\MC}{\multicolumn}
	\begin{tabular}{c|cccccccc|cccc}
		\hline
		\MR{2}{*}{Item} & \MC{8}{c|}{2023} & \MC{4}{c}{2024}                                                             \\ \cline{2-13}
		                & Mai              & Jun             & Jul & Ago & Set & Out & Nov & Dez & Jan & Fev & Mar & Abr \\ \hline
		\ref{M:2}       & \CM              & \CM             & \CM & \CM & \CM & \CM & \CM & \CM & \CM & \CM & \CM &     \\ \hline
		\ref{M:3}       & \CM              & \CM             &     &     &     &     &     &     &     &                 \\ \hline
		\ref{M:5}       & \CM              & \CM             & \CM & \CM & \CM & \CM & \CM & \CM & \CM &     &     &     \\ \hline
		\ref{M:6}       &                  &                 &     &     &     &     &     & \CM & \CM &     &     &     \\ \hline
		\ref{M:7}       &                  &                 &     &     &     &     &     &     & \CM & \CM &     &     \\ \hline
		\ref{M:8}       &                  &                 &     &     &     &     &     & \CM & \CM & \CM & \CM & \CM \\ \hline
	\end{tabular}
	\label{Cronograma}
\end{table}

\end{document}


