%==================================================================================================
\chapter{Introdução}
%==================================================================================================

% Você precisa apresentar o problema, contextualizar, identificar onde você pretende contribuir e dizer qual vai ser essa contribuição.

É notável o grande avanço tecnológico que a engenharia tem experimentado nas últimas décadas, o que tem permitido a realização de projetos cada vez mais complexos e desafiadores. Nesse contexto, a interação fluido-estrutura (IFE) tem se mostrado como um dos principais desafios a serem superados, uma vez que abrangem uma gama muito grande de problemas reais de engenharia e cuja solução ainda não é muito bem determinada. A interação entre fluidos e estruturas é um fenômeno que ocorre em diversas situações práticas, como em turbinas eólicas, pontes, edifícios de grande altitudes, aeronaves, estruturas \textit{offshore} entre outros. A análise desses problemas é de grande importância, uma vez que a interação entre fluidos e estruturas pode levar a falhas catastróficas, como o colapso de pontes ou edifícios, ou a perda de controle de uma aeronave.

Sendo assim, duas formas práticas são possíveis para se descrever o comportamento de um fluido em escoamento, \ie\ a construção de modelos em escalada reduzida para ensaios em túneis de ventos, canais ou tanques de ensaio, e o uso de modelagens matemáticas. A primeira opção, embora seja bastante utilizada, apresenta limitações, uma vez que demandam grandes investimentos com infraestrutura, assim como a construção de modelos em escala reduzida pode não capturar todos os fenômenos físicos que ocorrem em um escoamento real, além de possuir pouca flexibilidade quanto aos modelos ensaiados. Já a segunda opção, por sua vez, tem se mostrado como uma alternativa viável, pelo fato de ser capaz de contornar os problemas observados na primeira opção.

Contudo o estudo da interação fluido-estrutura por meio de modelagens matemáticas também apresenta desafios, uma vez que a resolução de problemas dessa natureza demanda a solução de sistemas de equações diferenciais parciais (EDP) cuja solução é desconhecida para a maioria dos problemas, e quando conhecida, é restrita a problemas pequenos e com muitas hipóteses simplificadoras, sem muita representatividade em problemas reais. Tais dificuldades são ainda potencializadas quando o problema trata de grandes deslocamentos da estrutura, ou alta convecção do escoamento, o que torna o caráter da solução altamente não-linear. Desse modo, a utilização de métodos numéricos robustos se mostra como uma alternativa viável, uma vez que são capazes de resolver tais sistemas de equações de maneira aproximada mantendo-se, mesmo assim, a boa representatividade de sua solução.

Das diversas formas existentes de se obter soluções aproximadas para o sistema de EDP em problemas de IFE destaca-se o método dos elementos finitos (MEF), o qual parte da subdivisão de domínios contínuos em elementos constituídos por nós, que por sua vez possuem intrinsecamente valores associados às variáveis do problema. Assim, ao invés de procurar uma solução na forma forte do problema, essa abordagem busca soluções na forma fraca (integral), normalmente partindo do método de resíduos ponderados. Com isso o problema resume-se à busca de uma solução de um sistema algébrico de equações.

No contexto da dinâmica dos fluidos computacional (CFD), o MEF tem se mostrado como uma alternativa viável para a resolução de problemas de escoamentos, uma vez que é capaz de lidar com geometrias complexas, além de permitir a aplicação de condições de contorno de maneira simples. No entanto, à medida que o número de Reynolds aumenta, o escoamento passa a apresentar um comportamento mais turbulento, surgindo, portanto, estruturas turbulentas, que são estruturas instáveis formadas desordenadamente, de caráter tridimensional e que podem se manifestar em diversas escalas diferentes. Por consequência, em problemas desse tipo, se torna necessária a adoção de malhas muito refinadas para capturar a ocorrência dessas estruturas, o que ocasiona um elevado custo computacional.

Tal custo torna, portanto, a aplicação do MEF inviável em problemas usuais de engenharia, que em muitos casos exigem respostas mais ágeis para serem utilizadas em projetos. Assim, motivados à reduzir o custo computacional, diversos modelos de turbulência têm sido propostos, como os baseados na decomposição de Reynolds (\RANS\ - RANS), cuja premissa básica está em se obter uma solução média das equações de Navier-Stokes, ou em grandes vórtices (\LES\ - LES), que busca resolver as escalas maiores do escoamento, enquanto as escalas menores são modeladas, obtendo, assim, uma solução transiente com a possibilidade de se utilizar malhas mais grosseiras para as análises, mantendo-se a boa representatividade dos resultados.

Outro problema enfrentado pelo MEF na CFD reside na utilização de elementos clássicos de Galerkin, o qual considera igualmente as parcelas de derivadas tanto à jusante quanto à montante, o que leva à resultados espúrios em problemas de convecção dominante. Sendo assim, uma possibilidade de se contornar esse problema está na aplicação de técnicas de estabilização, como a aplicação do termo SUPG (\SUPG), que, por meio de modificação nas funções ponderadoras, toma uma contribuição maior das derivadas à montante do escoamento, estabilizando a solução na direção das linhas de corrente.

Além disso, os espaços de aproximação para a pressão em problemas de escoamentos incompressíveis também podem ser fonte de instabilidade numérica. Dessa forma, a fim de se obter um sistema definido, é necessária a adoção de elementos mistos que atendam às condições de \LBB\ (LBB), as quais impedem a escolha dos mesmo espaços aproximadores para os campos de velocidades e pressões. Assim, alguns elementos mistos são possíveis, como os elementos Taylor-Hood, que são capazes de atender às condições LBB, e que são amplamente utilizados em problemas de escoamentos incompressíveis. No entanto, buscando uma maior flexibilidade na escolha dos espaços aproximadores, alguns trabalhos têm proposto a utilização de técnicas de estabilização para a pressão, como a técnica PSPG (\PSPG).

Já de maneira mais ampla, é possível partir para métodos de estabilização multiescala, como a técnica de estabilização VMS (\VMS), que busca estabilizar a solução nas diferentes escalas do escoamento. A partir da decomposição dos campos de velocidades e pressões em parcelas de grandes e pequenas escalas, a técnica VMS incorpora tanto os termos SUPG e PSPG quanto termos adicionais da própria formulação.

Sendo assim, visando a identificação de uma formulação eficiente para simular problemas de IFE com escoamentos turbulentos incompressíveis, o presente trabalho propõe o estudo e implementação do modelo de turbulência LES baseado no modelo de viscosidade de Smagorinsky, bem como a formulação estabilizada VMS em um programa de contornos móveis (empregando a descrição ALE), do acoplamento da ferramenta resultante com o programa para análise dinâmica de cascas com grandes deslocamentos, por meio do acoplamento forte tipo bloco-iterativo, e com a finalidade de se estudar o desempenho desses modelos em simulação de problemas numéricos para estudo e verificação das implementações. Os estudos numéricos deverão tomar como referência tanto a solução direta (sem modelo de turbulência) obtida pela ferramenta computacional desenvolvida, bem como resultados disponíveis na literatura.

Com isso, o trabalho busca trazer uma maior compreensão sobre os impactos que a utilização de modelos de turbulência pode ter em problemas de IFE, bem como a influência das técnicas de estabilização na obtenção de soluções numéricas. Além disso, espera-se desenvolver uma ferramenta computacional que seja capaz de alternar entre diferentes tipos de elementos, técnicas de estabilização e modelo de turbulência em problemas de IFE com escoamentos turbulentos incompressíveis.

%==================================================================================================
\section{Objetivos}
%==================================================================================================

Esta proposta tem como objetivo principal o estudo de formulações numéricas e a implementação computacional de modo a se obter ferramentas computacionais eficientes e precisas para a simulação de problemas de interação fluido-estrutura com elevados números de Reynolds, onde possa haver efeitos de turbulência. Dentro desse escopo, alguns objetivos específicos devem ser alcançados:

\begin{itemize}
    \item Estudo das formulações estabilizadas do método dos elementos finitos para escoamentos incompressíveis com contornos móveis, com destaque para as estabilizações SUPG, LSIC, PSPG e VMS;

    \item Estudo da formulação posicional do MEF para análise dinâmica de sólidos e cascas com grandes deslocamentos;

    \item Estudo das técnicas de acoplamento particionado fluido-estrutura com malhas móveis;

    \item Estudo dos diversos modelos de turbulência no contexto do MEF;

    \item Implementação da formulação estabilizada \VMS;

    \item Implementação do modelo \LES;

    \item Acoplamento do código para mecânica dos fluidos com programa para análise não linear de estruturas de cascas;

    \item Estudo comparativo dos modelos implementados através da simulação de exemplos disponíveis na literatura.
\end{itemize}

%==================================================================================================
\section{Justificativa}
%==================================================================================================

Os avanços na área da engenharia colocam em evidência a necessidade cada vez maior de se determinar de forma precisa as variáveis necessárias ao dimensionamento de estruturas, sejam referentes à resistência dos materiais utilizados, ou às solicitações atuantes. Nesse sentido, em diversas ocasiões, os efeitos advindos da interação fluido-estrutura (IFE) são responsáveis por submeter estruturas a esforços consideráveis, o que demanda que esses efeitos sejam adequadamente estudados. Assim, mesmo que diversos trabalhos tenham sido realizados, resultando em grandes avanços na área, alguns desafios se mantêm, como o elevado custo computacional relacionado à simulações de IFE, o que demanda muito tempo pra a obtenção de resultados, inviabilizando sua aplicação em muitos projetos usuais de engenharia.

Outro desafio que se pode destacar é referente aos problemas de acoplamento, uma vez que os modelos constitutivos que regem tanto os sólidos quanto os fluidos são distintos, assim como as variáveis empregadas para descrever o comportamento de cada um. Também verifica-se que o sistema de equações fundamental para o estudo desses problemas, ou não possui solução analítica, ou essa só pode ser obtida para casos simples e com hipóteses bastante simplificadoras, sem muita possibilidade de generalizações.

Dessa forma, os métodos numéricos são vistos como uma alternativa necessária. Dentre os métodos empregados, pode-se destacar o dos Elementos Finitos (MEF), com aplicações tanto para modelar o sólido como para o fluido. Esse método ganha atenção, uma vez que pode se adequar com certa facilidade à problemas de geometria complexa, garantindo ainda facilidade para aplicação de condições de contorno.

Embora no contexto da mecânica dos fluidos, a utilização do MEF segundo o método de Bubnov-Galerkin possa gerar oscilações espúrias, decorrentes dos termos convectivos, de ondas de choque nos casos compressíveis, ou da interpolação da pressão por espaços de funções inadequados nos casos incompressíveis, todos esses problemas já dispõem de técnicas consistentes para que sejam evitados. Como exemplos, destacam-se a técnica SUPG para estabilização da convecção, a técnica PSPG ou o uso de elementos Taylor-Hood para estabilização da pressão nos escoamentos incompressíveis e a introdução de operadores de captura de choque para os casos compressíveis.

Outra questão envolvendo a utilização do MEF nesse tipo de problema está relacionado ao custo computacional, que também se mostra como um desafio enfrentado por pesquisadores dessa área. Nesse cenário, problemas de escoamento turbulentos se tornam ainda mais custosos, devido à alguns fenômenos como a manifestação de vórtices, geralmente tridimensionais, de forma desordenada, instável e em uma grande amplitude de escalas. Assim necessita-se de malhas muito refinadas para capturar a ocorrência dessas estruturas. Para que a simulação desses problemas seja estável em qualquer nível de discretização, e possa resultar em respostas consistentes a custos computacionais aceitáveis, faz-se necessária a adoção de modelos de turbulência, como  aqueles baseados na decomposição de Reynolds (RANS) ou em grandes vórtices (LES), ou de técnicas multiescala adequadas.

Sendo assim, o presente trabalho é justificado ao propor o desenvolvimento de uma ferramenta computacional eficiente e precisa, aplicada a problemas de IFE com escoamentos turbulentos com a opção de se aplicar modelo de turbulência. Igualmente, fica justificado a proposta de estudar e comparar os diferentes modelos e estabilizações a serem implementadas quando da aplicação a problemas de IFE. A presente pesquisa visa ainda aprimorar as ferramentas computacionais que estão sendo desenvolvidas pelo grupo de pesquisa do SET, ampliando seu leque de aplicações.