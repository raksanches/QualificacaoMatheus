%==================================================================================================
\chapter{Introdução}
%==================================================================================================

%Sugestões gerais: Deixar menos revisão bibliográfica (usar menos referências aqui...) e introduzir melhor o conceito de escoamento turbulento e os modelos de turbulência bem como sua importância na análise de problemas de IFE.

Visto o constante avanço da engenharia, com o dimensionamento de estruturas cada vez mais leves e esbeltas, observa-se a necessidade de uma determinação cada vez mais precisa de todos as variáveis que podem influenciar na resistência e estabilidade da estrutura, bem como dos carregamentos atuantes. Nesse contexto, torna-se necessária a verificação precisa dos efeitos da interação entre as estruturas e os fluidos com elas em contato, uma vez que os fenômenos decorrentes dessa interação podem comprometer a segurança e o desempenho estrutural. Tais fenômenos possuem alta complexidade em sua descrição, uma vez que envolvem dois meios com constituição muito diferente, demandando diferentes suposições e descrições matemáticas. Dessa forma, a modelagem numérica dos problemas de interação fluido-estrutura (IFE) deve ser conduzida de forma que as condições de acoplamento sejam devidamente atendidas, levando em consideração as diferenças constitutivas e nas formulações individuais para sólido e fluido.

Como exemplos de problemas de IFE, destacam-se as ações do vento sobre estruturas de edifícios altos, a ação da água sobre barragens e estruturas \textit{offshore}, dentre outros. Além disso, percebe-se a presença desses efeitos em diversas outras áreas, como por exemplo, no escoamento do sangue em vasos sanguíneos, ou em problemas aeronáuticos \cite{sanches2014fluid, fernandes2020tecnica}.

\textcolor{red}{Uma modelagem matemática não é nada prática, muito menos viável, já que não você não vai conseguir achar solução! Reformule os parágrafos abaixo e me avise para que eu revise novamente.}

Duas formas práticas de se estudar os efeitos de IFE em uma dada estrutura são: o estudo experimental de modelos em escala real ou reduzida em túneis de vento, canais ou tanques de ensaio; ou a modelagem matemática do problema em questão. No primeiro caso há a verificação real do comportamento da estrutura e do fluido nas condições ensaiadas. No entanto, além de ficar restrito ao caso analisado, há, em diversos casos, dificuldade para a realização de ensaios em escala reduzida que sejam representativos do problema em escala real. Também, ressalta-se o fato da análise experimental demandar custos elevados e uma infraestrutura robusta para se analisar apenas alguns problemas específicos \cite{fernandes2020tecnica}.

Já a modelagem matemática desses problemas, se mostra uma opção mais viável, uma vez que dispensa grandes investimentos com material, espaço e equipamentos de ensaios, e possui grande flexibilidade de aplicações, podendo ser utilizado em diversos tipos de análises. Todavia, os modelos matemáticos conduzem a sistemas de equações que, ou não apresentam solução analítica, ou apresentam apenas para casos simples e com uso de hipóteses simplificadoras que deixam a solução menos geral, demandando assim soluções numéricas. Devido à complexidade e à não-linearidade de problemas da Dinâmica dos Sólidos Computacional (CSD) ligados à grandes deslocamentos e da Dinâmica dos Fluidos Computacional (CFD) ligados à problema de convecção dominante, além da natureza não linear do problema acoplado, o estudo matemático de IFE apresenta desafios às formulações numéricas, em muitos casos com custo computacional muito elevado, havendo ainda, a despeito do número de trabalhos que já foram desenvolvidos nessa área, demanda por formulações ainda mais robustas, precisas e eficientes.

Para se descrever matematicamente tanto a mecânica dos sólidos como a mecânica dos fluidos, é necessário o estabelecimento de hipóteses simplificadoras que reduzam as incógnitas envolvidas. Uma das principais hipóteses utilizada para esses problemas é a que considera os objetos de análise como meios contínuos, desprezando-se, dessa forma, os efeitos advindos da microestrutura da matéria \cite{lai2009introduction, mase2009continuum}. Assim, assume-se que o material pode ser subdividido em elementos cada vez menores sem que haja descontinuidades, independentemente do quão pequena seja essa divisão. Dessa forma, faz-se possível a modelagem de elementos com volume infinitesimal, permitindo a utilização de artifícios matemáticos, como o cálculo diferencial e integral. Portanto, podem ser empregadas funções contínuas sobre os domínios, facilitando a determinação das variáveis de interesse da análise \cite{irgens2008continuum, lai2009introduction, malvern1969introduction}.

Vale ressaltar que a consideração dos meios contínuos apresenta resultados muito satisfatórios nos estudos de engenharia, uma vez que os objetos de estudo possuem dimensões muito maiores que as distâncias moleculares do mesmo \cite{malvern1969introduction, mase2009continuum}.

Ainda assim, a complexidade das equações diferenciais que governam os problemas mecânicos, demanda métodos numéricos que possam reduzir as infinitas incógnitas de um meio contínuo a um número discreto. Dentre os métodos disponíveis para isso, destaca-se o Método dos Elementos Finitos (MEF), sendo atualmente o método mais empregado para a mecânica dos sólidos, e que tem ganhado cada vez mais espaço também na mecânica dos fluidos. Algumas características verificadas nesse método, que conferem eficiência, são a possibilidade de se modelar problemas de geometria complexa, aplicar condições de contorno de maneira facilitada, assim como poder trabalhar com diferentes materiais em uma mesma simulação \cite{anderson1995computational} \textcolor{red}{O Anderson não fala nada sobre elementos finitos! É um livro sobre diferenças finitas! Arrume essa citação.}.

Ainda existem diferentes formas de descrever matematicamente um problema mecânico em termos da referência adotada. Tradicionalmente são empregadas a descrição Lagrangiana (ou material) e a descrição Euleriana (ou espacial).

Na descrição Lagrangiana a referência adotada é a configuração inicial do contínuo. Essa descrição apresenta bom desempenho quando aplicada à mecânica dos sólidos, uma vez que a configuração inicial é bem definida e suas deformações são finitas, tendo como variáveis principais os deslocamentos do sólido \cite{sanches2014fluid, fernandes2019ale}. Já a descrição Euleriana toma como referência a configuração atual do contínuo, sendo muito utilizada para a descrição de fluidos newtonianos, pois esses não apresentam resistência às tensões de cisalhamento, distorcendo-se indefinidamente quando submetidos a tais solicitações \cite{sanches2014fluid, fernandes2019ale}.

Assim, é importante que a análise de interação fluido estrutura combine adequadamente a mecânica dos sólidos, convenientemente descrita na forma Lagrangiana, com a mecânica dos fluidos, convenientemente descrita na forma Euleriana. Uma forma robusta de se fazer isso é empregar a descrição Lagrangiana-Euleriana Arbitrária (ALE)  \cite{donea1982arbitrary} para o fluido. Essa descrição considera um domínio de referência com movimento arbitrário e independente das partículas. Isso permite que a malha do fluido seja movimentada dinamicamente para acomodar a movimentação da interface fluido-sólido.

No entanto, a aplicação do MEF a problemas de dinâmica dos fluidos, assumindo o método clássico de Galerkin, o qual considera igualmente as parcelas de derivadas tanto à jusante quanto à montante, leva à obtenção de resultados espúrios em problemas de convecção dominante. Sendo assim, uma possibilidade de se contornar esse problema é a aplicação do método SUPG (\textit{Streamline-Upwind/Petrov-Galerkin}), que, por meio de modificação nas funções ponderadoras, toma uma contribuição maior das derivadas à montante do escoamento, estabilizando a solução na direção das linhas de corrente.

% Outra fonte de instabilidade a ser observada é referente à ocorrência de descontinuidades nos escoamentos compressíveis, conhecidas como ondas de choque. Nesse cenário, empregam-se operadores de captura de choque para estabilizar o problema, sendo um processo muito comum constituído da adição de viscosidade artificial na região da descontinuidade, o qual pode ser determinado em função da derivada segunda da pressão, possuindo valor máximo no ponto de descontinuidade e sendo anulado conforme o ponto analisado se afasta da descontinuidade.

No caso de escoamentos incompressíveis, no contexto do MEF, há a necessidade de se empregar elementos mistos, tendo a pressão como grau de liberdade, além das velocidades. Isso termina por conduzir a um sistema algébrico do tipo ponto de sela com sub-blocos nulos na diagonal principal na matriz do problema. Dessa forma, a fim de se obter um sistema definido, observa-se a necessidade de se atender as condições de compatibilidade de \LBB\ (LBB) no momento da escolha dos espaços de aproximação de velocidade e de pressão. Contudo, tais condições restringem consideravelmente a escolha dos espaços de aproximação, além de possuírem como consequência a impossibilidade de se aproximar os campos de velocidades e pressões em um mesmo espaço. Sendo assim, objetivando uma flexibilização maior na escolha desses espaços surgem métodos estabilizados, como o \textit{Pressure-Stabilizing/Petrov-Galerkin} (PSPG), garantindo um campo de pressões estável.

Outra possibilidade de se obter soluções estáveis para problemas de escoamentos incompressíveis está na utilização de modelos de estabilização multiescala, como o \VMS\ (VMS), o qual considera a decomposição dos campos de velocidade e pressão em escalas macroscópicas e microscópicas, sendo que a escala macroscópica é resolvida diretamente pelo MEF, enquanto a escala microscópica é modelada. Nesse contexto, o VMS se mostra capaz de estabilizar não somente problemas de convecção dominante, como dispensa a satisfação das condições de LBB, uma vez que insere um termo estabilizador SUPS, que abrange tanto a estabilização SUPG, quanto a PSPG, além de um termo adicional de estabilização LSIC (\LSIC).

Já em relação à classificação dos escoamentos, estes podem ser classificados em função do número de Reynolds, o qual é um coeficiente adimensional definido como a razão entre as forças inerciais e as forças viscosas. Assim, os escoamentos podem ser classificados, a depender desse coeficiente, em 3 tipos: escoamento laminar; de transição; e turbulento. No primeiro caso, o escoamento se dá de forma similar ao escorregamento de lâminas paralelas entre si, sem que haja uma mistura macroscópica entre elas. Já no segundo caso começam a surgir algumas flutuações esporádicas no escoamento, porém ainda não se apresentam de forma tão significativa a considerá-lo turbulento. O último caso apresenta flutuações constantes em seu fluxo, resultando no surgimento de estruturas denominadas de vórtices. Tais estruturas se manifestam tridimensionalmente de maneira instável, desordenada e em diversas escalas diferentes. Nota-se que os escoamentos laminares ocorrem com baixos valores de número de Reynolds, enquanto os escoamentos turbulentos ocorrem com elevados valores de número de Reynolds. Isso faz com que a obtenção de soluções numéricas desse tipo de fluxo seja uma tarefa muito custosa computacionalmente, uma vez que, para capturar a formação de vórtices até mesmo nas menores escalas, é necessária uma malha muito refinada, assim como algumas simulações possuem um domínio computacional consideravelmente grande.

Entretanto os escoamentos turbulentos representam uma parcela considerável daqueles presentes em problemas de engenharia, sendo o seu comportamento determinado pelas equações de Navier-Stokes. No entanto, a aplicação direta dos métodos numéricos às equações governantes para a solução de problemas com escoamento turbulento demanda uma discretização muito refinada, dadas as escalas em que podem se dar a formação de vórtices.

%Descrever o problema de turbulência - o que é e que equações governam esses escoamentos.
%Descrever o desafio de simular problemas com escoamentos turbulentos no MEF
%Por que é desafiadora? Quais são os desafios? O que acontece em uma simulação quando há presença de turbulência?.

% Assim, isso tem sido tema de diversos pesquisadores que propuseram modelos para os problemas decorrentes da turbulência % Há trabalhos que classificam os modelos de turbulência em grupos. Tente encontrar e encaixar isso aqui.

No intuito de se obter uma melhor eficiência computacional, com a possibilidade de se capturar efeitos turbulentos mesmo com malhas menos refinadas e maiores intervalos discretos de tempos, surgem os modelos de turbulência, os quais geralmente se baseiam na decomposição de Reynolds (\RANS\ - RANS), ou na simulação de grandes vórtices (\LES\ - LES).

Os modelos de turbulência podem ser classificados em três grupos: $i$ - modelos mais simples, que empregam expressões algébricas para descrever a viscosidade turbulenta, assumindo que as estruturas turbulentas são geradas e dissipadas no mesmo local; $ii$ - modelos que adicionam equações diferenciais adicionais para o cálculo da viscosidade turbulenta; e $iii$ - modelos que adicionam equações diferenciais de transporte para o cálculo das tensões de Reynolds \cite{souza2011revisao,alfonsi2009reynolds,teixeira2001simulaccao}.

% No caso dos modelos RANS, as variáveis das equações de Navier-Stokes são decompostas em parcelas referentes à sua média e às suas flutuações, sendo que a tomada da média pode ser feita de diferentes formas a depender da natureza do escoamento. Os modelos RANS são mais comumente classificados em função da quantidade de equações diferenciais de transporte adicionadas ao problema, como os problemas de zero equações, uma equação, duas equações (por exemplo os modelos $\ke$ e $\kw$) e os modelos de tensões ($\te$). Nesse tipo de simulação observa-se a introdução de diversos novos termos e grandezas com as equações adicionais, tal como o tensor de Reynolds e os termos presentes nas equações de transporte escritos em função de campos de flutuação, os quais inserem constantes definidas empiricamente ao problema. Ainda assim relata-se que a utilização desse método possui custo computacional reduzido \cite{katopodes2019free}.

No caso dos modelos LES a decomposição das variáveis é feita a partir de um filtro que separa as grandes estruturas turbulentas das pequenas. A suposição básica desses modelos é que as grandes escalas são responsáveis pelo transporte da energia e da quantidade de movimento, sendo resolvidas diretamente pelas equações de Navier-Stokes filtradas, enquanto as escalas pequenas (\textit{Sub-Grid Scales} - SGS) possuem comportamento isotrópico que, apesar de não poderem ser resolvidas diretamente, sua influência na escala maior pode ser modelada. Dentre as propostas de se modelar as pequenas escalas, é tradicionalmente empregado o modelo de Smagorinsky, o qual relaciona a viscosidade turbulenta com o campo de velocidades em grandes escalas.
%Nesse cenário observa-se que o modelo LES introduz uma quantidade menor novos termos e grandezas (como o tensor SGS, modelado pelo tensor de Smagorinsky) em comparação com o RANS, no entanto é relatado que seu custo computacional é maior.

%Uma possível maneira de se obter resultados é pela simulação de \textit{Direct Numerical Simulation} (DNS), a qual é capaz de solucionar as equações de Navier-Stokes sem a utilização de modelos, descrevendo o escoamento em todas as escalas presentes no mesmo. Contudo, seu custo computacional é extremamente elevado, tornando sua utilização muito restrita a problemas de pequenas dimensões e para verificação de modelos \cite{olad2022towards}.

%Já para simulações mais representativas, se faz necessário o uso de modelos que descrevam adequadamente o comportamento dos escoamentos turbulentos. Dentre os possíveis modelos, destacam-se os modelos de \RANS\ (RANS), \LES\ (LES) e \VMS\ (VMS). Esses modelos são caracterizados por considerar uma separação dos parâmetros de análise em duas parcelas, onde, dependendo de como essa separação é feita, pode ocorrer aprimoramentos no tempo de processamento ou na qualidade dos resultados.

%Atenção: O VMS não é um modelo de turbulência, mas sim um modelo de estabilização multiescala... Você precisa conseguir definir muito bem isso...

Assim, visando a identificação de uma formulação eficiente para simular de problemas de interação fluido-estrutura com escoamentos turbulentos, esta proposta trata da implementação do modelo de turbulência \LES\ (LES) baseado no modelo de viscosidade de Smagorinsky, bem como da formulação \VMS\ (VMS) em um programa para escoamentos incompressíveis com contornos móveis (empregando descrição ALE), do acoplamento da ferramenta resultante com um programa para análise dinâmica de cascas com grandes deslocamentos, e com a finalidade de se estudar o desempenho desses modelos em simulação de problemas numéricos para estudo e verificação das implementações. Os estudos numéricos deverão tomar como referência tanto a solução direta (sem modelo de turbulência) obtida pela ferramenta computacional desenvolvida, bem como resultados disponíveis na literatura.

%Explicar melhor os modelos RANS e LES

%==================================================================================================
\section{Objetivos}
%==================================================================================================

Esta proposta tem como objetivo principal o estudo de formulações numéricas e a implementação computacional de modo a se obter ferramentas computacionais eficientes e precisas para a simulação de problemas de interação fluido-estrutura com elevados números de Reynolds, onde possa haver efeitos de turbulência. Dentro desse escopo, alguns objetivos específicos devem ser alcançados:

\begin{itemize}
    \item Estudo das formulações estabilizadas do método dos elementos finitos para escoamentos incompressíveis com contornos móveis, com destaque para as estabilizações SUPG, LSIC, PSPG e VMS;

    \item Estudo da formulação posicional do MEF para análise dinâmica de sólidos e cascas com grandes deslocamentos;

    \item Estudo das técnicas de acoplamento particionado fluido-estrutura com malhas móveis;

    \item Estudo dos diversos modelos de turbulência no contexto do MEF;

    \item Implementação da formulação estabilizada \VMS;

    \item Implementação do modelo \LES;

    \item Acoplamento do código para mecânica dos fluidos com programa para análise não linear de estruturas de cascas;

    \item Estudo comparativo dos modelos implementados através da simulação de exemplos disponíveis na literatura.
\end{itemize}

%==================================================================================================
\section{Justificativa}
%==================================================================================================


Os avanços na área da engenharia colocam em evidência a necessidade de se determinar, com precisão cada vez maior, as variáveis necessárias ao dimensionamento de estruturas, sejam referentes à resistência dos materiais utilizados, ou às solicitações atuantes. Em diversas ocasiões, os efeitos advindos da interação fluido-estrutura são responsáveis por submeter estruturas a esforços consideráveis, o que demanda que esses efeitos sejam adequadamente estudados durante o projeto. No entanto, tais estudos são muito desafiadores, a começar com os problemas de acoplamento, uma vez que os modelos constitutivos que regem tanto os sólidos quanto os fluidos são distintos, assim como as variáveis empregadas para descrever o comportamento de cada um. Também verifica-se que o sistema de equações fundamental para o estudo desses problemas, ou não possui solução analítica, ou se possui, essa só pode ser obtida para casos simples e com hipóteses bastante simplificadoras, sem muita possibilidade de generalizações.

Dessa forma, os métodos numéricos possuem grande demanda nesse contexto, e, embora já hajam muitos trabalhos importantes que trazem consideráveis avanços nessa área, ainda há obstáculos importantes, como o elevado custo computacional relacionado a simulações de IFE, demandando muito tempo de processamento pra a obtenção de resultados e inviabilizando sua aplicação em muitos projetos usuais de engenharia.

Os problemas de escoamento turbulentos se fazem muito presentes nos casos de ações do vento sobre estruturas, incluindo fenômenos como a manifestação de vórtices, geralmente tridimensionais, que podem ocorrer de forma desordenada, instável e em uma grande amplitude de escalas. Isso implica na necessidade de uma discretização espacial muito refinada para capturar a ocorrência dessas estruturas, que por sua vez, aumenta ainda mais o custo computacional. Assim, tornam-se muito importantes a adoção de técnicas que permitam a a simulação desses problemas de maneira estável em qualquer nível de discretização, resultando em respostas consistentes a custos computacionais aceitáveis, onde destacam-se os chamados  modelos de turbulência, como  aqueles baseados na decomposição de Reynolds (RANS) ou em grandes vórtices (LES), e também as técnicas baseadas em análise multiescala.

Sendo assim, o presente trabalho é justificado ao propor o desenvolvimento de uma ferramenta computacional eficiente e precisa, aplicada a problemas de IFE com escoamentos em elevados números de Reynolds com a opção de se aplicar modelo de turbulência. Igualmente, fica justificado a proposta de estudar e comparar os diferentes modelos e estabilizações a serem implementadas quando da aplicação a problemas de IFE. A presente pesquisa visa ainda aprimorar as ferramentas computacionais que estão sendo desenvolvidas pelo grupo de pesquisa do SET, ampliando seu leque de aplicações.





