%==================================================================================================
\chapter{Introdução}
%==================================================================================================

Visto o constante avanço da engenharia em poder se dimensionar estruturas cada vez mais leves e esbeltas, observa-se a necessidade de uma determinação cada vez mais precisa de todos os parâmetros que podem influenciar na resistência e estabilidade da estrutura. Nesse contexto torna-se de grande importância a verificação dos efeitos da interação entre os fluidos e a estrutura, uma vez que esses fenômenos podem comprometer a segurança estrutural. Exemplos de problemas envolvendo Interação Fluido-Estrutura (IFE) podem ser notados em ações do vento sobre estruturas de grandes altitudes, ação de marés sobre estruturas de barragens e estruturas \textit{offshore}, dentre outros. Além disso, também percebe-se a aplicação desse tipo de análise em demais áreas, como, por exemplo, no estudo de escoamento do sangue em vasos sanguíneos, ou em problemas envolvendo aerodinâmica \cite{sanches2014fluid, fernandes2020tecnica}.

Desse modo, é possível realizar a análise de IFE de duas formas principais: a construção de amostras em escala; ou a modelagem matemática do problema em questão. No primeiro caso há a verificação real do comportamento da estrutura e do fluido à um determinado problema. No entanto, dependendo da natureza do problema, esse comportamento pode variar de acordo com a escala, o que pode não apresentar resultados representativos. Também ressalta-se o fato de a construção das amostras demandar uma infraestrutura robusta para se analisar apenas alguns problemas específicos \cite{fernandes2020tecnica}.

Já a modelagem matemática de problemas se mostra como uma opção mais viável para análise de problemas envolvendo IFE, uma vez que dispensa grandes investimentos e possui grande flexibilidade de aplicações, podendo ser utilizado em diversos tipos de análises. Todavia, devido à alta não-linearidade de problemas da Dinâmica dos Sólidos Computacional (CSD) ligados à grandes deslocamentos e da Dinâmica dos Fluidos Computacional (CFD) ligados à problema de convecção dominante, o estudo matemático de IFE leva a modelagem de problemas muito complexos, com alto custo computacional, exigindo, assim, formulações mais robustas para sua aplicação.

Para se realizar o estudo, tanto de CSD quanto CFD, é necessário o estabelecimento de algumas simplificações. Uma das principais simplificações utilizada para esses problemas é a que considera os objetos de análise como meios contínuos, desprezando-se, dessa forma, os efeitos advindos da microestrutura da matéria \cite{lai2009introduction, mase2009continuum}.

Sabe-se que toda a matéria é composta por átomos e partículas subatômicas, entretanto a mecânica do contínuo desconsidera os efeitos dessas partículas e aponta que todo material pode ser subdividido em elementos cada vez menores sem que haja mudanças em suas propriedades físicas, independentemente do quão pequena seja essa divisão. Assim, faz-se possível a modelagem de elementos com volume infinitesimal, permitindo a utilização de artifícios matemáticos, como o cálculo diferencial e integral. Dessa forma, teorias baseadas na elasticidade e plasticidade podem ser empregadas utilizando funções contínuas, facilitando a determinação de parâmetros de interesse da análise \cite{irgens2008continuum, lai2009introduction, malvern1969introduction}.

Vale ressaltar que a consideração dos meios contínuos apresenta resultados muito satisfatórios nos estudos de engenharia, uma vez que os objetos de estudo possuem dimensões muito maiores que as distâncias moleculares do mesmo \cite{malvern1969introduction, mase2009continuum}.

Nesse sentido, ainda necessita-se de um método que seja capaz de descrever adequadamente o comportamento de estruturas, sendo capaz de prever seus parâmetros até em condições de grandes deslocamentos. Com isso nota-se a presença do Método dos Elementos Finitos (MEF), que se apresenta como uma ferramenta eficiente na determinação de tais parâmetros, em especial o MEF baseado em posições, o qual contorna alguns problemas que são observados em outros métodos, como aquele baseado em deslocamentos e rotações, por exemplo, que possui a dificuldade da falta de comutatividade de rotações.

Ainda existem diferentes formas de descrever matematicamente um problema mecânico, em termos da referência adotada para os cálculos. São estas: a descrição Lagrangiana (ou material) e a descrição Euleriana (ou espacial).

Na descrição Lagrangiana a referência adotada é a configuração inicial do contínuo. Essa descrição apresenta boa representatividade quando aplicada em modelos sólidos, uma vez que a configuração inicial dos elementos é bem definida, tendo como variáveis principais os deslocamentos do elemento \cite{sanches2014fluid, fernandes2019ale}.

Já a descrição Euleriana toma como referência a configuração atual do contínuo, sendo bem utilizada para a descrição de fluidos newtonianos, pois estes não apresentam resistência à esforços de cisalhamento, distorcendo-se indefinidamente quando submetidos à tais esforços \cite{sanches2014fluid, fernandes2019ale}.

Porém, uma nova descrição vêm se mostrado promissora para representação de simulações de IFE, denominada como Lagrangiana-Euleriana Arbitrária (ALE), a qual realiza a determinação dos parâmetros em função de um referencial arbitrário, que pode se mover independentemente das configurações do domínio analisado \cite{donea1982arbitrary}. Isso torna-se interessante, uma vez que facilita o processo de acoplamento entre os diferentes meios analisados (sólido e fluido).

Outra dificuldade verificada em problemas de CFD está relacionada à escoamentos com alto número de Reynolds, o que o classifica como um escoamento turbulento. Esse tipo de escoamento é muito comum na natureza, principalmente em problemas envolvendo engenharia, e é caracterizado pela formação de vórtices, que são estruturas formadas de maneira instável, desordenadas e em diversas escalas diferentes. A determinação dos parâmetros desse escoamento pode ser dado através das Equações de Navier-Stokes, entretanto a obtenção de resultados a partir dessas equações pode ser uma tarefa muito desafiadora e requer diferentes estratégias para esse fim.

Uma possível maneira de se obter resultados é pela simulação de \textit{Direct Numerical Simulation} (DNS), a qual é capaz de solucionar as equações de Navier-Stokes sem a utilização de modelos, descrevendo o escoamento em todas as escalas presentes no mesmo. Contudo, seu custo computacional é extremamente elevado, tornando sua utilização muito restrita a problemas de pequenas dimensões e para validação de modelos \cite{olad2022towards}.

Já para simulações mais representativas, se faz necessário o uso de modelos que descrevam adequadamente o comportamento dos escoamentos turbulentos. Dentre os possíveis modelos, destacam-se os modelos de \RANS\ (RANS), \LES\ (LES) e \VMS\ (VMS). Esses modelos são caracterizados por considerar uma separação dos parâmetros de análise em duas parcelas, onde, dependendo de como essa separação é feita, pode ocorrer aprimoramentos no tempo de processamento ou na qualidade dos resultados.

Nesse sentido, o presente trabalho busca avaliar os três modelos de turbulência em problemas de IFE, tendo em vista seu tempo de processamento e precisão dos resultados obtidos, os quais serão comparados com exemplos de alta fidelidade apresentados na literatura.

%==================================================================================================
\section{Objetivos}
%==================================================================================================

O presente trabalho visa o estudo e implementação na linguagem de programação C++ de modelos de turbulência para análise de problemas de interação fluido-estrutura com elevados números de Reynolds. Com isso, traça-se os seguintes objetivos específicos:

\begin{itemize}
    \item Estudo e implementação de técnicas de \RANS;

    \item Estudo e implementação de técnicas de \LES;

    \item Estudo e implementação de técnicas de \VMS;

    \item Estudo comparativo das características numéricas dos métodos \RANS, \LES, \VMS\ em problemas de Interação Fluido-Estrutura.
\end{itemize}

%==================================================================================================
\section{Justificativa}
%==================================================================================================

Observa-se os efeitos da interação fluido-estrutura em diversas estruturas, em especial os efeitos devido à escoamentos turbulentos em estruturas flexíveis, por exemplo, a atuação de ventos em edifícios. Nesse sentido, se faz necessário o estudo de técnicas eficazes para determinação dos impactos desses escoamentos sobre as mais diversas estruturas.

Duas formas possíveis de se obter esses resultados são: pela construção de modelos reais em escala, observando-se na prática o comportamento estrutural e do escoamento; ou pela modelagem matemática do problema. O primeiro cenário apresenta algumas desvantagens, pois os parâmetros que são obtidos nesse tipo de análise podem ser dependentes da escala da amostra, impactando diretamente na qualidade dos resultados, além de demandar uma infraestrutura muito robusta para se obter valores coerentes. Assim, o modelo matemático se mostra como uma solução mais eficiente, uma vez que não demanda um grande investimento em infraestrutura.

No entanto muitos tipos de análise envolvendo escoamentos turbulentos recaem em problema com um alto grau de complexidade, ocasionando, consequentemente, um custo computacional muito alto. Dessa maneira, o presente trabalho busca analisar entre as técnicas matemáticas qual se mostra mais eficiente para essas análises, dando continuidade aos trabalhos que já foram realizados pelo grupo de pesquisa.

Além dos métodos para determinação de parâmetros referentes ao escoamento, também é necessário se obter aqueles referentes à resposta da estrutura frente à esse escoamento. Assim, nota-se a presença de diversos métodos para a determinação de parâmetros relativos à estruturas flexíveis, em que se percebe o Método dos Elementos Finitos Posicional como uma alternativa interessante para tal análise. Esse método se destaca pela consideração das posições nodais como parâmetros de análise, evitando problemas devido à falta de comutatividade de rotações, que é observado, por exemplo, no Método dos Elementos Finitos Corrotacional, que, além de deslocamentos, também considera rotações nodais como parâmetros de análise. Além disso, também vale destacar que a matriz de massa em problemas dinâmicos permanece constante ao longo do tempo, possibilitando a implementação facilitada de métodos de integração temporal.