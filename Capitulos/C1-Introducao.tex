%==================================================================================================
\chapter{Introdução}
%==================================================================================================

%Sugestões gerais: Deixar menos revisão bibliográfica (usar menos referências aqui...) e introduzir melhor o conceito de escoamento turbulento e os modelos de turbulência bem como sua importância na análise de problemas de IFE.

%introduzir problemas de IFE e importância da IFE computacional
Visto o constante avanço da engenharia em poder se dimensionar estruturas cada vez mais leves e esbeltas, observa-se a necessidade de uma determinação cada vez mais precisa de todos os parâmetros que podem influenciar na resistência e estabilidade da estrutura, bem como os carregamentos atuantes. Nesse contexto, torna-se necessária a verificação precisa dos efeitos da interação entre as estruturas e os fluidos com elas em contato, uma vez que os fenômenos decorrentes dessa interação podem comprometer a segurança e o desempenho estrutural. Tais fenômenos possuem são complexos em sua descrição, uma vez que envolvem dois meios com constituição muito diferente, demandando diferentes suposições e descrições matemáticas. Dessa forma, a modelagem numérica dos problemas de IFE deve ser conduzida de forma que ambos os meios sejam devidamente acoplados.

Exemplos de problemas envolvendo interação fluido-estrutura (IFE) podem ser notados em ações do vento sobre estruturas de edifícios altos, ação da água sobre barragens e estruturas \textit{offshore}, dentre outros. Além disso, percebe-se a presença desses efeitos em diversas outras áreas, como por exemplo, no escoamento do sangue em vasos sanguíneos, ou em problemas aeronáuticos \cite{sanches2014fluid, fernandes2020tecnica}.

Duas formas práticas de se estudar os efeitos de IFE em uma dada estrutura são: o estudo experimental de modelos em escala real ou reduzida em túneis de vento, canais ou tanques de ensaio; ou a modelagem matemática do problema em questão. No primeiro caso há a verificação real do comportamento da estrutura e do fluido nas condições ensaiadas. No entanto, além de ficar restrito ao caso analisado, há, em diversos casos, dificuldade para a realização de ensaios em escala reduzida que sejam representativos do problema em escala real. Também, ressalta-se o fato da análise experimental demandar custos elevados e uma infraestrutura robusta para se analisar apenas alguns problemas específicos \cite{fernandes2020tecnica}.

Já a modelagem matemática desses problemas, se mostra uma opção mais viável, uma vez que dispensa grandes investimentos com material, espaço e equipamentos de ensaios, e possui grande flexibilidade de aplicações, podendo ser utilizado em diversos tipos de análises. Todavia, os modelos matemáticos conduzem a sistemas de equações que, ou não apresentam solução analítica, ou apresentam apenas para casos simples e com uso de hipóteses simplificadoras que deixam a solução menos geral, demandando assim soluções numéricas. Devido à complexidade e à não-linearidade de problemas da Dinâmica dos Sólidos Computacional (CSD) ligados à grandes deslocamentos e da Dinâmica dos Fluidos Computacional (CFD) ligados à problema de convecção dominante, além da natureza não linear do problema acoplado, o estudo matemático de IFE apresenta desafios às formulações numéricas, em muitos casos com custo computacional muito elevado, havendo ainda, a despeito do número de trabalhos que já foram desenvolvidos nessa área, demanda por formulações ainda mais robustas, precisas e eficientes.

%Descrever os métodos computacionais aplicados à IFE chegando no Método dos elementos finitos
Para se descrever matematicamente tanto a mecânica dos sólidos como a mecânica dos fluidos, é necessário o estabelecimento de hipóteses simplificadoras que reduzam as incógnitas envolvidas. Uma das principais hipóteses utilizada para esses problemas é a que considera os objetos de análise como meios contínuos, desprezando-se, dessa forma, os efeitos advindos da microestrutura da matéria \cite{lai2009introduction, mase2009continuum}. Assim, assume-se que o material pode ser subdividido em elementos cada vez menores sem que hajam descontinuidades, independentemente do quão pequena seja essa divisão. Assim, faz-se possível a modelagem de elementos com volume infinitesimal, permitindo a utilização de artifícios matemáticos, como o cálculo diferencial e integral. Dessa forma, podem ser empregadas funções contínuas sobre os domínios, facilitando a determinação de parâmetros de interesse da análise \cite{irgens2008continuum, lai2009introduction, malvern1969introduction}.

Vale ressaltar que a consideração dos meios contínuos apresenta resultados muito satisfatórios nos estudos de engenharia, uma vez que os objetos de estudo possuem dimensões muito maiores que as distâncias moleculares do mesmo \cite{malvern1969introduction, mase2009continuum}.

Ainda assim, a complexidade das equações diferenciais que governam os problemas mecânicos, demandam métodos numéricos que possam reduzir as infinitas incógnitas de um meio contínuo a um número discreto. Dentre os métodos disponíveis para isso, destaca-se o Método dos Elementos Finitos (MEF), sendo atualmente o método mais empregado para a mecânica dos sólidos, e que tem ganhado cada vez mais espaço também na mecânica dos fluidos. Algumas características verificadas nesse método, que conferem eficiência, são a possibilidade de se modelar problemas de geometria complexa, aplicar condições de contorno de maneira facilitada, assim como poder trabalhar com diferentes materiais em uma mesma simulação.

Ainda existem diferentes formas de descrever matematicamente um problema mecânico em termos da referência adotada. Tradicionalmente são empregadas a descrição Lagrangiana (ou material) e a descrição Euleriana (ou espacial).

Na descrição Lagrangiana a referência adotada é a configuração inicial do contínuo. Essa descrição apresenta bom desempenho quando aplicada à mecânica dos sólidos, uma vez que a configuração inicial é bem definida e suas deformações são finitas, tendo como variáveis principais os deslocamentos do sólido \cite{sanches2014fluid, fernandes2019ale}.

Já a descrição Euleriana toma como referência a configuração atual do contínuo, sendo bem utilizada para a descrição de fluidos newtonianos, pois estes não apresentam resistência às tensões de cisalhamento, distorcendo-se indefinidamente quando submetidos a tais solicitações \cite{sanches2014fluid, fernandes2019ale}.

Assim, a análise de interação fluido estrutura precisa combinar adequadamente a mecânica dos sólidos, convenientemente descrita na forma Lagrangiana, com a mecânica dos fluidos, convenientemente descrita na forma Euleriana. Uma forma robusta de se fazer isso é empregar a descrição Lagrangiana-Euleriana Arbitrária (ALE)  \cite{donea1982arbitrary} para o fluido. Essa descrição considera um domínio de referência com movimento arbitrário e independente das partículas. Isso permite que a malha do fluido seja movimentada dinamicamente para acomodar a movimentação da interface fluido-sólido.

\textcolor{red}{Falar primeiro sobre a convecção, que está presente tanto no compressível como no incompressível e é estabilizada com SUPG. Depois falar sobre as ondas de choque (no caso compressível apenas e que precisa de captura de choque para estabilizar) e por fim ir para o caso incompressível definindo as LBB e o PSPG. VMS em teoria captura todos os problemas citados.
}

No entanto, a aplicação do MEF em problemas da dinâmica dos fluidos, considerando a aplicação dos princípios variacionais diretamente nas equações governantes, conduzem a um sistema algébrico do tipo ponto de sela com sob-blocos nulos na diagonal principal na matriz do problema. Assim, na busca de um sistema positivo-definido, verifica-se que a escolha de espaços aproximadores para as variáveis do problema (velocidades e pressões) não deve ser feita inadvertidamente, sendo necessário obedecer as condições de compatibilidade de \LBB\ (LBB). Por outro lado, demais trabalhos objetivando uma flexibilização maior na escolha desses espaços adotam métodos estabilizados, como a aplicação dos termos estabilizadores \textit{Pressure-Stabilizing/Petrov-Galerkin} (PSPG), \textit{Streamline-Upwind/Petrov-Galerkin} (SUPG), ou na consideração de modelos estabilizados, como o \textit{Variational Multi-Scale} (VMS).

%Descrever os desafios do MEF na análise de IFE e soluções - convecção dominante - SUPG, escoamento incompressível - Babuska-Brezi

Já com relação ao escoamento, este pode ser classificado de três formas diferentes em função de seu Número de Reynolds: o escoamento laminar; de transição; e turbulento. No primeiro caso, o escoamento se dá de forma similar ao escorregamento de lâminas paralelas entre si, sem que haja uma mistura macroscópica entre elas. Já no segundo caso começam a surgir algumas flutuações esporádicas no escoamento, porém ainda não se apresentam de forma tão significativa a considerá-lo turbulento. Já o último caso apresenta flutuações constantes em seu fluxo, resultando no surgimento de estruturas denominadas de vórtices. Tais estruturas se manifestam de maneira instável, desordenada e em diversas escalas diferentes. Isso faz com que a obtenção de soluções numéricas desse tipo de fluxo seja uma tarefa muito custosa computacionalmente, uma vez que, para capturar a formação de vórtices até mesmo nas menores escalas, é necessária uma malha muito refinada, assim como algumas simulações possuem um domínio computacional consideravelmente grande.

Entretanto os escoamentos turbulentos representam uma parcela considerável daqueles presentes em problemas de engenharia e seu comportamento é bem determinado pelas equações de Navier-Stokes, as quais levam em consideração os princípios da conservação de massa, da conservação da quantidade de movimento e, caso necessário, da conservação da energia. No entanto, a aplicação direta dos métodos numéricos às equações governantes para a solução de problemas com escoamento turbulento demanda uma discretização muito refinada, dadas as escalas em que podem se dar a formação de vórtices.

%Descrever o problema de turbulência - o que é e que equações governam esses escoamentos.
%Descrever o desafio de simular problemas com escoamentos turbulentos no MEF
%Por que é desafiadora? Quais são os desafios? O que acontece em uma simulação quando há presença de turbulência?.

% Assim, isso tem sido tema de diversos pesquisadores que propuseram modelos para os problemas decorrentes da turbulência % Há trabalhos que classificam os modelos de turbulência em grupos. Tente encontrar e encaixar isso aqui.

No intuito de se obter uma melhor eficiência computacional, com a possibilidade de se capturar efeitos turbulentos mesmo com malhas menos refinadas e maiores intervalos discretos de tempos, surgem os modelos de turbulência, os quais geralmente se baseiam na decomposição de Reynolds (\RANS\ - RANS), ou na simulação de grandes escalas (\LES\ - LES).

Assim, diversos pesquisadores classificam os modelos de turbulência dependendo da maneira como tratam as equações de Navier-Stokes, sendo uma possibilidade a classificação em três grupos: Modelos mais simples, que empregam expressões algébricas para descrever a viscosidade turbulenta, assumindo que as estruturas turbulentas são geradas e dissipadas no mesmo local; Modelos que adicionam equações diferenciais adicionais para o cálculo da viscosidade turbulenta; E modelos que adicionam equações diferenciais de transporte para o cálculo das tensões de Reynolds \cite{souza2011revisao,alfonsi2009reynolds,teixeira2001simulaccao}.

No caso de simulações RANS as variáveis das equações de Navier-Stokes são decompostas em parcelas referentes à sua média e às suas flutuações, sendo que o valor da média pode ser tomado de diferentes formas a depender da natureza do escoamento. Além disso, problemas envolvendo RANS são mais comumente classificados em função da quantidade de equações diferenciais de transporte adicionais no problema, como os problemas de zero equações, uma equação, duas equações (por exemplo os modelos $\ke$ e $\kw$) e os modelos de tensões ($\te$). Nesse tipo de simulação observa-se o emprego de várias parcelas modeladas, ao passo que seu custo computacional é muito reduzido.

Já o modelo LES faz a decomposição das variáveis a partir de um filtro que faz a separação das grandes estruturas turbulentas das pequenas. A suposição básica desse tipo de modelo diz que as grandes escalas são responsáveis pelo transporte da energia e da quantidade de movimento, sendo resolvidas pelas equações de Navier-Stokes filtradas, enquanto as escalas pequenas (\textit{Sub-Grid Scales} - SGS) possuem comportamento isotrópico que, apesar de não poder ser resolvido diretamente, podem ser modeladas. Nesse cenário observa-se que o modelo LES possui uma quantidade menor de parcelas modeladas em comparação com o RANS, no entanto é relatado que seu custo computacional é maior. Dentre as propostas de se modelar as pequenas escalas, é tradicionalmente empregado o modelo de Smagorinsky, o qual relaciona a viscosidade turbulenta com o campo de velocidades em grandes escalas.

%Uma possível maneira de se obter resultados é pela simulação de \textit{Direct Numerical Simulation} (DNS), a qual é capaz de solucionar as equações de Navier-Stokes sem a utilização de modelos, descrevendo o escoamento em todas as escalas presentes no mesmo. Contudo, seu custo computacional é extremamente elevado, tornando sua utilização muito restrita a problemas de pequenas dimensões e para verificação de modelos \cite{olad2022towards}.

%Já para simulações mais representativas, se faz necessário o uso de modelos que descrevam adequadamente o comportamento dos escoamentos turbulentos. Dentre os possíveis modelos, destacam-se os modelos de \RANS\ (RANS), \LES\ (LES) e \VMS\ (VMS). Esses modelos são caracterizados por considerar uma separação dos parâmetros de análise em duas parcelas, onde, dependendo de como essa separação é feita, pode ocorrer aprimoramentos no tempo de processamento ou na qualidade dos resultados.

%Atenção: O VMS não é um modelo de turbulência, mas sim um modelo de estabilização multiescala... Você precisa conseguir definir muito bem isso...

Nesse sentido, propõe-se modelos de turbulência \RANS\ (RANS) e \LES\ (LES) em formulações estabilizadas do método dos elementos finitos em descrição ALE, realizando-se um estudo do desempenho desses modelos comparado com o desempenho da formulação estabilizada aplicada diretamente (sem modelos de turbulência) a problemas de interação fluido-estrutura, tendo em vista seu tempo de processamento e precisão dos resultados obtidos, os quais serão comparados com exemplos de alta fidelidade apresentados na literatura.

%Explicar melhor os modelos RANS e LES

%==================================================================================================
\section{Objetivos}
%==================================================================================================

Esta proposta tem como objetivo principal o estudo de formulações numéricas e a implementação computacional de modo a se obter ferramentas computacionais eficientes e precisas para a simulação de problemas de interação fluido-estrutura com elevados números de Reynolds, onde possam haver efeitos de turbulência. Dentro desse escopo, alguns objetivos específicos devem ser alcançados:

\begin{itemize}
    \item Estudo das formulações estabilizadas do método dos elementos finitos para escoamentos incompressíveis com contornos móveis, com destaque para as estabilizações SUPG, LSIC, PSPG e VMS;

    \item Estudo da formulação posicional do MEF para análise dinâmica de sólidos e cascas com grandes deslocamentos;

    \item Estudo das técnicas de acoplamento particionado fluido-estrutura com malhas móveis;

    \item Estudo dos diversos modelos de turbulência no contexto do MEF;

    \item Implementação da formulação estabilizada \VMS;

    \item Implementação dos métodos \RANS\ e \LES;

    \item Acoplamento do código para mecânica dos fluidos com programa para análise não linear de estruturas de cascas;

    \item Estudo comparativo dos modelos implementados através da simulação de exemplos disponíveis na literatura.
\end{itemize}

%==================================================================================================
\section{Justificativa}
%==================================================================================================

Os avanços na área da engenharia exigem determinações precisas dos parâmetros relacionados ao dimensionamento, sejam referentes à resistência dos materiais utilizados, ou aos esforços atuantes sobre a estrutura. Nesse cenário, os efeitos advindos da interação fluido-estrutura são responsáveis por submeter tais estruturas a esforços consideráveis, o que os tornam objetos de estudos aprofundados. Assim, mesmo que diversos trabalhos tenham sido realizados, resultando em grandes avanços na área, alguns desafios se mantêm de forma a inviabilizar sua prática no cotidiano da engenharia.

Dentre esses desafios pode-se destacar os problemas de acoplamento, uma vez que os princípios físicos que regem tanto os sólidos quanto os fluidos são distintos, além dos parâmetros estudados por cada um serem também diferentes. Também verifica-se que o sistema de equações fundamental para o estudo de fluidos, ou não possui solução analítica, ou esta foi obtida apenas para uma gama de casos simples, sem muita possibilidade de generalizações e aplicações.

Dessa forma, os métodos numéricos são vistos como uma alternativa necessária para o estudo de fluidos, nos quais pode-se destacar o uso recente do Método dos Elementos Finitos (MEF) para se realizar as análises. Esse método ganha atenção, uma vez que possui a propriedade de se adequar com certa facilidade à problemas de geometria complexa, suas condições de contorno são mais facilmente aplicadas, assim como possibilita a consideração de meios com propriedades diferentes, ou até problemas multifísicos.

Contudo a utilização do MEF em simulações de fluidos conduz a certas instabilidades, que são superadas pelo acato das condições de \LBB, que impedem o uso de espaços aproximadores iguais para velocidades e pressões. Nesse sentido, alguns elementos podem ser utilizados de maneira a obedecer essas condições, denominados de elementos de Taylor-Hood. Alternativamente a utilização de termos estabilizadores pode contornar esses problemas, como aqueles presentes em uma formulação SUPG/PSPG, ou em métodos estabilizados como o VMS.

Outra questão envolvendo a utilização do MEF nesse tipo de problema está relacionado ao custo computacional, que também de mostra como um desafio enfrentado por pesquisadores dessa área. Nesse cenário, problemas de escoamento turbulentos se tornam ainda mais custosos, devido à algumas propriedades de escoamentos turbulentos, como a manifestação de vórtices, geralmente tridimensionais, de forma desordenada, instável e em uma grande amplitude de escalas. Assim necessita-se de malhas muito refinadas para capturar a ocorrência dessas estruturas.

Com isso, foram desenvolvidos modelos de turbulência, que possibilitam a utilização de malhas menos refinadas, assim como a possibilidade de adotar passos de tempo maiores nas simulações, os quais destacam-se aqueles baseados na decomposição de Reynolds (RANS) e em grandes vórtices (LES).

Sendo assim, o presente trabalho é justificado pelo desenvolvimento de uma ferramenta computacional eficiente e precisa, aplicada a problemas de IFE, capaz de se utilizar de diferentes modelos de turbulência para redução do custo computacional. Nesse fim, serão comparados os diferentes modelos de turbulência, com diferentes elementos finitos, incluindo o elemento Taylor-Hood P2P1, assim como a influência que cada termo estabilizador causa na solução.

Desse modo, a presente pesquisa visa aprimorar as ferramentas computacionais que estão sendo desenvolvidas pelo grupo de pesquisa do SET, aplicando conceitos de otimização baseados em modelos de turbulência no código atual.


%1-chamar a atenção para as contribuições e a importância da mecânica computacional para a engenharia e especialmente para problemas como os de IFE. Deixar claro que apesar do elevado número de trabalhos de qualidade e da maturidade das pesquisas nesse tema, ainda há desafios que impedem que seja empregada no dia-a-dia da engenharia, e isso inclui os custos computacionais para problemas como os de IFE com turbulência (Jeferson, Giovane)

%2-Defender o emprego do MEF para IFE

%3-Mostar a necessidade de modelos de turbulência no MEF nos problemas de IFE

%4-Mencionar que isso já justifica uma proposta de trabalho que busque o desenvolvimento de ferramenta computacional mais eficiente e precisa para IFE

%5-Trazer para o contexto do departamento e do grupo de pesquisas, justificando a melhoria e a extensão da plataforma computacional já desenvolvida

\begin{comment}
Observa-se os efeitos da interação fluido-estrutura em diversas estruturas, em especial os efeitos devido à escoamentos turbulentos em estruturas flexíveis, por exemplo, a atuação de ventos em edifícios. Nesse sentido, se faz necessário o estudo de técnicas eficazes para determinação dos impactos desses escoamentos sobre as mais diversas estruturas.

Duas formas possíveis de se obter esses resultados são: pela construção de modelos reais em escala, observando-se na prática o comportamento estrutural e do escoamento; ou pela modelagem matemática do problema. O primeiro cenário apresenta algumas desvantagens, pois os parâmetros que são obtidos nesse tipo de análise podem ser dependentes da escala da amostra, impactando diretamente na qualidade dos resultados, além de demandar uma infraestrutura muito robusta para se obter valores coerentes. Assim, o modelo matemático se mostra como uma solução mais eficiente, uma vez que não demanda um grande investimento em infraestrutura.

No entanto muitos tipos de análise envolvendo escoamentos turbulentos recaem em problema com um alto grau de complexidade, ocasionando, consequentemente, um custo computacional muito alto. Dessa maneira, o presente trabalho busca analisar entre as técnicas matemáticas qual se mostra mais eficiente para essas análises, dando continuidade aos trabalhos que já foram realizados pelo grupo de pesquisa.

Além dos métodos para determinação de parâmetros referentes ao escoamento, também é necessário se obter aqueles referentes à resposta da estrutura frente à esse escoamento. Assim, nota-se a presença de diversos métodos para a determinação de parâmetros relativos à estruturas flexíveis, em que se percebe o Método dos Elementos Finitos Posicional como uma alternativa interessante para tal análise. Esse método se destaca pela consideração das posições nodais como parâmetros de análise, evitando problemas devido à falta de comutatividade de rotações, que é observado, por exemplo, no Método dos Elementos Finitos Corrotacional, que, além de deslocamentos, também considera rotações nodais como parâmetros de análise. Além disso, também vale destacar que a matriz de massa em problemas dinâmicos permanece constante ao longo do tempo, possibilitando a implementação facilitada de métodos de integração temporal.
\end{comment}