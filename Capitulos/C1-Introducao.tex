%==================================================================================================
\chapter{Introdução}\label{CapIntroducao}
%==================================================================================================

Problemas de interação fluido-estrutura (IFE) ocorrem quando o escoamento do fluido em contato com a estrutura afeta o comportamento da mesma, e vice-versa. Essa interação pode ser entendida levando-se em conta que as forças desenvolvidas na estrutura em decorrência do escoamento (pressão e tensões viscosas do fluido atuando na interface fluido-estrutura) causam a movimentação da estrutura (deformação), que por sua vez, altera o escoamento, modificando a pressão e as tensões viscosas na interface fluido-estrutura. Assim, o estudo desse tipo de problema envolve a análise multicampo, na qual os diferentes meios físicos (sólido e fluido), que embora sujeitos às mesmas leis da mecânica, apresentam relações constitutivas muito diferentes. Isso faz com que a análise numérica de problemas de IFE seja um tema bastante desafiador e interdisciplinar, envolvendo três tópicos principais: a mecânica dos sólidos, a mecânica dos fluidos e o problema de interação.

Esses problemas ocorrem em diversas situações, como tais como em turbinas eólicas, pontes, edifícios de grande altitudes, estruturas \textit{offshore}, aeronaves, escoamento em vasos sanguíneos, entre outros. A análise adequada desses problemas é de grande importância em diversos projetos de engenharia, uma vez que podem levar a falhas catastróficas, como o colapso de pontes ou edifícios, ou a perda de controle de uma aeronave.

Uma das formas de se estudar os problemas de IFE é através da abordagem experimental por meio de ensaios de modelos em escala real ou reduzida em túneis de vento, canais ou tanques de ensaio. No entanto, essa abordagem apresenta limitações, uma vez que demanda grandes investimentos com infraestrutura, a construção de modelos em escala reduzida pode não capturar todos os fenômenos físicos que ocorrem em um escoamento real, além de que os resultados aplicam-se especificamente e somente aos casos ensaiados \cite{fernandes2020tecnica}.

Outra forma é a modelagem matemática desses problemas, que se mostra como uma opção mais viável, uma vez que dispensa grandes investimentos com material, espaço e equipamentos de ensaios, e possui grande flexibilidade de aplicações, podendo ser utilizado em diversos tipos de análises. Todavia, os modelos matemáticos conduzem a sistemas de equações que, ou não apresentam solução analítica, ou apresentam apenas para casos simples e com uso de hipóteses simplificadoras que deixam a solução menos geral, demandando assim soluções numéricas.

A solução dos problemas de dinâmica dos sólidos computacional (CSD) e de dinâmica dos fluidos computacional (CFD), isoladamente já podem apresentar elevado grau de complexidade, de forma que a análise numérica do problema acoplado, a despeito do número de trabalhos que já foram desenvolvidos nessa área, ainda apresenta apresenta diversos desafios a serem superados, sendo o alto custo computacional de muitos problemas um desses desafios. Tal custo torna a simulação numérica da IFE impraticável para diversos problemas usuais de engenharia, que exigem respostas mais ágeis para serem utilizadas em elaborações de projetos.

Os fluidos se diferenciam dos sólidos por apresentarem pouca, ou, no caso dos fluidos Newtonianos, nenhuma resistência ao cisalhamento, de modo que qualquer valor de tensão de cisalhamento provoca escoamento. Dessa forma, tradicionalmente é empregada a descrição Euleriana (ou espacial) para as equações da mecânica dos fluidos, implicando em domínio espacial fixo e indeformável, enquanto para os sólidos é empregada a descrição Lagrangiana (ou material), implicando em domínio espacial deformável que segue os pontos materiais da estrutura.

O primeiro desafio da análise numérica de IFE é combinar essas duas descrições diferentes de maneira consistente. Uma forma robusta de se fazer isso é empregar a descrição Lagrangiana-Euleriana Arbitrária (ALE) \cite{donea1982arbitrary} para o fluido. Essa descrição considera um domínio de referência com movimento arbitrário e independente das partículas. Isso permite que a malha do fluido seja movimentada dinamicamente para acomodar a movimentação da interface fluido-sólido.

Os escoamentos de fluidos viscosos podem ser classificados, dentre outras formas, em função do número de Reynolds, o qual é um coeficiente adimensional definido como a razão entre as forças inerciais e as forças viscosas, resultando em três regimes de escoamento. O primeiro regime é denominado laminar, onde o escoamento se dá de forma similar ao escorregamento de lâminas paralelas entre si, sem que haja uma mistura macroscópica entre elas. Já o segundo regime é chamado regime de transição, caracterizado pelo surgimento de algumas flutuações esporádicas no campo de velocidade, porém ainda de forma não tão significativa. Já o último é denominado regime turbulento, apresentando flutuações constantes no campo de velocidade, resultando no surgimento de estruturas denominadas de vórtices.

Assim, em casos de escoamentos turbulentos, ocorrem desprendimentos de vórtices, que podem estar localizados em pontos da interface fluido-estrutura, causando vibrações induzidas pelas flutuações dos campos de velocidade e pressão. Também podem ocorrer, por exemplo, aumento da força de arrasto ou queda da força de sustentação em estruturas aerodinâmicas devido à formação de bolhas de desprendimento.

As estruturas turbulentas, podem se manifestar tridimensionalmente de maneira instável, desordenada e em diversas escalas. Isso torna a turbulência um fator com importante contribuição ao alto custo computacional das soluções numéricas de problemas de IFE, uma vez que, para capturar a formação de vórtices até nas menores escalas é necessária uma discretização muito refinada, ao mesmo tempo que algumas simulações demandam um domínio computacional consideravelmente grande.

Buscando soluções mais competitivas diversos trabalhos foram desenvolvidos de forma a reduzir o custo computacional, seja pelo aumento da ordem de convergência dos problemas, ou pela possibilidade de se utilizar malhas menos refinadas nas simulações. Dentre as possibilidades existentes destacam-se os métodos multiescala (VMS), os quais partem da premissa de decomposição dos campos de velocidade e de pressão em parcelas de pequenas e de grandes escalas, introduzindo termos estabilizadores capazes de estabilizar tanto problemas com convecção quanto com campo de pressão. Assim ocorre uma melhora na convergência, além de se dispensar o refinamento da malha em regiões de alta convecção.

Outros trabalhos também foram desenvolvidos no sentido de permitir o uso de malha mais grosseiras em regiões de alta vorticidade, onde estruturas turbulentas se formam. Dessa forma, surgem as simulações de grandes vórtices (LES) \cite{smagorinsky1963general}, as quais fazem a decomposição dos campos a partir da consideração de um filtro, fazendo com que a turbulência gerada na subescala seja considerada a partir da adição de uma viscosidade de vórtice no escoamento.

Assim, visando a identificação de uma abordagem eficiente para simular de problemas de interação fluido-estrutura em escoamentos com presença de vorticidade, este trabalho propõe o estudo de formulações numéricas e sua respectiva implementação, a fim de se obter uma ferramenta capaz de aplicar diferentes métodos para redução do custo computacional.

%==================================================================================================
\section{Organização do trabalho}
%==================================================================================================

A presente dissertação está estruturada em 6 capítulos, os quais são brevemente descritos a seguir.

    {
        \newcommand{\Capi}[2]{\textbf{Capítulo #1 - #2}:}

        \Capi{\ref{CapIntroducao}}{Introdução} Apresenta a motivação para o desenvolvimento do trabalho, os objetivos, a justificativa para a realização do mesmo e a metodologia adotada. Introduz-se resumidamente as principais características e particularidades dos problemas de interação fluido-estrutura, bem como as dificuldades encontradas para a obtenção de soluções numéricas precisas para esses problemas. Na sequência é apresentado o estado da arte dos principais assuntos envolvidos, de forma a situar o leitor quanto aos principais desenvolvimentos, dando uma visão do panorama científico atual em relação aos métodos numéricos para análise de interação fluido estrutura aplicados a escoamentos incompressíveis viscosos. Com isso o leitor estará apto a compreender a metodologia proposta e sua justificativa, as quais são apresentadas na sequência.

        \Capi{\ref{EGDF}}{Análise Numérica Para Mecânica Dos Fluidos} Inicialmente são apresentadas as equações governantes para escoamentos isotérmicos incompressíveis, tanto na descrição Euleriana quanto na descrição Lagrangiana-Euleriana Arbitrária, que permite a simulação de problemas de escoamentos com contornos móveis. Em seguida, desenvolve-se a formulação numérica adotada para a solução dos problemas de escoamentos incompressíveis, a qual trata-se de uma formulação estabilizada do método dos elementos finitos com integração temporal por meio do método implícito de marcha no tempo $\alpha$-generalizado. Na sequência se introduz o esquema adotado para a movimentação da malha para garantir a conformidade do domínio com os contornos móveis, o qual é baseado na equação de Laplace. Posteriormente é desenvolvida a formulação estabilizada de acordo com o VMS. Em seguida é introduzido o modelo de turbulência LES, com destaque para o emprego da viscosidade de Smagorinsky. Por fim, são apresentados exemplos de verificação para a formulação numérica proposta, tanto em problemas de contornos fixos, quanto de contornos móveis.

        \Capi{\ref{EGDS}}{Análise Numérica Para Dinâmica Não Linear Dos Sólidos} Apresenta-se primeiramente os conceitos básicos da dinâmica dos corpos deformáveis, onde se introduz as medidas fundamentais para o estudo desses meios, adotando-se uma descrição Lagrangiana Total. Em seguida, apresenta-se a formulação posicional do MEF para elementos de casca com cinemática de Reissner-Mindlin, empregando-se o método implícito $\alpha$-generalizado para integração temporal. Finalmente são apresentados alguns exemplos para verificação do código empregado.

        % \Capi{4}{Movimentação Da Malha Via Suavização De Laplace} Introduz-se uma modificação do esquema de movimentação de Laplace. Nesse esquema resultante os elementos menores da malha possuem uma rigidez maior em relação aos maiores, preservando, assim, a qualidade da malha. Com isso são apresentados exemplos de aplicação da técnica em problemas de IFE com movimento prescrito por parte da estrutura.

        \Capi{\ref{AFE}}{Acoplamento Fluido-Estrutura} São apresentadas as condições que devem ser atendidas para garantir o acoplamento fluido-estrutura, assim como as técnicas de acoplamento particionado (fraco e forte) e suas particularidades. Na sequeência, são estudados exemplos selecionados de problemas de IFE, explorando-se os efeitos de diferentes discretizações, diferentes espaços aproximadores, uso da formulação VMS e uso dos modelo de turbulência LES.

        \Capi{\ref{Conclusao}}{Conclusões} Com base nos resultados obtidos nos capítulos anteriores são apresentadas as conclusões do trabalho e, a partir disso, são propostas sugestões para trabalhos futuros.
    }

%==================================================================================================
% ESTADO DA ARTE
%==================================================================================================

% Introdução

\documentclass[_ArquivoPrincipal.tex]{subfiles}

\begin{document}

%==================================================================================================
\chapter{Estado Da Arte}
%==================================================================================================

No presente capítulo serão abordados os principais temas relacionados à problemas de Interação Fluido-Estrutura (IFE), tendo como ênfase os modos de turbulência. Serão destacados os principais métodos de cálculos envolvendo a dinâmica das estruturas computacional (\ref{CSD}), a dinâmica dos fluidos computacional (\ref{CFD}) e alguns modelos de turbulências (\ref{MT}).

%==================================================================================================
\section{Dinâmica Das Estruturas Computacional} \label{CSD}
%==================================================================================================

A mecânica dos sólidos visa a determinação de parâmetros referentes à elementos estruturais sujeitos a solicitações externas, tais como tensões e deformações, ou forças e deslocamentos de determinados pontos dos mesmos. Para isso são desenvolvidos diversos métodos com a finalidade de melhor descrever o comportamento das estruturas, como o Método dos Elementos Finitos (MEF), que se apresenta como a ferramenta computacional mais amplamente utilizada para esse fim.

Nesse sentido vale ressaltar os trabalhos de \citeonline{hughes1981nonlinear, argyris1982excursion, ibrahimbegovic2002role, pimenta2004fully, battini2006choice} e \citeonline{pimenta2008exact}, que realizaram análises através do MEF Corrotacional, que se trata de uma formulação onde os deslocamentos e rotações dos elementos são os parâmetros principais da análise.

No entanto a utilização de uma análise que considera tais parâmetros nodais só se mostra eficiente ao se estudar estruturas que desenvolvem pequenos deslocamentos e deformações. Isso deve-se ao fato de essa abordagem utilizar rotações finitas como parâmetros nodais, o que pode gerar controversas em problemas envolvendo grandes rotações, uma vez que não há comutatividade entre essas grandezas. Ainda observa-se que a avaliação dinâmica de estruturas reticuladas se torna problemática, do ponto de vista da conservação de energia, além da matriz de massa ser variável, tornando o processo de integração temporal muito complexo \cite{sanches2013unconstrained}.

Com isso, vale ressaltar os trabalhos de \citeonline{coda2004simple, coda2007alternative, coda2010improved, coda2009two, coda2009unconstrained} e \citeonline{carrazedo2010alternative} que utilizam de forma bem-sucedida o MEF Posicional para análise de estruturas reticuladas e de cascas aplicadas à grandes deslocamentos. Este método se diferencia dos demais ao considerar as posições nodais, obtidas a partir de vetores indeformados, como parâmetros de análise, facilitando o cálculo de efeitos de não-linearidade geométrica.

\citeonline{sanches2013unconstrained} apresentaram uma aplicação do método dos elementos finitos posicional em elementos de cascas e constataram a presença de uma matriz de massa constante, o que possibilita a utilização de métodos de integração temporal, como o método de Newmark $\beta$, conservando, assim, o momento linear e angular em análises dinâmicas. Além disso, esse trabalho também analisou problemas envolvendo IFE, obtendo resultados favoráveis, indicando a boa aplicação desse método em problemas dessa natureza.

No âmbito da IFE, destaca-se a necessidade dessa consideração, uma vez que são observadas aplicações como, por exemplo, grandes amplitudes de deslocamentos, como \textit{flutter}, aplicações biomecânicas, estruturas infláveis \cite{karagiozis2011computational}, simulações de turbinas \cite{bazilevs20113d}, dentre outras.

%==================================================================================================
\section{Dinâmica Dos Fluidos Computacional} \label{CFD}
%==================================================================================================

Ao contrário de problemas envolvendo elementos sólidos, que possuem um estado inicial bem definido, os fluidos, em especial os Newtonianos, não o possuem, uma vez que não incapazes de resistir à tensões desviadoras, deformando-se, assim, indefinidamente quando sujeitos à essas tensões. Dessa forma, torna-se apropriada a utilização de uma descrição Euleriana para descrever os fluidos, em que os parâmetros nodais são principalmente as velocidades do mesmo \cite{fernandes2020tecnica}.

Em geral, problemas envolvendo Dinâmica dos Sólidos (\textit{Computational Solid Dynamics} - CSD) partem do princípio da estacionariedade de energias, buscando a determinação do ponto em que a energia do sistema seja mínima. Esses métodos apresentam a particularidade de surgir um sistema de equações com uma matriz simétrica e, em alguns casos, de vetores cujas componentes possuem significado físico. Porém, em problemas que envolvem a Dinâmica dos Fluidos (\textit{Computational Fluid Dynamics} - CFD), o sistema de equações possui uma matriz, na maioria dos casos, assimétrica, devido à presença de termos convectivos nas equações governantes \cite{bazilevs2013computational,brooks1982streamline}.

Nesse sentido, são desenvolvidos novos métodos, em busca de uma solução mais representativa com uma malha menos refinada. Dentre os principais desenvolvidos, vale mencionar o \textit{Streamline-Upwind/Petrov-Galerkin} (SUPG) \cite{brooks1982streamline}, \textit{Galerkin Least-Squares} (GLS) \cite{hughes1989new,tezduyar1991stabilized}, \textit{Subgrid Scales} (SGS) e \cite{hughes1995multiscale}.

Um campo que vale destacar na CFD é o que estuda os fenômenos de turbulência, uma vez que esses fenômenos podem se apresentar nas mais variadas escalas, sendo necessário a geração de uma malha muito refinada para detectar a formação dos vórtices, o que ocasiona um aumento radical no custo computacional da análise. Assim, são desenvolvidas novas técnicas afim de se obter melhores resultados, sem que haja um grande aumento no volume de cálculos. Dentre esses métodos destacam-se os métodos multiescala (\textit{Variational Multi-Scale} - VMS), as simulações de grandes vórtices (\textit{Large Eddy Simulation} - LES) \cite{hughes1995multiscale,hughes1998variational,hughes2002variational,bazilevs2010large,vsekutkovski2021partitioned}, aproximações de \textit{Reynolds-Averaged Navier-Stokes} \cite{alfonsi2009reynolds} e os métodos de atualização de malha \cite{de1993petrov}. O capítulo \ref{FT} descreve de forma mais detalhada cada um desses métodos.

%==================================================================================================
\section{Interação Fluido-Estrutura} \label{IFE}
%==================================================================================================

Segundo \citeonline{sanches2014fluid} problemas numéricos envolvendo Interação Fluido-Estrutura (IFE) são divididos em três áreas, sendo elas a CFD, CDS e Problemas de Interação (\textit{Interaction Problem} - IP). Fazer essa interação entre CFD e CSD pode se tornar uma tarefa complexa, pois se caracteriza pela sua multidisciplinaridade \cite{hou2012numerical} além de acoplar esses dois problemas se tornar complicado, uma vez que pode-se utilizar descrições diferentes para cada problema, por exemplo a utilização de uma descrição Lagrangiana para modelar o sólido e uma descrição Euleriana para o fluido, além de que os parâmetros nodais determinados em cada um dos dois problemas são distintos, como deslocamentos ou posições em sólidos e velocidades e pressões no fluido.

Outra questão a ser observada em problemas de IFE é que, devido à movimentação da estrutura, deve-se realizar algum procedimento para que o domínio do fluido perceba essa movimentação. \citeonline{bazilevs2013computational} apontam duas técnicas possíveis de serem utilizadas para se considerar esse efeito: a técnica de malha conforme (ou \textit{interface-tracking}); e a técnica de malha não-conforme (ou \textit{interface-capturing}). Na técnica de malhas conformes a interface entre o sólido e o fluido é caracterizada pela presença de condições de contorno na mesma, necessitando, assim, que a malha do fluido se deforma para se acomodar à nova configuração, deformando-se no processo ou, caso necessário, passando por remalhamento \cite{terahara2020heart}. Esse tipo de técnica é interessante, já que permite a utilização de uma malha razoavelmente complexa próxima à interface, para se obter resultados mais precisos nessa região. No entanto percebe-se que alguns casos de deformações excessivas do domínio do fluido, pode ocorrer de não ser viável ou meramente possível essa movimentação. Por sua vez, a técnica de malhas não-conformes considera as condições de contorno diretamente nas equações governantes, permitindo que os problemas sejam resolvidos separadamente sem a necessidade de movimentação da malha, evitando, assim, os problemas relacionados à movimentação excessiva da malha. Porém, em certos casos de problemas de geometria complexa, os custos de remalhamento podem ser compensados com o aumento de precisão obtido próximo à interface \cite{bazilevs2013computational,hou2012numerical,bazilevs2015ale}.

Dentro do âmbito de malhas não-conformes, nota-se a utilização de técnicas baseadas em contornos imersos em diversos trabalhos, como: o de \citeonline{zhao2016numerical}, o qual analisou os resultados da técnica por comparação com respostas analíticas, numéricas e experimentais, tendo obtido bons resultados; o de \citeonline{zheng2020numerical}, que utilizou uma formulação modificada de um método de contornos imersos, comparando os resultados com os obtidos experimentalmente em situações simétricas e assimétricas, obtendo resultados satisfatórios; o de \citeonline{xiao2022immersed}, sendo estudado escoamentos com transferência de massa, calor e momento, e também são apontadas pelos autores as dificuldades provenientes, dentre outras causas, da não conformidade da malha, especialmente em problemas com alto número de Reynolds; dentre outros, como de \citeonline{wang2011algorithms,ruess2013weakly,yan2021three}.

Tendo em vista a técnica de malha conforme, destaca-se a utilização formulação Lagrangiana-Euleriana Arbitrária (\textit{Arbitrary Lagrangian-Eulerian} - ALE) \cite{donea1982arbitrary,kanchi20073d,fernandes2019ale} e a formulação Espaço Tempo (\textit{Space-Time} - ST) \cite{takizawa2011multiscale,terahara2020heart,takizawa2011stabilized}. Assim, o problema pode ser subdividido na determinação dos parâmetros referentes ao: sólido; fluido; e malha.

Como mencionado por \citeonline{hou2012numerical}, problemas de IFE podem ser ainda classificados em termos de sua abordagem: o modelo monolítico; e o modelo particionado, o qual também pode ser subdividido em particionado forte e particionado fraco. No modelo monolítico todos os parâmetros do problema são calculados diretamente no mesmo bloco de equações. Uma grande vantagem dessa abordagem é a obtenção direta dos valores de interesse, além de alcançar uma precisão maior nos resultados, às custas de um maior custo computacional, uma vez que o sistema a ser resolvido é consideravelmente maior. Já o modelo particionado resolve de forma independente os diferentes problemas envolvidos, o que se mostra como uma grande vantagem desse método, já que é possível reutilizar códigos funcionais para diferentes modelos e apenas integrá-los em um processo que fará a interação, flexibilizando o código à novos modelos com uma probabilidade menor de erros no código \cite{roux2008domain,hou2012numerical}. O modelo particionado fraco é suficientemente satisfatório em simulações com intervalos pequenos de passo de tempo, dispensando, assim, a interação em \textit{loop}, sendo classificado, portanto, como um método explícito, ao contrário do modelo particionado forte, onde ocorrem interações em \textit{loop} para corrigir eventuais erros na IFE, caracterizando-se como um método implícito \cite{fernandes2020tecnica}.

Dentre os trabalhos que se utilizam do modelo monolítico cita-se os trabalhos de \citeonline{michler2004monolithic,hron2007fluid,wick2021optimization}. Já os trabalhos que utilizam o modelo particionado cita-se \citeonline{sanches2013unconstrained,sanches2014fluid,fernandes2019ale}

\textcolor{red}{comentar sobre efeitos de massa-adicionada?}

%==================================================================================================
\section{Modelos De Turbulência} \label{MT}
%==================================================================================================

Um fluido pode apresentar um escoamento de duas formas distintas, a depender do número de Reynolds que este apresenta. Em casos cujo número de Reynolds é baixo, o escoamento é considerado laminar, ou seja, o escoamento se dá de forma semelhante ao movimento de lâminas independentes, não havendo mistura macroscópica entre as mesmas. Esse tipo de escoamento possui soluções muito mais simples de se obter, no entanto representam uma ocorrência baixa na maioria dos problemas observados na natureza. Já em casos cujo número de Reynolds é muito elevado, o escoamento é considerado turbulento. Nesse cenário há a formação de vórtices sobre o escoamento, que podem ocorrer de forma instável, desordenada e em várias escalas diferentes \cite{popiolek2005analise,shaughnessy2005introduction}. Esse fenômeno pode ser descrito de acordo com as equações de Navier-Stokes, no entanto sua resolução se apresenta com um alto grau de complexidade, uma vez que possui termos não lineares em sua composição. Assim são necessárias técnicas de solução, para que se possa obter uma solução de forma aproximada à essas equações.

Uma possível forma de simulação de escoamentos turbulentos é o denominado \textit{Direct Numerical Simulation} (DNS), o qual resolve as equações de Navier-Stokes diretamente em todas as escalas de presentes, sendo, portanto, o método mais preciso de se simular escoamentos turbulentos. No entanto possui um alto custo computacional, restringindo sua aplicação à problemas pequenos e de geometria simples para validação de outros modelos de turbulência que sejam mais viáveis \cite{piomelli1999large,yokokawa200216}. Alguns trabalhos realizados que utilizam o DNS são: \citeonline{yokokawa200216,picano2015turbulent,olad2022towards,frey2021machine}.

Outra forma de se simular escoamentos turbulentos é o \textit{Reynolds-Averaged Navier-Stokes} (RANS). A principal premissa desse modelo diz que é possível separar as propriedades do escoamento em duas partes: uma parte relacionada à média temporal da propriedade, sendo essa a parcela predominante e a principal a ser determinada, e outra relacionada à variações no espaço-tempo da mesma. Por meio dessa consideração, a equação de Navier-Stokes se transforma de tal forma a surgir um termo adicional relacionado à interações entre as parcelas flutuantes, a qual representa a interferência dos efeitos turbulentos na propriedade média. Existem diversos modelos para descrever o comportamento, em que o mais comum de se encontrar é baseado no tensor de tensões de Reynolds, levando em consideração efeitos de viscosidade turbulenta \cite{piomelli1999large,alfonsi2009reynolds,bazilevs2010large,ling2015evaluation}.

Por sua vez, o modelo de \textit{Large-Eddy Simulation} (LES), que considera também a separação dos parâmetros em duas parcelas: uma parcela filtrada; e outra não-filtrada. Nesse cenário se faz necessária a modelagem dos termos não-filtrados, enquanto a parcela filtrada é determinada diretamente por meio da resolução das equações de Navier-Stokes. Uma possível forma de se modelar os termos não-filtrado baseia-se no modelo de \textit{Sub-Grid Scale} (SGS), que faz a interação entre os campos de grandes e de pequenas escalas, sendo aprimorada de forma a capturar os efeitos turbulentos em função do tamanho da malha \cite{ghosal1995basic,hughes2000large,moeng2015large}.
\citeonline{olad2022towards} aponta que a formulação a base de LES se mostrou mais preciso na determinação dos campos de velocidades e de tensões desviadoras, porém com um custo computacional consideravelmente superior ao RANS.

No entanto, as simulações baseadas em RANS e LES possuem a dificuldade de trabalharem com sistemas de equações nas quais surgem sub-blocos nulos na matriz do problema. Assim, surgem modelos como o \textit{Variational Multi-Scale} (VMS), introduzido por \citeonline{hughes1995multiscale,hughes1998variational,hughes2000large}. que contornam essa dificuldade. Esse modelo faz uso dos princípios variacionais, em que tanto os espaços tentativas quanto os espaços testes são divididos em: parcela de grandes escalas; e parcela de pequenas escalas. Com isso se faz a modelagem do espaço de pequenas escalas em termos de resíduos das equações de conservação de massa e de conservação da quantidade de movimento. Tal consideração preenche os termos da diagonal principal, fazendo com que o problema se torne positivo-definido, facilitando a escolha dos espaços de aproximação de velocidades e pressões, além da resolução do problema se mostrar de forma mais estável \cite{bazilevs2013computational,sondak2015new}.

\end{document}

%==================================================================================================
\section{Objetivos}
%==================================================================================================

Esta proposta tem como objetivo principal o estudo de formulações numéricas e a implementação computacional de modo a se obter ferramentas computacionais eficientes e precisas para a simulação de problemas de interação fluido-estrutura com elevados números de Reynolds, onde possa haver efeitos de turbulência. Dentro desse escopo, alguns objetivos específicos devem ser alcançados:

\begin{itemize}
    \item Estudo das formulações estabilizadas do método dos elementos finitos para escoamentos incompressíveis com contornos móveis, com destaque para as estabilizações SUPG, LSIC, PSPG e VMS;

    \item Estudo da formulação posicional do MEF para análise dinâmica de sólidos e cascas com grandes deslocamentos;

    \item Estudo das técnicas de acoplamento particionado fluido-estrutura com malhas móveis;

    \item Estudo dos diversos modelos de turbulência no contexto do MEF;

    \item Implementação da formulação estabilizada \VMS;

    \item Implementação do modelo \LES;

    \item Acoplamento do código para mecânica dos fluidos com programa para análise não linear de estruturas de cascas;

    \item Estudo comparativo dos modelos implementados através da simulação de exemplos disponíveis na literatura.
\end{itemize}

%==================================================================================================
% METODOLOGIA
%==================================================================================================

%==================================================================================================
\section{Metodologia} \label{MetodologiaCronograma}
%==================================================================================================

Tendo em vista a complexidade dos problemas em estudo, delimita-se os estudos dos problemas de interação fluido-estrutura considerando escoamento incompressível viscoso interagindo com estruturas de casca com grandes deslocamentos. Adota-se acoplamento particionado forte do tipo bloco-iterativo com método de malha móvel para o fluido empregando-se a descrição Lagrangiana-Euleriana Arbitrária (ALE). Essa abordagem é adotada por ser reconhecidamente robusta, modular e amplamente utilizada para análises de IFE.

Assim como os problemas a serem estudados, as implementações necessárias também são bastante complexas, sendo importante a adoção de uma metodologia de programação que aproveite os códigos disponíveis e ao mesmo tempo facilite alterações de tipos de elementos, modelos constitutivos, métodos de solução de sistemas e operações algébricas, e mais importante no contexto desta proposta, diferentes métodos de estabilização e modelos de turbulência.

Dessa forma adota-se a linguagem de programação C++ orientada a objeto, uma vez que se é possível aproveitar diversos códigos já desenvolvidos pelo grupo de pesquisa, \ie\ programa de elementos finitos para análise de escoamentos incompressíveis empregando-se elementos triangulares (2D) e tetraédricos (3D) com aproximações linear ou quadrática e programa para análise dinâmica não linear geométrica de estruturas de casca finas ou espessas empregando elementos triangulares cúbicos.

Para a solução da mecânica dos sólidos será empregado o MEF baseado em posições, aproveitando-se um programa para análise de cascas com cinemática de Reissner-Mindlin, já desenvolvido pelo grupo de pesquisa. A formulação baseada em posições emprega uma descrição Lagrangiana Total, sendo adotado o modelo constitutivo de Saint-Venant Kirchhoff  com a medida de deformação de Green-Lagrange. O elemento de casca utilizado possui 7 graus de liberdade por nó, \ie\ 3 coordenadas de posição, 3 coordenadas de um vetor generalizado, inicialmente perpendicular à superfície média da casca, e 1 parâmetro de enriquecimento para permitir variação linear da deformação na direção da espessura.% A aproximação das derivadas temporais do problema é feita com integrador temporal $\alpha$-generalizado. O processo de solução do sistema não linear é obtida por meio do método iterativo de Newton-Raphson.

Já para a solução da mecânica dos fluidos, parte-se de um código computacional para escoamentos bidimensionais e tridimensionais empregando a formulação estabilizada SUPG/PSPG do MEF e a descrição ALE, o que possibilita a representação de contornos móveis. Assim como no sólido, a aproximação temporal é dada pelo integrador $\alpha$-generalizado e o método de Newton-Raphson é empregado para a solução do sistema não linear. Neste código é parte desta proposta a implementação de elementos Taylor-Hood, com aproximação quadrática para velocidades e linear para pressão, a implementação da técnica de estabilização \LSIC\ (LSIC) para captura de vórtices, assim como o modelo de turbulência LES e o modelo de estabilização VMS.

Após o estudo e verificação isolada das ferramentas para solução do fluido e do sólido, parte-se para o acoplamento. Para isso será adotado um modelo de acoplamento particionado forte do tipo bloco-iterativo, sendo a movimentação da malha do fluido realizada por meio de uma técnica de movimentação suave de Laplace, em que os elementos menores possuirão rigidez elevada, para evitar distorções excessivas, enquanto os elementos maiores absorverão a maior parte das deformações.

Será empregado o protocolo de processamento paralelo \textit{Message Passing Interface} (MPI), com utilização da biblioteca PETSc (\textit{Portable, Extensible Toolkit for Scientific Computation}) \cite{petsc-web-page}, a qual é reconhecidamente eficiente para solução de sistemas esparsos que necessitem de alto desempenho computacional, tendo suporte para C++ e MPI.  As malhas serão geradas a partir do programa Gmsh \cite{geuzaine2009gmsh}, e os resultados serão interpretados graficamente por meio do visualizador Paraview \cite{ahrens2005paraview} e por meio do programa gnuPlot, no caso da geração de gráficos de linha 2D ou 3D. Nota-se que se prioriza o uso de ferramentas computacionais livres e de código aberto para facilitar a distribuição e as atualizações da ferramenta computacional resultante.

%==================================================================================================
\section{Justificativa}
%==================================================================================================

Os avanços na área da engenharia colocam em evidência a necessidade cada vez maior de se determinar de forma precisa as variáveis necessárias ao dimensionamento de estruturas, sejam referentes à resistência dos materiais utilizados, ou às solicitações atuantes. Nesse sentido, em diversas ocasiões, os efeitos advindos da interação fluido-estrutura (IFE) são responsáveis por submeter estruturas a esforços consideráveis, o que demanda que esses efeitos sejam adequadamente estudados. Dessa forma, a análise desse tipo de problema via métodos computacionais se mostra promissor, uma vez que estudos experimentais normalmente são dispendiosos e pouco flexíveis quanto às suas aplicações.

Dentre os métodos empregados, pode-se destacar o dos Elementos Finitos (MEF), com aplicações tanto para modelar o sólido como para o fluido. Esse método ganha atenção, uma vez que pode se adequar com certa facilidade à problemas de geometria complexa, garantindo ainda facilidade para aplicação de condições de contorno.

Embora no contexto da mecânica dos fluidos, a utilização do MEF segundo o método de Bubnov-Galerkin possa gerar oscilações espúrias, decorrentes dos termos convectivos, de ondas de choque nos casos compressíveis, ou da interpolação da pressão por espaços de funções inadequados nos casos incompressíveis, todos esses problemas já dispõem de técnicas consistentes para que sejam evitados. Como exemplos, destacam-se a técnica SUPG para estabilização da convecção, a técnica PSPG ou o uso de elementos Taylor-Hood para estabilização da pressão nos escoamentos incompressíveis e a introdução de operadores de captura de choque para os casos compressíveis.

Entretanto, mesmo que diversos trabalhos tenham sido realizados, resultando em grandes avanços na área, alguns desafios se mantêm, como o elevado custo computacional relacionado à simulações de IFE, o que demanda muito tempo pra a obtenção de resultados, inviabilizando sua aplicação na etapa de elaboração de muitos projetos usuais de engenharia. Nesse cenário, problemas de escoamento turbulentos se tornam ainda mais custosos, devido à alguns fenômenos como a manifestação de vórtices, geralmente tridimensionais, de forma desordenada, instável e em uma grande amplitude de escalas. Assim necessita-se de malhas muito refinadas para capturar a ocorrência dessas estruturas. Para que a simulação desses problemas seja estável em qualquer nível de discretização, e possa resultar em respostas consistentes a custos computacionais aceitáveis, faz-se necessária a adoção de modelos de turbulência, como  aqueles baseados na decomposição de Reynolds (RANS) ou em grandes vórtices (LES), ou de técnicas multiescala adequadas.

Já no contexto da mecânica dos sólidos, diversas técnicas são possíveis para abordar o problema, \eg\ o MEF corrotacional. No entanto considerar rotações nodais finitas e uma matriz de massa variável podem dificultar a aplicação dessa abordagem. Sendo assim, a utilização do MEF baseado em posições se torna interessante, uma vez que considera intrinsecamente a não linearidade geométrica, além de possuir uma matriz de massa constante, facilitando o processo de integração temporal.

Para o acoplamento entre fluido e estrutura, o processo de acoplamento particionado se torna viável na presente proposta, uma vez que se pode utilizar diferentes códigos que funcionam independentemente entre si, tanto para solução de problemas de fluidos, quanto para problemas de sólidos, e que podem ser acoplados de maneira eficiente. Isso se justifica pela presença de códigos já desenvolvidos pelo grupo de pesquisa do Departamento de Engenharia de Estruturas (SET), capazes de resolver isoladamente problemas da mecânica dos sólidos e dos fluidos. Também se opta pelo processo de acoplamento particionado forte, uma vez que se mostra eficaz na resolução de problemas fortemente acoplados.

Além disso, a utilização de malhas móveis para o fluido, por meio da descrição ALE, se mostra como uma alternativa viável, uma vez que permite que a malha do fluido se movimente dinamicamente para acomodar a movimentação da interface fluido-sólido. Sendo a movimentação dada pelo esquema de movimentação suave de Laplace, com uma modificação que confere maior rigidez aos elementos menores da malha, o que previne distorções excessivas de elementos pequenos, mantendo, consequentemente, a qualidade da malha.

Sendo assim, o presente trabalho é justificado ao propor o desenvolvimento de uma ferramenta computacional eficiente e precisa, aplicada a problemas de IFE com escoamentos turbulentos com a opção de se aplicar modelo de turbulência para redução do custo computacional. Igualmente, fica justificado a proposta de estudar e comparar os diferentes modelos e estabilizações a serem implementadas quando da aplicação a problemas de IFE. A presente pesquisa visa ainda aprimorar as ferramentas computacionais que estão sendo desenvolvidas pelo grupo de pesquisa do SET da Escola de Engenharia de São Carlos (EESC) da Universidade de São Paulo (USP), ampliando seu leque de aplicações.

% Outro desafio que se pode destacar é referente aos problemas de acoplamento, uma vez que os modelos constitutivos que regem tanto os sólidos quanto os fluidos são distintos, assim como as variáveis empregadas para descrever o comportamento de cada um. Também verifica-se que o sistema de equações fundamental para o estudo desses problemas, ou não possui solução analítica, ou essa só pode ser obtida para casos simples e com hipóteses bastante simplificadoras, sem muita possibilidade de generalizações.