%==================================================================================================
\chapter{Conclusão} \label{Conclusao}
%==================================================================================================

O presente trabalho trata-se do estudo de ferramenta computacional que seja eficiente e precisa para a simulação numérica de problemas de interação fluido-estrutura com elevados números de Reynolds. Para tanto, tomou-se como referência uma formulação do método dos elementos para escoamentos incompressíveis com estabilização SUPG/PSPG em descrição ALE. Foram implementados a formulação variacional multiescala VMS, e o modelo de turbulência LES, ambos em descrição ALE. Os métodos de solução dos escoamentos incompressíveis foram acoplados de forma particionada forte com um programa para a análise não linear geométrica de estruturas de casca com formulação baseada em posições.

Partiu-se de códigos já desenvolvidos pelo grupo de pesquisa, sendo esses um programa para análise de escoamentos Newtonianos incompressíveis em descrição ALE, empregando formulação estabilizada do método dos elementos finitos, e um programa para análise dinâmica não linear geométrica de estruturas de casca com cinemática de Reissner, o qual emprega uma formulação dos elementos finitos baseada em posições.

A formulação adotada para a simulação do escoamento incompressível, baseia-se no Método dos Elementos Finitos (MEF), obtida partindo-se do empregando o método dos resíduos ponderados para a obtenção da forma fraca das equações governantes. Nesse ponto, observa-se que o emprego do método clássico de Galerkin (Bubnov-Galerkin) pode conduzir a soluções com variações espúrias e problemas de instabilidades. Uma das fontes de instabilidades advém da escolha dos espaços aproximadores para velocidade e pressão, a qual deve atender as condições de \LBB\ (LBB) para garantir a estabilidade do campo de pressão. No entanto, a satisfação dessas condições restringe consideravelmente a escolha dos espaços aproximadores, o que motiva o uso de elementos estabilizados, como a estabilização \PSPG\ (PSPG). Outra fonte de instabilidade surge quando o termo convectivo é dominante sobre o termo dissipativo, aumentando o caráter hiperbólico da equação, sendo necessário o uso de métodos de estabilização, como o \SUPG\ (SUPG).

Uma outra técnica que pode contornar essas instabilidades, e ainda introduzir termos adicionais, que atuam inclusive sobre problemas decorrentes de vorticidade, é o método variacional multiescala (VMS). Esse modelo faz a separação das grandes escalas do problema das pequenas escalas, e produz uma formulação estabilizada que engloba os mesmos termos decorrentes da estabilização SUPG/PSPG, alé de outros termos.

A discretização temporal do problema é obtida por meio do integrador $\alpha$-generalizado, o qual se torna interessante uma vez que é possível controlar facilmente as dissipações de alta-frequência, por meio de um parâmetro único fornecido pelo usuário.

A movimentação da malha é baseada na equação de Laplace, modificada de maneira a tornar os elementos menores mais rígidos em comparação com os maiores. Essa consideração é feita para manter a qualidade da malha, uma vez que evita distorções elevadas nos elementos menores, fazendo com que os elementos maiores sejam os principais responsáveis por absorver as deformações.

Por sua vez, o modelo de turbulência LES faz a decomposição dos campos envolvidos em parcelas filtrada e não filtrada, a partir da consideração de um filtro, aplicado sobre as variáveis do problema como uma convolução integral. O modelo faz a consideração que as variáveis não filtradas do escoamento possuem características isotrópicas e homogêneas, permitindo assim a sua modelagem. Dessa forma, o modelo adotado é o modelo de viscosidade de Smagorinsky, adicionando um termo de viscosidade de vórtice ao problema, em função da taxa de deformação do campo filtrado.

Em análises numéricas de escoamento com contornos fixos, verificou-se a influência da utilização do modelo VMS em comparação ao SUPG/PSPG, ambos com e sem a aplicação de LES, assim como a utilização de elementos finitos Taylor-Hood P2P1 que dispensam a estabilização da pressão. Notou-se que a aplicação do VMS melhorou ligeiramente os resultados em relação à formulação SUPG/PSPG. Já a aplicação do LES melhorou significativamente a qualidade dos resultados, principalmente à medida em que o número de Reynolds aumenta. Além disso também constatou-se que o LES melhorou a estabilidade do método em problemas com discretização espacial mais grosseira. Também foi possível observar que a utilização de elementos P2P1 levou à resultados de menor qualidade em comparação com a formulação PSPG, demandando uma discretização mais refinada para chegar a resultados semelhantes. 

Foram realizados também testes realizados em problemas com contornos móveis, no entanto, os benchmarks para problemas com contornos móveis encontrados na literatura não apresentam número de Reynolds muito elevados, de forma que não se observou grande influência do LES, sendo possível notar resultados levemente mais estáveis com o emprego do LES do que com a simulação direta. 

Os exemplos numéricos simulados foram comparados com aqueles presentes na literatura e com simulações feitas no \textit{software} ANSYS, empregando o elemento \textit{shell} 281, o qual também emprega a cinemática de Reissner-Mindlin. Assim todos os problemas, tanto em análise estática quanto dinâmica, resultaram em valores muito próximos aos valores de referência, mostrando a boa implementação do método.
Estudando-se a abordagem posicional do método dos elementos finitos aplicado à análise dinâmica de estruturas de cascas com grandes deslocamentos,  observou-se esta formulação é adequada para o tipo de problema de IFE que é abordado aqui, sendo, por meio das análise, concluído que o elemento triangular com funções de forma quadráticas do tipo polinômios de Lagrange, com 7 parâmetros nodais por nó, é adequado para as análises além de permitir acoplamento com nós coincidentes com os do fluido, ou seja, discretização da interface fluido-estrutura totalmente coincidente para fluido e sólido. Embora nos exemplos analisados a casca seja esbelta, com a adoção da cinemática de Reissner-Mindlin, é possível considerar também elementos espessos.

%Se tratando da formulação dos problemas da dinâmica dos sólidos, foram apresentados os fundamentos da cinemática dos corpos deformáveis, levando à obtenção da medida de deformação de Green-Lagrange, a qual é utilizada no presente estudo, assim como algumas relações interessantes para se chegar à descrição Lagrangiana do problema. Já a abordagem adotada para se obter soluções numéricas parte do princípio energético, em que as parcelas de energia envolvidas no problema são devidas à energia potencial das forças externas, a energia de deformação elástica e a energia cinética. Assim, utilizando os princípios variacionais, se busca determinar o estado de equilíbrio do sólido a partir da minimização da energia potencial total do sistema. Dessa forma, a formulação é feita por meio do MEF baseado em posições para elementos de casca com cinemática de Reissner-Mindlin. Os graus de liberdade do problema são posições nodais, componentes do vetor generalizado e uma variável adicionar inseria como enriquecimento do formulação, evitando o travamento volumétrico do elemento.


Por fim, o acoplamento dos códigos foi feito a partir do acoplamento particionado forte, uma vez que permite a utilização de códigos independentes de maneira facilitada. Esse tipo de acoplamento resolve todos os problemas envolvidos, da dinâmica dos fluidos, da estrutura e da movimentação da malha, dentro de um mesmo bloco iterativo. Esse processo se distingue do acoplamento fraco, pois permite a consideração de passos de tempo maiores, além de ser capaz de resolver problemas mais fortemente acoplados.

Nos exemplos numéricos percebe-se que a aplicação do LES no problema de cavidade com fundo flexível não influenciou significativamente nos resultados obtidos. Isso pode ser explicado pela necessidade de se ter um refinamento maior da malha na região de entrada e saída do escoamento, onde se tem a ocorrência dos maiores gradientes de velocidade. Isso faz com que a resposta se mantenha representativa mesmo sem o uso de LES. Já o problema de \textit{flutter} em painel flexível mostrou que, em simulação com malha grosseira, a simulação SUPG/PSPG sem a aplicação do LES não atingiu o equilíbrio dinâmico, enquanto, ao se aplicar o LES ao problema, essa solução já se tornou mais representativa. Por sua vez, em simulação VMS, a aplicação do LES atrasou um pouco a estacionariedade da solução, porém ainda se manteve coerente com o esperado.

Portanto se pode verificar que a aplicação do LES em problemas de iteração-fluido estrutura permite a utilização de malhas mais grosseiras, sem perda significativa de qualidade nos resultados, assim como uma melhora na estabilidade do método, reduzindo, consequentemente, o custo computacional. Além disso, a aplicação do LES em problemas de contornos móveis manteve a qualidade dos resultados em malhas com bom refinamento em regiões de maior gradiente de velocidade, enquanto em malhas mais grosseiras, a aplicação do LES melhorou a convergência do método, permitindo a obtenção de soluções mais representativas.

%==================================================================================================
%\section{Sugestões para trabalhos futuros} \label{Sugestoes}
%==================================================================================================

Para a continuidade da pesquisa, sugere-se para trabalhos futuros a implementação de um modelo de turbulência mais sofisticado, como o \textit{Dynamic Smagorinsky Model} (DSM), o qual trata o coeficiente de Smagorinsky não mais como uma constante, mas como função do espaço e do tempo, por meio da adoção de um novo filtro aplicado também sobre as funções teste. Assim esse coeficiente se adapta dinamicamente ao problema analisado, melhorando ainda mais os resultados.

Também se sugere a implementação de outros tipos de elementos sólidos, uma vez que só foi realizado o acoplamento com elementos de casca, tornando as análises de problemas de IFE mais gerais, ampliando a gama de problemas que podem ser simulados.