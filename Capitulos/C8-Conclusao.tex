%==================================================================================================
\chapter{Conclusão} \label{Conclusao}
%==================================================================================================

A proposta do presente estudo está no desenvolvimento de uma ferramenta computacional que seja eficiente e precisa para a simulação numérica de problemas de interação fluido-estrutura. Para tanto, foi realizado o estudo do modelo de turbulência LES, capaz de capturar estruturas turbulentas presentes na subescala do problema, bem como a implementação de um método numérico que fosse capaz de resolver as equações governantes do escoamento e da estrutura de maneira acoplada.

Para isso, utilizou-se códigos já desenvolvidos pelo grupo de pesquisa do Departamento de Engenharia de Estruturas (SET) da Escola de Engenharia de São Carlos (EESC) da Universidade de São Paulo (USP), os quais são capazes de resolver isoladamente tanto problemas de escoamentos incompressíveis quanto da dinâmica das estruturas.

Primeiramente observam-se as equações que governam os escoamentos isotérmicos incompressíveis em descrição Euleriana, para problemas de contornos fixos, tendo em vista os princípios da conservação de massa e quantidade de movimento, resultando, assim, nas equações de Navier-Stokes. Nesse cenário o modelo constitutivo adotado é o de fluido Newtoniano, caracterizado pela presença de viscosidade constante do fluido. Posteriormente a formulação é estendida para a descrição Lagrangiana-Euleriana Arbitrária (ALE), permitindo a movimentação independente do domínio de referência. Esse processo se torna interessante, uma vez que se torna possível a movimentação da malha computacional para se adequar aos deslocamentos sofridos pela estrutura submetida ao escoamento.

A obtenção de uma solução numérica do escoamento adotada, baseia-se no Método dos Elementos Finitos (MEF), onde se partiu da forma fraca das equações governantes empregando o método dos resíduos ponderados. Com isso são apresentadas as possíveis fontes de instabilidades que podem ocorrer ao utilizar tal abordagem no contexto de escoamentos incompressíveis. Uma das fontes de instabilidades advém da escolha dos espaços aproximadores, a qual deve atender as condições de \LBB\ (LBB) para garantir a estabilidade do campo de pressão. No entanto, a satisfação dessas condições restringe consideravelmente a escolha dos espaços aproximadores, o que motiva o uso de elementos estabilizados, como a estabilização \PSPG\ (PSPG). Outra fonte de instabilidade provém da domínio do termo convectivo sobre o termo dissipativo, aumentando o caráter hiperbólico da equação, sendo necessário o uso de métodos de estabilização, como o \SUPG\ (SUPG).

Uma outra possibilidade de se contornar essas instabilidades está no uso de modelos de estabilização, em que, no presente trabalho, foi adotado a formulação variacional multiescala (VMS). Esse modelo faz a separação das grandes escalas do problema das pequenas escalas. Uma característica interessante observada nesse modelo está na presença tanto dos termos SUPG/PSPG quanto termos adicionais próprios da formulação.

Já a discretização temporal do problema é feita a partir do integrador $\alpha$-generalizado, o qual se torna interessante uma vez que é possível controlar facilmente as dissipações de alta-frequência, por meio de um parâmetro único fornecido pelo usuário.

A movimentação da malha é feita a partir da solução do problema de Laplace, o qual foi modificado de maneira a tornar os elementos menores mais rígidos em comparação com os maiores. Essa consideração é feita para manter a qualidade da malha, uma vez que evita distorções elevadas nos pequenos elementos, fazendo com que os elementos grandes sejam os principais responsáveis por absorver as deformações.

Por sua vez, o modelo de turbulência LES faz a decomposição dos campos envolvidos em parcelas filtradas e não filtradas, a partir da consideração de um filtro, aplicado sobre as variáveis do problema como uma convolução integral. O modelo faz a consideração que as variáveis não filtradas do escoamento possuem características isotrópicas e homogêneas, permitindo, assim, sua modelagem. Dessa forma, o modelo adotado é o modelo de viscosidade de Smagorinsky, adicionando um termo de viscosidade de vórtice ao problema, calculado em função da taxa de deformação do campo filtrado.

Em análises numéricas de escoamento com contornos fixos, verificou-se a influência da utilização do modelo VMS em comparação ao SUPG/PSPG, ambos com e sem a aplicação de LES, assim como a utilização de elementos Taylor-Hood P2P1. Notou-se que a aplicação do VMS melhorou ligeiramente os resultados em relação à estabilização SUPG/PSPG. Já a aplicação do LES melhorou significativamente a qualidade dos resultados, principalmente conforme o número de Reynolds aumenta. Além disso também constatou-se que o LES melhorou a estabilidade do método em problemas com malha grosseira. Além disso, também foi possível observar que a utilização de elementos P2P1 levou à resultados com oscilações espúrias no campo de pressões, além de uma perda considerável de qualidade ao observar os coeficientes de arrasto e de sustentação em simulações com malha grosseira. Já em problemas com contornos móveis observou-se que a aplicação do LES modificou levemente os resultados obtidos. Essa pequena variação pode ser explicada pelos baixos números de Reynolds desenvolvidos, assim como uma malha ainda refinada nos pontos de maior gradiente de velocidades.

Se tratando da formulação dos problemas da dinâmica dos sólidos, foram apresentados os fundamentos da cinemática dos corpos deformáveis, levando à obtenção da medida de deformação de Green-Lagrange, a qual é utilizada no presente estudo, assim como algumas relações interessantes para se chegar à descrição Lagrangiana do problema. Já a abordagem adotada para se obter soluções numéricas parte do princípio energético, em que as parcelas de energia envolvidas no problema são devidas à energia potencial das forças externas, a energia de deformação elástica e a energia cinética. Assim, utilizando os princípios variacionais, se busca determinar o estado de equilíbrio do sólido a partir da minimização da energia potencial total do sistema. Dessa forma, a formulação é feita por meio do MEF baseado em posições para elementos de casca com cinemática de Reissner-Mindlin. Os graus de liberdade do problema são posições nodais, componentes do vetor generalizado e uma variável adicionar inseria como enriquecimento do formulação, evitando o travamento volumétrico do elemento.

Os exemplos numéricos simulados foram comparados com aqueles presentes na literatura e com simulações feitas no \textit{software} ANSYS, empregando o elemento \textit{shell} 281, o qual também emprega a cinemática de Reissner-Mindlin. Assim todos os problemas, tanto em análise estática quanto dinâmica, resultaram em valores muito próximos aos valores de referência, mostrando a boa implementação do método.

Por fim, o acoplamento dos códigos foi feito a partir do acoplamento particionado forte, uma vez que permite a utilização de códigos independentes de maneira facilitada. Esse tipo de acoplamento resolve todos os problemas envolvidos, da dinâmica dos fluidos, da estrutura e da movimentação da malha, dentro de um mesmo bloco iterativo. Esse processo se distingue do acoplamento fraco, pois permite a consideração de passos de tempo maiores, além de ser capaz de resolver problemas mais fortemente acoplados.

Nos exemplos numéricos percebe-se que a aplicação do LES no problema de cavidade com fundo flexível não influenciou significativamente nos resultados obtidos. Isso pode ser explicado pela necessidade de se ter um refinamento maior da malha na região de entrada e saída do escoamento, onde se tem a ocorrência dos maiores gradientes de velocidade. Isso faz com que a resposta se mantenha representativa mesmo sem o uso de LES. Já o problema de \textit{flutter} em painel flexível mostrou que, em simulação com malha grosseira, a simulação SUPG/PSPG sem a aplicação do LES não atingiu o equilíbrio dinâmico, enquanto, ao se aplicar o LES ao problema, essa solução já se tornou mais representativa. Por sua vez, em simulação VMS, a aplicação do LES atrasou um pouco a estacionariedade da solução, porém ainda se manteve coerente com o esperado.

Portanto se pode verificar que a aplicação do LES em problemas de iteração-fluido estrutura permite a utilização de malhas mais grosseiras, sem perda significativa de qualidade nos resultados, assim como uma melhora na estabilidade do método, reduzindo, consequentemente, o custo computacional. Além disso, a aplicação do LES em problemas de contornos móveis manteve a qualidade dos resultados em malhas com bom refinamento em regiões de maior gradiente de velocidade, enquanto em malhas mais grosseiras, a aplicação do LES melhorou a convergência do método, permitindo a obtenção de soluções mais representativas.

%==================================================================================================
%\section{Sugestões para trabalhos futuros} \label{Sugestoes}
%==================================================================================================

Para a continuidade da pesquisa, sugere-se para trabalhos futuros a implementação de um modelo de turbulência mais sofisticado, como o \textit{Dynamic Smagorinsky Model} (DSM), o qual trata o coeficiente de Smagorinsky não mais como uma constante, mas como função do espaço e do tempo, por meio da adoção de um novo filtro aplicado também sobre as funções teste. Assim esse coeficiente se adapta dinamicamente ao problema analisado, melhorando ainda mais os resultados.

Também se sugere a implementação de outros tipos de elementos sólidos, uma vez que só foi realizado o acoplamento com elementos de casca, tornando as análises de problemas de IFE mais gerais, ampliando a gama de problemas que podem ser simulados.