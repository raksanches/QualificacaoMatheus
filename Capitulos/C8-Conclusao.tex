%==================================================================================================
\chapter{Conclusão} \label{Conclusao}
%==================================================================================================

O presente trabalho objetivou o estudo de ferramentas computacionais eficientes e precisas para a simulação numérica de problemas de interação fluido-estrutura com elevados números de Reynolds. Especificamente, são considerados os casos de escoamentos incompressíveis Newtonianos interagindo com estruturas com comportamento elástico em grandes deslocamentos. Embora os problemas de interação fluido-estrutura estejam muito presentes na engenharia, e as pesquisas relativas aos métodos numéricos para os estudos desses problemas tenham atingido certa maturidade, há ainda diversos desafios que geram custo computacional muito elevando para diversas situações, sendo os escoamentos com turbulência uma das fontes deste custo computacional. Neste trabalho tomou-se como referência uma formulação do método dos elementos para escoamentos incompressíveis com estabilização da convecção SUPG e com estabilização da pressão PSPG, em descrição ALE. Foram então implementados a formulação variacional multiescala VMS, e o modelo de turbulência LES, ambos em descrição ALE. Para a modelagem da estrutura, empregou-se uma abordagem do método dos elementos finitos baseada em posições para a análise não linear geométrica de estruturas de casca, sendo o acoplamento com o escoamento feito de forma particionada forte.

Partiu-se de códigos já desenvolvidos pelo grupo de pesquisa, sendo esses um programa básico para análise de escoamentos Newtonianos incompressíveis em descrição ALE, e um programa para análise dinâmica não linear geométrica de estruturas de casca com cinemática de Reissner.

Inicia-se com o estudo da formulação estabilizada do MEF para a simulação de escoamentos incompressíveis, a qual é obtida partindo-se do método dos resíduos ponderados para a obtenção da forma fraca das equações governantes. Nesse ponto, observa-se que o emprego do método clássico de Galerkin (Bubnov-Galerkin) pode conduzir a soluções com variações espúrias e problemas de instabilidades. Uma das fontes de instabilidades advém da escolha dos espaços aproximadores e teste para velocidade e pressão, a qual deve atender as condições de \LBB\ (LBB) para garantir a estabilidade do campo de pressão. No entanto, a satisfação dessas condições restringe consideravelmente a escolha dos espaços aproximadores, o que motiva o uso de elementos estabilizados, como a estabilização \PSPG\ (PSPG). Outra fonte de instabilidade surge quando o termo convectivo é dominante sobre o termo dissipativo, aumentando o caráter hiperbólico da equação, sendo necessário o uso de métodos de estabilização, como o \SUPG\ (SUPG). Combinando-se a formulação estabilizada SUPG/PSPG, aplicada à descrição ALE das equações de Navier-Stokes para escoamentos incompressíveis, com o método $\alpha$-generalizado de marcha no tempo, obtém-se a formulação numérica de referência adotada para a mecânica dos fluidos neste trabalho.

A descrição ALE demanda um modelo de movimentação da malha que garanta a conformidade do domínio computacional do fluido com os contornos móveis, bem como consistência da malha no interior do domínio. Assim, emprega-se um modelo baseado equação de Laplace, a qual é modificada de maneira a tornar os elementos menores mais rígidos em comparação com os maiores. Essa modificação visa manter a qualidade da malha, uma vez que evita distorções elevadas nos elementos menores, fazendo com que os elementos maiores sejam os principais responsáveis por absorver as deformações.

Na sequência, implementa-se o método variacional multiescala (VMS). Esse modelo faz a separação das grandes escalas do problema das pequenas escalas, e produz uma formulação estabilizada que engloba os mesmos termos decorrentes da estabilização SUPG/PSPG, além de introduzir termos adicionais, que atuam inclusive sobre problemas decorrentes de vorticidade.

Visando a obtenção de uma formulação mais eficiente para tratar os problemas decorrentes da turbulência, é implementado o modelo LES, que faz a decomposição dos campos envolvidos em parcelas filtrada e não filtrada, a partir da consideração de um filtro aplicado sobre as variáveis do problema como uma convolução integral. O modelo faz a consideração de que as variáveis não filtradas do escoamento possuem características isotrópicas e homogêneas, permitindo assim a sua modelagem. O modelo de filtro adotado é o modelo de viscosidade de Smagorinsky, adicionando-se um termo de viscosidade de vórtice ao problema, dada em função da taxa de deformação do campo filtrado.

Por meio do estudo de exemplos de análises numéricas de escoamento com contornos fixos, verificou-se a influência da utilização do modelo VMS em comparação ao SUPG/PSPG, ambos com e sem a aplicação de LES, assim como a utilização de elementos finitos Taylor-Hood P2P1 que dispensam a estabilização da pressão. Nota-se que a aplicação do VMS melhora ligeiramente os resultados em relação à formulação SUPG/PSPG. Já a aplicação do LES melhora significativamente a qualidade dos resultados, principalmente à medida em que o número de Reynolds aumenta. Além disso também constata-se que o LES melhora a estabilidade do método em problemas com discretização espacial mais grosseira. Também é possível observar que a utilização de elementos P2P1 leva a resultados de menor qualidade em comparação com a formulação PSPG, demandando uma discretização mais refinada para chegar a resultados semelhantes. Com relação a problemas com contornos móveis, são considerados os exemplos de um cilindro com deslocamento prescrito e de um aerofólio com movimento oscilatório prescrito. Esses problemas são muito empregados para a verificação de programas para escoamentos com contornos móveis, e permitiram notar que as formulações implementadas são eficientes na simulação desses problemas. No entanto, por não apresentarem número de Reynolds muito elevados, não se observa grande influência do LES, sendo possível apenas notar de maneira qualitativa resultados levemente mais estáveis com o emprego do LES do que com a simulação direta. Já a simulação VMS apresentou uma diferença sutil nos resultados em relação ao SUPG/PSPG, no entanto, mais próximo aos resultados de referência.

Para permitir a compreensão da abordagem adotada para a solução da mecânica das estruturas, permitindo um acoplamento adequado, a formulação posicional do MEF para cascas de Reissner-Mindlin é descrita. Essa formulação baseia-se na aplicação do princípio da estacionariedade da energia considerando-se as posições nodais e vetores generalizados como parâmetros nodais. Escolhe-se o elementos finito triangular com funções de forma quadráticas do tipo polinômios de Lagrange. Exemplos numéricos são simulados e os resultados comparados com a literatura e com simulações feitas no programa \citeonline{ansys2016ansys}, empregando o elemento \textit{shell} 281, o qual também emprega a cinemática de Reissner-Mindlin. Nota-se que todos os problemas, tanto em análise estática quanto dinâmica, resultam em valores muito próximos aos de referência. Conclui-se que a formulação e o elemento adotados são adequados para o tipo de problema de IFE que é abordado aqui. Embora nos exemplos analisados a casca seja esbelta, com a adoção da cinemática de Reissner-Mindlin, é possível considerar também elementos espessos.

%Se tratando da formulação dos problemas da dinâmica dos sólidos, foram apresentados os fundamentos da cinemática dos corpos deformáveis, levando à obtenção da medida de deformação de Green-Lagrange, a qual é utilizada no presente estudo, assim como algumas relações interessantes para se chegar à descrição Lagrangiana do problema. Já a abordagem adotada para se obter soluções numéricas parte do princípio energético, em que as parcelas de energia envolvidas no problema são devidas à energia potencial das forças externas, a energia de deformação elástica e a energia cinética. Assim, utilizando os princípios variacionais, se busca determinar o estado de equilíbrio do sólido a partir da minimização da energia potencial total do sistema. Dessa forma, a formulação é feita por meio do MEF baseado em posições para elementos de casca com cinemática de Reissner-Mindlin. Os graus de liberdade do problema são posições nodais, componentes do vetor generalizado e uma variável adicionar inseria como enriquecimento do formulação, evitando o travamento volumétrico do elemento.

Por fim, seguindo \citeonline{bazilevs2013computational}, parte-se do sistema de equações esperado para o problema monolítico de IFE, e, desprezando-se parcelas da matriz tangente, constrói-se um modelo particionado de acoplamento forte. Esse modelo resulta em solução do tipo Gauss-Seidel do problema acoplado, com 3 blocos que são resolvidos sequencialmente em cada iteração, um para o fluido, outro para a estrutura e outro para a malha, permitindo a utilização de códigos independentes de maneira facilitada. Nota-se que, como o resíduo do problema acoplado não é modificado, havendo convergência, ela ocorre sempre para a solução exata do modelo numérico, no entanto, como a matriz tangente é modificada, não há garantia sobre a ordem de convergência, podendo, inclusive, não haver convergência a depender das características físicas dos meios fluido e sólido considerado, que podem conduzir a um problema fortemente acoplado (quando pequenas perturbação em um dos meios gera grandes perturbações no outro). Há na literatura maneiras eficientes de se tratar esse problema, mas que não fazem parte do escopo deste trabalho, notando-se o método \textit{Augmented} A22, ou relaxações de Aitken.

Considerou-se três exemplos numéricos de interação fluido-estrutura, todos com formação de vórtices, iniciando-se com problemas mais simples e partindo-se para casos mais complexos. No problema \ref{FSI-Cavity2D}, que trata-se de uma cavidade com fundo flexível, notou-se que a formulação apenas com as estabilizações SUPG e PSPG é suficiente para a simulação precisa e estável do problema, sendo que nem o modelo multiescala VMS e nem o modelo de turbulência LES influenciaram  significativamente nos resultados obtidos. Já para o problema de \textit{flutter} em painel flexível (\ref{FSI-prism}) foi possível observar que, em simulação com malhas grosseiras, a simulação SUPG/PSPG, sem a aplicação do LES, não atingiu a convergência, enquanto, ao se aplicar o LES ao problema, essa solução já se aproximou mais dos valores de referência. Por sua vez, em simulação LES-VMS houve um atraso para que o ciclo limite fosse atingido em relação à simulação somente com VMS, porém ainda se manteve coerente com o esperado. Já, comparando-se as simulações sem aplicação de LES, a simulação VMS apresentou resultados mais próximos dos valores de referência, enquanto a SUPG/PSPG não atingiu a convergência.

Conclui-se que aplicação do LES em problemas de iteração-fluido estrutura permite a utilização de malhas mais grosseiras, sem perda significativa de qualidade nos resultados, assim como uma melhora na estabilidade do método, reduzindo, consequentemente, o custo computacional.
%{\color{green}Além disso, a aplicação do LES em problemas de contornos móveis manteve a qualidade dos resultados em malhas com bom refinamento em regiões de maior gradiente de velocidade, enquanto em malhas mais grosseiras, a aplicação do LES melhorou a convergência do método, permitindo a obtenção de soluções mais representativas.} \textcolor{red}{Não está bem explicado o que você quer dizer aqui. O que você está acrescentando em relação à primeira parte do parágrafo?}

Para a continuidade da pesquisa, sugere-se como trabalhos futuros a implementação de um modelo de turbulência mais sofisticado, como o \textit{Dynamic Smagorinsky Model} (DSM) \cite{germano1991dynamic}, o qual trata o coeficiente de Smagorinsky não mais como uma constante, mas como função do espaço e do tempo, por meio da adoção de um novo filtro aplicado também sobre as funções teste. Assim esse coeficiente se adapta dinamicamente ao problema analisado, sem a necessidade de calibração por parte do usuário e minimizando a influência do LES próximo à paredes, ou em regiões de relaminarização.

Sugere-se ainda a consideração de outas técnicas de discretização, como o uso da análise isogeométrica, tanto para o fluido como para o sólido, além do emprego de outros elementos para representar a estrutura, tal como elementos sólidos, barras e outros elementos de casca, torando a ferramenta computacional mais geral. Por fim, sugere-se a aplicação dos modelos de turbulência juntamente com métodos de malhas fixas (com contorno imerso) para interação fluido-estrutura, ampliando-se a gama de problemas que podem ser simulados.