%==================================================================================================
\chapter{Acoplamento Fluido-Estrutura} \label{AFE}
%==================================================================================================

Para se descrever o problema acoplado, denota-se $\Omega_F$ o domínio do fluido, $\Omega_E$ o domínio da estrutura, $\Omega_{IFE}=\Omega_S\cup\Omega_E$ o domínio do problema e $\Gamma_{IFE}=\Omega_S\cap\Omega_E$ a interface de IFE.

\citeonline{richter2017fluid} apontam três condições que devem ser satisfeitas no acoplamento: a Condição Cinemática, que diz respeito à movimentação dos domínios analisados, devendo ser compatíveis em $\Gamma_{IFE}$, ou seja, a componente normal ao movimento deve ser igual em ambos os meios, assim como a componente tangencial em caso de condição de aderência do fluido à estrutura; a Condição Dinâmica, que aponta a continuidade das forças internas, observadas no tensor de tensões de Cauchy; e a Condição Geométrica, que exige a necessidade de ambos os domínios coincidirem em $\Gamma_{IFE}$, não havendo sobreposições nem formação de vazios.

Numericamente, há diversas possibilidades para que essas condições sejam impostas de forma exata ou aproximada, sendo possível acoplamento monolítico ou acoplamento particionado, com o último ainda subdividido em particionado fraco e forte, como mencionado em \ref{IFE}. % \textcolor{red}{(Citar trecho no estado da arte)}.

Nas formulações adotadas para CFD e CSD, é comumente empregado o Método de Newton-Raphson para o cálculo das variáveis incógnitas. O mesmo pode ser aplicado quando do acoplamento direto, monolítico, entre os dois meios. Para isso a matriz tangente ($\BB{H}$) pode ser obtida por \cite{bazilevs2013computational,sanches2022metodos}:

\begin{equation}
    H_{ij}^k=\Dder{\script{G}}{\beta_i^k}{\alpha_j^k}\text{,}
\end{equation}

\noindent sendo $\script{G}$ a soma de todas as equações diferenciais do problema em sua forma fraca, $\beta_i^k$ o vetor com todos os parâmetros nodais das funções teste e $\alpha_j^k$ o vetor com todos os parâmetros nodais incógnitas do problema. Assim, obtém-se a correção dos valores das variáveis ($\Delta\BB{\alpha}^k$) por meio da solução do sistema:

\begin{equation}
    \BB{H}^k\cdot\Delta\BB{\alpha}^k=-\BB{h}^k\text{,}
\end{equation}

\noindent em que $\BB{h}=\partial\script{G}/\partial\beta_i^k$ é o vetor resíduo.

Expandindo a matriz em submatrizes a fim de visualizar a contribuição de cada parcela no sistema global tem-se:

\begin{equation}
    \begin{bmatrix}
        \BB{H}_{11}^k & \BB{H}_{12}^k & \BB{H}_{13}^k \\
        \BB{H}_{21}^k & \BB{H}_{22}^k & \BB{H}_{23}^k \\
        \BB{H}_{31}^k & \BB{H}_{32}^k & \BB{H}_{33}^k
    \end{bmatrix}\cdot\begin{bmatrix}
        \Delta\BB{\alpha}_1^k \\\Delta\BB{\alpha}_2^k\\\Delta\BB{\alpha}_3^k
    \end{bmatrix}=-\begin{bmatrix}
        \BB{h}_1^k \\\BB{h}_2^k\\\BB{h}_3^k
    \end{bmatrix}\text{,}
\end{equation}

\noindent sendo os subíndices $1$, $2$ e $3$ referentes às variáveis do fluido, da estrutura e da malha, respectivamente.

Já o acoplamento particionado trata o sistema de forma a eliminar os termos cruzados da matriz tangente, resultando em blocos de sistema que podem ser resolvidos independentemente:

\begin{equation}
    \begin{bmatrix}
        \BB{H}_{11}^k & 0             & 0             \\
        0             & \BB{H}_{22}^k & 0             \\
        0             & 0             & \BB{H}_{33}^k
    \end{bmatrix}\cdot\begin{bmatrix}
        \Delta\BB{\alpha}_1^k \\\Delta\BB{\alpha}_2^k\\\Delta\BB{\alpha}_3^k
    \end{bmatrix}=-\begin{bmatrix}
        \BB{h}_1^k \\\BB{h}_2^k\\\BB{h}_3^k
    \end{bmatrix}\text{.}
\end{equation}

Note que o método continua sendo consistente, uma vez que apenas a matriz tangente do método de Newton-Raphson foi modificada, no entanto, a convergência não é garantida.
Para aprimorar os resultados, em cada iteração $k$ do processo de Newton-Raphson, pode-se resolver sequencialmente os blocos e atualizar os valores calculados em um bloco para o cálculo do próximo, sendo obtido pelo procedimento apresentado no Algoritmo \ref{alg:AcoplamentoParticionado}. Isso resulta em um acoplamento forte, onde, havendo convergência, a mesma resposta do problema fortemente acoplado deve ser alcançada.


\begin{algorithm}[h!]
    \caption{Cálculo dos valores das variáveis incógnitas}
    \label{alg:AcoplamentoParticionado}
    Resolver o sistema: $\BB{H}_{11}^k(\BB{\alpha}_1^k,\BB{\alpha}_2^k,\BB{\alpha}_3^k)\cdot\Delta\BB{\alpha}_1^k=-\BB{h}_1^k(\BB{\alpha}_1^k,\BB{\alpha}_2^k,\BB{\alpha}_3^k)$\;
    Atualizar valores: $\BB{\alpha}_1^{k+1}\gets\BB{\alpha}_1^k+\Delta\BB{\alpha}_1^k$\;
    Resolver o sistema: $\BB{H}_{22}^k(\BB{\alpha}_1^{k+1},\BB{\alpha}_2^k,\BB{\alpha}_3^k)\cdot\Delta\BB{\alpha}_2^k=-\BB{h}_2^k(\BB{\alpha}_1^{k+1},\BB{\alpha}_2^k,\BB{\alpha}_3^k)$\;
    Atualizar valores: $\BB{\alpha}_2^{k+1}\gets\BB{\alpha}_2^k+\Delta\BB{\alpha}_2^k$\;
    Resolver o sistema: $\BB{H}_{33}^k(\BB{\alpha}_1^{k+1},\BB{\alpha}_2^{k+1},\BB{\alpha}_3^k)\cdot\Delta\BB{\alpha}_3^k=-\BB{h}_3^k(\BB{\alpha}_1^{k+1},\BB{\alpha}_2^{k+1},\BB{\alpha}_3^k)$\;
    Atualizar valores: $\BB{\alpha}_3^{k+1}\gets\BB{\alpha}_3^k+\Delta\BB{\alpha}_3^k$\;
\end{algorithm}


Caso os blocos sejam resolvidos em processos iterativos separados, sem que haja atualização das variáveis de um problema para o outro dentro de cada processo iterativo, fazendo a atualização apenas ao final de cada passo no tempo, tem-se um modelo de acoplamento particionado fraco. Para construir um acoplamento particionado fraco, pode-se assumir que o problema do fluido depende das variáveis do sólido e da malha em um instante $t$ e das variáveis do fluido em um instante $t+1$, já o problema do sólido depende do valor das variáveis do fluido e do sólido em um tempo $t+1$ e, por fim, o problema da malha depende das variáveis do sólido e da malha em um instante $t+1$. Assim, o cálculo é realizado segundo o Algoritmo \ref{alg:PartFraco}, no qual $k_f$, $k_s$ e $k_m$ são as iterações das soluções dos problemas de fluido, estrutura e malha, respectivamente \cite{sanches2022metodos}. Ainda é observado que a movimentação da malha segue um problema linear, sendo seu resultado obtido diretamente em um iteração.

%\textcolor{red}{Você precisa modificar o algoritmo para identificar que 1 e 2 estão dentro do processo iterativo para resolver o fluido, 3 e 4 dentro do processo iterativo para resolver o sólido, e 5 e 6 dentro do processo iterativo para resolver a malha}

\begin{algorithm}[h!]
    \caption{Cálculo das variáveis em um acoplamento particionado fraco}
    \label{alg:PartFraco}

    \While{erro>tol}{
    Resolver o sistema: $\BB{H}_{11}((\BB{\alpha}_1^{k_f})^{t+1},(\BB{\alpha}_2)^t,(\BB{\alpha}_3)^t)\cdot\Delta(\BB{\alpha}_1^{k_f})^{t+1}=-\BB{h}_1((\BB{\alpha}_1^{k_f})^{t+1},(\BB{\alpha}_2)^t,(\BB{\alpha}_3)^t)$\;
    Atualizar valores: $(\BB{\alpha}_1^{k_f+1})^{t+1}\gets(\BB{\alpha}_1^{k_f})^{t+1}+(\Delta\BB{\alpha}_1^{k_f})^{t+1}$\;
    $k_f\gets k_f+1$
    }

    \While{erro>tol}{
    Resolver o sistema: $\BB{H}_{22}((\BB{\alpha}_1)^{t+1},(\BB{\alpha}_2^{k_s})^{t+1})\cdot\Delta(\BB{\alpha}_2^{k_s})^{t+1}=-\BB{h}_2((\BB{\alpha}_1)^{t+1},(\BB{\alpha}_2^{k_s})^t)$\;
    Atualizar valores: $(\BB{\alpha}_2^{k_s+1})^{t+1}\gets(\BB{\alpha}_2^{k_s})^{t+1}+(\Delta\BB{\alpha}_2^{k_s})^{t+1}$\;
    $k_s\gets k_s+1$
    }

    \While{erro>tol}{
    Resolver o sistema: $\BB{H}_{3}((\BB{\alpha}_3^{k_m})^{t+1},(\BB{\alpha}_2)^t)\cdot\Delta(\BB{\alpha}_3^{k_s})^{t+1}=-\BB{h}_3((\BB{\alpha}_3^{k_m})^{t+1},(\BB{\alpha}_2)^t)$\;
    Atualizar valores: $(\BB{\alpha}_3^{k_m+1})^{t+1}\gets(\BB{\alpha}_3^{k_m})^{t+1}+(\Delta\BB{\alpha}_3^{k_m})^{t+1}$\;
    $k_m\gets k_m+1$
    }
\end{algorithm}

Por dispensar etapas de correção, esse método possui um custo computacional menor em relação ao particionado forte, no entanto exige a adoção de passos de tempo pequenos e fica sujeito a problemas de instabilidades como mencionado em \citeonline{Felippaetal2001}. Nota-se que esse tipo de acoplamento é mais adequado a problemas com escoamento compressível, onde um passo de tempo pequeno é necessário para se capturar a propagação de ondas de choque ao mesmo tempo em que a massa específica do fluido costuma ser muito menor que a do sólido.

Por outro lado, a convergência do processo de acoplamento particionado forte pode ficar prejudicada, especialmente à medida em que a massa específica do fluido e do sólido se aproximam, ou em problemas fortemente acoplados, onde uma pequena perturbação do fluido pode produzir grandes perturbações na estrutura e vice-versa. Como meios de garantir convergência nessa situação, mantendo o acoplamento particionado, pode-se adotar relaxações de Aitken, destinadas melhorar a convergência no método de Gauss-Seidel, na atualização das variáveis \cite{fernandes2019ale} ou aplicar um fator de escala sobre a matriz de massa do sólido (técnica \textit{Augmented $A_{22}$}) \cite{bazilevs2013computational}.