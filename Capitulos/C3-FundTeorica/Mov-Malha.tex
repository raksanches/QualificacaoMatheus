%==================================================================================================
\chapter{Movimentação da malha via suavização de Laplace} \label{MovMalha}
%==================================================================================================

Para a solução de problemas de IFE por meio da descrição ALE, se faz necessária a movimentação do domínio computacional do fluido de forma a acomodar a movimentação de estrutura. Tal tarefa deve ser realizada mantendo-se a qualidade da malha utilizada para o cálculo dos passos de tempo posteriores, evitando, assim, distorções excessivas nos elementos. Como mencionado por \citeonline{kanchi20073d}, essa etapa deve produzir uma malha de boa qualidade ao longo do tempo, ao mesmo tempo que não deve ser custoso computacionalmente. Assim, uma possibilidade para se realizar a movimentação da malha parte da suavização de Laplace, o qual foi estudado tanto em casos bidimensionais \cite{masud2007adaptive}, quanto tridimensionais \cite{kanchi20073d}.

Nesse esquema, considere o domínio limitado aberto do fluido $\Omega_F$, com uma fronteira suave por partes $\Gamma$. Logo, $\Gamma$ pode ser decomposta em partes de fronteira fixa $\Gamma_f$ e partes de fronteira móvel $\Gamma_m$, tais que:

\begin{subequations}
    \begin{equation}
        \Gamma=\Gamma_m\cup\Gamma_f\text{ e}
    \end{equation}
    \begin{equation}
        \varnothing=\Gamma_m\cap\Gamma_f\text{.}
    \end{equation}
\end{subequations}

Dessa forma, define-se as condições de contorno do problema como:

\begin{equation}
    \left\{
    \begin{array}{ll}
        \BB{z}=\BB{z}_m & \text{ em }\Gamma_m\text{ e} \\
        \BB{z}=\BB{0}   & \text{ em }\Gamma_f\text{,}
    \end{array}
    \right.
\end{equation}

\noindent em que $\BB{z}$ é o campo de deslocamentos da malha e $\BB{z}_m$ é o deslocamento prescrito em $\Gamma_f$.

Em seu trabalho, \citeonline{masud2007adaptive} aprontam que a simples utilização de uma equação do tipo $\Lapl\BB{z}=\BB{0}$ em $\Omega_F$ apresenta uma boa movimentação da malha em situação onde os elementos possuem tamanhos semelhantes entre si, assim como $\BB{z}_m$ não apresenta uma ordem de grandeza muito superior ao tamanho dos elementos na interface $\Gamma_m$. No entanto, como destacado pelos autores, a malha do fluido possui um refinamento maior em áreas de interesse da análise, o que coincide, em muitos casos, com a região próxima à $\Gamma_m$. Assim, se propõe uma modificação na formulação para que os elementos menores não sofram grandes distorções e inversões, ao passo que os elementos maiores absorverão tais distorções.

Assim, propõe-se um esquema de movimentação dado pelo problema de valor de contorno:

\begin{equation}
    \left\{
    \begin{array}{ll}
        \Ny\cdot([1+\eta]\Ny)\BB{z}=0 & \text{ em }\Omega_F\text{,}  \\
        \BB{z}=\BB{z}_m               & \text{ em }\Gamma_m\text{ e} \\
        \BB{z}=\BB{0}                 & \text{ em }\Gamma_f\text{,}
    \end{array}
    \right.\label{eq:LaplMov}
\end{equation}

\noindent em que:

\begin{equation}
    \eta^e=\frac{V_\mathrm{máx}-V_\mathrm{mín}}{V^e}
    \label{eq:movStiff}
\end{equation}

\noindent é um termo adicionado ao problema de maneira a aumentar a rigidez dos elementos menores, enquanto possibilita a movimentação dos elementos maiores, $V_\mathrm{máx}$, $V_\mathrm{mín}$ e $V^e$ são os volumes atuais do maior elemento, do menor elemento e do elemento $e$ calculado.

Para a solução do problema \eqref{eq:LaplMov}, parte-se para o método dos resíduos ponderados, em que $\BB{\upsilon}$ é a função teste:

\begin{equation}
    \SintDom{[1+\eta^e]\BB{\upsilon}\cdot\Lapl\BB{z}}=0\text{.}
\end{equation}

Integrando por partes e aplicando o teorema da divergência tem-se que:

\begin{equation}
    \SintFront{[1+\eta^e]\BB{\upsilon}\cdot(\Ny\BB{z}\cdot\BB{n})}-\SintDom{[1+\eta]\Ny\BB{\upsilon}:\Ny\BB{z}}=0\text{.}
\end{equation}

Por se tratar de um problema de fronteira de Dirichlet, a primeira parcela é nula, o que leva ao seguinte problema:

\begin{equation}
    \SintDom{[1+\eta^e]\Ny\BB{\upsilon}:\Ny\BB{z}}=0\text{.}
\end{equation}

Fazendo a aproximação da função teste por funções de forma, tem-se:

\begin{equation}
    \sum_{e=1}^{n_{el}}{\BB{\Upsilon}_a\cdot\intDome{[1+\eta^e]\Ny N_a\cdot\Ny\BB{z}}}\text{,}
\end{equation}

\noindent sendo $\BB{\Upsilon}_a$ e $N_a$ o valor nodal da função teste e a função de forma sobre o nó $a$.

Devido à arbitrariedade da função teste, tem-se, portanto, a definição do vetor resíduo do problema de Laplace:

\begin{equation}
    \res_L^a=\intDome{[1+\eta^e]\Ny N_a\cdot\Ny\BB{z}}=0\text{.}
    \label{eq:resLapl}
\end{equation}

Para se determinar o vetor $\BB{Z}$, referente aos deslocamentos nodais da malha, parte-se para o método de Newton-Raphson. Assim, toma-se como variáveis a serem corrigidas as acelerações nodais, que, a partir de uma aproximação por $\alpha$-generalizado tem-se a seguinte matriz do problema:

\begin{equation}
    \BB{B}^{ab}=\der{\res_L^a}{\ddBB{Y}^b}=\alpha_f\beta\Delta t^2\intDome{[1+\eta^e]\Ny N_a\cdot\Ny N_b}\BB{I}\text{.}
    \label{eq:movMatriz}
\end{equation}

Portanto o problema de movimentação da malha no contexto do MEF pode ser dado por:

\begin{equation}
    \BB{B}\Delta\ddBB{Y}=-\res_L\text{,}
    \label{eq:sistLapl}
\end{equation}

\noindent sujeito às seguintes condições de contorno:

\begin{equation}
    \left\{
    \begin{array}{ll}
        \ddBB{y}=\ddBB{y}_m & \text{ em }\Gamma_m \\
        \ddBB{y}=\BB{0}     & \text{ em }\Gamma_f
    \end{array}
    \right.
\end{equation}

Assim, o procedimento utilizado para a movimentação da malha é apresentado no algoritmo \ref{alg:movMalha}.

\begin{algorithm}[h!]
    \caption{Algoritmo utilizado para realizar a movimentação da malha.}
    \label{alg:movMalha}
    \ForEach{\textnormal{passo de tempo}}{
    Previsão dos valores nodais:\newline
    $
        \begin{bmatrix}
            \BB{y}^{n+1} \\\dBB{y}^{n+1}\\\ddBB{y}^{n+1}
        \end{bmatrix}^0\gets
        \begin{bmatrix}
            \BB{y}^n                                                                                    \\
            \bigpar{1-\frac{\gamma}{\beta}}\dBB{y}^n+\Delta t\bigpar{1-\frac{\gamma}{2\beta}}\ddBB{y}^n \\
            -\frac{\dBB{y}^n}{\beta\Delta t}+\bigpar{1-\frac{1}{2\beta}}\ddBB{y}^n
        \end{bmatrix}
    $\\
    Atualização dos valores da fronteira\;
    \ForEach{\textnormal{iteração de Newton-Raphson}}{
    \ForEach{\textnormal{elemento}}{
        Interpolação das variáveis: $\BB{z}^{n+\alpha_f}\gets\alpha_f\BB{y}^{n+1}+(1-\alpha_f)\BB{y}^n-\BB{x}$\\
        Cálculo da matriz tangente \eqref{eq:movMatriz} e do vetor resíduo \eqref{eq:resLapl}\;
    }
    Montagem do sistema global e aplicação das condições de contorno\;
    Determinar as correções nodais pela solução do sistema \eqref{eq:sistLapl}\;
    Correção das acelerações: $(\ddBB{y}^{n+1})^{k+1}\gets(\ddBB{y}^{n+1})^k+\Delta(\ddBB{y}^{n+1})^k$\;
    Correção das posições e velocidades:\newline
    $
        \begin{bmatrix}
            \BB{y}^{n+1} \\\dBB{y}^{n+1}
        \end{bmatrix}^{k+1}\gets
        \begin{bmatrix}
            \BB{y}^n+\dBB{y}^n\Delta t+\Delta t^2\bigpar{\bigpar{\frac{1}{2}-\beta}\ddBB{y}^n+\beta\ddBB{y}^{n+1}} \\
            \dBB{y}^n+\Delta t\bigpar{(1-\gamma)\ddBB{y}^n+\gamma\ddBB{y}^{n+1}}
        \end{bmatrix}^{k+1}
    $\\
    Cálculo da medida de convergência $\epsilon$\;
    \lIf{$\epsilon<\mathrm{tol}$}{Sair do \textit{loop}}
    }

    Atualização dos valores passados\;
    }
\end{algorithm}