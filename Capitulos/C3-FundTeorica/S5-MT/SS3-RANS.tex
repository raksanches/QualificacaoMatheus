%==================================================================================================
\subsection{\textit{Reynolds-Averaged Navier-Stokes}} \label{RANS}
%==================================================================================================

A aproximação das equações diferenciais denominada \textit{Reynolds-Averaged Navier-Stokes} (RANS) busca encontrar uma solução a partir da chamada decomposição de Reynolds. Nesse contexto, observa-se que simulações feitas utilizando RANS são mais eficientes computacionalmente que aquelas feitas a partir de LES, além de possuírem uma implementação mais simples \cite{alfonsi2009reynolds, ling2015evaluation}.

Considera-se que uma propriedade $\phi$ pode ser decomposta em duas parcelas: uma referente à média, denotada por uma barra ($\bar{\phi}(\BB{y},t)$), e uma referente à flutuações no espaço-tempo, denotada por $\phi'(\BB{y},t)$, ou seja, $\phi(\BB{y},t)=\bar{\phi}+\phi'$.

No entanto, há diferentes formas de se considerar a média de uma propriedade, tal como: média temporal, comumente utilizada em escoamentos estacionários; média espacial, utilizada em escoamentos turbulentos homogêneos; e média de um conjunto de experimentos \cite{tennekes1972first,speziale1991analytical,alfonsi2009reynolds}. Como apontado por \citeonline{speziale1991analytical,alfonsi2009reynolds}, a última opção é adequada para descrever escoamentos que não são nem estacionários, nem homogêneos, sendo calculada como:

\begin{equation}
    \bar{\phi}(\BB{y},t)=\lim_{n\to\infty}{\frac{1}{n}\sum_{k=1}^n{\phi^{k}(\BB{y},t)}}\text{,}
\end{equation}

\noindent em que $n$ é o número de experimentos realizados.

Vale ressaltar algumas propriedades interessantes relacionadas à média de uma propriedade ($\phi$, $\phi_1$ ou $\phi_2$ quaisquer), tais como:

\begin{enumerate}[label=\alph*.]
    \item $\bar{\phi}'=0$;
    \item $\bar{\bar{\phi}}=\bar{\phi}$;
    \item $\bar{\phi_1+\phi_2}=\bar{\phi}_1+\bar{\phi}_2$;
    \item $\bar{\phi_1\bar{\phi}_2}=\bar{\phi}_1\bar{\phi}_2$;
    \item $\bar{\phi_1\phi_2}=\bar{\phi}_1\bar{\phi}_2+\bar{\phi_1'\phi_2'}$;
    \item $\bar{\der{\phi}{y_i}}=\der{\bar{\phi}}{y_i}$; e
    \item $\bar{\der{\phi}{t}}=\der{\bar{\phi}}{t}$.
\end{enumerate}

Assim, pode-se tomar a média das Equações de Navier-Stokes, obtendo-se:

\begin{equation}
    \left\{
    \begin{array}{ll}
        \rho\bigpar{\bar{\dot{u}_i}+\bar{(u_ju_i)_{,j}}-\bar{f_i}}-\bar{\sigma_{ji,j}}=0 & \text{ em }\Omega\text{,} \\
        \bar{u_{i,i}}=0                                                                  & \text{ em }\Omega\text{.}
    \end{array}
    \right.
\end{equation}

Aplicando também o modelo constitutivo apresentado em \ref{MC}, o problema se torna:

\begin{equation}
    \left\{
    \begin{array}{ll}
        \rho\bigpar{(\dot{\bar{u}}_i+\bar{u_ju_i})_{,j}-\bar{f}_i}-\mu(\bar{u_{i,j}+u_{j,i}})_{,j}+\bar{p}_{,i}=0 & \text{ em }\Omega\text{,} \\
        \bar{u}_{i,i}=0                                                                                           & \text{ em }\Omega\text{.}
    \end{array}
    \right.
\end{equation}

Realizando a separação das variáveis em suas respectivas parcelas na equação da conservação de movimento, tem-se que:

\begin{equation}
    \rho\bigpar{(\dot{\bar{u}}_i+\bar{(\bar{u}_j+u'_j)(\bar{u}_i+u'_i)})_{,j}-\bar{f}_i}-\mu\bigpar{\bar{(\bar{u}_i+u'_i)}_{,j}+\bar{(\bar{u}_j+u'_j)}_{,i}}+\bar{p}_{,i}=0\text{,}
\end{equation}

\noindent que leva à seguinte expressão simplificada:

\begin{equation}
    \rho\bigpar{\dot{\bar{u}}_i+\bar{u}_j\bar{u}_{i,j}+(\bar{u'_iu'_j})_{,j}-\bar{f}_i}-\mu\bar{u}_{i,jj}+\bar{p}_{,i}=0\text{,}
    \label{eq:RANS-estac}
\end{equation}

\noindent a qual representa a equação RANS da quantidade de movimento para escoamentos incompressíveis \cite{chou1945velocity,alfonsi2009reynolds}.

Nesse contexto vale mencionar o tensor de tensões de Reynolds (dividido pela densidade), dado por $\tau_{ij}=-\bar{u'_iu'_j}$, traz a interferência que os efeitos turbulentos que a parcela de flutuação causa no movimento médio \cite{chou1945velocity,alfonsi2009reynolds}. Porém, verifica-se que, ao assumir essa separação de variáveis, o problema conta com mais incógnitas do que equações para as determinar. Logo, uma forma de se obter equações adicionais que auxiliem na resolução do problema se encontra na modelagem do tensor de tensões de Reynolds de forma a relacionar as flutuações de velocidades com as velocidades médias.

\citeonline{alfonsi2009reynolds} aponta que problemas envolvendo RANS podem ser classificados dependendo da quantidade de equações diferenciais resolvidas, em que cada equação adicionada se refere ao transporte de uma propriedade relativa à turbulência, alterando, assim, a maneira com que a viscosidade de vórtice é determinada. Segundo o autor, as classes de modelos RANS de turbulência são:

\begin{enumerate}[label=\alph*.]
    \item Modelos de Zero Equações: Apenas as equações referentes ao campo médio são resolvidas, sendo $l_0$ e $\tau_0$ determinados empiricamente;
    \item Modelos de Uma Equação: É adicionada uma equação de transporte em termos da energia cinética média $k$ para auxiliar na determinação do campo de flutuações de velocidades;
    \item Modelos de Duas Equações: Uma segunda equação é adicionada para determinação das escalas de comprimento turbulento;
    \item Modelos de Tensões: A respeito aos Modelos de Zero Equações, adiciona-se duas equações de transporte referentes ao tensor de Reynolds e à $\varepsilon$, tal modelo também pode ser chamado de modelo $\te$.
\end{enumerate}

Sobre os Modelos de Duas Equações, destacam-se os modelos $\ke$ \cite{haakansson2012experimental,davidson2014pans,parente2011improved}, nos quais as equações adicionais são dadas em função da energia cinética média gerada pelo campo de flutuações ($k$) e da taxa de dissipação da energia cinética turbulenta ($\varepsilon$) e $\kw$ \cite{larsen2018over,bassi2005discontinuous}, que ao invés de adicionar uma equação em termos de $\varepsilon$, esta de dá em função da escala de tempo turbulento recíproco ($\omega=\varepsilon/K$).

Dentre os modelos $\ke$, $\kw$ e $\te$, observa-se que o primeiro possui menor custo computacional e maior facilidade de implementação, sendo uma boa alternativa para as simulações \cite{koutsourakis2012evaluation,adanta2020comparison}.

Para se obter as equações adicionais, faz-se a simples substituição de $\BB{u}=\umed+\BB{u}'$ e $p=\bar{p}+p'$ na equação da quantidade de movimento, o que leva a:

\begin{equation}
    \begin{split}
        &\rho\bigpar{\dot{\bar{u}}_i+\dot{u}'_i+\bar{u}_j\bar{u}_{i,j}+\bar{u}_ju'_{i,j}+u'_j\bar{u}_{i,j}+u'_ju'_{i,j}-\bar{f}_i}-\mu\bar{u}_{i,jj}-\mu u'_{i,jj}+\bar{p}_{,i}+p'_{,i}=0\text{.}
    \end{split}
    \label{eq:NS-RANS1}
\end{equation}

Subtraindo \eqref{eq:RANS-estac} de \eqref{eq:NS-RANS1} obtém-se:

\begin{equation}
    \rho\bigpar{\dot{u}'_i+\bar{u}_ju'_{i,j}+u'_j\bar{u}_{i,j}+u'_ju'_{i,j}-(\bar{u'_iu'_j})_{,j}}-\mu u'_{i,jj}+p'_{,i}=0\text{.}
    \label{eq:NS-RANS2}
\end{equation}

Seguindo o mesmo procedimento para a equação da conservação da massa obtém-se que:

\begin{equation}
    u'_{i,i}=0\text{,}
\end{equation}

\noindent sendo equivalente à condição de incompressibilidade no campo de flutuações de velocidade, a qual será interessante para algumas simplificações que virão posteriormente.

Fazendo a contração do índice $i$ em \eqref{eq:NS-RANS2} por meio de $u'_i$, tomando a média e dividindo por $\rho$, tem-se:

\begin{equation}
    \bar{u'_i\dot{u}'_i}+\bar{u'_i\bar{u}_ju'_{i,j}}+\bar{u'_iu'_j\bar{u}_{i,j}}+\bar{u'_iu'_ju'_{i,j}}+\bar{u'_i\tau_{ij,j}}-\bar{\nu u'_iu'_{i,jj}}+\bar{u'_i\pr'_{,i}}=0\text{.}
\end{equation}

\noindent em que $\pr'=p'/\rho$.

Fazendo as devidas simplificações, tem-se a equação que descreve de forma exata o comportamento da energia cinética média gerada pelo campo de flutuações ($K=\bar{u'_iu'_i}/2$) \cite{alfonsi2009reynolds}:

\begin{equation}
    \dot{K}+\bar{u}_jK_{,j}=\tau_{ij}\bar{u}_{i,j}-\nu\bar{u'_{i,j}u'_{i,j}}-\bigpar{\frac{\bar{u'_iu'_iu'_j}}{2}+\bar{u'_j\pr'}}_{,j}+\nu K_{,jj}\text{,}
\end{equation}

\noindent a qual pode ser reescrita de forma mais compacta, considerando-se o efeito de cada parcela:

\begin{equation}
    \dot{K}+\bar{u}_jK_{,j}=P_K-\varepsilon-D_{j,j}+\nu K_{,jj}\text{,}
\end{equation}

\noindent em que $P_K$ é a produção da energia cinética turbulenta, a qual não precisa de modelagem, $\varepsilon$ é a taxa de dissipação da energia cinética turbulenta, $D_j$ é o gradiente de difusão turbulenta e o último termo, que dispensa modelagem, representa a difusão viscosa. Tais termos são dados por:

\begin{subequations}
    \begin{align}
         & P_K=\tau_{ij}\bar{u}_{i,j}\text{,}                      \\
         & \varepsilon=\nu\bar{u'_{i,j}u'_{i,j}}\text{ e}          \\
         & D_j=\frac{\bar{u'_iu'_iu'_j}}{2}+\bar{u'_j\pr'}\text{.}
    \end{align}
\end{subequations}

\citeonline{alfonsi2009reynolds} apresenta que uma forma de se modelar os termos $\varepsilon$ e $D_j$ como:

\begin{subequations}
    \begin{equation}
        \varepsilon=C^*\frac{K^{3/2}}{l_0}\text{ e}\\
    \end{equation}
    \begin{equation}
        D_j=-\frac{\nu_T}{\sigma_K}K_{,j}\text{,}
    \end{equation}
\end{subequations}

\noindent sendo $l_0$ o comprimento de vórtice, $C^*=0,166$ e $\sigma_K=1,0$ constantes adimensionais.

Há ainda a necessidade de se modelar o tensor de Reynolds, o qual pode ser subdividido em duas parcelas: uma parte isotrópica ($\tau_{ij}^I$) e outra desviadora ($\tau_{ij}^D$), ou seja:

\begin{equation}
    \tau_{ij}=\tau_{ij}^I+\tau_{ij}^D\text{,}
\end{equation}

\noindent em que:

\begin{subequations}
    \begin{align}
         & \tau_{ij}^I=-\frac{2}{3}K\delta_{ij}\text{ e} \\
         & \tau_{ij}^D=2\nu_T\epmean_{ij}\text{,}
    \end{align}
\end{subequations}

\noindent $\epmean_{ij}$ é a taxa de deformação do campo de velocidades média:

\begin{equation}
    \epmean_{ij}=\frac{\bar{u}_{i,j}+\bar{u}_{j,i}}{2}
\end{equation}

\noindent e $\nu_T$ é a viscosidade de vórtice, dada por

\begin{equation}
    \nu_T=C_\mu\frac{K^2}{\varepsilon}\text{,}
\end{equation}

\noindent sendo $C_\mu=0,09$.

Já a equação exata do transporte que descreve a taxa de dissipação também é proveniente da equação \eqref{eq:NS-RANS2}, fazendo a troca dos índices mudos $j\to k$, derivando ambos os lados em uma direção $j$ e contraindo-se os índices $i$ e $j$ por meio de $\nu u'_{i,j}/\rho$:

\begin{equation}
    \begin{split}
        &\nu u'_{i,j}\dot{u}'_{i,j}+\nu u'_{i,j}\bigpar{\bar{u}_ku'_{i,k}}_{,j}+\nu u'_{i,j}\bigpar{u'_k\bar{u}_{i,k}}_{,j}+\nu u'_{i,j}\bigpar{u'_ku'_{i,k}}_{,j}-\\
        &\nu u'_{i,j}(\bar{u'_iu'_k})_{,jk}-\nu^2u'_{i,j}u'_{i,jkk}+\nu u'_{i,j}\pr'_{,ij}=0\text{.}
    \end{split}
\end{equation}

Tomando-se a média dos termos e realizando as devidas simplificações tem-se a equação exata para $\ep$:

\begin{equation}
    \begin{split}
        &\dot{\ep}+\bar{u}_k\ep_{,k}=-2\nu\bar{u'_{i,j}u'_{k,j}}\bar{u}_{i,k}-2\nu\bar{u'_{i,j}u'_{i,k}}\bar{u}_{k,j}-2\nu\bar{u'_ku'_{i,j}}\bar{u}_{i,jk}-\\
        &2\nu\bar{u'_{i,j}u'_{i,k}u'_{k,j}}-\nu(\bar{u'_ku'_{i,j}u'_{i,j}})_{,k}-2\nu(\bar{\pr'_{,j}u'_{i,j}})_{,i}-2\nu^2\bar{u'_{i,jk}u'_{i,jk}}+\nu\varepsilon_{,kk}\text{.}
    \end{split}
\end{equation}

Fazendo a simplificação em termos dos efeitos de cada parcela tem-se:

\begin{equation}
    \dot{\ep}+\bar{u}_i\ep_{,i}=P_\ep+D_\ep-\Phi_\ep+\nu\ep_{,ii}\text{,}
\end{equation}

\noindent em que $P_\ep$ é a produção de dissipação, $D_\ep$ é a difusão turbulenta de dissipação, $\Phi_\ep$ é a destruição turbulenta da dissipação e o último termo é o transporte viscoso. Essas parcelas são dadas de forma exata por:

\begin{subequations}
    \begin{align}
         & P_\ep=-2\nu\bar{u'_{i,j}u'_{k,j}}\bar{u}_{i,k}-2\nu\bar{u'_{i,j}u'_{i,k}}\bar{u}_{k,j}-2\nu\bar{u'_ku'_{i,j}}\bar{u}_{i,jk}-2\nu\bar{u'_{i,j}u'_{i,k}u'_{k,j}} \text{,} \\
         & D_\ep=-\nu(\bar{u'_ku'_{i,j}u'_{i,j}})_{,k}-2\nu(\bar{\pr'_{,j}u'_{i,j}})_{,i} \text{ e}                                                                                \\
         & \Phi_\ep=2\nu^2\bar{u'_{i,jk}u'_{i,jk}}\text{.}
    \end{align}
\end{subequations}

Assim, realiza-se uma modelagem sobre os termos necessários como:

\begin{subequations}
    \begin{align}
         & P_\ep=C_{\ep 1}\frac{\ep}{k}\tau_{ij}\bar{u}_{i,j}\text{,}   \\
         & D_\ep=\bigpar{\frac{\nu_T}{\sigma_\ep}\ep_{,i}}_{,i}\text{e} \\
         & \Phi_\ep=C_{\ep 2}\frac{\ep^2}{K}\text{,}
    \end{align}
\end{subequations}

\noindent em que $C_{\ep 1}=1,44$, $C_{\ep 2}=1,92$ e $\sigma_\ep=1,3$ são constantes adimensionais.

Assim, o problema RANS fica descrito pelas equações:

\begin{equation}
    \left\{
    \begin{array}{l}
        \rho\bigpar{\dot{\bar{u}}_i+\bar{u}_j\bar{u}_{i,j}-\tau_{ij,j}-\bar{f}_i}-\mu\bar{u}_{i,jj}+\bar{p}_{,i}=0    \\
        \bar{u}_{i,i}=0                                                                                               \\
        \dot{K}+\bar{u}_jK_{,j}=P_K-\varepsilon-D_{j,j}+\nu K_{,jj}                                                   \\
        \dot{\varepsilon}+\bar{u}_i\varepsilon_{,i}=P_\varepsilon+D_\varepsilon-\Phi_\varepsilon+\nu\varepsilon_{,ii} \\
        \tau_{ij}=-\frac{2}{3}K\delta_{ij}+\nu_T(\bar{u}_{i,j}+\bar{u}_{j,i})                                         \\
        \nu_T=C_\mu\frac{K^2}{\ep}
    \end{array}
    \right.\text{.}
\end{equation}