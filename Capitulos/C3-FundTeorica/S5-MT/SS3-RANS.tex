% Introdução

\documentclass[_ArquivoPrincipal.tex]{subfiles}

\begin{document}

%==================================================================================================
\subsection{\textit{Reynolds-Averaged Navier-Stokes}} \label{RANS}
%==================================================================================================

Na tentativa de se obter soluções aproximativas para as equações de Navier-Stokes, são desenvolvidas técnicas de aproximação, baseadas em equações diferenciais. Uma dessas aproximações, denominada de \textit{Reynolds-Averaged Navier-Stokes} (RANS) busca encontrar uma solução a partir da decomposição de Reynolds. Nesse contexto, observa-se que simulações feitas utilizando RANS são mais eficientes computacionalmente que aquelas feitas a partir de LES, além de possuírem uma implementação mais facilitada \cite{alfonsi2009reynolds, ling2015evaluation}. Ainda segundo \citeonline{alfonsi2009reynolds}, problemas envolvendo RANS podem ser classificados dependendo da quantidade de equações diferenciais resolvidas, em que cada equação adicionada refere-se ao transporte de uma propriedade relativa à turbulência.

Em uma descrição Euleriana, sua obtenção parte da equação \ref{eq:NS-Euler} e considera-se que uma propriedade $\phi$ pode ser decomposta em duas parcelas: uma referente à média temporal, denotada por uma barra ($\bar{\phi}(\BB{y})$), dada por \cite{tennekes1972first,speziale1991analytical}:

\begin{equation}
    \bar{\phi}(\BB{y})=\lim_{T\to\infty}{\frac{1}{T}\int_{t_0}^{t_0+T}{\phi(\BB{y},t)dt}}\text{,}
\end{equation}

\noindent e uma referente à flutuações no espaço-tempo, denotada por $\phi'(\BB{y},t)$, ou seja, $\phi(\BB{y},t)=\bar{\phi}(\BB{y})+\phi'(\BB{y},t)$.

Vale ressaltar algumas propriedades interessantes relacionadas à média de uma propriedade ($\phi$, $\phi_1$ ou $\phi_2$ quaisquer), tais como:

\begin{enumerate}[label=\alph*.]
    \item $\bar{\phi'}(\BB{y})=0$;
    \item $\bar{\bar{\phi}}=\bar{\phi}$;
    \item $\bar{\phi_1+\phi_2}=\bar{\phi}_1+\bar{\phi}_2$;
    \item $\bar{\phi_1\bar{\phi}_2}=\bar{\phi}_1\bar{\phi}_2$;
    \item $\bar{\phi_1\phi_2}=\bar{\phi}_1\bar{\phi}_2+\bar{\phi_1'\phi_2'}$; e
    \item $\bar{\der{\phi}{y_i}}=\der{\bar{\phi}}{y_i}$.
\end{enumerate}

Assim, pode-se tomar a média das Equações de Navier-Stokes, obtendo-se:

\begin{equation}
    \left\{
    \begin{array}{ll}
        \rho\bigpar{\bar{\dot{u}_i}+\bar{(u_ju_i)_{,j}}-\bar{f_i}}-\bar{\sigma_{ji,j}}=0 & \text{ em }\Omega\text{,} \\
        \bar{u_{i,i}}=0                                                                  & \text{ em }\Omega\text{.}
    \end{array}
    \right.
\end{equation}

Note que, ao tomar a média temporal, o problema se torna independente do tempo. Portanto o termo referente à derivada temporal de $\BB{u}$ desaparece. Assim, aplicando também o modelo constitutivo apresentado em \ref{MC}, o problema se torna:

\begin{equation}
    \left\{
    \begin{array}{ll}
        \rho\bigpar{(\bar{u_ju_i})_{,j}-\bar{f}_i}-\mu(\bar{u_{i,j}+u_{j,i}})_{,j}+\bar{p}_{,i}=0 & \text{ em }\Omega\text{,} \\
        \bar{u}_{i,i}=0                                                                           & \text{ em }\Omega\text{.}
    \end{array}
    \right.
\end{equation}

Realizando a separação dos parâmetros em suas respectivas parcelas na equação da conservação de movimento, tem-se que:

\begin{equation}
    \rho\bigpar{(\bar{(\bar{u}_j+u'_j)(\bar{u}_i+u'_i)})_{,j}-\bar{f}_i}-\mu(\bar{(\bar{u}_i+u'_i)}_{,j}+\bar{(\bar{u}_j+u'_j)}_{,i})+\bar{p}_{,i}=0\text{,}
\end{equation}

\noindent que leva à seguinte expressão simplificada:

\begin{equation}
    \rho\bigpar{\bar{u}_j\bar{u}_{i,j}+(\bar{u'_iu'_j})_{,j}-\bar{f}_i}-\mu\bar{u}_{i,jj}+\bar{p}_{,i}=0\text{,}
    \label{eq:RANS-estac}
\end{equation}

\noindent a qual representa a equação de Navier-Stokes para escoamentos incompressíveis em regime estacionário em uma formulação RANS \cite{chou1945velocity,alfonsi2009reynolds}.

Além disso, é possível fazer a substituição de $\BB{u}=\umed+\BB{u}'$ e $p=\bar{p}+p'$ na equação da quantidade de movimento, o que leva a:

\begin{equation}
    \begin{split}
        &\rho\bigpar{\dot{u}'_i+\bar{u}_j\bar{u}_{i,j}+\bar{u}_ju'_{i,j}+u'_j\bar{u}_{i,j}+u'_ju'_{i,j}-\bar{f}_i}-\mu\bar{u}_{i,jj}-\mu u'_{i,jj}+\bar{p}_{,i}+p'_{,i}=0\text{.}
    \end{split}
    \label{eq:NS-RANS1}
\end{equation}

Subtraindo \ref{eq:RANS-estac} de \ref{eq:NS-RANS1} obtém-se:

\begin{equation}
    \rho\bigpar{\dot{u}'_i+\bar{u}_ju'_{i,j}+u'_j\bar{u}_{i,j}+u'_ju'_{i,j}-(\bar{u'_iu'_j})_{,j}}-\mu u'_{i,jj}+p'_{,i}=0\text{.}
\end{equation}

\noindent e de forma similar tem-se:

\begin{equation}
    u'_{i,i}=0\text{.}
\end{equation}

Nesse contexto vale mencionar o tensor de tensões de Reynolds (dividido pela densidade), dado por $\tau_{ij}=-\bar{u'_iu'_j}$, que traz a interferência que os efeitos turbulentos da parcela de flutuação causa no movimento médio \cite{chou1945velocity,alfonsi2009reynolds}. Porém, verifica-se que, ao assumir essa separação de variáveis, o problema conta com mais incógnitas que equações para determiná-las. Logo, uma forma de se obter equações adicionais que auxiliem na resolução do problema se encontra na modelagem do tensor de tensões de Reynolds de forma a relacionar as flutuações de velocidades com as velocidades médias.

Segundo \citeonline{alfonsi2009reynolds} o tensor de Reynolds pode ser subdividido em uma parte isotrópica ($\tau_{ij}^I$) e outra desviadora ($\tau_{ij}^D$):

\begin{equation}
    \tau_{ij}=\tau_{ij}^I+\tau_{ij}^D\text{,}
\end{equation}

\noindent em que:

\begin{subequations}
    \begin{equation}
        \tau_{ij}^I=-\frac{2}{3}K\delta_{ij}\text{ e}
    \end{equation}
    \begin{equation}
        \tau_{ij}^D=2\nu_T\epmean_{ij}\text{,}
    \end{equation}
\end{subequations}

\noindent sendo $K$ a energia cinética média gerada pelo campo de flutuações de velocidade:

\begin{equation}
    K=\frac{1}{2}\bar{u'_iu'_i}\text{,}
\end{equation}

\noindent $\epmean_{ij}$ é a taxa de deformação do campo de velocidades média:

\begin{equation}
    \epmean_{ij}=\frac{\bar{u}_{i,j}+\bar{u}_{j,i}}{2}
\end{equation}

\noindent e $\nu_T$ é a viscosidade de vórtice.

\end{document}