% Introdução

\documentclass[_ArquivoPrincipal.tex]{subfiles}

\begin{document}

%==================================================================================================
\subsection{\textit{Reynolds-Averaged Navier-Stokes}} \label{RANS}
%==================================================================================================

Na tentativa de se obter soluções aproximativas para as equações de Navier-Stokes, são desenvolvidas técnicas de aproximação, baseadas em equações diferenciais. Uma dessas aproximações, denominada de \textit{Reynolds-Averaged Navier-Stokes} (RANS) busca encontrar uma solução a partir da decomposição de Reynolds. Nesse contexto, observa-se que simulações feitas utilizando RANS são mais eficientes computacionalmente que aquelas feitas a partir de LES, além de possuírem uma implementação mais facilitada \cite{alfonsi2009reynolds, ling2015evaluation}.

Em uma descrição Euleriana, sua obtenção parte da equação \ref{eq:NS-Euler} e considera-se que uma propriedade $\phi$ pode ser decomposta em duas parcelas: uma referente à média temporal, denotada por uma barra ($\bar{\phi}(\BB{y})$), dada por \cite{tennekes1972first,speziale1991analytical}:

\begin{equation}
    \bar{\phi}(\BB{y})=\lim_{T\to\infty}{\frac{1}{T}\int_{t_0}^{t_0+T}{\phi(\BB{y},t)dt}}\text{,}
\end{equation}

\noindent e uma referente à flutuações no espaço-tempo, denotada por $\phi'(\BB{y},t)$, ou seja, $\phi(\BB{y},t)=\bar{\phi}(\BB{y})+\phi'(\BB{y},t)$.

Vale ressaltar algumas propriedades interessantes relacionadas à média de uma propriedade ($\phi$, $\phi_1$ ou $\phi_2$ quaisquer), tais como:

\begin{enumerate}[label=\alph*.]
    \item $\bar{\phi'}(\BB{y})=0$;
    \item $\bar{\bar{\phi}}=\bar{\phi}$;
    \item $\bar{\phi_1+\phi_2}=\bar{\phi}_1+\bar{\phi}_2$;
    \item $\bar{\phi_1\bar{\phi}_2}=\bar{\phi}_1\bar{\phi}_2$;
    \item $\bar{\phi_1\phi_2}=\bar{\phi}_1\bar{\phi}_2+\bar{\phi_1'\phi_2'}$; e
    \item $\bar{\der{\phi}{y_i}}=\der{\bar{\phi}}{y_i}$.
\end{enumerate}

Assim, pode-se tomar a média das Equações de Navier-Stokes, obtendo-se:

\begin{equation}
    \left\{
    \begin{array}{ll}
        \rho\bigpar{\bar{\dot{u}_i}+\bar{(u_ju_i)_{,j}}-\bar{f_i}}-\bar{\sigma_{ji,j}}=0 & \text{ em }\Omega\text{,} \\
        \bar{u_{i,i}}=0                                                                  & \text{ em }\Omega\text{.}
    \end{array}
    \right.
\end{equation}

Note que, ao tomar a média temporal, o problema se torna independente do tempo. Portanto o termo referente à derivada temporal de $\BB{u}$ desaparece. Assim, aplicando também o modelo constitutivo apresentado em \ref{MC}, o problema se torna:

\begin{equation}
    \left\{
    \begin{array}{ll}
        \rho\bigpar{(\bar{u_ju_i})_{,j}-\bar{f}_i}-\mu(\bar{u_{i,j}+u_{j,i}})_{,j}+\bar{p}_{,i}=0 & \text{ em }\Omega\text{,} \\
        \bar{u}_{i,i}=0                                                                           & \text{ em }\Omega\text{.}
    \end{array}
    \right.
\end{equation}

Realizando a separação dos parâmetros em suas respectivas parcelas na equação da conservação de movimento, tem-se que:

\begin{equation}
    \rho\bigpar{(\bar{(\bar{u}_j+u'_j)(\bar{u}_i+u'_i)})_{,j}-\bar{f}_i}-\mu\bigpar{\bar{(\bar{u}_i+u'_i)}_{,j}+\bar{(\bar{u}_j+u'_j)}_{,i}}+\bar{p}_{,i}=0\text{,}
\end{equation}

\noindent que leva à seguinte expressão simplificada:

\begin{equation}
    \rho\bigpar{\bar{u}_j\bar{u}_{i,j}+(\bar{u'_iu'_j})_{,j}-\bar{f}_i}-\mu\bar{u}_{i,jj}+\bar{p}_{,i}=0\text{,}
    \label{eq:RANS-estac}
\end{equation}

\noindent a qual representa a equação de Navier-Stokes para escoamentos incompressíveis em regime estacionário em uma formulação RANS \cite{chou1945velocity,alfonsi2009reynolds}.

Além disso, é possível fazer a substituição de $\BB{u}=\umed+\BB{u}'$ e $p=\bar{p}+p'$ na equação da quantidade de movimento, o que leva a:

\begin{equation}
    \begin{split}
        &\rho\bigpar{\dot{u}'_i+\bar{u}_j\bar{u}_{i,j}+\bar{u}_ju'_{i,j}+u'_j\bar{u}_{i,j}+u'_ju'_{i,j}-\bar{f}_i}-\mu\bar{u}_{i,jj}-\mu u'_{i,jj}+\bar{p}_{,i}+p'_{,i}=0\text{.}
    \end{split}
    \label{eq:NS-RANS1}
\end{equation}

Subtraindo \ref{eq:RANS-estac} de \ref{eq:NS-RANS1} obtém-se:

\begin{equation}
    \rho\bigpar{\dot{u}'_i+\bar{u}_ju'_{i,j}+u'_j\bar{u}_{i,j}+u'_ju'_{i,j}-(\bar{u'_iu'_j})_{,j}}-\mu u'_{i,jj}+p'_{,i}=0\text{.}
\end{equation}

\noindent e de forma similar tem-se:

\begin{equation}
    u'_{i,i}=0\text{.}
\end{equation}

Nesse contexto vale mencionar o tensor de tensões de Reynolds (dividido pela densidade), dado por $\tau_{ij}=-\bar{u'_iu'_j}$, que traz a interferência que os efeitos turbulentos da parcela de flutuação causa no movimento médio \cite{chou1945velocity,alfonsi2009reynolds}. Porém, verifica-se que, ao assumir essa separação de variáveis, o problema conta com mais incógnitas que equações para determiná-las. Logo, uma forma de se obter equações adicionais que auxiliem na resolução do problema se encontra na modelagem do tensor de tensões de Reynolds de forma a relacionar as flutuações de velocidades com as velocidades médias.

\citeonline{alfonsi2009reynolds} aponta que problemas envolvendo RANS podem ser classificados dependendo da quantidade de equações diferenciais resolvidas, em que cada equação adicionada refere-se ao transporte de uma propriedade relativa à turbulência, alterando, assim, a maneira com que a viscosidade de vórtice é determinada. Segundo o autor, as classes de modelos de turbulencia em RANS são:

\begin{enumerate}[label=\alph*.]
    \item Modelos de Zero Equações: Apenas as equações referentes ao campo médio são resolvidas, sendo $l_0$ e $\tau_0$ determinados empiricamente;
    \item Modelos de Uma Equação: É adicionada uma equação de transporte em termos da energia cinética média $k$ para auxiliar na determinação do campo de flutuações de velocidades;
    \item Modelos de Duas Equações: Uma segunda equação é adicionada para determinação das escalas de comprimento turbulentos;
    \item Modelos de Tensões: A respeito aos Modelos de Zero Equações, adiciona-se duas equações de transporte referentes ao tensor de Reynolds e à $\varepsilon$, tal modelo também pode ser chamado de modelo \te.
\end{enumerate}

A respeito dos Modelos de Duas Equações, destacam-se os modelos $\ke$ \cite{haakansson2012experimental,davidson2014pans,parente2011improved}, em que as equações adicionais são dadas em função de $k$ e da taxa de dissipação da energia cinética turbulenta ($\varepsilon$) e $\kw$ \cite{larsen2018over,bassi2005discontinuous}, que ao invés de adicionar uma equação em termos de $\varepsilon$, esta de dá em função da escala de tempo turbulento recíproco ($\omega=\varepsilon/k$).

Contudo, \citeonline{alfonsi2009reynolds} aponta que o modelo $\ke$ não é naturalmente capaz de ser integrado na fronteira em o sólido, sendo necessário um tratamento adicional para que isso se torne possível, assim como o modelo $\kw$ se torna mais singular próximo à fronteira. Assim, o modelo \te\ se mostra promissor, pois, como mencionado pelo autor, é possível aplicar efeitos de relaxação, com diferentes tempos característicos e diferentes componentes do tensor de Reynolds em resposta para mudanças de geometria.

Nesse cenário, apresenta-se a equação que descreve de forma exata o comportamento do tensor de Reynolds:

\begin{equation}
    \begin{split}
        &-\dot{\tau}_{ij}-\bar{u}_k\tau_{ij,k}=\tau_{ik}\bar{u}_{j,k}+\tau_{jk}\bar{u}_{i,k}+\bar{p'(u'_{i,j}+u'_{j,i})}-2\nu\bar{u'_{i,k}u'_{j,k}}-\\
        &(\bar{u'_iu'_ju'_k}+\bar{p'u'_i}\delta_{jk}+\bar{p'u'_j}\delta_{ik})_{,k}+\nu\tau_{ij,kk}\text{,}
    \end{split}
\end{equation}

\noindent que pode ser reescrito de forma mais compacta, considerando-se o efeito de cada parcela:

\begin{equation}
    -\dot{\tau}_{ij}-\bar{u}_k\tau_{ij,k}=P_{ij}+\Pi_{ij}-\varepsilon_{ij}-C_{ijk,k}+\nu\tau_{ij,kk}\text{,}
\end{equation}

\noindent em que $P_{ij}$ é a produção do tensor de Reynolds, a qual não precisa de modelagem, $\Pi_{ij}$ é a correlação entre a pressão e o tensor de taxa de deformação do campo de flutuações, $\varepsilon_{ij}$ é a taxa de dissipação, $C_{ijk}$ é o gradiente de difusão turbulenta e o último termo, que dispensa qualquer modelagem representa o transporte viscoso. Tais parâmetros são dados por:

\begin{subequations}
    \begin{align}
         & P_{ij}=\tau_{ik}\bar{u}_{j,k}-\tau_{jk}\bar{u}_{i,k}                               \\
         & \Pi_{ij}=\bar{p'(u'_{i,j}+u'_{j,i})}                                               \\
         & \varepsilon_{ij}=2\nu\bar{u'_{i,k}u'_{j,k}}                                        \\
         & C_{ijk}=\bar{u'_iu'_ju'_k}+\bar{p'u'_i}\delta_{jk}+\bar{p'u'_j}\delta_{ik}\text{.}
    \end{align}
\end{subequations}

Já a equação exata do transporte que descreve a taxa de dissipação é dada por:

\begin{equation}
    \begin{split}
        &\dot{\varepsilon}+\bar{u}_i\varepsilon_{,i}=-2\nu\bar{u'_{i,j}u'_{k,j}}\bar{u}_{i,k}-2\nu\bar{u'_{j,i}u'_{j,k}}\bar{u}_{i,k}-2\nu\bar{u'_ku'_{i,j}}\bar{u}_{i,kj}-\\
        &2\nu\bar{u'_{i,k}u'_{i,m}u'_{k,m}}-\nu(\bar{u'_ku'_{i,m}u'_{i,m}})_{,k}-2\nu(\bar{p'_{,m}u'_{k,m}})_{,k}-2\nu^2\bar{u'_{i,km}u'_{i,km}}+\nu\varepsilon_{,ii}\text{.}
    \end{split}
\end{equation}

Fazendo a simplificação em termos dos efeitos de cada parcela tem-se:

\begin{equation}
    \dot{\varepsilon}+\bar{u}_i\varepsilon_{,i}=P_\varepsilon+D_\varepsilon-\Phi_\varepsilon+\nu\varepsilon_{,ii}\text{,}
\end{equation}

\noindent em que $P_\varepsilon$ é a produção de dissipação, $D_\varepsilon$ é a difusão turbulenta de dissipação, $\Phi_\varepsilon$ é a destruição turbulenta da dissipação e o último termo é o transporte viscoso. Essas parcelas são dadas de forma exata por:

\begin{subequations}
    \begin{align}
        &P_\varepsilon=-2\nu\bar{u'_{i,j}u'_{k,j}}\bar{u}_{i,k}-2\nu\bar{u'_{j,i}u'_{j,k}}\bar{u}_{i,k}-2\nu\bar{u'_ku'_{i,j}}\bar{u}_{i,kj}-2\nu\bar{u'_{i,k}u'_{i,m}u'_{k,m}}\\
        &D_\varepsilon=-\nu(\bar{u'_ku'_{i,m}u'_{i,m}})_{,k}-2\nu(\bar{p'_{,m}u'_{k,m}})_{,k}\\
        &\Phi_\varepsilon=2\nu^2\bar{u'_{i,km}u'_{i,km}}\text{.}
    \end{align}
\end{subequations}

Já o tensor de Reynolds pode ser subdividido em duas parcelas: uma parte isotrópica ($\tau_{ij}^I$) e outra desviadora ($\tau_{ij}^D$), ou seja:

\begin{equation}
    \tau_{ij}=\tau_{ij}^I+\tau_{ij}^D\text{,}
\end{equation}

\noindent em que:

\begin{subequations}
    \begin{equation}
        \tau_{ij}^I=-\frac{2}{3}k\delta_{ij}\text{ e}
    \end{equation}
    \begin{equation}
        \tau_{ij}^D=2\nu_T\epmean_{ij}\text{,}
    \end{equation}
\end{subequations}

\noindent sendo $k$ a energia cinética média gerada pelo campo de flutuações de velocidade:

\begin{equation}
    k=\frac{1}{2}\bar{u'_iu'_i}\text{,}
\end{equation}

\noindent $\epmean_{ij}$ é a taxa de deformação do campo de velocidades média:

\begin{equation}
    \epmean_{ij}=\frac{\bar{u}_{i,j}+\bar{u}_{j,i}}{2}
\end{equation}

\noindent e $\nu_T$ é a viscosidade de vórtice.

Assim, realiza-se uma modelagem sobre os termos necessários como:

\begin{subequations}
    \begin{align}
        &P_\ep=-C_{\ep 1}\frac{\ep}{k}\tau_{ij}\bar{u}_{i,j}\text{,}\\
        &D_\ep=C_\ep\bigpar{\frac{k}{\ep}\tau_{ij}\ep_{,j}}_{,i}\text{,}\\
        &\Phi_\ep=C_{\ep 2}\frac{\ep^2}{k}\text{ e}\\
        &\nu_T=C_\mu\frac{k^2}{\ep}\text{,}
    \end{align}
\end{subequations}

\noindent em que $C_{\ep 1}=1,44$, $C_\ep=0,15$, $C_{\ep 1}=1,92$ e $C_\mu=0,09$ são constantes adimensionais.

\end{document}