%==================================================================================================
\section{Formulação variacional multiescala} \label{VMS}
%==================================================================================================

O Método Variacional Multiescala, introduzido por \citeonline{hughes1995multiscale,hughes1998variational,hughes2000large}, é um modelo capaz de produzir solução estabilizada baseado na separação dos espaços tentativa e teste em subespaços que representam as escalas grosseiras, que se tratam de subespaços de dimensão finita, denotadas neste texto por uma barra sobrescrita, e subespaços que representam as escalas refinadas, que são subespaços de dimensão infinita e denotadas neste texto por um apóstrofe ($'$), ou seja:

\begin{subequations}
    \begin{align}
         & \script{S}_u=\bar{\script{S}}_u\oplus\script{S}'_u\text{,}  \\
         & \script{S}_p=\bar{\script{S}}_p\oplus\script{S}'_p\text{,}  \\
         & \script{V}_u=\bar{\script{V}}_u\oplus\script{V}'_u\text{ e} \\
         & \script{V}_p=\bar{\script{V}}_p\oplus\script{V}'_p\text{.}
    \end{align}
\end{subequations}

Assim, o sistema a ser resolvido parte do apresentado na Eq. \eqref{eq:NS-Euler}, que em sua forma fraca se encontra em \eqref{eq:WeakForm2}. Primeiramente realiza-se a separação dos membros em:

\begin{subequations}
    \begin{align}
         & \BB{u}=\BBB{u}+\BB{u}'\text{,}  \\
         & p=\bar{p}+p'\text{,}            \\
         & \BB{w}=\BBB{w}+\BB{w}'\text{ e} \\
         & q=\bar{q}+q'\text{.}
    \end{align}
\end{subequations}

Seguindo \citeonline{bazilevs2007variational} adota-se $\BB{w}=\BBB{w}$, $q=\bar{q}$ e assume-se que as parcelas em escala fina $\BB{u}'$ e $p'$ são modeladas como:

\begin{subequations}
    \begin{equation}
        \BB{u}'=-\frac{\tau_{\sups}}{\rho}\rM\text{ e}
    \end{equation}
    \begin{equation}
        p'=-\rho\nu_{\lsic}\rC\text{,}
    \end{equation}
    \label{eq:VMS-fine}
\end{subequations}

\noindent onde $\tau_{\sups}$ e $\nu_{\lsic}$ são termos estabilizadores, propostos por \citeonline{bazilevs2013computational} como:
%página 80 do pdf de hughes2013 e 65 de fernandes2020

% \textcolor{red}{Lembre-se de substituir as derivadas temporais (trocar o ponto pela derivada parcial no tempo).}

\begin{subequations}
    \begin{equation}
        \tau_{\sups}=\bigpar{\frac{4}{\Delta t^2}+(\BBB{u}-\HBB{u})\cdot\BB{G}(\BBB{u}-\HBB{u})+C_I\nu^2\BB{G}:\BB{G}}^{-1/2}\text{ e}
    \end{equation}
    \begin{equation}
        \nu_{\lsic}=(\tr{\BB{G}}\tau_{\sups})^{-1}\text{,}
    \end{equation}
    \label{eq:TermEstab}
\end{subequations}

Já os termos $\rM$ e $\rC$ são os resíduos da equação de conservação da quantidade de movimento e da equação da continuidade, respectivamente, ou seja,

\begin{subequations}
    \begin{equation}
        \rM=\rho\bigpar{\dTime{\BBB{u}}+((\BBB{u}-\HBB{u})\cdot\Ny)\BBB{u}-\BBB{f}}-\Ny\cdot\bar{\tens}\text{ e}
    \end{equation}
    \begin{equation}
        \rC=\Ny\cdot\BBB{u}\text{.}
    \end{equation}
\end{subequations}

Outra forma de se determinar $ \tau_{\sups}$ e $\nu_{\lsic}$, também apresentada em \citeonline{bazilevs2013computational}, consiste em fazer:

\begin{subequations}
    \begin{equation}
        \tau_{\sups}=\bigpar{\frac{1}{\tau_{\sugn 1}^2}+\frac{1}{\tau_{\sugn 2}^2}+\frac{1}{\tau_{\sugn 3}^2}}^{-1/2}\text{ e}
    \end{equation}
    \begin{equation}
        \nu_{\lsic}=\tau_{\sups}\norm{\BBB{u}-\HBB{u}}^2\text{.}
    \end{equation}
    \label{eq:stabilizatingTerms}
\end{subequations}

Assim, obtém-se o problema do Método Variacional Multiescala Baseado em Resíduos (\textit{Residual-Based Variational Multi-Scale} - RBVMS) que consiste em: determinar $\BBB{u}\in\bar{\script{S}}_u$ e $\bar{p}\in\bar{\script{S}}_p$, tais que para todo $\BBB{w}\in\bar{\script{V}}_u$ e $\bar{q}\in\bar{\script{V}}_p$:

\begin{equation}
    \begin{split}
        &\intDom{\BBB{w}\cdot\rho\bigpar{\dTime{\BBB{u}}+((\BBB{u}-\HBB{u})\cdot\Ny)\BBB{u}-\BBB{f}}}+\intDom{\Ny\BBB{w}:\bar{\tens}}-\intNeumann{\BBB{w}\cdot\BBB{h}}+\\
        &\intDom{\bar{q}\Ny\cdot\BBB{u}}-\SintDom{\rho\bigpar{((\BBB{u}-\HBB{u})\cdot\Ny)\BBB{w}+\frac{\Ny\bar{q}}{\rho}}\cdot\BB{u}'}-\\
        &\SintDom{\Ny\cdot\BBB{w}p'}+\SintDom{\rho\BBB{w}\cdot((\BB{u}'\cdot\Ny)\BBB{u})}-\\
        &\SintDom{\rho\Ny\BBB{w}:(\BB{u}'\otimes\BB{u}')}=0
        \text{,}
        \label{eq:RBVMS1}
    \end{split}
\end{equation}

\noindent o que pode ser reescrito ainda, substituindo-se $\BB{u}'$ e $p'$ por \eqref{eq:VMS-fine}:

\begin{equation}
    \begin{split}
        &\intDom{\BBB{w}\cdot\rho\bigpar{\dTime{\BBB{u}}+((\BBB{u}-\HBB{u})\cdot\Ny)\BBB{u}-\BBB{f}}}+\intDom{\Ny\BBB{w}:\bar{\tens}}-\intNeumann{\BBB{w}\cdot\BBB{h}}+\\
        &\intDom{\bar{q}\Ny\cdot\BBB{u}}+\SintDom{\tau_\sups\bigpar{((\BBB{u}-\HBB{u})\cdot\Ny)\BBB{w}+\frac{\Ny\bar{q}}{\rho}}\rM}+\\
        &\SintDom{\rho\nu_\lsic\Ny\cdot\BBB{w}\rC}-\SintDom{\tau_\sups\BBB{w}\cdot((\rM\cdot\Ny)\BBB{u})}-\\
        &\SintDom{\frac{\Ny\BBB{w}}{\rho}:((\tau_\sups\rM)\otimes(\tau_\sups\rM))}=0
        \text{.}
        \label{eq:RBVMS1}
    \end{split}
\end{equation}

Fazendo a aproximação por função de forma e levando em conta a arbitrariedade das funções teste, obtém-se os seguintes vetores resíduo:

\begin{subequations}
    \begin{equation}
        \begin{split}
            \NM^a=&
            \intDom{N_a^u\rho\bigpar{\dTime{\BBB{u}}+((\BBB{u}-\HBB{u})\cdot\Ny)\BBB{u}-\BBB{f}}}+\intDom{\bar{\tens}\cdot\Ny N_a^u}-\\
            &\intNeumann{N_a^u\BBB{h}}+\SintDom{\tau_\sups\bigpar{(\BBB{u}-\HBB{u})\cdot\Ny N_a^u}\rM}+\\
            &\SintDom{\rho\nu_\lsic\Ny N_a^u\rC}-\SintDom{\tau_\sups N_a^u((\rM\cdot\Ny)\BBB{u})}-\\
            &\SintDom{\frac{\Ny N_a^u}{\rho}\cdot((\tau_\sups\rM)\otimes(\tau_\sups\rM))}
            =\BB{0}\text{,}
        \end{split}
    \end{equation}
    \begin{equation}
        \NC^a=\intDom{N_a^p\Ny\cdot\BBB{u}}+\SintDom{\tau_\sups\frac{\Ny N_a^p}{\rho}\cdot\rM}=0
    \end{equation}
    \label{Eq:Residuos-Euler}
\end{subequations}

Logo o problema a ser resolvido consiste em: encontrar $\BB{U}$, $\dot{\BB{U}}$ e $\BB{P}$, tais que:

\begin{subequations}
    \begin{align}
         & \NM(\BB{U},\dot{\BB{U}},\BB{P})=\BB{0}\text{ e} \\
         & \NC(\BB{U},\dot{\BB{U}},\BB{P})=\BB{0}\text{.}
    \end{align}
\end{subequations}

Note que omitindo-se os dois últimos termos da equação \eqref{eq:RBVMS1} e adotando-se valores diferentes de $\tau$ para as equações de conservação da quantidade de movimento e da continuidade, chega-se à formulação estabilizada SUPG/PSPG \cite{tezduyar2000finite,tezduyar2003computation,catabriga2005compressible,catabriga2006compressible}, dada por:

\begin{equation}
    \begin{split}
        &\intDom{\BBB{w}\cdot\rho\bigpar{\dTime{\BBB{u}}+((\BBB{u}-\HBB{u})\cdot\Ny)\BBB{u}-\BBB{f}}}+\intDom{\Ny\BBB{w}:\bar{\tens}}-\intNeumann{\BBB{w}\cdot\BBB{h}}+\\
        &\intDom{\bar{q}\Ny\cdot\BBB{u}}+\SintDom{\tau_\supg((\BBB{u}-\HBB{u})\cdot\Ny)\BBB{w}\cdot\rM}+\\
        &\SintDom{\tau_\pspg\frac{\Ny\bar{q}}{\rho}\cdot\rM}+\SintDom{\rho\nu_\lsic\Ny\cdot\BBB{w}\rC}=0
        \text{,}
        \label{eq:SUPG-PSPG}
    \end{split}
\end{equation}

\noindent onde, segundo \citeonline{bazilevs2013computational}, adota-se $\tau_\pspg=\tau_\supg=\tau_\sups$ com sucesso para uma boa variedade de problemas.

%==================================================================================================
\subsection{Procedimento iterativo} \label{Comp-VMS}
%==================================================================================================

O procedimento para minimizar os vetores resíduo obtido parte do método de Newton-Raphson, no qual os valores a serem corrigidos são os vetores de acelerações nodais ($\dot{\BB{U}}$) e de pressões nodais ($\BB{P}$). Dessa forma, o problema a ser resolvido para a correção dessas variáveis é:

\begin{equation}
    \begin{bmatrix}
        \der{\NM}{\dBB{U}^{n+1}} & \der{\NM}{\BB{P}^{n+1}} \\
        \der{\NC}{\dBB{U}^{n+1}} & \der{\NC}{\BB{P}^{n+1}}
    \end{bmatrix}
    \begin{bmatrix}
        \Delta\dBB{U}^{n+1} \\
        \Delta\BB{P}^{n+1}
    \end{bmatrix}=-
    \begin{bmatrix}
        \NM \\
        \NC
    \end{bmatrix}\text{,}
    \label{Eq:NR-NS}
\end{equation}

\noindent em que a matriz tangente elementar em descrição ALE é dada por:

\begin{subequations}
    \begin{equation}
        \begin{split}
            \der{\NM^a}{(\dBB{U}^b)^{n+1}}=&\am\intDoma{\rho\kr{a}N_b}\TS{I}+\agdt\intDoma{\rho\kr{a}\dconv{b}}\TS{I}+\\
            &\agdt\intDoma{\mu \Ny N_a\cdot\Ny N_b}\TS{I}+\agdt\intDoma{\rho\kr{a}N_b\Ny\BBB{u}}+\\
            &\agdt\intDoma{(\mu(\Ny N_b\otimes\Ny N_a)+\rho\nlsic(\Ny N_a\otimes\Ny N_b))}-\\
            &\am\intDoma{\rho\tsups N_aN_b\Ny\BBB{u}}-\\
            &\agdt\intDoma{\rho\tsups N_a(\dconv{b}\TS{I}+N_b\Ny\BBB{u})\cdot\Ny\BBB{u}}\text{,}
        \end{split}
    \end{equation}
    \begin{equation}
        \der{\NM^a}{(P^b)^{n+1}}=-\intDoma{N_b\Ny N_a}+\intDoma{\tsups(\dconv{a}\TS{I}-N_a\Ny\BBB{u})\cdot\Ny N_b}\text{,}
    \end{equation}
    \begin{equation}
        \begin{split}
            \der{\NC^a}{(\dBB{U}^b)^{n+1}}=&\agdt\intDoma{N_a\Ny N_b}+\am\intDoma{\tsups N_b\Ny N_a}+\\
            &\agdt\intDoma{\tsups(\dconv{b}\TS{I}+N_b\NyT\BBB{u})\cdot\Ny N_a}\text{ e}
        \end{split}
    \end{equation}
    \begin{equation}
        \der{\NC^a}{(P^b)^{n+1}}=\intDoma{\frac{\tsups}{\rho}\Ny N_a\cdot\Ny N_b}\text{,}
    \end{equation}
    \label{Eq:MatrizTangente-VMS}
\end{subequations}

\noindent em que $\dconv{a}=\UUB\cdot\Ny N_{a}$, $\kr{a}=N_a+\tsups\dconv{a}$ e:

\begin{equation}
    \Omega_\alpha=\left\{\bar{\BB{x}}|\bar{\BB{x}}(\hat{\BB{x}},t^{n+\alpha_f})=\alpha_f\bar{\BB{x}}(\hat{\BB{x}},t^{n+1})+(1-\alpha_f)\bar{\BB{x}}(\hat{\BB{x}},t^n)\right\}\text{.}
\end{equation}

Para uma descrição Euleriana, considera-se a velocidade da malha como nula na formulação apresentada.

Sendo assim, o algoritmo \ref{alg:MEF-VMS} mostra o procedimento para obtenção da solução aproximada.

\begin{algorithm}[h!]
    \caption{Algoritmo para a solução de escoamento incompressíveis via VMS.}
    \label{alg:MEF-VMS}
    \KwResult{Vetor de velocidades, acelerações e pressões nodais}
    \ForEach{\textnormal{passo de tempo}}{
    Previsão dos valores nodais:\newline
    $
        \begin{bmatrix}
            \dBB{U}^{n+1} \\\BB{U}^{n+1}\\\BB{P}^{n+1}
        \end{bmatrix}^0\gets
        \begin{bmatrix}
            ((\gamma-1)/\gamma)\dBB{U}^n \\
            \BB{U}^n                     \\
            \BB{P}^n
        \end{bmatrix}
    $\\
    \ForEach{\textnormal{iteração de Newton-Raphson}}
    {
    \ForEach{\textnormal{elemento}}{
    Interpolação das variáveis:\newline
    $
        \begin{bmatrix}
            \dBB{U}^{n+\alpha_m} \\\BB{U}^{n+\alpha_f}\\\BB{P}^{n+1}
        \end{bmatrix}^{k+1}\gets
        \begin{bmatrix}
            \alpha_m \BB{\dot{U}}^{n+1}+(1-\alpha_m)\BB{\dot{U}}^n \\
            \alpha_f \BB{U}^{n+1}+(1-\alpha_f)\BB{U}^n             \\
            \BB{P}^n
        \end{bmatrix}^{k-1}
    $\\
    Cálculo dos termos estabilizadores $\tau_\sups$ e $\nu_\lsic$ \eqref{eq:stabilizatingTerms}\;
    Cálculo da matriz tangente \eqref{Eq:MatrizTangente-VMS} e dos vetores de resíduos \eqref{Eq:Residuos-Euler}\;
    }
    Montagem do sistema global e aplicação das condições de contorno\;
    Determinar as correções nodais pela solução do sistema \eqref{Eq:NR-NS}\;
    Correção dos valores:\newline
    $
        \begin{bmatrix}
            \dBB{U}^{n+1} \\\BB{U}^{n+1}\\\BB{P}^{n+1}
        \end{bmatrix}^{k+1}\gets
        \begin{bmatrix}
            \dBB{U}^{n+1} \\\BB{U}^{n+1}\\\BB{P}^{n+1}
        \end{bmatrix}^{k}+
        \begin{bmatrix}
            \Delta\dBB{U}^{n+1} \\\gamma\Delta t\Delta\dBB{U}^{n+1}\\\Delta\BB{P}^{n+1}
        \end{bmatrix}^{k}
    $\\
    Cálculo da medida de convergência: $\epsilon=\norm{\NM^k}_{L^2}$
    }
    }
\end{algorithm}