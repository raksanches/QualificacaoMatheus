%==================================================================================================
\section{\textit{Variational Multi-Scale}} \label{VMS}
%==================================================================================================

O Método Variacional Multiescala, introduzido por \citeonline{hughes1995multiscale,hughes1998variational,hughes2000large}, é um modelo de estabilização baseado na separação dos espaços tentativa e teste em subespaços que representam as escalas grosseiras, que se tratam de subespaços de dimensão finita, denotadas por uma barra sobrescrita, e subespaços que representam as escalas refinadas, que são subespaços de dimensão infinita e denotadas por $'$, ou seja:

\begin{subequations}
    \begin{align}
         & \script{S}_u=\bar{\script{S}}_u\oplus\script{S}'_u\text{,}  \\
         & \script{S}_p=\bar{\script{S}}_p\oplus\script{S}'_p\text{,}  \\
         & \script{V}_u=\bar{\script{V}}_u\oplus\script{V}'_u\text{ e} \\
         & \script{V}_p=\bar{\script{V}}_p\oplus\script{V}'_p\text{.}
    \end{align}
\end{subequations}

Assim, o sistema a ser resolvido parte do apresentado na Eq. \eqref{eq:NS-Euler}, que em sua forma fraca se encontra em \eqref{eq:WeakForm2}. Primeiramente realiza-se a separação dos membros em:

\begin{subequations}
    \begin{align}
         & \BB{u}=\BBB{u}+\BB{u}'\text{,}  \\
         & p=\bar{p}+p'\text{,}            \\
         & \BB{w}=\BBB{w}+\BB{w}'\text{ e} \\
         & q=\bar{q}+q'\text{.}
    \end{align}
\end{subequations}

Seguindo \citeonline{bazilevs2007variational} adota-se $\BB{w}=\BBB{w}$, $q=\bar{q}$ e assume-se que a parcela em escala fina $\BB{u}'$ e $p'$ são modeladas como:

\begin{subequations}
    \begin{equation}
        \BB{u}'=-\frac{\tau_{\sups}}{\rho}\rM\text{ e}
    \end{equation}
    \begin{equation}
        p'=-\rho\nu_{\lsic}\rC\text{,}
    \end{equation}
    \label{eq:VMS-fine}
\end{subequations}

\noindent onde $\tau_{\sups}$ e $\nu_{\lsic}$ são termos estabilizadores, propostos por \citeonline{bazilevs2013computational} como:
%página 80 do pdf de hughes2013 e 65 de fernandes2020

\begin{subequations}
    \begin{equation}
        \tau_{\sups}=\bigpar{\frac{4}{\Delta t^2}+(\BBB{u}-\HBB{u})\cdot\BB{G}(\BBB{u}-\HBB{u})+C_I\nu^2\BB{G}:\BB{G}}^{-1/2}\text{ e}
    \end{equation}
    \begin{equation}
        \nu_{\lsic}=(\tr{\BB{G}}\tau_{\sups})^{-1}\text{,}
    \end{equation}
    \label{eq:TermEstab}
\end{subequations}

\noindent onde $C_I$ é uma constante e:

\begin{equation}
    \BB{G}=\frac{\partial\BB{\xi}}{\partial\BB{y}}^T\frac{\partial\BB{\xi}}{\partial\BB{y}}\text{.}
\end{equation}

Já os termos $\rM$ e $\rC$ são os resíduos da equação de conservação da quantidade de movimento e da equação da continuidade, respectivamente, ou seja,

\begin{subequations}
    \begin{equation}
        \rM=\rho\bigpar{\dot{\BBB{u}}+((\BBB{u}-\HBB{u})\cdot\Ny)\BBB{u}-\BBB{f}}-\Ny\cdot\bar{\tens}\text{ e}
    \end{equation}
    \begin{equation}
        \rC=\Ny\cdot\BBB{u}\text{.}
    \end{equation}
\end{subequations}

Outra forma de se determinar $ \tau_{\sups}$ e $\nu_{\lsic}$, também apresentada em \citeonline{bazilevs2013computational}, consiste em fazer:

\begin{subequations}
    \begin{equation}
        \tau_{\sups}=\bigpar{\frac{1}{\tau_{\sugn 1}^2}+\frac{1}{\tau_{\sugn 2}^2}+\frac{1}{\tau_{\sugn 3}^2}}^{-1/2}\text{ e}
    \end{equation}
    \begin{equation}
        \nu_{\lsic}=\tau_{\sups}\norm{\BBB{u}-\HBB{u}}^2\text{,}
    \end{equation}
    \label{eq:stabilizatingTerms}
\end{subequations}

\noindent sendo:

\begin{subequations}
    \begin{align}
         & \tau_{\sugn 1}=\bigpar{\sum_{a=1}^{n_{en}}{\abs{(\BBB{u}-\HBB{u})\cdot\Ny N_a}}}^{-1}\text{,} \\
         & \tau_{\sugn 2}=\frac{\Delta t}{2}\text{,}                                                     \\
         & \tau_{\sugn 3}=\frac{h_{\rgn}^2}{4\nu}\text{,}                                                \\
         & h_\rgn=2\bigpar{\sum_{a=1}^{n_{en}}{\abs{\BB{r}\cdot\Ny N_a}}}^{-1}\text{ e}                  \\
         & \BB{r}=\frac{\Ny\norm{\BBB{u}-\HBB{u}}}{\norm{\Ny\norm{\BBB{u}-\HBB{u}}}}\text{.}
    \end{align}
\end{subequations}

Assim, obtém-se o problema do Método Variacional Multiescala Baseado em Resíduos (\textit{Residual-Based Variational Multi-Scale} - RBVMS) que consiste em: determinar $\BBB{u}\in\bar{\script{S}}_u$ e $\bar{p}\in\bar{\script{S}}_p$, tais que para todo $\BBB{w}\in\bar{\script{V}}_u$ e $\bar{q}\in\bar{\script{V}}_p$:

\begin{equation}
    \begin{split}
        &\intDom{\BBB{w}\cdot\rho\bigpar{\dot{\BBB{u}}+((\BBB{u}-\HBB{u})\cdot\Ny)\BBB{u}-\BBB{f}}}+\intDom{\Ny\BBB{w}:\bar{\tens}}-\intNeumann{\BBB{w}\cdot\BBB{h}}+\\
        &\intDom{\bar{q}\Ny\cdot\BBB{u}}-\SintDom{\rho\bigpar{((\BBB{u}-\HBB{u})\cdot\Ny)\BBB{w}+\frac{\Ny\bar{q}}{\rho}}\cdot\BB{u}'}-\\
        &\SintDom{\Ny\cdot\BBB{w}p'}+\SintDom{\rho\BBB{w}\cdot((\BB{u}'\cdot\Ny)\BBB{u})}-\\
        &\SintDom{\rho\Ny\BBB{w}:(\BB{u}'\otimes\BB{u}')}=0
        \text{,}
        \label{eq:RBVMS1}
    \end{split}
\end{equation}

\noindent o que pode ser reescrito ainda, substituindo-se $\BB{u}'$ e $p'$ por \eqref{eq:VMS-fine}:

\begin{equation}
    \begin{split}
        &\intDom{\BBB{w}\cdot\rho\bigpar{\dot{\BBB{u}}+((\BBB{u}-\HBB{u})\cdot\Ny)\BBB{u}-\BBB{f}}}+\intDom{\Ny\BBB{w}:\bar{\tens}}-\intNeumann{\BBB{w}\cdot\BBB{h}}+\\
        &\intDom{\bar{q}\Ny\cdot\BBB{u}}+\SintDom{\tau_\sups\bigpar{((\BBB{u}-\HBB{u})\cdot\Ny)\BBB{w}+\frac{\Ny\bar{q}}{\rho}}\rM}+\\
        &\SintDom{\rho\nu_\lsic\Ny\cdot\BBB{w}\rC}-\SintDom{\tau_\sups\BBB{w}\cdot((\rM\cdot\Ny)\BBB{u})}-\\
        &\SintDom{\frac{\Ny\BBB{w}}{\rho}:((\tau_\sups\rM)\otimes(\tau_\sups\rM))}=0
        \text{.}
        \label{eq:RBVMS1}
    \end{split}
\end{equation}

Para a discretização do problema em descrição Euleriana pode-se realizar a separação da dependência espacial e temporal para as funções tentativas e testes como:

\begin{subequations}
    \begin{align}
         & \BBB{u}(\BB{y},t)=\sum_{\BB{\eta}^s}{\BB{U}^a(t)N_a^u(\BB{y})}\text{,}\label{eq:uh} \\
         & \bar{p}(\BB{y},t)=\sum_{\BB{\eta}^s}{P^a(t)N_a^p(\BB{y})}\text{,}                   \\
         & \BBB{w}(\BB{y})=\sum_{\BB{\eta}^w}{\BB{W}^aN_a^u(\BB{y})}\text{ e}\label{eq:w-sep}  \\
         & \bar{q}(\BB{y})=\sum_{\BB{\eta}^w}{Q^aN_a^p(\BB{y})}\text{,}\label{eq:q-sep}
    \end{align}
\end{subequations}

\noindent onde $N_a^u$ e $N_a^p$ são as funções de forma para aproximação do campo de velocidades e de pressões, respectivamente, associadas ao nó $a$ da malha e $\BB{\eta}^s$ o respectivo índice desse nó, assim como $\BB{\eta}^w$ são os índices referentes aos nós da malha para as funções teste.

Já para a descrição ALE as funções de forma $N_a(\BB{y},t)$ são definidas com base na configuração de referência da malha como:

\begin{equation}
    N_a(\BB{y},t)=\hat{N}_a(\hfmc^{-1}(\BB{y},t))\text{,}
\end{equation}

\noindent onde $\hfmc(\BB{y},t)$ é a função de mudança de configuração da configuração de referência da malha para a configuração atual $\hat{\Omega}\to\Omega$, conforme apresentado no item \ref{CFD-ALE}, dada em sua forma discreta por:

\begin{equation}
    \hfmc(\HBB{x},t)=\sum_{a\in\BB{\eta}^s}{(\HBB{x}_a+\Delta\HBB{x}_a(t))\hat{N}_a(\HBB{x})}\text{,}
\end{equation}

\noindent sendo $\HBB{x}_a$ as posições nodais em $\hat{\Omega}$, $\Delta\HBB{x}(t)$ o deslocamento nodal e $\hat{N}_a$ é a função de forma fixa da discretização de $\hat{\Omega}$. Nota-se, portanto, que as funções $N_a(\BB{y},t)$ possuem dependência temporal devido à movimentação da malha.

Substituindo-se \eqref{eq:w-sep} e \eqref{eq:q-sep} em \eqref{eq:RBVMS1}, e levando em conta a arbitrariedade de $W^a$ e $Q^a$, a forma fraca espacialmente discreta fica:

\begin{subequations}
    \begin{equation}
        \begin{split}
            \NM^a=&
            \intDom{N_a^u\rho\bigpar{\dot{\BBB{u}}+((\BBB{u}-\HBB{u})\cdot\Ny)\BBB{u}-\BBB{f}}}+\intDom{\bar{\tens}\cdot\Ny N_a^u}-\\
            &\intNeumann{N_a^u\BBB{h}}-\SintDom{\rho\bigpar{(\BBB{u}-\HBB{u})\cdot\Ny N_a^u}\BB{u}'}-\SintDom{\Ny N_a^up'}+\\
            &\SintDom{\rho N_a^u((\BB{u}'\cdot\Ny)\BBB{u})}-\SintDom{\rho\Ny N_a^u\cdot(\BB{u}'\otimes\BB{u}')}
            =\BB{0}\text{,}
        \end{split}
    \end{equation}
    \begin{equation}
        \NC^a=\intDom{N_a^p\Ny\cdot\BBB{u}}-\SintDom{\Ny N_a^p\cdot\BB{u}'}=0
    \end{equation}
    \label{Eq:Residuos-Euler}
\end{subequations}

Sendo os vetores $\BB{U}=[\BB{u}_B]$, $\dot{\BB{U}}=[\dot{\BB{u}}_B]$ e $\BB{P}=[p_B]$, que representam, respectivamente, os valores nodais de velocidade, sua primeira derivada temporal e os valores nodais de pressão, então o problema a ser resolvido consiste em: encontrar $\BB{U}$, $\dot{\BB{U}}$ e $\BB{P}$, tais que:

\begin{subequations}
    \begin{align}
         & \NM(\BB{U},\dot{\BB{U}},\BB{P})=\BB{0}\text{ e} \\
         & \NC(\BB{U},\dot{\BB{U}},\BB{P})=\BB{0}\text{.}
    \end{align}
\end{subequations}

Omitindo-se os dois últimos termos da equação \eqref{eq:RBVMS1} e adotando-se valores diferentes de $\tau$ para as equações de conservação da quantidade de movimento e da continuidade, chega-se à formulação estabilizada SUPG/PSPG \cite{tezduyar2000finite,tezduyar2003computation,catabriga2005compressible,catabriga2006compressible}, dada por:

\begin{equation}
    \begin{split}
        &\intDom{\BBB{w}\cdot\rho\bigpar{\dot{\BBB{u}}+((\BBB{u}-\HBB{u})\cdot\Ny)\BBB{u}-\BBB{f}}}+\intDom{\Ny\BBB{w}:\bar{\tens}}-\intNeumann{\BBB{w}\cdot\BBB{h}}+\\
        &\intDom{\bar{q}\Ny\cdot\BBB{u}}+\SintDom{\tau_\supg((\BBB{u}-\HBB{u})\cdot\Ny)\BBB{w}\cdot\rM}+\\
        &\SintDom{\tau_\pspg\frac{\Ny\bar{q}}{\rho}\cdot\rM}+\SintDom{\rho\nu_\lsic\Ny\cdot\BBB{w}\rC}=0
        \text{,}
        \label{eq:SUPG-PSPG}
    \end{split}
\end{equation}

\noindent onde, segundo \citeonline{bazilevs2013computational}, adota-se $\tau_\pspg=\tau_\supg=\tau_\sups$ com sucesso para uma boa variedade de problemas.

%==================================================================================================
\subsection{Integração temporal} \label{IT-VMS}
%==================================================================================================

Até o momento, apresentou-se uma forma semi-discreta das equações governantes (discreta no espaço, porém contínua no tempo) (Equações \eqref{Eq:Residuos-Euler}). Dessa forma torna-se necessária a devida discretização temporal das variáveis, que pode ocorrer de diferentes formas. Tal como apontado por \citeonline{reddy2010finite}, tem-se, por exemplo, integradores explícitos, como o integrador baseado em diferenças adiantadas, implícitos, como o baseado em diferenças finitas atrasadas, e os denominados semi-explícitos. Segundo o autor os integradores implícitos possuem vantagens sobre os explícitos, uma vez que se observa a implicidade natural da pressão em escoamentos incompressíveis e é possível obtenção de estabilidade incondicional do integrador temporal.

O integrador temporal utilizado no presente trabalho é o denominado integrador $\alpha$-generalizado, desenvolvido por \citeonline{chung1993time}, que possui a capacidade de representar adequadamente problema de escoamentos incompressíveis, além de permitir a introdução de difusão numérica ao processo de forma controlada \cite{fernandes2020tecnica}.

Esse integrador parte da consideração de valores intermediários de aceleração e velocidade em um intervalo $[t_n,t_{n+1}]$ no $n$-ésimo passo de tempo, representados respectivamente por $\dBB{U}^{n+\alpha_m}$ e $\BB{U}^{n+\alpha_f}$:

\begin{subequations}
    \begin{equation}
        \dBB{U}^{n+\alpha_m}=\dBB{U}^n+\alpha_m(\dBB{U}^{n+1}-\dBB{U}^n)\text{ e}
    \end{equation}
    \begin{equation}
        \BB{U}^{n+\alpha_f}=\BB{U}^n+\alpha_f(\BB{U}^{n+1}-\BB{U}^n)\text{.}
    \end{equation}
\end{subequations}

Já para se relacionar a velocidade à aceleração, emprega-se a aproximação de Newmark \cite{bazilevs2013computational}:

\begin{equation}
    \BB{U}^{n+1}=\BB{U}^n+\Delta t_n\bigpar{(1-\gamma)\dBB{U}^n+\gamma\dBB{U}^{n+1}}\text{,}
\end{equation}

\noindent sendo $\alpha_m$, $\alpha_f$ e $\gamma$ valores escolhidos arbitrariamente, observando-se as necessidades de estabilidade e precisão do método.

De acordo com \citeonline{chung1993time,jansen2000generalized,bazilevs2013computational}, a precisão de segunda ordem dessa aproximação pode ser atingida uma vez que:

\begin{equation}
    \gamma=\frac{1}{2}+\alpha_m-\alpha_f\text{,}
\end{equation}

\noindent enquanto a estabilidade incondicional é obtida caso:

\begin{equation}
    \alpha_m\geq\alpha_f\geq\frac{1}{2}\text{.}
\end{equation}

Ainda é possível escrever, a partir da Equação \eqref{eq:one-par-stable}, $\alpha_m$ e $\alpha_f$ em termos de um parâmetro arbitrário único  ($0\leq\rho_\infty\leq1$), que representa o raio espectral da matriz amplificação para $\Delta t\to\infty$, o qual é utilizado para controlar as dissipações de alta-frequência.

\begin{subequations}
    \begin{equation}
        \alpha_m=\frac{1}{2}\bigpar{\frac{3-\rho_\infty}{1+\rho_\infty}}\text{ e}
    \end{equation}
    \begin{equation}
        \alpha_f=\frac{1}{1+\rho_\infty}\text{.}
    \end{equation}
    \label{eq:one-par-stable}
\end{subequations}

Para o caso de $\rho_\infty=1$ não ocorre a introdução de difusão numérica, enquanto para $\rho_\infty=0$ se tem a máxima dissipação de altas frequências \cite{fernandes2020tecnica}.

Sendo assim, os resíduos obtidos anteriormente podem ser escritos em termos dos valores intermediários como:

\begin{subequations}
    \begin{equation}
        \NM(\dot{\BB{U}}^{n+\alpha_m},\BB{U}^{n+\alpha_f},\BB{P}^{n+1})=\BB{0}
    \end{equation}
    \begin{equation}
        \NC(\dot{\BB{U}}^{n+\alpha_m},\BB{U}^{n+\alpha_f},\BB{P}^{n+1})=\BB{0}
    \end{equation}
\end{subequations}

%==================================================================================================
\subsection{Procedimento iterativo} \label{Comp-VMS}
%==================================================================================================

O procedimento para minimizar os vetores resíduo obtido parte do método de Newton-Raphson, no qual os valores a serem corrigidos são os vetores de acelerações nodais ($\dot{\BB{U}}$) e de pressões nodais ($\BB{P}$). Dessa forma, o problema a ser resolvido para a correção dessas variáveis é:

\begin{equation}
    \begin{bmatrix}
        \der{\NM}{\dBB{U}^{n+1}} & \der{\NM}{\BB{P}^{n+1}} \\
        \der{\NC}{\dBB{U}^{n+1}} & \der{\NC}{\BB{P}^{n+1}}
    \end{bmatrix}\cdot
    \begin{bmatrix}
        \Delta\dBB{U}^{n+1} \\
        \Delta\BB{P}^{n+1}
    \end{bmatrix}=-
    \begin{bmatrix}
        \NM \\
        \NC
    \end{bmatrix}\text{,}
    \label{Eq:NR-NS}
\end{equation}

\noindent em que a matriz tangente elementar em descrição ALE é dada por:

\begin{subequations}
    \begin{equation}
        \begin{split}
            \der{\NM^a}{(\dBB{U}^b)^{n+1}}=&\am\intDoma{\rho\kr{a}N_b}\BB{I}+\agdt\intDoma{\rho\kr{a}\dconv{b}}\BB{I}+\\
            &\agdt\intDoma{\mu \Ny N_a\cdot\Ny N_b}\BB{I}+\agdt\intDoma{\rho\kr{a}N_b\Ny\BBB{u}}+\\
            &\agdt\intDoma{(\mu(\Ny N_b\otimes\Ny N_a)+\rho\nlsic(\Ny N_a\otimes\Ny N_b))}-\\
            &\am\intDoma{\rho\tsups N_aN_b\Ny\BBB{u}}-\\
            &\agdt\intDoma{\rho\tsups N_a(\dconv{b}\BB{I}+N_b\Ny\BBB{u})\cdot\Ny\BBB{u}}\text{,}
        \end{split}
    \end{equation}
    \begin{equation}
        \der{\NM^a}{(P^b)^{n+1}}=-\intDoma{N_b\Ny N_a}+\intDoma{\tsups(\dconv{a}\BB{I}-N_a\Ny\BBB{u})\cdot\Ny N_b}\text{,}
    \end{equation}
    \begin{equation}
        \begin{split}
            \der{\NC^a}{(\dBB{U}^b)^{n+1}}=&\agdt\intDoma{N_a\Ny N_b}+\am\intDoma{\tsups N_b\Ny N_a}+\\
            &\agdt\intDoma{\tsups(\dconv{b}\BB{I}+N_b\NyT\BBB{u})\cdot\Ny N_a}\text{ e}
        \end{split}
    \end{equation}
    \begin{equation}
        \der{\NC^a}{(P^b)^{n+1}}=\intDoma{\frac{\tsups}{\rho}\Ny N_a\cdot\Ny N_b}\text{,}
    \end{equation}
    \label{Eq:MatrizTangente-VMS}
\end{subequations}

\noindent em que $\dconv{a}=\UUB\cdot\Ny N_{a}$, $\kr{a}=N_a+\tsups\dconv{a}$ e:

\begin{equation}
    \Omega_\alpha=\left\{\bar{\BB{x}}|\bar{\BB{x}}(\hat{\BB{x}},t^{n+\alpha_f})=\alpha_f\bar{\BB{x}}(\hat{\BB{x}},t^{n+1})+(1-\alpha_f)\bar{\BB{x}}(\hat{\BB{x}},t^n)\right\}\text{.}
\end{equation}

Para uma descrição Euleriana, considera-se a velocidade da malha como nula na formulação apresentada.

Sendo assim, o pseudocódigo apresentado no algoritmo \ref{alg:MEF-VMS} mostra o procedimento para obtenção da solução aproximada.

\begin{algorithm}[h!]
    \caption{Algoritmo para a solução de escoamento incompressíveis via VMS.}
    \label{alg:MEF-VMS}
    \KwResult{Vetor de velocidades, acelerações e pressões nodais}
    \ForEach{\textnormal{passo de tempo}}{
    Previsão dos valores nodais:\newline
    $
        \begin{bmatrix}
            \dBB{U}^{n+1} \\\BB{U}^{n+1}\\\BB{P}^{n+1}
        \end{bmatrix}^0\gets
        \begin{bmatrix}
            ((\gamma-1)/\gamma)\dBB{U}^n \\
            \BB{U}^n                     \\
            \BB{P}^n
        \end{bmatrix}
    $\\
    \ForEach{\textnormal{iteração de Newton-Raphson}}
    {
    \ForEach{\textnormal{elemento}}{
    Interpolação das variáveis:\newline
    $
        \begin{bmatrix}
            \dBB{U}^{n+\alpha_m} \\\BB{U}^{n+\alpha_f}\\\BB{P}^{n+1}
        \end{bmatrix}^{k+1}\gets
        \begin{bmatrix}
            \alpha_m \BB{\dot{U}}^{n+1}+(1-\alpha_m)\BB{\dot{U}}^n \\
            \alpha_f \BB{U}^{n+1}+(1-\alpha_f)\BB{U}^n             \\
            \BB{P}^n
        \end{bmatrix}^{k-1}
    $\\
    Cálculo dos termos estabilizadores $\tau_\sups$ e $\nu_\lsic$ \eqref{eq:stabilizatingTerms}\;
    Cálculo da matriz tangente \eqref{Eq:MatrizTangente-VMS} e dos vetores de resíduos \eqref{Eq:Residuos-Euler}\;
    }
    Montagem do sistema global e aplicação das condições de contorno\;
    Determinar as correções nodais pela solução do sistema \eqref{Eq:NR-NS}\;
    Correção dos valores:\newline
    $
        \begin{bmatrix}
            \dBB{U}^{n+1} \\\BB{U}^{n+1}\\\BB{P}^{n+1}
        \end{bmatrix}^{k+1}\gets
        \begin{bmatrix}
            \dBB{U}^{n+1} \\\BB{U}^{n+1}\\\BB{P}^{n+1}
        \end{bmatrix}^{k}+
        \begin{bmatrix}
            \Delta\dBB{U}^{n+1} \\\gamma\Delta t\Delta\dBB{U}^{n+1}\\\Delta\BB{P}^{n+1}
        \end{bmatrix}^{k}
    $\\
    Cálculo do erro: $\epsilon=\norm{\NM^k}_{L^2}$
    }
    }
\end{algorithm}