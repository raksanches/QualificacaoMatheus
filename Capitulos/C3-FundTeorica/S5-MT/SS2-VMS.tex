\documentclass[_ArquivoPrincipal.tex]{subfiles}

\begin{document}

%==================================================================================================
\subsection{\textit{Variational Multi-Scale}} \label{VMS}
%==================================================================================================

O Método Variacional Multiescala foi introduzido por \citeonline{hughes1995multiscale,hughes1998variational,hughes2000large}, o qual faz a separação dos espaços de tentativas e de testes em subespaços que representem as escalas grosseiras, que se tratam de subespaços de dimensões finitas e denotadas por uma barra, e as escalas finas, que são subespaços de infinitas dimensões e denotadas por $'$, ou seja:

\begin{subequations}
    \begin{align}
         & \script{S}_u=\bar{\script{S}}_u\oplus\script{S}'_u\text{,}  \\
         & \script{S}_p=\bar{\script{S}}_p\oplus\script{S}'_p\text{,}  \\
         & \script{V}_u=\bar{\script{V}}_u\oplus\script{V}'_u\text{ e} \\
         & \script{V}_p=\bar{\script{V}}_p\oplus\script{V}'_p\text{.}
    \end{align}
\end{subequations}

Inicialmente será abordado uma técnica baseada em uma descrição Euleriana com domínio fixo, para maior familiarização com o método, e na sequência será apresentada uma formulação em descrição ALE utilizando domínio móvel.

O sistema a ser resolvido parte do apresentado em \ref{eq:NS-Euler}, que em sua forma fraca se encontra em \ref{eq:WeakForm2}. Primeiramente realiza-se a separação dos membros em:

\begin{subequations}
    \begin{align}
         & u_i=\bar{u}_i+u'_i\text{,}  \\
         & p=\bar{p}+p'\text{,}        \\
         & w_i=\bar{w}_i+w'_i\text{ e} \\
         & q=\bar{q}+q'\text{,}
    \end{align}
\end{subequations}

\noindent em que se adota $w_i=\bar{w}_i$ e $q=\bar{q}$ e as escalas finas $u'_i$ e $p'$ podem ser modeladas como:

\begin{subequations}
    \begin{equation}
        \BB{u}'=-\frac{\tau_{\sups}}{\rho}\rM\text{ e}
    \end{equation}
    \begin{equation}
        p'=-\rho\nu_{\lsic}\rC\text{,}
    \end{equation}
\end{subequations}

\noindent nas quais $\tau_{\sups}$ e $\nu_{\lsic}$ são termos estabilizadores, dados por \cite{bazilevs2013computational}:
%página 80 do pdf de hughes2013 e 65 de fernandes2020

\begin{subequations}
    \begin{equation}
        \tau_{\sups}=\bigpar{\frac{4}{\Delta t^2}+\BBB{u}\cdot\BB{G}\BBB{u}+C_I\nu^2\BB{G}:\BB{G}}^{-1/2}\text{ e}
    \end{equation}
    \begin{equation}
        \nu_{\lsic}=(\tr{\BB{G}}\tau_{\sups})^{-1}\text{,}
    \end{equation}
    \label{eq:TermEstab}
\end{subequations}

\noindent onde $C_I$ é uma constante e:

\begin{equation}
    \BB{G}=\frac{\partial\BB{\xi}}{\partial\BB{y}}^T\frac{\partial\BB{\xi}}{\partial\BB{y}}\text{.}
\end{equation}

Já os termos $\rMi{i}$ e $\rCi$ são os resíduos associados à equação de conservação da quantidade de movimento e da continuidade, respectivamente:

\begin{subequations}
    \begin{equation}
        \rMi{i}=\rho\bigpar{\dot{\bar{u}}_i+\bar{u}_j\bar{u}_{i,j}-\bar{f}_i}-\sigma_{ji,j}\text{ e}
    \end{equation}
    \begin{equation}
        \rCi=\bar{u}_{i,i}\text{.}
    \end{equation}
\end{subequations}

Outra forma de se modelar os parâmetros estabilizadores pode ser dado por \cite{bazilevs2013computational}:

\begin{subequations}
    \begin{equation}
        \tau_{\sups}=\bigpar{\frac{1}{\tau_{\sugn 1}^2}+\frac{1}{\tau_{\sugn 2}^2}+\frac{1}{\tau_{\sugn 3}^2}}^{-1/2}\text{ e}
    \end{equation}
    \begin{equation}
        \nu_{\lsic}=\tau_{\sups}\norm{\BBB{u}}^2\text{,}
    \end{equation}
\end{subequations}

\noindent tal que:

\begin{subequations}
    \begin{align}
         & \tau_{\sugn 1}=\bigpar{\sum_{a=1}^{n_{en}}{\abs{\BBB{u}\cdot\NN N_a}}}^{-1}\text{,} \\
         & \tau_{\sugn 2}=\frac{\Delta t}{2}\text{,}                                           \\
         & \tau_{\sugn 3}=\frac{h_{\rgn}^2}{4\nu}\text{,}                                      \\
         & h_\rgn=2\bigpar{\sum_{a=1}^{n_{en}}{\abs{\BB{r}\cdot\NN N_a}}}^{-1}\text{ e}        \\
         & \BB{r}=\frac{\NN\norm{\BBB{u}}}{\norm{\NN\norm{\BBB{u}}}}\text{.}
    \end{align}
\end{subequations}

Assim, obtém-se o problema do Método Variacional Multiescala Baseado em Resíduos (\textit{Residual-Based Variational Multi-Scale} - RBVMS) que busca determinar $\BBB{u}\in\bar{\script{S}}_u$ e $\bar{p}\in\bar{\script{S}}_p$, tais que para todo $\BBB{w}\in\bar{\script{V}}_u$ e $\bar{q}\in\bar{\script{V}}_p$ \cite{bazilevs2013computational}:

\begin{equation}
    \begin{split}
        &\intDom{\bar{w}_i\rho\bigpar{\dot{\bar{u}}_i+\bar{u}_j\bar{u}_{i,j}-\bar{f}_i}}+\intDom{\bar{w}_{i,j}\bar{\sigma}_{ij}}-\intNeumann{\bar{w}_i\bar{h}_i}+\intDom{\bar{q}\bar{u}_{i,i}}-\\
        &\SintDom{\rho\bigpar{\bar{u}_j\bar{w}_{i,j}+\frac{\bar{q}_{,i}}{\rho}}u'_i}-\SintDom{\bar{w}_{i,i}p'}+\\
        &\SintDom{\rho\bar{w}_iu'_j\bar{u}_{i,j}}-\SintDom{\rho\bar{w}_{i,j}u'_iu'_j}=0
        \text{.}
        \label{eq:RBVMS1}
    \end{split}
\end{equation}

Para a discretização do problema pode-se realizar a separação da dependência espacial e temporal para os espaços tentativas e testes como:

\begin{subequations}
    \begin{align}
         & \bar{u}_i(\BB{y},t)=\sum_{\BB{\eta}^s}{U_i^a(t)N_a(\BB{y})}\text{,}             \\
         & \bar{p}(\BB{y},t)=\sum_{\BB{\eta}^s}{P^a(t)N_a(\BB{y})}\text{,}                 \\
         & \bar{w}_i(\BB{y})=\sum_{\BB{\eta}^w}{W_i^aN_a(\BB{y})}\text{ e}\label{eq:w-sep} \\
         & \bar{q}(\BB{y})=\sum_{\BB{\eta}^w}{Q^aN_a(\BB{y})}\text{.}\label{eq:q-sep}
    \end{align}
\end{subequations}

Substituindo \ref{eq:w-sep} e \ref{eq:q-sep} em \ref{eq:RBVMS1}, tais que $W^a$ e $Q^a$ são valores arbitrários, obtém-se dois vetores ($\NM=[(N_\mathrm{M})_i^a]$ e $\NC=[(N_\mathrm{C})^a]$) que representam resíduos a serem minimizados:

\begin{subequations}
    \begin{equation}
        \begin{split}
            (N_\mathrm{M})_i^a=&
            \intDom{N_a\rho\bigpar{\dot{\bar{u}}_i+\bar{u}_j\bar{u}_{i,j}-\bar{f}_i}}+\intDom{N_{a,j}\bar{\sigma}_{ij}}-\intNeumann{N_a\bar{h}_i}-\\
            &\SintDom{\rho N_{a,j}\bar{u}_ju'_i}-\SintDom{N_{a,i}p'}+\\
            &\SintDom{\rho N_a\bar{u}_{i,j}u'_j}-\SintDom{\rho N_{a,j}u'_iu'_j}
            \text{,}
        \end{split}
    \end{equation}
    \begin{equation}
        (N_\mathrm{C})^a=\intDom{N_a\bar{u}_{i,i}}-\SintDom{N_{a,i}u'_i}
    \end{equation}
    \label{Eq:Residuos-Euler}
\end{subequations}

Sendo os vetores $\BB{U}=[\BB{u}_B]$, $\dot{\BB{U}}=[\dot{\BB{u}}_B]$ e $\BB{P}=[p_B]$, que representam, respectivamente, os graus de liberdade em velocidades, primeira derivada temporal das velocidades e pressões nodais, então o problema a ser resolvido será dado por: encontrar $\BB{U}$, $\dot{\BB{U}}$ e $\BB{P}$, tais que:

\begin{subequations}
    \begin{align}
         & \NM(\BB{U},\dot{\BB{U}},\BB{P})=\BB{0}\text{ e} \\
         & \NC(\BB{U},\dot{\BB{U}},\BB{P})=\BB{0}\text{.}
    \end{align}
\end{subequations}

Uma formulação alternativa à apresentada é apresentada por \citeonline{bazilevs2013computational}, denominada como SUPG/PSPG, onde se omite os dois últimos termos da equação \ref{eq:RBVMS1} e se utiliza de valores diferentes de $\tau$ para as equações de conservação da quantidade de movimento e da continuidade, resultando em:

\begin{equation}
    \begin{split}
        &\intDom{\bar{w}_i\rho\bigpar{\dot{\bar{u}}_i+\bar{u}_j\bar{u}_{i,j}-\bar{f}_i}}+\intDom{\bar{w}_{i,j}\bar{\sigma}_{ij}}-\intNeumann{\bar{w}_i\bar{h}_i}+\intDom{\bar{q}\bar{u}_{i,i}}-\\
        &\SintDom{\tau_\supg\bar{w}_{i,j}\bar{u}_j\rMi{i}}-\SintDom{\tau_\pspg\frac{q_{,i}}{\rho}\rMi{i}}-\\
        &\SintDom{\rho\nu_\lsic\bar{w}_{i,i}\rCi}=0
        \text{,}
        \label{eq:SUPG-PSPG}
    \end{split}
\end{equation}

\noindent na qual, segundo os autores, adota-se $\tau_\pspg=\tau_\supg=\tau_\sups$ para uma boa variedade de problemas.

Por sua vez, a formulação baseada em uma descrição ALE parte das equações apresentadas em \ref{eq:NS-ALE}, cujo problema semi-discreto pode ser dado por:

\begin{equation}
    \begin{split}
        &\intDom{w_i\rho\bigpar{\dot{u}_i+(u_j-\hat{u}_j)u_{i,j}-f_j}}+\intDom{w_{i,j}\sigma_{ij}}-\\
        &\intNeumann{w_ih_i}+\intDom{qu_{i,i}}=0
    \end{split}
\end{equation}

Portanto o problema semi-discreto em formulação RBVMS de escoamentos incompressíveis segundo uma descrição ALE será: encontrar $\BBB{u}\in\bar{\script{S}}_u$ e $\bar{p}\in\bar{\script{S}}_p$, tais que para todo $\BBB{w}\in\bar{\script{V}}_u$ e $\bar{q}\in\bar{\script{V}}_p$ \cite{bazilevs2013computational}:

\begin{equation}
    \begin{split}
        &\intDom{\bar{w}_i\rho\bigpar{\dot{\bar{u}}_i+(\bar{u}_j-\hat{u}_j)\bar{u}_{i,j}-\bar{f}_i}}+\intDom{\bar{w}_{i,j}\bar{\sigma}_{ij}}-\intNeumann{\bar{w}_i\bar{h}_i}+\intDom{\bar{q}\bar{u}_{i,i}}-\\
        &\SintDom{\bigpar{\rho(\bar{u}_j-\hat{u}_j)\bar{w}_{i,j}+\bar{q}_{,i}}u'_i}-\SintDom{\bar{w}_{i,i}p'}+\\
        &\SintDom{\rho\bar{w}_i\bar{u}_{i,j}u'_j}-\SintDom{\rho\bar{w}_{i,j}u'_iu'_j}=0
        \text{,}
        \label{eq:RBVMS-ALE}
    \end{split}
\end{equation}

\noindent em que os termos estabilizadores $\tau_\sups$ e $\nu_\lsic$ são alterados das equações \ref{eq:TermEstab}, onde se considera, ao invés da velocidade do fluido $\bar{u}_i$, a velocidade relativa à malha ($\bar{u}_i-\hat{u}_i$), ou seja, $\bar{u}_i\gets\bar{u}_i-\hat{u}_i$.

Para o problema discretizado utiliza-se as seguintes expressões de aproximação dos espaços tentativas e testes:

\begin{subequations}
    \begin{align}
         & \bar{u}_i(\BB{y},t)=\sum_{\BB{\eta}^s}{U_i^a(t)N_a(\BB{y},t)}\text{,} \\
         & \bar{p}(\BB{y},t)=\sum_{\BB{\eta}^s}{P^a(t)N_a(\BB{y},t)}\text{,}     \\
         & \bar{w}_i(\BB{y})=\sum_{\BB{\eta}^w}{W_i^aN_a(\BB{y},t)}\text{ e}     \\
         & \bar{q}(\BB{y})=\sum_{\BB{\eta}^w}{Q^aN_a(\BB{y},t)}\text{,}
    \end{align}
\end{subequations}

\noindent onde as funções de forma $N_a(\BB{y},t)$ são definidas como:

\begin{equation}
    N_a(\BB{y},t)=\hat{N}_a(\bhat{f}^{-1}(\BB{y},t))\text{,}
\end{equation}

\noindent em que $\bhat{f}(\BB{y},t)$ é a função de mudança de configuração de $\hat{\Omega}\to\Omega$, conforme apresentado no item \ref{CFD-ALE}, dada em sua forma discreta por:

\begin{equation}
    \bhat{f}(\bhat{x},t)=\sum_{a\in\BB{\eta}^s}{(\bhat{x}_a+\Delta\bhat{x}_a(t))\hat{N}_a(\bhat{x})}\text{,}
\end{equation}

\noindent sendo $\bhat{x}_a$ as posições nodais em $\hat{\Omega}$, $\Delta\bhat{x}(t)$ o deslocamento nodal e $\hat{N}_a$ é a função de forma fixa da discretização de $\hat{\Omega}$. Nota-se, portanto, que as funções $N_a(\BB{y},t)$ possuem dependência temporal devido à movimentação da malha.

Com isso, define-se os vetores de resíduos da conservação de quantidade de movimento de da continuidade como:

\begin{subequations}
    \begin{equation}
        \NM=[(N_\mathrm{M})_i^a]\text{,}
    \end{equation}
    \begin{equation}
        \NC=[(N_\mathrm{C})^a]\text{,}
    \end{equation}
    \begin{equation}
        \begin{split}
            (N_\mathrm{M})_i^a=&
            \intDom{N_a\rho\bigpar{\dot{\bar{u}}_i+(\bar{u}_j-\hat{u}_j)\bar{u}_{i,j}-\bar{f}_i}}+\intDom{N_{a,j}\bar{\sigma}_{ij}}-\intNeumann{N_a\bar{h}_i}-\\
            &\SintDom{\rho N_{a,j}(\bar{u}_j-\hat{u}_j)u'_i}-\SintDom{N_{a,i}p'}+\\
            &\SintDom{\rho N_a\bar{u}_{i,j}u'_j}-\SintDom{\rho N_{a,j}u'_iu'_j}
            \text{,}
        \end{split}
    \end{equation}
    \begin{equation}
        (N_\mathrm{C})^a=\intDom{N_a\bar{u}_{i,i}}-\SintDom{N_{a,i}u'_i}
    \end{equation}
    \label{Eq:Residuos-ALE}
\end{subequations}

Assim, pretende-se determinar os vetores $\BB{U}$, $\dot{\BB{U}}$ e $\BB{P}$, tais que:

\begin{subequations}
    \begin{align}
         & \NM(\BB{U},\dot{\BB{U}},\BB{P})=\BB{0}\text{ e} \\
         & \NC(\BB{U},\dot{\BB{U}},\BB{P})=\BB{0}\text{.}
    \end{align}
\end{subequations}

%==================================================================================================
\subsubsection{Integração temporal} \label{IT-VMS}
%==================================================================================================

Como pôde-se verificar, em ambas as descrições, Euleriana e ALE, chega-se a um problema discreto no espaço, porém contínuo no tempo (Equações \ref{Eq:Residuos-Euler} e \ref{Eq:Residuos-ALE}). Dessa forma torna-se necessária a devida discretização temporal dos parâmetros, que pode ocorrer de diferentes formas, como apontado por \citeonline{reddy2010finite}, tem-se, por exemplo, o surgimento de integradores explícitos, como o integrador baseado em diferenças adiantadas, implícitos, como em diferenças finitas atrasadas, e o denominado semi-explícito (ou da regra de trapézios). Segundo o autor os integradores implícitos possuem vantagens sobre os explícitos, uma vez que: se observa a implicidade natural da pressão em escoamentos incompressíveis; deve-se ter um cuidado extra para garantir a estabilidade do integrador; apresentar problemas para a diagonalização de matrizes de massa; e perda de precisão na diagonalização.

O integrador temporal utilizado no presente trabalho é o denominado integrador $\alpha$-generalizado, desenvolvido por \citeonline{chung1993time}, que possui a capacidade de representar adequadamente problema de escoamentos incompressíveis, além de permitir a introdução de difusão numérica ao processo \cite{fernandes2020tecnica}.

Esse integrador parte da consideração de valores intermediários de aceleração e velocidade em um intervalo de tempo $[t_n,t_{n+1}]$ no $n$-ésimo passo de tempo, representados respectivamente por $\dot{\BB{U}}^{n+\alpha_m}$ e $\BB{U}^{n+\alpha_f}$:

\begin{subequations}
    \begin{equation}
        \dot{\BB{U}}^{n+\alpha_m}=\dot{\BB{U}}^n+\alpha_m(\dot{\BB{U}}^{n+1}-\dot{\BB{U}}^n)\text{ e}
    \end{equation}
    \begin{equation}
        \BB{U}^{n+\alpha_f}=\BB{U}^n+\alpha_f(\BB{U}^{n+1}-\BB{U}^n)\text{.}
    \end{equation}
\end{subequations}

Já para se relacionar a velocidade à aceleração, pode-se proceder com a aproximação de Newmark \cite{bazilevs2013computational}:

\begin{equation}
    \BB{U}^{n+1}=\BB{U}^n+\Delta t_n\bigpar{(1-\gamma)\dot{\BB{U}}^n+\gamma\dot{\BB{U}}^{n+1}}\text{,}
\end{equation}

\noindent sendo $\alpha_m$, $\alpha_f$ e $\gamma$ valores escolhidos arbitrariamente observando as necessidades de estabilidade e precisão do método.

De acordo com \citeonline{chung1993time,jansen2000generalized,bazilevs2013computational}, a precisão de segunda ordem dessa aproximação pode ser atingida uma vez que:

\begin{equation}
    \gamma=\frac{1}{2}+\alpha_m-\alpha_f\text{,}
\end{equation}

\noindent enquanto a estabilidade incondicional pode ser obtida caso:

\begin{equation}
    \alpha_m\geq\alpha_f\geq\frac{1}{2}\text{.}
\end{equation}

Ainda é possível escrever, a partir da Equação \ref{eq:one-par-stable}, $\alpha_m$ e $\alpha_f$ em termos de um parâmetro arbitrário único  ($0\leq\rho_\infty\leq1$), que representa o raio espectral de amplificação da matriz para $\Delta t\to\infty$, o qual é utilizado para controlar as dissipações de alta-frequência.

\begin{subequations}
    \begin{equation}
        \alpha_m=\frac{1}{2}\bigpar{\frac{3-\rho_\infty}{1+\rho_\infty}}\text{ e}
    \end{equation}
    \begin{equation}
        \alpha_f=\frac{1}{1+\rho_\infty}\text{.}
    \end{equation}
    \label{eq:one-par-stable}
\end{subequations}

Para o caso de $\rho_\infty=1$ não ocorre a introdução de difusão numérica, enquanto para $\rho_\infty=0$ se tem a máxima dissipação de altas frequências \cite{fernandes2020tecnica}.

Sendo assim, os resíduos obtidos anteriormente podem ser escritos em termos dos valores intermediários como:

\begin{subequations}
    \begin{equation}
        \NM(\dot{\BB{U}}^{n+\alpha_m},\BB{U}^{n+\alpha_f},\BB{P}^{n+1})=0
    \end{equation}
    \begin{equation}
        \NC(\dot{\BB{U}}^{n+\alpha_m},\BB{U}^{n+\alpha_f},\BB{P}^{n+1})=0
    \end{equation}
\end{subequations}

%==================================================================================================
\subsubsection{Procedimento iterativo} \label{Comp-VMS}
%==================================================================================================

O procedimento para minimizar os vetores resíduo obtido parte do método de Newton-Raphson, no qual os parâmetros a serem corrigidos são os vetores de acelerações nodais ($\dot{\BB{U}}$) e de pressões nodais ($\BB{P}$). Dessa forma, o problema a ser resolvido para a correção desses parâmetros é:

\begin{equation}
    \begin{bmatrix}
        \der{(N_M)_i^a}{(\dot{U}_j^b)^{n+1}} & \der{(N_M)_i^a}{(P^b)^{n+1}} \\
        \der{(N_C)^a}{(\dot{U}_j^b)^{n+1}}   & \der{(N_C)^a}{(P^b)^{n+1}}
    \end{bmatrix}
    \begin{bmatrix}
        \Delta(\dot{U}_j^b)^{n+1} \\
        \Delta (P^b)^{n+1}
    \end{bmatrix}=-
    \begin{bmatrix}
        (N_M)_i^a \\
        (N_C)^a
    \end{bmatrix}\text{,}
    \label{Eq:Newton-Raphson}
\end{equation}

\noindent em que, para uma descrição ALE:

\begin{subequations}
    \begin{equation}
        \begin{split}
            \der{(N_M)_i^a}{(\dot{U}_j^b)^{n+1}}=&\am\intDomna{\rho N_aN_b}\dij+\am\intDomna{\rho\tsups N_{a,k}N_{b}\uub{k}}\dij+\\
            &\agdt\intDomna{\rho N_aN_{b,k}\uub{k}}\dij+\agdt\intDomna{\mu N_{a,k}N_{b,k}}\dij+\\
            &\agdt\intDomna{\mu N_{a,j}N_{b,i}}+\agdt\intDomna{\rho\nlsic N_{a,i}N_{b,j}}+\\
            &\agdt\intDomna{\rho\tsups N_{a,k}N_{b,m}\uub{k}\uub{m}}\dij+\\
            &\agdt\intDomna{\rho N_aN_b\bar{u}_{i,j}}+\agdt\intDomna{\rho\tsups N_{a,k}N_b\uub{k}\bar{u}_{i,j}}-\\
            &\am\intDomna{\rho\tsups N_aN_b\bar{u}_{i,j}}-\\
            &\agdt\intDomna{\rho\tsups N_aN_{b,m}\uub{m}\bar{u}_{i,j}}-\\
            &\agdt\intDomna{\rho\tsups N_aN_b\bar{u}_{i,k}\bar{u}_{k,j}}\text{,}
        \end{split}
    \end{equation}
    \begin{equation}
        \begin{split}
            \der{(N_M)_i^a}{(P^b)^{n+1}}=&-\intDomna{N_{a,i}N_b}+\intDomna{\tsups N_{a,j}N_{b,i}\uub{j}}-\\
            &\intDomna{\tsups N_aN_{b,j}\bar{u}_{i,j}}\text{,}
        \end{split}
    \end{equation}
    \begin{equation}
        \begin{split}
            \der{(N_C)^a}{(\dot{U}_j^b)^{n+1}}=&\agdt\intDomna{N_aN_{b,j}}+\am\intDomna{\tsups N_{a,j}N_b}+\\
            &\agdt\intDomna{\tsups N_{a,j}N_{b,m}\uub{m}}+\\
            &\agdt\intDomna{\tsups N_{a,i}N_b\bar{u}_{i,j}}\text{ e}
        \end{split}
    \end{equation}
    \begin{equation}
        \der{(N_C)^a}{(P^b)^{n+1}}=\intDomna{\frac{\tsups}{\rho}N_{a,i}N_{b,i}}\text{,}
    \end{equation}
\end{subequations}

\noindent em que $\agdt=\alpha_f\gamma\Delta t$ e:

\begin{equation}
    \Omega^{n+\alpha_f}=\left\{\bar{\BB{x}}|\bar{\BB{x}}(\hat{\BB{x}},t^{n+\alpha_f})=\alpha_f\bar{\BB{x}}(\hat{\BB{x}},t^{n+1})+(1-\alpha_f)\bar{\BB{x}}(\hat{\BB{x}},t^n)\right\}\text{.}
\end{equation}

Para uma descrição Euleriana, considera-se a velocidade da malha como nula na formulação apresentada.

\end{document}