%% USPSC-Abstract.tex
%\autor{Silva, M. J.}
\begin{resumo}[Abstract]
	\begin{otherlanguage*}{english}
		\begin{flushleft}
			\setlength{\absparsep}{0pt} % ajusta o espaçamento dos parágrafos do resumo		
			\SingleSpacing  		\imprimirautorabr~~\textbf{\imprimirtitleabstract}.	\imprimirdata.  \pageref{LastPage} p.
			%Substitua p. por f. quando utilizar oneside em \documentclass
			%\pageref{LastPage} f.
			\imprimirtipotrabalhoabs~-~\imprimirinstituicao, \imprimirlocal, 	\imprimirdata.
		\end{flushleft}
		\OnehalfSpacing

		Fluid-structure interaction (FSI) problems are widely present in engineering and must be adequately considered during structural design, however, they pose several challenges to computational simulation. With the main objective of developing and evaluating methodologies to address these challenges, this work presents a study on computational tools for numerical simulation of fluid-structure interaction (FSI) problems focusing on incompressible flows with vorticity and turbulence effects interacting with elastic structures subjected to large deformations. Firstly, a formulation of the Finite Element Method (FEM) for incompressible flows with convection stabilization (SUPG) and pressure stabilization (PSPG) is explored, employing the Arbitrary Lagrangian-Eulerian (ALE) description to enable the movement of the fluid-structure interface. Subsequently, a variational multiscale (VMS) formulation and Large Eddy Simulation (LES) model are implemented to capture turbulence effects, aiming for more realistic and efficient simulations. The structures are modeled using Reissner-Mindlin shell elements, allowing the simulation of various problems with slender or thick structures. A position-based FEM formulation is employed, which naturally incorporates the effects of geometric nonlinearity. The fluid mesh is dynamically deformed using a Laplace equation-based model to maintain conformity with the fluid-structure interface while preserving mesh quality. The fluid-structure coupling is performed in a strongly partitioned manner, enabling the solution of the coupled nonlinear problem in a block-iterative fashion and ensuring modularity of the computational code. The obtained results are compared with literature results, demonstrating the effectiveness and applicability of the studied methodologies in the context of structural and fluid dynamics engineering. It is noted that the use of VMS slightly improves results compared to direct simulation employing only convection and pressure stabilizations. On the other hand, the application of LES significantly enhances result quality, particularly in problems with high Reynolds numbers, enabling stable solutions even with less refined discretization in regions of higher velocity gradients.		\vspace{\onelineskip}

		\noindent
		\textbf{Keywords}: fluid-structure interaction; turbulent flow; finite element method; large eddy simulation; variational multi-scale.
	\end{otherlanguage*}
\end{resumo}
