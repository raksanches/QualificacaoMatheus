%% USPSC-Resumo.tex
\setlength{\absparsep}{18pt} % ajusta o espaçamento dos parágrafos do resumo		
\begin{resumo}
	\begin{flushleft}
		\setlength{\absparsep}{0pt} % ajusta o espaçamento da referência	
		\SingleSpacing
		\imprimirautorabr~~\textbf{\imprimirtituloresumo}.	\imprimirdata. \pageref{LastPage} p.
		%Substitua p. por f. quando utilizar oneside em \documentclass
		%\pageref{LastPage} f.
		\imprimirtipotrabalho~-~\imprimirinstituicao, \imprimirlocal, \imprimirdata.
	\end{flushleft}
	\OnehalfSpacing

	Problemas de Interação Fluido-Estrutura (IFE) ocorrem quando um escoamento afeta o comportamento da estrutura nele imerso e vice-versa. A análise desse tipo de problema possui sua importância, uma vez que os efeitos dessa interação pode levar à falhas catastróficas, como o colapso de uma estrutura, ou perda de controle de aeronaves. No entanto a simulação numérica da IFE possui desafios a serem superados, como o alto custo computacional envolvido nas análises, principalmente em problemas de escoamentos turbulentos, trazendo a necessidade de se utilizar malhas mais refinadas para capturar a ocorrência de vórtices. Nesse contexto, o presente trabalho busca estudar e implementar o modelo de turbulência \LES\ (LES), observando sua influência em análises pouco discretizadas. Para tanto, são apresentadas as equações que governam os ecoamentos isotérmicos incompressíveis, focando na descrição Lagrangiana-Euleriana Arbitrária (ALE), a qual permite a movimentação independente do domínio de referência. A obtenção da solução numérica do escoamento é feita a partir do Método dos Elementos Finitos (MEF), podendo ser estabilizada pela formulação SUPG/PSPG ou pelo modelo variacional multiescala (VMS). O integrador temporal utilizado é o $\alpha$-generalizado, o qual permite o controle da dissipação de alta-frequência do problema de forma facilitada. A movimentação da malha é feita a partir da solução do problema de Laplace modificado, de forma a preservar a qualidade da malha. Por sua vez os problemas da dinâmica dos sólidos computacional são analisados seguindo uma abordagem do MEF baseado em posições para elementos de casca utilizando cinemática de Reissner-Mindlin. Por fim o acoplamento entre os diferentes meios é dado por meio do acoplamento particionado forte, uma vez que permite a utilização de códigos independentes que resolvam apropriadamente cada meio envolvido. Dessa maneira, os exemplos numéricos são comparados com aqueles presentes na literatura, empregando as diferentes possibilidades de estabilizações, podendo se aplicar ou não o LES. Assim verifica-se que a estabilização por VMS melhora ligeiramente os resultados em comparação com a estabilização SUPG/PSPG. Já a aplicação do LES melhora significativamente a qualidade dos resultados, principalmente em problemas com número de Reynolds elevados. Em problemas de IFE o LES não surtiu muito efeito em problemas com malhas refinadas na região de maior gradiente de velocidade, no entanto ainda melhorou a convergência do método em malhas mais grosseiras.

	\textbf{Palavras-chave:} interação fluido-estrutura; escoamento turbulento; método dos elementos finitos; \textit{large eddy simulation}; \textit{variational multi-scale}.
\end{resumo}