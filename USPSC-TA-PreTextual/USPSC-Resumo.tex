%% USPSC-Resumo.tex
\setlength{\absparsep}{18pt} % ajusta o espaçamento dos parágrafos do resumo		
\begin{resumo}
	\begin{flushleft}
		\setlength{\absparsep}{0pt} % ajusta o espaçamento da referência	
		\SingleSpacing
		\imprimirautorabr~~\textbf{\imprimirtituloresumo}.	\imprimirdata. \pageref{LastPage} p.
		%Substitua p. por f. quando utilizar oneside em \documentclass
		%\pageref{LastPage} f.
		\imprimirtipotrabalho~-~\imprimirinstituicao, \imprimirlocal, \imprimirdata.
	\end{flushleft}
	\OnehalfSpacing

	Os problemas de interação fluido-estrutura estão amplamente presentes na engenharia e devem ser adequadamente considerados durante o projeto estrutural, no entanto, apresentam diversos desafios à simulação computacional. Com o objetivo principal de desenvolver e avaliar metodologias para lidar com esses desafios, este trabalho apresenta um estudo sobre ferramentas computacionais para simulação numérica de problemas de interação fluido-estrutura (IFE) com foco em escoamentos incompressíveis com efeitos de vorticidade e turbulência interagindo com estruturas elásticas sujeitas a grandes deslocamentos. Primeiramente, explora-se uma formulação do Método dos Elementos Finitos (MEF) para escoamentos incompressíveis com estabilização dos termos convectivos (SUPG) e da pressão (PSPG), empregando a descrição Lagrangiana-Euleriana arbitrária (ALE) para permitir a movimentação da interface fluido-estrutura. Na sequência, implementa-se a formulação variacional multiescala (VMS) e o modelo \textit{Large Eddy Simulation} (LES) para capturar os efeitos de turbulência, buscando simulações mais realistas e eficientes. As estruturas são modeladas por elementos de casca de Reissner-Mindlin, permitindo a simulação de diversos problemas com estruturas esbeltas ou espessas. Emprega-se uma formulação do MEF baseada em posições, a qual naturalmente engloba os efeitos da não linearidade geométrica. A malha do fluido é deformada dinamicamente empregando-se um modelo baseado na equação de Laplace, de modo a permitir a conformidade com a interface fluido-estrutura preservando a qualidade da malha. O acoplamento fluido-estrutura é realizado de forma particionada forte, permitindo a solução do problema não linear acoplado de forma bloco-iterativa e garantindo modularidade ao código computacional. Os resultados obtidos foram comparados com resultados da literatura, demonstrando a eficácia e a aplicabilidade das metodologias estudadas no contexto da engenharia estrutural e fluidodinâmica. Nota-se que o emprego do VMS melhora ligeiramente os resultados em comparação com a simulação direta empregado SUPG/PSPG. Já a aplicação do LES melhora significativamente a qualidade dos resultados, principalmente em problemas com número de Reynolds elevados, permitindo a obtenção de solução estável mesmo com uma discretização menos refinada na região de maiores gradientes de velocidade.

	\textbf{Palavras-chave:} interação fluido-estrutura; escoamento turbulento; método dos elementos finitos; \textit{large eddy simulation}; \textit{variational multi-scale}.
\end{resumo}