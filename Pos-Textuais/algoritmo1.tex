\chapter[APÊNDICE \ref{Ap:Shell-alg}]{Algoritmo utilizado para determinação das posições de elemento de casca}
\label{Ap:Shell-alg}

\begin{algorithm}[h!]
    \caption{Algoritmo utilizado para o cálculo das posições nodais.}
    \label{alg:MEF-casca}
    \KwResult{Vetor global de parâmetros nodais}
    $\BB{Y}\gets\BB{X}$\;
    \For{$t_i\gets 0$ \KwTo $t_f$}{
    Calcular $Q^n$ e $R^n$ (Eq. \ref{eq:CSD-QnRn})\;
    \ForEach{\textnormal{passo de carga}}{
    \tcp{Cálculo das forças externas}
    \ForEach{\textnormal{elemento}}{
    \ForEach{\textnormal{ponto de Hammer }$i_h$}{
    Calcular $\BB{A}^0_{2\times 2}$ e $J_0=\det{(A^0)}$\;
    \ForEach{\textnormal{Nó $a$ do elemento na direção $i$}}{
        Calcular $\BB{F}^c$ e $\BB{F}^{nc}$ (Eq. \ref{eq:CSD-Fc} e \ref{eq:CSD-Fnc})\;
        Somar a contribuição no vetor global: $((F^c)_i^a+(F^{nc})_i^a)\cdot J_0\cdot w_{i_h}$\;
    }
    }
    }
    Somar a contribuição das forças concentradas no vetor global\;
    \While{erro>tol}{
    \ForEach{\textnormal{elemento}}{
    \ForEach{\textnormal{ponto de Hammer }$i_h$}{
    Calcular $\BB{A}^0_{3\times 3}$ (Eq. \ref{eq:CSD-A0}), $J_0=\det{(A^0)}$, $(\BB{A}^0)^{-1}_{3\times 3}$ e $\BB{A}^1_{3\times 3}$ (Eq. \ref{eq:CSD-A1})\;
    Calcular $\BB{A}_{3\times 3}=\BB{A}^1\cdot(\BB{A}^0)^{-1}$\;
    Calcular $\BB{C}_{3\times 3}=\BB{A}^T\cdot\BB{A}$\;
    Calcular $\mathbb{E}_{3\times 3}$ (Eq. \ref{eq:CSD-DefGreenLagr})\;
    Calcular $\BB{S}_{3\times 3}$ (Eq. \ref{eq:CSD-S})\;
    \ForEach{\textnormal{Nó $a$ do elemento na direção $i$}}{
        Calcular $r_i^a$ (Eq. \ref{eq:CSD-g} e \ref{eq:CSD-Fint} a \ref{eq:CSD-Finerc})\;
        Somar a contribuição no vetor global: $r_i^a\cdot J_0\cdot w_{i_h}$\;
        \ForEach{\textnormal{Nó $b$ do elemento na direção $j$}}{
            Calcular $H_{ij}^{ab}$ (Eq. \ref{eq:CSD-Hijab} e \ref{eq:CSD-HMC})\;
            Somar a contribuição na matriz global: $H_{ij}^{ab}\cdot J_0\cdot w_{ih}$\;
        }
    }
    }
    }
    Resolver o sistema global: $\BB{H}\cdot\Delta\BB{g}=-\BB{r}$\;
    Atualizar os parâmetros nodais: $\BB{g}\gets\BB{g}+\Delta\BB{g}$\;
    Atualizar derivadas temporais (Eq. \ref{eq:CSD-Newmark2})\;
    Cálculo do erro via \ref{eq:CSD-erro}\;
    }
    }
    Atualização dos valores passados\;
    }
\end{algorithm}