% Resumo

\documentclass[_ArquivoPrincipal.tex]{subfiles}

\begin{document}
	\setlength{\absparsep}{18pt} % ajusta o espaçamento dos parágrafos do resumo		
	\begin{resumo}
		\begin{flushleft} 
			\setlength{\absparsep}{0pt} % ajusta o espaçamento da referência	
			\SingleSpacing 
			\imprimirautorabr~ ~\textbf{\imprimirtitulo}.	\imprimirdata. \pageref{LastPage}f. 
			%Substitua p. por f. quando utilizar oneside em \documentclass
			%\pageref{LastPage}f.
			\imprimirtipotrabalho~-~\imprimirinstituicao, \imprimirlocal, \imprimirdata. 
		\end{flushleft}
		\OnehalfSpacing
		
		O estudo de Interação Fluido-Estrutura está tendo um grau de importância cada vez maior, uma vez que as estruturas estão se tornando cada vez mais leves e esbeltas, devido aos constantes avanços nas diversas áreas da engenharia. Dentre essas interações, enfatiza-se aquelas em que o fluido se encontra em escoamento turbulento, o que ocorre, por exemplo, em edifícios sujeitos à ação de ventos. Sendo assim, se torna necessário o desenvolvimento de técnicas cada vez mais eficientes na determinação do comportamento tanto da estrutura, quanto do escoamento dos fluidos que interagem com a mesma. Nesse sentido, observa-se a existência de uma grande variedade técnicas para determinação do comportamento de estruturas flexíveis, sendo as mais notáveis aquelas baseadas no Método dos Elementos Finitos. Dentre esses métodos, um que está ganhando destaque é aquele que considera posições nodais como parâmetros de análise, denominado de Método dos Elementos Finitos Posicional. Da mesma forma, observa-se que análise de escoamentos também pode ser realizada de diversas maneiras, tais como: construção de amostras em escalas para ensaios práticos; e a modelagem de problemas via métodos matemáticos. Nesse contexto, verifica-se que a construção de amostras reais em escala é muito dispendiosa no ponto de vista de ser necessária uma grande infraestrutura para obtenção de dados válidos, assim como, em alguns casos, os dados obtidos serem dependentes da escala da amostra, não apontando resultados reais, por consequência. Sendo assim, os modelos matemáticos se tornam a melhor solução para se analisar esses problemas. Porém, a depender do grau de complexidade do problema e/ou do modelo empregado, essa análise leva a um custo computacional muito elevado, inviabilizando sua resolução. Dessa forma, o presente trabalho busca realizar um estudo comparativo entre algumas das técnicas mais comuns de análise de Interação Fluido-Estrutura, considerando escoamentos turbulentos em estruturas flexíveis, empregando o Método dos Elementos Finitos Posicional para essa avaliação.
		
		\textbf{Palavras-chave:} Interação Fluido-Estrutura. Escoamento Turbulento. Método dos Elementos Finitos Posicional. \textit{Large Eddy Simulation}. \textit{Variational Multiscale Methods}. \textit{Reynolds-Averaged Navier-Stokes}.
	\end{resumo}
\end{document}