% Introdução

\documentclass[_ArquivoPrincipal.tex]{subfiles}

\begin{document}

\chapter{Estado Da Arte}
	
No presente capítulo serão abordados os principais temas relacionados à problemas de Interação Fluido-Estrutura (IFE), tendo como ênfase os modos de turbulência. Serão destacados os principais métodos de cálculos envolvendo a dinâmica das estruturas computacional (\ref{DEC}), a dinâmica dos fluidos computacional (\ref{DFC}), modo de acoplamento Lagrangiana-Euleriana Arbitrária (\textit{Arbitrary Lagrangian-Eulerian} - ALE) (\ref{ALE}) e alguns modelos de turbulências (\ref{MT}).

\section{Dinâmica Das Estruturas Computacional} \label{DEC}

A mecânica dos sólidos visa a determinação de parâmetros referentes à elementos estruturais sujeitos a solicitações externas, tais como tensões e deformações, ou forças e deslocamentos de determinados pontos dos mesmos. Para isso são desenvolvidos diversos métodos com a finalidade de melhor descrever o comportamento das estruturas, como o Método dos Elementos Finitos (MEF), que se apresenta como a ferramenta computacional mais amplamente utilizada para esse fim.

Nesse sentido vale ressaltar os trabalhos de \citeonline{hughes1981nonlinear, argyris1982excursion, ibrahimbegovic2002role, pimenta2004fully, battini2006choice} e \citeonline{pimenta2008exact}, que realizaram análises através do MEF Corrotacional, que se trata de uma formulação onde os deslocamentos e rotações dos elementos são os parâmetros principais da análise.

No entanto a utilização de uma análise que considera tais parâmetros nodais só se mostra eficiente ao se estudar estruturas que desenvolvem pequenos deslocamentos e deformações, uma vez que não há comutatividade de rotações \textcolor{red}{(REFERÊNCIAS)}. Ainda observa-se a avaliação dinâmica de estruturas reticuladas se torna problemática, do ponto de vista da conservação de energia \textcolor{red}{(COMENTAR MAIS) (REFERÊNCIAS)}, além da matriz de massa ser variável, tornando o processo de integração temporal muito complexo.

Com isso, vale ressaltar os trabalhos de \citeonline{coda2004simple, coda2007alternative, coda2010improved, coda2009two, coda2009unconstrained} e \citeonline{carrazedo2010alternative} que utilizam de forma bem-sucedida o MEF Posicional para análise de estruturas reticuladas e de cascas aplicadas à grandes deslocamentos. Este método se diferencia dos demais ao considerar as posições nodais como parâmetros de análise, facilitando o cálculo de efeitos de não-linearidade geométrica.

No âmbito da IFE, destaca-se a necessidade dessa consideração, uma vez que são observadas aplicações como, por exemplo, grandes amplitudes de deslocamentos, como \textit{flutter} \textcolor{red}{(REFERÊNCIAS)}, aplicações biomecânicas \textcolor{red}{(REFERÊNCIAS)}, estruturas infláveis \cite{karagiozis2011computational}, simulações de turbinas \cite{bazilevs20113d}, dentre outras.

\section{Dinâmica Dos Fluidos Computacional} \label{DFC}

\section{Acomplamento ALE} \label{ALE}

\section{Modelo De Turbulência} \label{MT}

\subsection{\textit{Variational Multiscale Methods}} \label{VMS}

\subsection{\textit{Large Eddy Simulation}} \label{LES}

\subsection{\textit{Averaged Navier-Stokes}} \label{ANS}

\subsection{\textit{Remeshing} e enriquecimento} \label{ReE}

\end{document}
	
	
